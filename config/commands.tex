%isTable=true, weil das Symbol meist vor Tabellen benutzt wird
\newboolean{isTable}
\setboolean{isTable}{true}

\newsavebox{\litbox}

\newcommand{\chaptertoc}[1][Inhaltsangabe]{
    \begingroup
        \parfillskip 0pt 
        \leftskip 0cm 
        \rightskip 1cm
        
        \etocsettocstyle{
            \addsec*{#1\linebreak
                \rule{\textwidth}{0.4pt}
            }
            \vspace{-1.5\baselineskip}
        }{}
        \setcounter{secnumdepth}{2}
        \etocsettocdepth{2}

        \etocsetstyle{section}
            {}
            {%
                \leavevmode
                \leftskip 0.5cm
                \relax
            }
            {
                \bfseries\normalsize
                \makebox[2.5em][l]{\etocnumber}%
                \etocname\nobreak
                \leaders\hbox{
                    \bfseries\normalsize
                    \hbox to 0ex {\hss.\hss}
                }
                \hfil\nobreak
                \rlap{
                    \makebox[0.75\rightskip][r]{\etocpage}
                }
                \par
                \vspace{-1\baselineskip}
            }
            {}
        
        \etocsetstyle{subsection}
        {}
        {%
            \leavevmode
            \leftskip 1cm
            \relax
        }
        {
            \mdseries\normalsize
            \makebox[3em][l]{\etocnumber}%
            \etocname\nobreak
            \leaders\hbox{
                \mdseries\normalsize
                \hbox to 0ex {\hss.\hss}
            }
            \hfill\nobreak
            \rlap{
                \makebox[0.75\rightskip][r]{\etocpage}
                }
            \par
            \vspace{-1\baselineskip}
        }
    {}

        \localtableofcontents
        \rule{\textwidth}{0.4pt}
        \vspace{1\baselineskip}
    \endgroup
}

\newenvironment{merke}{%
    \par
    \vspace{.5\baselineskip}
    \marginpar{
        \Ifthispageodd{
            \hspace*{0em}
        }
        {
            \hspace*{3em}
        }
        \raisebox{-2\baselineskip}{
            \includegraphics[scale=0.85]{lightbulb-on}
        }
    }
    \begin{lrbox}{\litbox}
        \begin{minipage}{0.949\linewidth}
            \smallskip
}{
            \smallskip
        \end{minipage}
    \end{lrbox}
    \fcolorbox{black}{lightyellow}{
        \usebox{\litbox}
    }
    \par
    \vspace{.5\baselineskip}
    \ignorespacesafterend
}


\newenvironment{oraclesql}[1][\boolean{isTable}]{
    \renewcommand*{\boolparam}{#1}
    \par
    \marginpar{
        \Ifthispageodd{
            \hspace*{0em}
        }
        {
            \hspace*{3em}
        }
        \raisebox{-3\baselineskip}{
            \includegraphics[scale=1]{oracle_11g}
        }
    }
    \begin{lrbox}{\litbox}
        \begin{minipage}{.949\linewidth}
            \smallskip
}{
            \smallskip
        \end{minipage}
    \end{lrbox}
    \usebox{\litbox}
}

\newenvironment{mssql}[1][\boolean{isTable}]{
    \renewcommand*{\boolparam}{#1}
    \par
    \marginpar{
        \Ifthispageodd{
            \hspace*{0em}
        }
        {
            \hspace*{3em}
        }
        \raisebox{-3\baselineskip}{
            \includegraphics[scale=1]{ms_sql}
        }
    }
    \begin{lrbox}{\litbox}
        \begin{minipage}{.949\linewidth}
            \smallskip
}{
            \smallskip
        \end{minipage}
    \end{lrbox}
    \usebox{\litbox}
}

\newenvironment{msoraclesql}[1][\boolean{isTable}]{
    \renewcommand*{\boolparam}{#1}
    \par
    \marginpar{
        \Ifthispageodd{
            \hspace*{0em}
        }
        {
            \hspace*{2.8em}
        }
        \raisebox{-3\baselineskip}{
            \includegraphics[scale=1]{ms_sql_oracle}
        }
    }
    \begin{lrbox}{\litbox}
        \begin{minipage}{.949\linewidth}
            \smallskip
}{
            \smallskip
        \end{minipage}
    \end{lrbox}
    \usebox{\litbox}
}
\newenvironment{literaturinternet}{
    \par
    \vspace{.5\baselineskip}
    \marginpar{
        \Ifthispageodd{
            \hspace*{0em}
        }
        {
            \hspace*{2.8em}
        }
        \raisebox{-2\baselineskip}{
            \includegraphics[scale=1]{globus}
        }
    }
    \begin{lrbox}{\litbox}
        \begin{minipage}{.949\linewidth}
            \smallskip
            \begin{small}
                \begin{itemize}
}{
                \end{itemize}
            \end{small}
            \smallskip
        \end{minipage}
    \end{lrbox}
    \fbox{\usebox{\litbox}}
    \par
    \vspace{.5\baselineskip}
    \ignorespacesafterend
}

% \newenvironment{literaturinternet}{
%   \par
%   \leaders\vbox to 2\baselineskip{%

%   }\vskip2\baselineskip
%   \marginpar{\vspace*{-1.5em}\Ifthispageodd{\hspace*{1em}}{\hspace*{3em}}\includegraphics[scale=1]{globus}}
%   \vspace{-1.5em}
%   \begin{lrbox}{\litbox}
%     \begin{minipage}{.96\linewidth}
%       \begin{small}
%         \begin{itemize}
% }{
%         \end{itemize}
%       \end{small}
%     \end{minipage}
%   \end{lrbox}
%   \fbox{\usebox{\litbox}}
%   \par
%   \addvspace{\baselineskip}
% }

\newcommand{\bild}[3]{
  \begingroup
    \par
    \setcapindent*{-0em}
    \setcapwidth[o]{0.15\linewidth}
    \settoheight{\bildhoehe}{\includegraphics[scale=#3]{#2}}
    \addtolength{\bildhoehe}{-3em}
    \addvspace{\baselineskip}
  \begin{figure}[h!t]
    \begin{captionbeside}{#1}[o][\linewidth][4.3em]*
      \parbox[t][\bildhoehe][b]{0.85\linewidth}{
      \centering\includegraphics[scale=#3]{#2}}
    \end{captionbeside}
    \label{#2}
  \end{figure}
    \par
  \endgroup
}

\newcommand{\identifier}[1]{\textsc{#1}}
\newcommand{\languageorasql}[1]{\lstinline[language=oracle_sql]{#1}}
\newcommand{\languagemssql}[1]{\lstinline[language=ms_sql]{#1}}
\newcommand{\languagerman}[1]{\lstinline[language=rman]{#1}}
\newcommand{\languageplsql}[1]{\lstinline[language=plsql]{#1}}
\newcommand{\languagesqlplus}[1]{\lstinline[language=sqlplus]{#1}}
\newcommand{\languageconfigfile}[1]{\lstinline[language=configfile]{#1}}
\newcommand{\languageexpdpimpdp}[1]{\lstinline[language=expdp_impdp]{#1}}
\newcommand{\languagepowershell}[1]{\lstinline[language=powershell]{#1}}
\newcommand{\oscommand}[1]{\texttt{#1}}
\newcommand{\privileg}[1]{\texttt{#1}}
\newcommand{\parameter}[1]{\MakeLowercase{\textsf{#1}}}

\newcommand{\SELECT}{\languageorasql{SELECT}}
\newcommand{\FROM}{\languageorasql{FROM}}
\newcommand{\WHERE}{\languageorasql{WHERE}}
\newcommand{\GROUPBY}{\languageorasql{GROUP BY}}
\newcommand{\HAVING}{\languageorasql{HAVING}}
\newcommand{\ORDERBY}{\languageorasql{ORDER BY}}
\newcommand{\CHECK}{\languageorasql{CHECK}}
\newcommand{\NOTNULL}{\languageorasql{NOT NULL}}
\newcommand{\UNIQUE}{\languageorasql{UNIQUE}}
\newcommand{\PRIMARYKEY}{\languageorasql{PRIMARY KEY}}
\newcommand{\FOREIGNKEY}{\languageorasql{FOREIGN KEY}}
\newcommand{\INSERT}{\languageorasql{INSERT}}
\newcommand{\UPDATE}{\languageorasql{UPDATE}}
\newcommand{\DELETE}{\languageorasql{DELETE}}
\newcommand{\COMMIT}{\languageorasql{COMMIT}}
\newcommand{\ROLLBACK}{\languageorasql{ROLLBACK}}
\newcommand{\GRANT}{\languageorasql{GRANT}}
\newcommand{\REVOKE}{\languageorasql{REVOKE}}
\newcommand{\DENY}{\languageorasql{DENY}}

\newenvironment{literaturbuch}{
  \marginpar{\vspace{1em}\Ifthispageodd{\hspace*{-4em}}{\hspace*{3em}}\includegraphics[scale=1]{buch}}
  \begin{lrbox}{\litbox}
    \begin{minipage}{.975\linewidth}
      \begin{small}
        \begin{itemize}
}{
        \end{itemize}
      \end{small}
    \end{minipage}
  \end{lrbox}
  \par\fbox{\usebox{\litbox}}\par
}

\newcommand{\pk}[1]{\underline{#1}}
\newcommand{\fk}[1]{$\Uparrow$#1$\Uparrow$}
\newcommand{\nn}[1]{#1 [NN]}
\newcommand{\un}[1]{#1 [UN]}

\newcommand{\changefont}[3]{\fontfamily{#1} \fontseries{#2} \fontshape{#3} \selectfont}
\newcommand{\beispiel}[1]{\hyperref[#1]{Beispiel~\ref*{#1}}}
\newcommand{\abschnitt}[1]{\hyperref[#1]{Abschnitt~\ref*{#1}}}
\newcommand{\tabelle}[1]{\hyperref[#1]{Tabelle~\ref*{#1}}}
\newcommand{\abbildung}[1]{\hyperref[#1]{Abbildung~\ref*{#1}}}
\makeatletter
\newcommand{\myhref}[2]{\hyper@linkurl{#2}{#1}}
\makeatother

