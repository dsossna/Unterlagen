\clearpage
    \section{Lösungen - Unterabfragen}
      \begin{enumerate}
        \item Schreiben Sie eine Abfrage, die für alle Eigenkunden, die keinen
        Berater haben (die nicht in der Tabelle
        \identifier{EigenkundeMitarbeiter} enthalten sind), den Vor- und den
        Nachnamen anzeigt.
        \begin{itemize}
          \item Lösen Sie die Aufgabe mit Hilfe des \languageorasql{EXISTS}-Operators!
          \item Lösen Sie die Aufgabe mit Hilfe des \languageorasql{IN}-Operators!
        \end{itemize}
        \begin{msoraclesql}[\FALSE]
        \end{msoraclesql}
        \begin{lstlisting}[language=oracle_sql]
SELECT k.Vorname, k.Nachname
FROM   Kunde k INNER JOIN Eigenkunde ek ON (k.Kunden_ID = ek.Kunden_ID)
WHERE  NOT EXISTS (SELECT 1
                   FROM   EigenkundeMitarbeiter ekm
                   WHERE  ekm.Kunden_ID = ek.Kunden_ID);

SELECT k.Vorname, k.Nachname
FROM   Kunde k INNER JOIN Eigenkunde ek ON (k.Kunden_ID = ek.Kunden_ID)
WHERE  ek.Kunden_ID NOT IN (SELECT Kunden_ID
                            FROM   EigenkundeMitarbeiter);
        \end{lstlisting}
        \item Erstellen Sie eine Abfrage, die ermittelt, ob es Mitarbeiter gibt
        (Vorname und Nachname), die keine Kundenberatung durchführen.
        Ausgenommen sind leitende Mitarbeiter (Mitarbeiter die in keiner
        Bankfiliale arbeiten) und Filialleiter.
        \begin{itemize}
          \item Lösen Sie die Aufgabe mit Hilfe des \languageorasql{EXISTS}-Operators!
          \item Lösen Sie die Aufgabe mit Hilfe des \languageorasql{IN}-Operators!
        \end{itemize}
        \begin{msoraclesql}[\FALSE]
        \end{msoraclesql}
        \begin{lstlisting}[language=oracle_sql]
SELECT m.Vorname, m.Nachname
FROM Mitarbeiter m
INNER JOIN Mitarbeiter m1 ON (m.Vorgesetzter_ID = m1.Mitarbeiter_ID)
INNER JOIN Bankfiliale b ON (m1.Bankfiliale_ID = b.Bankfiliale_ID)
WHERE NOT EXISTS (SELECT 1
                  FROM EigenkundeMitarbeiter ekm
                  WHERE m.Mitarbeiter_ID = ekm.Mitarbeiter_ID)
ORDER BY m.Vorname;


SELECT m.Vorname, m.Nachname
FROM Mitarbeiter m
INNER JOIN Mitarbeiter m1 ON (m.Vorgesetzter_ID = m1.Mitarbeiter_ID)
INNER JOIN Bankfiliale b ON (m1.Bankfiliale_ID = b.Bankfiliale_ID)
WHERE m.Mitarbeiter_ID NOT IN (SELECT Mitarbeiter_ID
                               FROM EigenkundeMitarbeiter ekm)
ORDER BY m.Vorname;
        \end{lstlisting}
\clearpage
        \item Schreiben Sie eine Abfrage, die den häufigsten Vornamen der
        Bankmitarbeiter anzeigt und wie oft dieser in der Tabelle
        \identifier{Mitarbeiter} vorkommt.
        \begin{oraclesql}[\FALSE]
        \end{oraclesql}
        \begin{lstlisting}[language=oracle_sql]
SELECT Vorname, Anzahl
FROM   (SELECT   Vorname, COUNT(Vorname) AS Anzahl
        FROM     Mitarbeiter
        GROUP BY Vorname
        ORDER BY COUNT(Vorname) DESC
       )
WHERE  rownum = 1;
        \end{lstlisting}
        \begin{mssql}[\FALSE]
        \end{mssql}
        \begin{lstlisting}[language=ms_sql]
SELECT   TOP 1 Vorname, COUNT(Vorname) AS Anzahl
FROM     Mitarbeiter
GROUP BY Vorname
ORDER BY COUNT(Vorname) DESC;
        \end{lstlisting}
        \item Schreiben Sie eine Abfrage, welche die drei Eigenkunden mit den
        niedrigsten Guthaben auf den Girokonten anzeigt.
        \begin{oraclesql}[\FALSE]
        \end{oraclesql}
        \begin{lstlisting}[language=oracle_sql]
SELECT *
FROM   (SELECT Vorname, Nachname, Guthaben
        FROM   Kunde k INNER JOIN EigenkundeKonto ekk
                 ON (k.Kunden_ID = ekk.Kunden_ID)
               INNER JOIN Girokonto g ON (ekk.Konto_ID = g.Konto_ID)
        ORDER BY Guthaben)
WHERE  rownum <= 3;
        \end{lstlisting}
        \begin{mssql}[\FALSE]
        \end{mssql}
        \begin{lstlisting}[language=ms_sql]
SELECT TOP 3 Vorname, Nachname, Guthaben
FROM   Kunde k INNER JOIN EigenkundeKonto ekk
         ON (k.Kunden_ID = ekk.Kunden_ID)
       INNER JOIN Girokonto g ON (ekk.Konto_ID = g.Konto_ID)
ORDER BY Guthaben;
        \end{lstlisting}
        \item Verändern Sie die Abfrage aus der vorangegangenen Aufgabe so,
        dass die drei Eigenkunden mit dem niedrigsten Guthaben (Girokonto +
        Sparbuch) angezeigt werden. Es müssen auch diejenigen Kunden angezeigt
        werden, die nur ein Girokonto oder nur ein Sparbuch haben!
\clearpage
        \begin{oraclesql}[\FALSE]
        \end{oraclesql}
        \begin{lstlisting}[language=oracle_sql]
SELECT *
FROM   (SELECT k.Vorname, k.Nachname, SUM(Guthaben)
        FROM   (SELECT Kunden_ID, Guthaben
                FROM   EigenkundeKonto ekk INNER JOIN Girokonto g
                         ON (ekk.Konto_ID = g.Konto_ID)
                UNION
                SELECT Kunden_ID, Guthaben
                FROM   EigenkundeKonto ekk INNER JOIN Sparbuch s
                         ON (ekk.Konto_ID = s.Konto_ID)) gut 
                       INNER JOIN Kunde k
                ON (k.Kunden_ID = gut.Kunden_ID)
        GROUP BY k.Kunden_ID, k.Vorname, k.Nachname
        ORDER BY 3)
WHERE rownum <= 3;
        \end{lstlisting}
        \begin{mssql}[\FALSE]
        \end{mssql}
        \begin{lstlisting}[language=ms_sql]
SELECT TOP 3 k.Vorname, k.Nachname, SUM(Guthaben)
FROM   (SELECT Kunden_ID, Guthaben
        FROM   EigenkundeKonto ekk INNER JOIN Girokonto g
                 ON (ekk.Konto_ID = g.Konto_ID)
        UNION
        SELECT Kunden_ID, Guthaben
        FROM   EigenkundeKonto ekk INNER JOIN Sparbuch s
                 ON (ekk.Konto_ID = s.Konto_ID)) gut
        INNER JOIN Kunde k ON (k.Kunden_ID = gut.Kunden_ID)
GROUP BY k.Kunden_ID, k.Vorname, k.Nachname
ORDER BY 3;
        \end{lstlisting}
        \item Schreiben Sie eine Abfrage, die alle Eigenkunden anzeigt, welche
        im Jahr 1985 keine Buchungen verursacht haben.
        \begin{oraclesql}[\FALSE]
        \end{oraclesql}
        \begin{lstlisting}[language=oracle_sql]
SELECT Vorname, Nachname
FROM   Kunde k INNER JOIN EigenkundeKonto ekk
         ON (k.Kunden_ID = ekk.Kunden_ID)
WHERE  NOT EXISTS (SELECT 1
                   FROM   Buchung b INNER JOIN EigenkundeKonto ekk1
                            ON (b.Konto_ID = ekk1.Konto_ID)
                   WHERE  ekk1.Kunden_ID = ekk.Kunden_ID
                     AND  Buchungsdatum BETWEEN
                          TO_DATE('01.01.1985') AND
                          TO_DATE('31.12.1985'))
GROUP BY k.Kunden_ID, Vorname, Nachname
ORDER BY Nachname, Vorname;
        \end{lstlisting}
\clearpage
        \begin{mssql}[\FALSE]
        \end{mssql}
        \begin{lstlisting}[language=ms_sql]
SELECT DISTINCT Vorname, Nachname
FROM   Kunde k INNER JOIN EigenkundeKonto ekk
         ON (k.Kunden_ID = ekk.Kunden_ID)
WHERE  k.Kunden_ID NOT IN (SELECT ek.kunden_id
                   FROM   Buchung b1 INNER JOIN EigenkundeKonto ek
													ON (b1.Konto_ID = ek.Konto_ID)
									 WHERE b1.Buchungsdatum BETWEEN
                          CONVERT(DATETIME2, '01.01.1985', 104) AND
                          CONVERT(DATETIME2, '31.12.1985', 104))
ORDER BY Nachname, Vorname;
        \end{lstlisting}
        \item Schreiben Sie eine Abfrage, die für jede Bankfiliale den
        Mitarbeiter mit dem höchsten Gehalt ausgibt.
        \begin{oraclesql}[\FALSE]
        \end{oraclesql}
        \begin{lstlisting}[language=oracle_sql]
SELECT b.Strasse || ' ' || b.Hausnummer || ' ' || b.PLZ || ' ' ||
       b.Ort AS Bankfiliale,
       m.Vorname, m.Nachname, m.Gehalt
FROM   Mitarbeiter m INNER JOIN Bankfiliale b
         ON (m.Bankfiliale_ID = b.Bankfiliale_ID)
WHERE  Gehalt = (SELECT MAX(Gehalt)
                 FROM   Mitarbeiter m1
                 WHERE  m.Bankfiliale_ID = m1.Bankfiliale_ID);
        \end{lstlisting}
        \begin{mssql}[\FALSE]
        \end{mssql}
        \begin{lstlisting}[language=ms_sql]
SELECT b.Strasse + ' ' + b.Hausnummer + ' ' + b.PLZ + ' ' +
       b.Ort AS Bankfiliale,
       m.Vorname, m.Nachname, m.Gehalt
FROM   Mitarbeiter m INNER JOIN Bankfiliale b
         ON (m.Bankfiliale_ID = b.Bankfiliale_ID)
WHERE  Gehalt = (SELECT MAX(Gehalt)
                 FROM   Mitarbeiter m1
                 WHERE  m.Bankfiliale_ID = m1.Bankfiliale_ID);
        \end{lstlisting}
        \item Schreiben Sie eine Abfrage, die für jeden Wohnort
        (\identifier{Eigenkunde.Ort}) den Kunden anzeigt, der im Jahr 1987 das
        höchste Einkommen hatte (Das Einkommen ist die Summe aller Beträge
        eines Kunden, in der Tabelle \identifier{Buchung}). Sortieren Sie die
        Abfrage nach den Wohnorten.
\clearpage
        \begin{oraclesql}[\FALSE]
        \end{oraclesql}
        \begin{lstlisting}[language=oracle_sql]
SELECT b1.Ort, b1.Vorname, b1.Nachname, b1.betrag
FROM   (SELECT ek.Kunden_ID, ek.Ort, k.Vorname, k.Nachname,
               SUM(Betrag) AS betrag
        FROM   EigenkundeKonto ekk INNER JOIN Buchung b
                 ON (ekk.Konto_ID = b.Konto_ID)
               INNER JOIN Eigenkunde ek
                 ON (ekk.Kunden_ID = ek.Kunden_ID)
               INNER JOIN Kunde k
                 ON (k.Kunden_ID = ek.Kunden_ID)
        WHERE  Buchungsdatum BETWEEN
               TO_DATE('01.01.1987', 'DD.MM.YYYY') AND
               TO_DATE('31.12.1987', 'DD.MM.YYYY')
        GROUP BY ek.Kunden_ID, ek.Ort, k.Vorname, k.Nachname) b1
WHERE   betrag = (SELECT MAX(betrag)
                  FROM   (SELECT ek.Kunden_ID, ek.Ort, SUM(Betrag) AS betrag
                            FROM   EigenkundeKonto ekk INNER JOIN Buchung b
                                     ON (ekk.Konto_ID = b.Konto_ID)
                                   INNER JOIN Eigenkunde ek
                                     ON (ekk.Kunden_ID = ek.Kunden_ID)
                            WHERE  Buchungsdatum BETWEEN
                              TO_DATE('01.01.1987', 'DD.MM.YYYY') AND
                              TO_DATE('31.12.1987', 'DD.MM.YYYY')
                            GROUP BY ek.Kunden_ID, ek.Ort) b2
                  WHERE   b1.Ort = b2.Ort)
ORDER BY b1.Ort;
        \end{lstlisting}
\clearpage
        \begin{mssql}[\FALSE]
        \end{mssql}
        \begin{lstlisting}[language=ms_sql]
SELECT b1.Ort, b1.Vorname, b1.Nachname, b1.betrag
FROM   (SELECT ek.Kunden_ID, ek.Ort, k.Vorname, k.Nachname,
               SUM(Betrag) AS betrag
        FROM   EigenkundeKonto ekk INNER JOIN Buchung b
                 ON (ekk.Konto_ID = b.Konto_ID)
               INNER JOIN Eigenkunde ek
                 ON (ekk.Kunden_ID = ek.Kunden_ID)
               INNER JOIN Kunde k
                 ON (k.Kunden_ID = ek.Kunden_ID)
        WHERE  Buchungsdatum BETWEEN
               CONVERT(DATETIME2, '01.01.1987', 104) AND
               CONVERT(DATETIME2, '31.12.1987', 104)
        GROUP BY ek.Kunden_ID, ek.Ort, k.Vorname, k.Nachname) b1
WHERE   betrag = (SELECT MAX(betrag)
                  FROM   (SELECT ek.Kunden_ID, ek.Ort, SUM(Betrag) AS betrag
                            FROM   EigenkundeKonto ekk INNER JOIN Buchung b
                                     ON (ekk.Konto_ID = b.Konto_ID)
                                   INNER JOIN Eigenkunde ek
                                     ON (ekk.Kunden_ID = ek.Kunden_ID)
                            WHERE  Buchungsdatum BETWEEN
                              CONVERT(DATETIME2, '01.01.1987', 104) AND
                              CONVERT(DATETIME2, '31.12.1987', 104)
                            GROUP BY ek.Kunden_ID, ek.Ort) b2
                  WHERE   b1.Ort = b2.Ort)
ORDER BY b1.Ort;
        \end{lstlisting}
        \item Erstellen Sie eine Abfrage, die die Umsätze der Bank
        (SUM(Buchung.Betrag)) für die Jahre 1985 bis einschließlich 1989
        als Pivottabelle anzeigt.
        \begin{oraclesql}[\FALSE]
        \end{oraclesql}
        \begin{lstlisting}[language=oracle_sql]
SELECT *
FROM   (SELECT TO_CHAR(Buchungsdatum, 'YYYY') AS Datum, Betrag
        FROM   Buchung)
PIVOT  (SUM(Betrag) FOR Datum IN ('1985', '1986', '1987', '1988', '1989'));
        \end{lstlisting}
        \begin{mssql}[\FALSE]
        \end{mssql}
        \begin{lstlisting}[language=ms_sql]
SELECT *
FROM   (SELECT DATEPART(YEAR, Buchungsdatum) AS Datum, Betrag
        FROM   Buchung) AS Sourcetable
PIVOT  (SUM(Betrag)
  FOR Datum IN ([1985], [1986], [1987], [1988], [1989])) AS Pivottable;
        \end{lstlisting}
        \item Verändern Sie die Abfrage aus der vorangegangenen Aufgabe so,
        dass die Beträge innerhalb der einzelnen Jahre nach Quartalen
        aufgeteilt werden.
\clearpage
        \begin{oraclesql}[\FALSE]
        \end{oraclesql}
        \begin{lstlisting}[language=oracle_sql]
SELECT *
FROM   (SELECT TO_CHAR(Buchungsdatum, 'Q') AS Quartal,
               TO_CHAR(Buchungsdatum, 'YYYY') AS Datum, Betrag
        FROM   Buchung)
PIVOT  (SUM(Betrag) FOR Datum IN ('1985', '1986', '1987', '1988', '1989'));
        \end{lstlisting}
        \begin{mssql}[\FALSE]
        \end{mssql}
        \begin{lstlisting}[language=oracle_sql]
SELECT *
FROM   (SELECT DATENAME(QUARTER, Buchungsdatum) AS Quartal,
               DATENAME(YEAR, Buchungsdatum) AS Datum, Betrag
        FROM   Buchung) AS Sourcetable
PIVOT  (SUM(Betrag)
  FOR Datum IN ([1985], [1986], [1987], [1988], [1989])) AS Pivottable;
        \end{lstlisting}
        \item Verändern Sie die Abfrage aus der vorangegangenen Aufgabe so,
        dass eine Summenzeile, unterhalb der Pivottabelle angezeigt wird.
        \begin{oraclesql}[\FALSE]
        \end{oraclesql}
        \begin{lstlisting}[language=oracle_sql]
SELECT *
FROM   (SELECT TO_CHAR(Buchungsdatum, 'Q') AS Quartal,
               TO_CHAR(Buchungsdatum, 'YYYY') AS Datum, Betrag
        FROM   Buchung)
PIVOT  (SUM(Betrag) FOR Datum IN ('1985', '1986', '1987', '1988', '1989'))
UNION ALL
SELECT *
FROM   (SELECT 'Summe' AS Quartal, TO_CHAR(Buchungsdatum, 'YYYY') AS Datum,
               Betrag
        FROM   Buchung)
PIVOT  (SUM(Betrag) FOR Datum IN ('1985', '1986', '1987', '1988', '1989'));
        \end{lstlisting}
        \begin{mssql}[\FALSE]
        \end{mssql}
        \begin{lstlisting}[language=ms_sql]
SELECT *
FROM   (SELECT DATENAME(QUARTER, Buchungsdatum) AS Quartal,
               DATENAME(YEAR, Buchungsdatum) AS Datum, Betrag
        FROM   Buchung) AS Sourcetable
PIVOT  (SUM(Betrag) FOR Datum IN ([1985], [1986], [1987], [1988], [1989]))
			 AS Pivottable
UNION ALL
SELECT *
FROM   (SELECT 'Summe' AS Quartal, DATENAME(YEAR, Buchungsdatum) AS Datum,
               Betrag
        FROM   Buchung) AS Sourcetable
PIVOT  (SUM(Betrag) FOR Datum IN ([1985], [1986], [1987], [1988], [1989]))
			 AS Pivottable;
        \end{lstlisting}
      \end{enumerate}
\clearpage
