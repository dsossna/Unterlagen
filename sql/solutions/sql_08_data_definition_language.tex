\clearpage
    \section{Lösungen - Erstellen von Views}
      \begin{enumerate}
        \item Erstellen Sie die View \identifier{v\_Arbeitsort}. Diese muss für
        jeden Mitarbeiter den Vorname, den Nachnamen, die Bankfiliale\_ID und
        den Ort anzeigen, an dem sich die Filiale befindet.
        \begin{msoraclesql}[\FALSE]
        \end{msoraclesql}
        \begin{lstlisting}[language=oracle_sql]
CREATE VIEW v_Arbeitsort
AS
  SELECT m.Vorname, m.Nachname, m.Bankfiliale_ID, b.Ort
  FROM   Mitarbeiter m INNER JOIN Bankfiliale b
           ON (b.Bankfiliale_ID = m.Bankfiliale_ID);
        \end{lstlisting}
        \item Erstellen Sie die View \identifier{v\_Depotbesitzer}, die zu jedem
        Eigenkunden, der ein Depot besitzt, seinen Vor- und Nachnamen, die
        Strasse mit der Hausnummer, sowie PLZ und Ort anzeigt.
        \begin{oraclesql}[\FALSE]
        \end{oraclesql}
        \begin{lstlisting}[language=oracle_sql]
CREATE VIEW v_Depotbesitzer
AS
  SELECT k.Vorname, k.Nachname, ek.Strasse || ' ' || 
         ek.Hausnummer AS Strasse,
         ek.PLZ, ek.Ort
  FROM   Kunde k INNER JOIN Eigenkunde ek
           ON (k.Kunden_ID = ek.Kunden_ID)
         INNER JOIN EigenkundeKonto ekk
           ON (ek.Kunden_ID = ekk.Kunden_ID)
         INNER JOIN Depot d
           ON (ekk.Konto_ID = d.Konto_ID);
        \end{lstlisting}
        \begin{mssql}[\FALSE]
        \end{mssql}
        \begin{lstlisting}[language=ms_sql]
CREATE VIEW v_Depotbesitzer
AS
  SELECT k.Vorname, k.Nachname, ek.Strasse + ' ' + ek.Hausnummer AS Strasse,
         ek.PLZ, ek.Ort
  FROM   Kunde k INNER JOIN Eigenkunde ek
           ON (k.Kunden_ID = ek.Kunden_ID)
         INNER JOIN EigenkundeKonto ekk
           ON (ek.Kunden_ID = ekk.Kunden_ID)
         INNER JOIN Depot d
           ON (ekk.Konto_ID = d.Konto_ID);
        \end{lstlisting}
\clearpage
        \item Erstellen Sie die View \identifier{v\_Finanzberater}, die für
        alle Eigenkunden deren Vor- und Nachnamen anzeigt, sowie den Vor- und
        den Nachnamen ihres persönlichen Finanzberaters (Tabelle
        \identifier{EigenkundeMitarbeiter}).
        \begin{msoraclesql}[\FALSE]
        \end{msoraclesql}
        \begin{lstlisting}[language=oracle_sql]
CREATE VIEW v_Finanzberater
("Vorname Kunde", "Nachname Kunde", "Vorname Berater", "Nachname Berater")
AS
  SELECT   k.Vorname, k.Nachname, m.Vorname, m.Nachname
  FROM     Kunde k INNER JOIN Eigenkunde ek ON (k.Kunden_ID = ek.Kunden_ID)
           LEFT OUTER JOIN EigenkundeMitarbeiter ekm
             ON (ek.Kunden_ID = ekm.Kunden_ID)
           LEFT OUTER JOIN Mitarbeiter m 
             ON (ekm.Mitarbeiter_ID = m.Mitarbeiter_ID);
        \end{lstlisting}
        \item Erstellen Sie die View \identifier{v\_Unterstellungsverhaeltnis}, die zu jedem Mitarbeiter (Vorname, Nachname) den Vor- und den Nachnamen seines Vorgesetzten anzeigt. Wichtig ist, dass alle Mitarbeiter, auch Herr Max Winter, der keinen Vorgestzten hat, angezeigt werden.
        \begin{msoraclesql}[\FALSE]
        \end{msoraclesql}
        \begin{lstlisting}[language=oracle_sql]
CREATE VIEW v_Unterstellungsverhaeltnis
AS
  SELECT   m.Vorname, m.Nachname, v.Vorname, v.Nachname
  FROM     Mitarbeiter m LEFT OUTER JOIN Mitarbeiter v
             ON (m.Vorgesetzter_ID = v.Mitarbeiter_ID)
        \end{lstlisting}
        \item Erstellen Sie die View \identifier{v\_Innendienstmitarbeiter}, die ermittelt, ob es Mitarbeiter gibt (Vorname und Nachname), die keine Kundenberatung durchführen. Ausgenommen sind leitende Mitarbeiter (Mitarbeiter die in keiner Bankfiliale arbeiten) und Filialleiter.
        \begin{oraclesql}[\FALSE]
        \end{oraclesql}
        \begin{lstlisting}[language=oracle_sql]
CREATE VIEW v_Innendienstmitarbeiter
AS
  SELECT m.Vorname, m.Nachname
  FROM   Mitarbeiter m LEFT OUTER JOIN EigenkundeMitarbeiter ekm
           ON (m.Mitarbeiter_ID = ekm.Mitarbeiter_ID)
  WHERE  ekm.Mitarbeiter_ID IS NULL
  MINUS
  SELECT DISTINCT v.Vorname, v.Nachname
  FROM   Mitarbeiter m INNER JOIN Mitarbeiter v
           ON (m.Vorgesetzter_ID = v.Mitarbeiter_ID);
        \end{lstlisting}
\clearpage
        \begin{mssql}[\FALSE]
        \end{mssql}
        \begin{lstlisting}[language=ms_sql]
CREATE VIEW v_Innendienstmitarbeiter
AS
  SELECT m.Vorname, m.Nachname
  FROM   Mitarbeiter m LEFT OUTER JOIN EigenkundeMitarbeiter ekm
           ON (m.Mitarbeiter_ID = ekm.Mitarbeiter_ID)
  WHERE  ekm.Mitarbeiter_ID IS NULL
  EXCEPT
  SELECT DISTINCT v.Vorname, v.Nachname
  FROM   Mitarbeiter m INNER JOIN Mitarbeiter v
           ON (m.Vorgesetzter_ID = v.Mitarbeiter_ID);
        \end{lstlisting}
        \item Erstellen Sie die View \identifier{v\_Girokontoinhaber}, die alle Eigenkunden anzeigt, die nur ein Girokonto besitzen.
        \begin{oraclesql}[\FALSE]
        \end{oraclesql}
        \begin{lstlisting}[language=oracle_sql]
CREATE VIEW v_Girokontoinhaber
AS
  SELECT k.Vorname, k.Nachname
  FROM   Kunde k INNER JOIN Eigenkunde ek ON (k.Kunden_ID = ek.Kunden_ID)
         INNER JOIN EigenkundeKonto ekk ON (ek.Kunden_ID = ekk.Kunden_ID)
         INNER JOIN Girokonto g ON (ekk.Konto_ID = g.Konto_ID)
  MINUS
  SELECT k.Vorname, k.Nachname
  FROM   Kunde k INNER JOIN Eigenkunde ek ON (k.Kunden_ID = ek.Kunden_ID)
         INNER JOIN EigenkundeKonto ekk ON (ek.Kunden_ID = ekk.Kunden_ID)
         INNER JOIN Sparbuch s ON (ekk.Konto_ID = s.Konto_ID)
  MINUS
  SELECT k.Vorname, k.Nachname
  FROM   Kunde k INNER JOIN Eigenkunde ek ON (k.Kunden_ID = ek.Kunden_ID)
         INNER JOIN EigenkundeKonto ekk ON (ek.Kunden_ID = ekk.Kunden_ID)
         INNER JOIN Depot d ON (ekk.Konto_ID = d.Konto_ID);
        \end{lstlisting}
        \begin{mssql}[\FALSE]
        \end{mssql}
        \begin{lstlisting}[language=ms_sql]
CREATE VIEW v_Girokontoinhaber
AS
  SELECT k.Vorname, k.Nachname
  FROM   Kunde k INNER JOIN Eigenkunde ek ON (k.Kunden_ID = ek.Kunden_ID)
         INNER JOIN EigenkundeKonto ekk ON (ek.Kunden_ID = ekk.Kunden_ID)
         INNER JOIN Girokonto g ON (ekk.Konto_ID = g.Konto_ID)
  EXCEPT
  SELECT k.Vorname, k.Nachname
  FROM   Kunde k INNER JOIN Eigenkunde ek ON (k.Kunden_ID = ek.Kunden_ID)
         INNER JOIN EigenkundeKonto ekk ON (ek.Kunden_ID = ekk.Kunden_ID)
         INNER JOIN Sparbuch s ON (ekk.Konto_ID = s.Konto_ID)
  EXCEPT
  SELECT k.Vorname, k.Nachname
  FROM   Kunde k INNER JOIN Eigenkunde ek ON (k.Kunden_ID = ek.Kunden_ID)
         INNER JOIN EigenkundeKonto ekk ON (ek.Kunden_ID = ekk.Kunden_ID)
         INNER JOIN Depot d ON (ekk.Konto_ID = d.Konto_ID);
        \end{lstlisting}
      \end{enumerate}
\clearpage
