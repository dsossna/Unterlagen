\clearpage
    \section{Lösungen - Selektieren und sortieren}
      \begin{enumerate}
        \item Erstellen Sie eine Abfrage, die die Konto\_ID und das aktuelle
        Guthaben des Girokontos der Bankkunden anzeigt, die weniger als 1000 EUR
        Guthaben besitzen.
        \begin{msoraclesql}[\FALSE]
        \end{msoraclesql}
        \begin{lstlisting}[language=oracle_sql]
SELECT Konto_ID, Guthaben
FROM   Girokonto
WHERE  Guthaben < 1000;
        \end{lstlisting}
        \item Erstellen Sie eine Abfrage, die die Mitarbeiter\_ID und den
        Nachnamen der Mitarbeiter mit der Vorgesetzter\_ID \enquote{2} anzeigt.
        \begin{msoraclesql}[\FALSE]
        \end{msoraclesql}
        \begin{lstlisting}[language=oracle_sql]
SELECT Mitarbeiter_ID, Nachname
FROM   Mitarbeiter
WHERE  Vorgesetzter_ID = 2;
        \end{lstlisting}
        \item Erstellen Sie eine Abfrage, die die Konto\_ID und das aktuelle
        Guthaben des Girokontos der Bankkunden anzeigt, deren Guthaben nicht
        zwischen 1000 EUR und 1500 EUR liegt.
        \begin{msoraclesql}[\FALSE]
        \end{msoraclesql}
        \begin{lstlisting}[language=oracle_sql]
SELECT Konto_ID, Guthaben
FROM   Girokonto
WHERE  Guthaben NOT BETWEEN 1000 AND 1500;
        \end{lstlisting}
        \item Lassen Sie sich die Kunden\_ID und das Geburtsdatum aller
        Eigenkunden anzeigen, die zwischen dem 20. Februar 1980 und dem 02.
        März 1988 geboren sind. Zusätzlich soll die Abfrage nach
        dem Geburtsdatum in aufsteigender Reihenfolge sortiert werden.
        \begin{msoraclesql}[\FALSE]
        \end{msoraclesql}
        \begin{lstlisting}[language=oracle_sql]
SELECT   Kunden_ID, Geburtsdatum
FROM     Eigenkunde
WHERE    Geburtsdatum BETWEEN '20.02.1980' AND '02.03.1988'
ORDER BY Geburtsdatum;
        \end{lstlisting}
         \item Zeigen Sie, in alphabetischer Reihenfolge, die Mitarbeiter\_ID
         und den Nachnamen der Mitarbeiter an, die die Vorgesetzter\_ID
         \enquote{5} oder \enquote{6} haben.
        \begin{msoraclesql}[\FALSE]
        \end{msoraclesql}
        \begin{lstlisting}[language=oracle_sql]
SELECT   Mitarbeiter_ID, Nachname
FROM     Mitarbeiter
WHERE    Vorgesetzter_ID IN (5, 6)
ORDER BY Nachname;
        \end{lstlisting}
        \item Erstellen Sie eine Abfrage, die den Nachnamen und die
        Bankfiliale\_ID der Mitarbeiter ausgibt, die die Vorgesetzten\_ID
        \enquote{5} oder \enquote{6} haben und deren Bankfiliale\_ID zwischen
        \enquote{10} und \enquote{20} ist. Die Spalten sollen mit
        \enquote{Mitarbeiter} und \enquote{Bankfiliale} benannt werden.
        \begin{msoraclesql}[\FALSE]
        \end{msoraclesql}
        \begin{lstlisting}[language=oracle_sql]
SELECT Nachname AS "Mitarbeiter", Bankfiliale_ID AS "Bankfiliale"
FROM   Mitarbeiter
WHERE  Vorgesetzter_ID IN (5, 6)
  AND  Bankfiliale_ID BETWEEN 10 AND 20;
        \end{lstlisting}
        \item Zeigen Sie die Mitarbeiter\_ID und den Nachnamen des Mitarbeiters
        an, der keinen Vorgesetzten hat.
        \begin{msoraclesql}[\FALSE]
        \end{msoraclesql}
        \begin{lstlisting}[language=oracle_sql]
SELECT Mitarbeiter_ID, Nachname
FROM   Mitarbeiter
WHERE  Vorgesetzter_ID IS NULL;
        \end{lstlisting}
        \item Zeigen Sie die Kunden\_ID und das Geburtsdatum derjenigen
        Eigenkunden an, die im Jahre 1980 geboren sind.
        \begin{msoraclesql}[\FALSE]
        \end{msoraclesql}
        \begin{lstlisting}[language=oracle_sql]
SELECT Kunden_ID, Geburtsdatum
FROM   Eigenkunde
WHERE  Geburtsdatum BETWEEN '01.01.1980' AND '31.12.1980';
        \end{lstlisting}
        \item Erstellen Sie eine Abfrage, die den Nachnamen, das Gehalt und die
        Provision für alle Mitarbeiter anzeigt, die eine Provision erhalten.
        Sortieren Sie die Ausgabe in absteigender Reihenfolge nach dem Gehalt.
        \begin{msoraclesql}[\FALSE]
        \end{msoraclesql}
        \begin{lstlisting}[language=oracle_sql]
SELECT   Nachname, Gehalt, Provision
FROM     Mitarbeiter
WHERE    Provision IS NOT NULL
ORDER BY Gehalt DESC;
        \end{lstlisting}
        \item Zeigen Sie die Nachnamen aller Mitarbeiter an, in deren Nachname
        an dritter Stelle ein \enquote{a} vorkommt.
        \begin{msoraclesql}[\FALSE]
        \end{msoraclesql}
        \begin{lstlisting}[language=oracle_sql]
SELECT Nachname
FROM   Mitarbeiter
WHERE  Nachname LIKE '__a%';
        \end{lstlisting}
\clearpage
        \item Zeigen Sie die Nachnamen aller Mitarbeiter an, deren Nachname ein
        kleines \enquote{a} und ein kleines \enquote{e} enthält.
        \begin{msoraclesql}[\FALSE]
        \end{msoraclesql}
        \begin{lstlisting}[language=oracle_sql]
SELECT Nachname
FROM   Mitarbeiter
WHERE  Nachname LIKE '%a%'
  AND  Nachname LIKE '%e%';
        \end{lstlisting}
      \end{enumerate}
