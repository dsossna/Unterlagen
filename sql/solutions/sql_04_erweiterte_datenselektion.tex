\clearpage
    \section{Lösungen - Erweiterte Datenselektion}
      \begin{enumerate}
        \item Schreiben Sie eine Abfrage, die für jeden Mitarbeiter den
        Vornamen, den Nachnamen, die Bankfiliale\_ID und den Ort anzeigt, an dem
        sich seine Filiale befindet.
        \begin{msoraclesql}[\FALSE]
        \end{msoraclesql}
        \begin{lstlisting}[language=oracle_sql]
SELECT m.Vorname, m.Nachname, m.Bankfiliale_ID, b.Ort
FROM   Mitarbeiter m INNER JOIN Bankfiliale b
         ON (b.Bankfiliale_ID = m.Bankfiliale_ID);
        \end{lstlisting}
        \item Schreiben Sie eine Abfrage, welche die Mitarbeiternummer, den
        Nachnamen, das Gehalt und ein um 3,5 \% erhöhtes Gehalt für alle
        Mitarbeiter anzeigt, die in einer Filiale in \enquote{Aschersleben}
        arbeiten. Das erhöhte Gehalt soll als ganze Zahl und mit dem
        Spaltenalias \enquote{Neues Gehalt} ausgegeben werden.
        \begin{msoraclesql}[\FALSE]
        \end{msoraclesql}
        \begin{lstlisting}[language=oracle_sql]
SELECT m.Mitarbeiter_ID, m.Nachname, m.Gehalt,
       ROUND(m.Gehalt * 1.035, 0) AS "Neues Gehalt"
FROM   Mitarbeiter m INNER JOIN Bankfiliale b
         ON (m.Bankfiliale_ID = b.Bankfiliale_ID)
WHERE  LOWER(b.Ort) LIKE 'aschersleben';
        \end{lstlisting}
        \item Erstellen Sie eine Abfrage, die zu jedem Eigenkunden, der ein
        Depot besitzt, seinen Vor- und Nachnamen, die Strasse mit der
        Hausnummer, sowie PLZ und Ort anzeigt.
        \begin{oraclesql}[\FALSE]
        \end{oraclesql}
        \begin{lstlisting}[language=oracle_sql]
SELECT k.Vorname, k.Nachname, ek.Strasse || ' ' || ek.Hausnummer AS Strasse,
       ek.PLZ, ek.Ort
FROM   Kunde k INNER JOIN Eigenkunde ek
         ON (k.Kunden_ID = ek.Kunden_ID)
       INNER JOIN EigenkundeKonto ekk
         ON (ek.Kunden_ID = ekk.Kunden_ID)
       INNER JOIN Depot d
         ON (ekk.Konto_ID = d.Konto_ID);
        \end{lstlisting}
\clearpage
        \begin{mssql}[\FALSE]
        \end{mssql}
        \begin{lstlisting}[language=ms_sql]
SELECT k.Vorname, k.Nachname, ek.Strasse + ' ' + ek.Hausnummer AS Strasse,
       ek.PLZ, ek.Ort
FROM   Kunde k INNER JOIN Eigenkunde ek
         ON (k.Kunden_ID = ek.Kunden_ID)
       INNER JOIN EigenkundeKonto ekk
         ON (ek.Kunden_ID = ekk.Kunden_ID)
       INNER JOIN Depot d
         ON (ekk.Konto_ID = d.Konto_ID);
        \end{lstlisting}

        \item Schreiben Sie eine Abfrage, die für alle Eigenkunden deren Vor-
        und Nachnamen anzeigt, sowie den Vor- und den Nachnamen ihres
        persönlichen Finanzberaters (Tabelle
        \identifier{EigenkundeMitarbeiter}). Sortieren Sie die Abfrage nach den
        Nachnamen der Finanzberater.
        \begin{msoraclesql}[\FALSE]
        \end{msoraclesql}
        \begin{lstlisting}[language=oracle_sql]
SELECT   k.Vorname AS "Vorname Kunde", k.Nachname AS "Nachname Kunde",
         m.Vorname AS "Vorname Berater", m.Nachname AS "Nachname Berater"
FROM     Kunde k INNER JOIN Eigenkunde ek
           ON (k.Kunden_ID = ek.Kunden_ID)
         INNER JOIN EigenkundeMitarbeiter ekm
           ON (ek.Kunden_ID = ekm.Kunden_ID)
         INNER JOIN Mitarbeiter m
           ON (ekm.Mitarbeiter_ID = m.Mitarbeiter_ID)
ORDER BY m.Nachname;
        \end{lstlisting}
        \item Schreiben Sie eine Abfrage, die für alle Eigenkunden, die keinen
        Berater haben (die nicht in der Tabelle
        \identifier{EigenkundeMitarbeiter} enthalten sind), den Vor- und den
        Nachnamen anzeigt.
        \begin{msoraclesql}[\FALSE]
        \end{msoraclesql}
        \begin{lstlisting}[language=oracle_sql]
SELECT k.Vorname, k.Nachname
FROM   Kunde k INNER JOIN Eigenkunde ek
         ON (k.Kunden_ID = ek.Kunden_ID)
       LEFT OUTER JOIN EigenkundeMitarbeiter ekm
         ON (ek.Kunden_ID = ekm.Kunden_ID)
WHERE  ekm.Kunden_ID IS NULL;
        \end{lstlisting}
        \item Schreiben Sie eine Abfrage, die zu jedem Mitarbeiter (Vorname,
        Nachname) den Vor- und den Nachnamen seines Vorgesetzten anzeigt.
\clearpage
        \begin{msoraclesql}[\FALSE]
        \end{msoraclesql}
        \begin{lstlisting}[language=oracle_sql]
SELECT m.Vorname AS Vorname_M, m.Nachname AS Nachname_M,
       v.Vorname AS Vorname_V, v.Nachname AS Nachname_V
FROM   Mitarbeiter m INNER JOIN Mitarbeiter v
         ON (m.Vorgesetzter_ID = v.Mitarbeiter_ID);
        \end{lstlisting}
        \item Verändern Sie die Abfrage aus der vorangegangenen Aufgabe so,
        dass alle Mitarbeiter, einschließlich des Mitarbeiters
        \enquote{Winter}, der keinen Vorgesetzten hat, angezeigt werden.
        Sortieren Sie das Ergebnis aufsteigend nach der Vorgesetzten\_ID. Der
        Mitarbeiter \enquote{Winter} soll ganz oben auf der Liste stehen.
        \begin{oraclesql}[\FALSE]
        \end{oraclesql}
        \begin{lstlisting}[language=oracle_sql]
SELECT   m.Vorname AS Vorname_M, m.Nachname AS Nachname_M,
         v.Vorname AS Vorname_V, v.Nachname AS Nachname_V
FROM     Mitarbeiter m LEFT OUTER JOIN Mitarbeiter v
           ON (m.Vorgesetzter_ID = v.Mitarbeiter_ID)
ORDER BY m.Vorgesetzter_ID NULLS FIRST;
        \end{lstlisting}
        \begin{mssql}[\FALSE]
        \end{mssql}
        \begin{lstlisting}[language=ms_sql]
SELECT   m.Vorname AS Vorname_M, m.Nachname AS Nachname_M,
         v.Vorname AS Vorname_V, v.Nachname AS Nachname_V
FROM     Mitarbeiter m LEFT OUTER JOIN Mitarbeiter v
           ON (m.Vorgesetzter_ID = v.Mitarbeiter_ID)
ORDER BY m.Vorgesetzter_ID;
        \end{lstlisting}
        \item Erstellen Sie eine Abfrage, die ermittelt, ob es Mitarbeiter gibt,
        die keine Kundenberatung durchführen. Ausgenommen sind leitende
        Mitarbeiter (Mitarbeiter die in keiner Bankfiliale arbeiten).
        \begin{msoraclesql}[\FALSE]
        \end{msoraclesql}
        \begin{lstlisting}[language=oracle_sql]
SELECT m.Vorname, m.Nachname
FROM   Mitarbeiter m LEFT OUTER JOIN EigenkundeMitarbeiter ekm
         ON (m.Mitarbeiter_ID = ekm.Mitarbeiter_ID)
WHERE  ekm.Mitarbeiter_ID IS NULL
  AND  m.Bankfiliale_ID IS NOT NULL;
        \end{lstlisting}
        \item Schreiben Sie eine Abfrage, die für alle Mitarbeiter, die
        höchstens 3 Jahre älter, aber keinesfalls jünger sind als ihr
        Vorgesetzter, den Vornamen, den Nachnamen, das Geburtsdatum und das
        Geburtsdatum des Vorgesetzten anzeigt.
\clearpage
        \begin{oraclesql}[\FALSE]
        \end{oraclesql}
        \begin{lstlisting}[language=oracle_sql]
SELECT m.Vorname, m.Nachname, m.Geburtsdatum,
       v.Geburtsdatum AS "Geburtstag Chef"
FROM   Mitarbeiter m INNER JOIN Mitarbeiter v
         ON (m.Vorgesetzter_ID = v.Mitarbeiter_ID)
WHERE  m.Geburtsdatum BETWEEN v.Geburtsdatum - INTERVAL '3' YEAR AND
                              v.Geburtsdatum;
        \end{lstlisting}
        \begin{mssql}[\FALSE]
        \end{mssql}
        \begin{lstlisting}[language=ms_sql]
SELECT m.Vorname, m.Nachname, m.Geburtsdatum,
       v.Geburtsdatum AS "Geburtstag Chef"
FROM   Mitarbeiter m INNER JOIN Mitarbeiter v
         ON (m.Vorgesetzter_ID = v.Mitarbeiter_ID)
WHERE  m.Geburtsdatum BETWEEN DATEADD(YEAR, -3, v.Geburtsdatum) AND
                              v.Geburtsdatum;
        \end{lstlisting}
        \item Schreiben Sie eine Abfrage, die für alle Mitarbeiter, die am
        gleichen Ort arbeiten, an dem sie auch wohnen, deren Vorname, Nachname
        den Wohnort und den Arbeitsort anzeigt. Beschriften Sie die Spalten, wie
        es in der Lösung zu sehen ist. Sortieren Sie die Abfragen in
        absteigender Reihenfolge nach dem Wohnort.
        \begin{msoraclesql}[\FALSE]
        \end{msoraclesql}
        \begin{lstlisting}[language=oracle_sql]
SELECT   m.Vorname, m.Nachname, m.Ort AS "Wohnort", b.Ort AS "Arbeitsort"
FROM     Mitarbeiter m INNER JOIN Bankfiliale b
           ON (m.Bankfiliale_ID = b.Bankfiliale_ID)
WHERE    m.Ort = b.Ort
ORDER BY m.Ort DESC;
        \end{lstlisting}
\clearpage
        \item Erstellen Sie eine Abfrage, die ermittelt, ob es Mitarbeiter gibt
        (Vorname und Nachname), die keine Kundenberatung durchführen.
        Ausgenommen sind leitende Mitarbeiter (Mitarbeiter die in keiner
        Bankfiliale arbeiten) und Filialleiter.
        \begin{oraclesql}[\FALSE]
        \end{oraclesql}
        \begin{lstlisting}[language=oracle_sql]
SELECT m.Vorname, m.Nachname
FROM   Mitarbeiter m LEFT OUTER JOIN EigenkundeMitarbeiter ekm
         ON (m.Mitarbeiter_ID = ekm.Mitarbeiter_ID)
WHERE  ekm.Mitarbeiter_ID IS NULL
MINUS
SELECT DISTINCT v.Vorname, v.Nachname
FROM   Mitarbeiter m INNER JOIN Mitarbeiter v
         ON (m.Vorgesetzter_ID = v.Mitarbeiter_ID);
        \end{lstlisting}
        \begin{mssql}[\FALSE]
        \end{mssql}
        \begin{lstlisting}[language=ms_sql]
SELECT m.Vorname, m.Nachname
FROM   Mitarbeiter m LEFT OUTER JOIN EigenkundeMitarbeiter ekm
         ON (m.Mitarbeiter_ID = ekm.Mitarbeiter_ID)
WHERE  ekm.Mitarbeiter_ID IS NULL
EXCEPT
SELECT DISTINCT v.Vorname, v.Nachname
FROM   Mitarbeiter m INNER JOIN Mitarbeiter v
         ON (m.Vorgesetzter_ID = v.Mitarbeiter_ID);
        \end{lstlisting}
        \item Erstellen Sie eine Abfrage, die alle Eigenkunden anzeigt, die nur
        Girokonten aber keine anderen Konten besitzen.
        \begin{oraclesql}[\FALSE]
        \end{oraclesql}
        \begin{lstlisting}[language=oracle_sql]
SELECT k.Vorname, k.Nachname
FROM   Kunde k INNER JOIN Eigenkunde ek ON (k.Kunden_ID = ek.Kunden_ID)
       INNER JOIN EigenkundeKonto ekk ON (ek.Kunden_ID = ekk.Kunden_ID)
       INNER JOIN Girokonto g ON (ekk.Konto_ID = g.Konto_ID)
MINUS
SELECT k.Vorname, k.Nachname
FROM   Kunde k INNER JOIN Eigenkunde ek ON (k.Kunden_ID = ek.Kunden_ID)
       INNER JOIN EigenkundeKonto ekk ON (ek.Kunden_ID = ekk.Kunden_ID)
       INNER JOIN Sparbuch s ON (ekk.Konto_ID = s.Konto_ID)
MINUS
SELECT k.Vorname, k.Nachname
FROM   Kunde k INNER JOIN Eigenkunde ek ON (k.Kunden_ID = ek.Kunden_ID)
       INNER JOIN EigenkundeKonto ekk ON (ek.Kunden_ID = ekk.Kunden_ID)
       INNER JOIN Depot d ON (ekk.Konto_ID = d.Konto_ID);
        \end{lstlisting}
\clearpage
        \begin{mssql}[\FALSE]
        \end{mssql}
        \begin{lstlisting}[language=ms_sql]
SELECT k.Vorname, k.Nachname
FROM   Kunde k INNER JOIN Eigenkunde ek ON (k.Kunden_ID = ek.Kunden_ID)
       INNER JOIN EigenkundeKonto ekk ON (ek.Kunden_ID = ekk.Kunden_ID)
       INNER JOIN Girokonto g ON (ekk.Konto_ID = g.Konto_ID)
EXCEPT
SELECT k.Vorname, k.Nachname
FROM   Kunde k INNER JOIN Eigenkunde ek ON (k.Kunden_ID = ek.Kunden_ID)
       INNER JOIN EigenkundeKonto ekk ON (ek.Kunden_ID = ekk.Kunden_ID)
       INNER JOIN Sparbuch s ON (ekk.Konto_ID = s.Konto_ID)
EXCEPT
SELECT k.Vorname, k.Nachname
FROM   Kunde k INNER JOIN Eigenkunde ek ON (k.Kunden_ID = ek.Kunden_ID)
       INNER JOIN EigenkundeKonto ekk ON (ek.Kunden_ID = ekk.Kunden_ID)
       INNER JOIN Depot d ON (ekk.Konto_ID = d.Konto_ID);
        \end{lstlisting}
        \item Erstellen Sie mit Hilfe einer Abfrage eine Liste, die den Vor- und
        den Nachnamen aller Kunden enthält, die sowohl ein Sparbuch, als auch
        ein Depot besitzten. Ob die Kunden ein Girokonto haben oder nicht ist
        irrelevant.
        \begin{msoraclesql}[\FALSE]
        \end{msoraclesql}
        \begin{lstlisting}[language=oracle_sql]
SELECT k.Vorname, k.Nachname
FROM   Kunde k INNER JOIN Eigenkunde ek ON (k.Kunden_ID = ek.Kunden_ID)
       INNER JOIN EigenkundeKonto ekk ON (ek.Kunden_ID = ekk.Kunden_ID)
       INNER JOIN Sparbuch s ON (ekk.Konto_ID = s.Konto_ID)
INTERSECT
SELECT k.Vorname, k.Nachname
FROM   Kunde k INNER JOIN Eigenkunde ek ON (k.Kunden_ID = ek.Kunden_ID)
       INNER JOIN EigenkundeKonto ekk ON (ek.Kunden_ID = ekk.Kunden_ID)
       INNER JOIN Depot d ON (ekk.Konto_ID = d.Konto_ID);
        \end{lstlisting}
        \clearpage
        \item Schreiben Sie eine Abfrage, die eine Liste aller Eigenkunden
        ausgibt, die ein Girokonto und ein Sparbuch besitzten, aber kein Depot.
        \begin{oraclesql}[\FALSE]
        \end{oraclesql}
        \begin{lstlisting}[language=oracle_sql]
SELECT k.Vorname, k.Nachname
FROM   Kunde k INNER JOIN Eigenkunde ek ON (k.Kunden_ID = ek.Kunden_ID)
       INNER JOIN EigenkundeKonto ekk ON (ek.Kunden_ID = ekk.Kunden_ID)
       INNER JOIN Girokonto g ON (ekk.Konto_ID = g.Konto_ID)
INTERSECT
SELECT k.Vorname, k.Nachname
FROM   Kunde k INNER JOIN Eigenkunde ek ON (k.Kunden_ID = ek.Kunden_ID)
       INNER JOIN EigenkundeKonto ekk ON (ek.Kunden_ID = ekk.Kunden_ID)
       INNER JOIN Sparbuch s ON (ekk.Konto_ID = s.Konto_ID)
MINUS
SELECT k.Vorname, k.Nachname
FROM   Kunde k INNER JOIN Eigenkunde ek ON (k.Kunden_ID = ek.Kunden_ID)
       INNER JOIN EigenkundeKonto ekk ON (ek.Kunden_ID = ekk.Kunden_ID)
       INNER JOIN Depot d ON (ekk.Konto_ID = d.Konto_ID);
        \end{lstlisting}
        \begin{mssql}[\FALSE]
        \end{mssql}
        \begin{lstlisting}[language=ms_sql]
SELECT k.Vorname, k.Nachname
FROM   Kunde k INNER JOIN Eigenkunde ek ON (k.Kunden_ID = ek.Kunden_ID)
       INNER JOIN EigenkundeKonto ekk ON (ek.Kunden_ID = ekk.Kunden_ID)
       INNER JOIN Girokonto g ON (ekk.Konto_ID = g.Konto_ID)
INTERSECT
SELECT k.Vorname, k.Nachname
FROM   Kunde k INNER JOIN Eigenkunde ek ON (k.Kunden_ID = ek.Kunden_ID)
       INNER JOIN EigenkundeKonto ekk ON (ek.Kunden_ID = ekk.Kunden_ID)
       INNER JOIN Sparbuch s ON (ekk.Konto_ID = s.Konto_ID)
EXCEPT
SELECT k.Vorname, k.Nachname
FROM   Kunde k INNER JOIN Eigenkunde ek ON (k.Kunden_ID = ek.Kunden_ID)
       INNER JOIN EigenkundeKonto ekk ON (ek.Kunden_ID = ekk.Kunden_ID)
       INNER JOIN Depot d ON (ekk.Konto_ID = d.Konto_ID);
        \end{lstlisting}
      \end{enumerate}
\clearpage
