\clearpage
    \section{Lösungen - Funktionen}
      \begin{enumerate}
        \item Lassen Sie das aktuelle Datum auf dem Bildschirm ausgeben und
        benennen Sie die Spalte mit \enquote{Datum}.
        \begin{oraclesql}[\FALSE]
        \end{oraclesql}
        \begin{lstlisting}[language=oracle_sql]
SELECT SYSDATE AS "Datum"
FROM   dual;
        \end{lstlisting}
        \begin{mssql}[\FALSE]
        \end{mssql}
        \begin{lstlisting}[language=ms_sql]
SELECT GETDATE() AS "Datum";
        \end{lstlisting}
        \item Lassen Sie das aktuelle Datum mit Uhrzeit auf dem Bildschirm
        ausgeben und benennen Sie die Spalte mit \enquote{Datum/Uhrzeit}.
        \begin{oraclesql}[\FALSE]
        \end{oraclesql}
        \begin{lstlisting}[language=oracle_sql]
SELECT SYSTIMESTAMP AS "Datum/Uhrzeit"
FROM   dual;
        \end{lstlisting}
        \begin{mssql}[\FALSE]
        \end{mssql}
        \begin{lstlisting}[language=ms_sql]
SELECT GETDATE() AS "Datum/Uhrzeit";
        \end{lstlisting}
        \item Schreiben Sie eine Abfrage, welche die Mitarbeiternummer, den
        Nachnamen, das Gehalt und ein um 3,5 \% erhöhtes Gehalt für jeden
        Mitarbeiter anzeigt. Das erhöhte Gehalt soll als ganze Zahl und mit
        dem Spaltenalias \enquote{Neues Gehalt} ausgegeben werden!
        \begin{oraclesql}[\FALSE]
        \end{oraclesql}
        \begin{lstlisting}[language=oracle_sql]
SELECT Mitarbeiter_ID, Nachname, Gehalt,
       ROUND(Gehalt * 1.035, 0) AS "Neues Gehalt"
FROM   Mitarbeiter;
        \end{lstlisting}
        \begin{mssql}[\FALSE]
        \end{mssql}
        \begin{lstlisting}[language=ms_sql]
SELECT Mitarbeiter_ID, Nachname, Gehalt,
       CEILING(ROUND(Gehalt * 1.035, 0)) AS "Neues Gehalt"
FROM   Mitarbeiter;
        \end{lstlisting}
        \item Verändern Sie die Abfrage, aus der vorangegangenen Aufgabe so,
        dass eine zusätzliche Spalte hinzugefügt wird, die die Differenz
        zwischen dem alten und dem erhöhten Gehalt anzeigt. Benennen Sie die
        Spalte mit \enquote{Gehaltserhoehung}.
        \begin{oraclesql}[\FALSE]
        \end{oraclesql}
        \begin{lstlisting}[language=oracle_sql]
SELECT Mitarbeiter_ID, Nachname, Gehalt,
       ROUND(Gehalt * 1.035, 0) AS "Neues Gehalt",
       ROUND(Gehalt * 1.035, 0) - Gehalt AS "Gehaltserhoehung"
FROM   Mitarbeiter;
        \end{lstlisting}
\clearpage
        \begin{mssql}[\FALSE]
        \end{mssql}
        \begin{lstlisting}[language=ms_sql]
SELECT Mitarbeiter_ID, Nachname, Gehalt,
       CEILING(ROUND(Gehalt * 1.035, 0)) AS "Neues Gehalt",
       CEILING(ROUND(Gehalt * 1.035, 0)) - Gehalt AS "Gehaltserhoehung"
FROM   Mitarbeiter;
        \end{lstlisting}
        \item Zeigen Sie die Nachnamen und die Länge der Nachnamen aller
        Mitarbeiter an, deren Nachname mit einem der Buchstaben \enquote{J},
        \enquote{M} oder \enquote{S} beginnt. Die Spalten sollen, wie in der
        Lösung zu sehen ist, beschriftet sein. Die Nachnamen müssen in
        Großbuchstaben ausgegeben werden. Sortieren Sie die Abfrage in
        absteigender Reihenfolge nach den Nachnamen! 
        \begin{oraclesql}[\FALSE]
        \end{oraclesql}
        \begin{lstlisting}[language=oracle_sql]
SELECT  UPPER(Nachname) AS "Nachname",
        LENGTH(Nachname) AS "Laenge"
FROM    Mitarbeiter
WHERE   (UPPER(Nachname) LIKE 'J%'
   OR    UPPER(Nachname) LIKE 'M%'
   OR    UPPER(Nachname) LIKE 'S%')
ORDER BY Nachname DESC;
        \end{lstlisting}
        \begin{mssql}[\FALSE]
        \end{mssql}
        \begin{lstlisting}[language=ms_sql]
SELECT  UPPER(Nachname) AS "Nachname",
        LEN(Nachname) AS "Laenge"
FROM    Mitarbeiter
WHERE   (Nachname LIKE 'J%'
   OR    Nachname LIKE 'M%'
   OR    Nachname LIKE 'S%')
ORDER BY Nachname DESC;
        \end{lstlisting}
        \item Zeigen Sie für jeden Mitarbeiter den Nachnamen an, sein
        Geburtsdatum und seit wie vielen Monaten dieser bereits 18 Jahre alt ist
        (gerundet auf zwei Stellen, nach dem Komma). Benennen Sie die Spalte mit
        den Monaten: \enquote{Alter in Monaten}. Sortieren Sie die Abfrage in
        aufsteigender Reihenfolge nach der Spalte \enquote{Alter in Monaten}.
        \begin{oraclesql}[\FALSE]
        \end{oraclesql}
        \begin{lstlisting}[language=oracle_sql]
SELECT   Nachname, Geburtsdatum,
         ROUND(MONTHS_BETWEEN(
           SYSDATE, Geburtsdatum + 
           INTERVAL '18' YEAR), 2) AS "Alter in Monaten"
FROM     Mitarbeiter
ORDER BY 3;
       \end{lstlisting}
\clearpage
        \begin{mssql}[\FALSE]
        \end{mssql}
        \begin{lstlisting}[language=ms_sql]
SELECT   Nachname, Geburtsdatum,
         ROUND(DATEDIFF(MONTH, DATEADD(YEAR, 18, Geburtsdatum),
                 GETDATE()), 2) AS "Alter in Monaten"
FROM     Mitarbeiter
ORDER BY 3;
        \end{lstlisting}
        \item Ermitteln Sie Vorname, Nachname und Geburtsdatum der Mitarbeiter,
        die mindestens 1 Jahr und 4 Monate nach dem \enquote{07.05.1978} geboren
        sind. Sortieren Sie die Abfrage in absteigender Reihenfolge nach dem
        Geburtsdatum.
        \begin{oraclesql}[\FALSE]
        \end{oraclesql}
        \begin{lstlisting}[language=oracle_sql]
SELECT   Vorname, Nachname, Geburtsdatum
FROM     Mitarbeiter
WHERE    Geburtsdatum >
           TO_DATE('07.05.1978', 'DD.MM.YYYY') + 
           INTERVAL '1-4' YEAR TO MONTH
ORDER BY 3 DESC;
        \end{lstlisting}
        \begin{mssql}[\FALSE]
        \end{mssql}
        \begin{lstlisting}[language=ms_sql]
SELECT   Vorname, Nachname, Geburtsdatum
FROM     Mitarbeiter
WHERE    Geburtsdatum > DATEADD(MONTH, 4, DATEADD(YEAR, 1, '07.05.1978'))
ORDER BY 3 DESC;
        \end{lstlisting}
        \item Zeigen Sie für jeden Mitarbeiter, der zum Zeitpunkt der
        Ausführung dieser Abfrage mindestens 35 Jahre alt ist, dessen
        Mitarbeiter\_ID, das Geburtsdatum und den Wochentag seiner Geburt an.
        Beschriften Sie die Spalten, wie in der Lösung vorgegeben. Ordnen Sie
        die Abfrage in aufsteigender Reihenfolge nach dem Wochentag, beginnend
        beim ersten Tag der Woche!
        \begin{oraclesql}[\FALSE]
        \end{oraclesql}
        \begin{lstlisting}[language=oracle_sql]
SELECT   Mitarbeiter_ID, Geburtsdatum,
         TO_CHAR(Geburtsdatum, 'DAY') AS "Wochentag"
FROM     Mitarbeiter
WHERE    SYSDATE > Geburtsdatum + INTERVAL '35' YEAR
ORDER BY TO_CHAR(Geburtsdatum, 'D');
        \end{lstlisting}
        \begin{mssql}[\FALSE]
        \end{mssql}
        \begin{lstlisting}[language=ms_sql]
SELECT   Mitarbeiter_ID, Geburtsdatum,
         DATENAME(WEEKDAY, Geburtsdatum) AS "Wochentag"
FROM     Mitarbeiter
WHERE    GETDATE() > DATEADD(YEAR, 35, Geburtsdatum)
ORDER BY DATEPART(WEEKDAY, Geburtsdatum);
        \end{lstlisting}
\clearpage
        \item Schreiben Sie eine Abfrage, die für alle Mitarbeiter deren
        Nachnamen und die Bankfiliale\_ID anzeigt. Wenn ein Mitarbeiter in
        keiner Bankfiliale tätig ist, soll \enquote{Keine Bankfiliale}
        angezeigt werden.
        \begin{oraclesql}[\FALSE]
        \end{oraclesql}
        \begin{lstlisting}[language=oracle_sql]
SELECT   Nachname, NVL(
           TO_CHAR(Bankfiliale_ID), 'Keine Bankfiliale') AS Bankfiliale
FROM     Mitarbeiter
ORDER BY Bankfiliale_ID;
        \end{lstlisting}
        \begin{mssql}[\FALSE]
        \end{mssql}
        \begin{lstlisting}[language=ms_sql]
SELECT Nachname,
       ISNULL(CONVERT(VARCHAR, Bankfiliale_ID),
         'Keine Bankfiliale') AS Bankfiliale
FROM   Mitarbeiter
ORDER BY Bankfiliale_ID;
        \end{lstlisting}
    \end{enumerate}
\clearpage
