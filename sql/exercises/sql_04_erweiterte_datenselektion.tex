\clearpage
    \section{Übungen - Erweiterte Datenselektion}
      \begin{enumerate}
        \item Schreiben Sie eine Abfrage, die für jeden Mitarbeiter den
        Vornamen, den Nachnamen, die Bankfiliale\_ID und den Ort anzeigt, an dem
        sich seine Filiale befindet.
        \begin{center}
          \begin{small}
            \changefont{pcr}{m}{n}
            \tablefirsthead {
              \multicolumn{1}{l}{\textbf{VORNAME}} &
              \multicolumn{1}{l}{\textbf{NACHNAME}} &
              \multicolumn{1}{r}{\textbf{BANKFILIALE\_ID}} &
              \multicolumn{1}{l}{\textbf{ORT}} \\
              \cmidrule(l){1-1}\cmidrule(l){2-2}\cmidrule(r){3-3}\cmidrule(l){4-4}
            }
            \tablehead{}
            \tabletail {
              \multicolumn{4}{l}{\textbf{93 Zeilen ausgewählt}} \\
            }
            \tablelasttail {
              \multicolumn{4}{l}{\textbf{93 Zeilen ausgewählt}} \\
            }
            \begin{msoraclesql}
              \begin{supertabular}{llrl}
                Marie & Kipp & 1 & Aschersleben \\
                Louis & Schmitz & 1 & Aschersleben \\
                Johannes & Lehmann & 1 & Aschersleben \\
                Dirk & Peters & 1 & Aschersleben \\
                Amelie & Krüger & 1 & Aschersleben \\
                Martin & Schacke & 2 & Aschersleben \\
              \end{supertabular}
            \end{msoraclesql}
          \end{small}
        \end{center}
        \item Schreiben Sie eine Abfrage, welche die Mitarbeiternummer, den
        Nachnamen, das Gehalt und ein um 3,5 \% erhöhtes Gehalt für alle
        Mitarbeiter anzeigt, die in einer Filiale in \enquote{Aschersleben}
        arbeiten. Das erhöhte Gehalt soll als ganze Zahl und mit dem
        Spaltenalias \enquote{Neues Gehalt} ausgegeben werden.
        \begin{center}
          \begin{small}
            \changefont{pcr}{m}{n}
            \tablefirsthead {
              \multicolumn{1}{r}{\textbf{MITARBEITER\_ID}} &
              \multicolumn{1}{l}{\textbf{NACHNAME}} &
              \multicolumn{1}{r}{\textbf{GEHALT}} &
              \multicolumn{1}{r}{\textbf{Neues Gehalt}} \\
              \cmidrule(r){1-1}\cmidrule(l){2-2}\cmidrule(r){3-3}\cmidrule(r){4-4}
            }
            \tablehead{}
            \tabletail {
              \multicolumn{4}{l}{\textbf{10 Zeilen ausgewählt}} \\
            }
            \tablelasttail {
              \multicolumn{4}{l}{\textbf{10 Zeilen ausgewählt}} \\
            }
            \begin{msoraclesql}
              \begin{supertabular}{rlrr}
                8 & Peters & 12000 & 12420 \\
                9 & Winter & 12000 & 12420 \\
                28 & Lehmann & 2000 & 2070 \\
                29 & Schmitz & 2000 & 2070 \\
                30 & Kipp & 2000 & 2070 \\
                31 & Krüger & 2500 & 2588 \\
                32 & Beck & 1500 & 1553 \\
                33 & Schacke & 1000 & 1035 \\
                34 & Oswald & 1500 & 1553 \\
                35 & Wolf & 1000 & 1035 \\
              \end{supertabular}
            \end{msoraclesql}
          \end{small}
        \end{center}
        \item Erstellen Sie eine Abfrage, die zu jedem Eigenkunden, der ein
        Depot besitzt, seinen Vor- und Nachnamen, die Strasse mit der
        Hausnummer, sowie PLZ und Ort anzeigt.
        \begin{center}
          \begin{small}
            \changefont{pcr}{m}{n}
            \tablefirsthead {
              \multicolumn{1}{l}{\textbf{VORNAME}} &
              \multicolumn{1}{l}{\textbf{NACHNAME}} &
              \multicolumn{1}{l}{\textbf{STRASSE}} &
              \multicolumn{1}{l}{\textbf{PLZ}} &
              \multicolumn{1}{l}{\textbf{ORT}} \\
              \cmidrule(l){1-1}\cmidrule(l){2-2}\cmidrule(l){3-3}\cmidrule(l){4-4}\cmidrule(l){5-5}
            }
            \tablehead{}
            \tabletail {
            }
            \tablelasttail {
              \multicolumn{5}{l}{\textbf{239 Zeilen ausgewählt}} \\
            }
            \begin{msoraclesql}
              \begin{supertabular}{lllll}
                Sophie & Junge & Plutoweg 3 & 39435 & Bördeaue \\
                Hanna & Beck & Beimsstraße 9 & 39439 & Güsten \\
                Sebastian & Peters & Steinigstraße 3 & 39240 & Staßfurt \\
                Tina & Berger & Bundschuhstraße 1 & 04177 & Leipzig \\
              \end{supertabular}
            \end{msoraclesql}
          \end{small}
        \end{center}
\clearpage
        \item Schreiben Sie eine Abfrage, die für alle Eigenkunden deren Vor-
        und Nachnamen anzeigt, sowie den Vor- und den Nachnamen ihres
        persönlichen Finanzberaters (Tabelle
        \identifier{EigenkundeMitarbeiter}). Sortieren Sie die Abfrage nach den
        Nachnamen der Finanzberater.
        \begin{center}
          \begin{small}
            \changefont{pcr}{m}{n}
            \tablefirsthead {
              \multicolumn{1}{l}{\textbf{Vorname Kunde}} &
              \multicolumn{1}{l}{\textbf{Nachname Kunde}} &
              \multicolumn{1}{l}{\textbf{Vorname Berater}} &
              \multicolumn{1}{l}{\textbf{Nachname Berater}} \\
              \cmidrule(l){1-1}\cmidrule(l){2-2}\cmidrule(l){3-3}\cmidrule(l){4-4}
            }
            \tablehead{}
            \tabletail {
              \multicolumn{4}{l}{\textbf{384 Zeilen ausgewählt}} \\
            }
            \tablelasttail {
              \multicolumn{4}{l}{\textbf{384 Zeilen ausgewählt}} \\
            }
            \begin{msoraclesql}
              \begin{supertabular}{llll}
                Amelie & Fuchs & Leonie & Bauer \\
                Sarah & Becker & Leonie & Bauer \\
                Pia & Zimmermann & Leonie & Bauer \\
                Hanna & Schreiber & Leonie & Bauer \\
                Frank & Zimmermann & Leonie & Bauer \\
                Chris & Wagner & Leonie & Bauer \\
                Petra & Berger & Leonie & Bauer \\
                Maximilian & Junge & Leonie & Bauer \\
              \end{supertabular}
            \end{msoraclesql}
          \end{small}
        \end{center}
        \item Schreiben Sie eine Abfrage, die für alle Eigenkunden, die keinen
        Berater haben (die nicht in der Tabelle
        \identifier{EigenkundeMitarbeiter} enthalten sind), den Vor- und den
        Nachnamen anzeigt.
        \begin{center}
          \begin{small}
            \changefont{pcr}{m}{n}
            \tablefirsthead {
              \multicolumn{1}{l}{\textbf{VORNAME}} &
              \multicolumn{1}{l}{\textbf{NACHNAME}} \\
              \cmidrule(l){1-1}\cmidrule(l){2-2}
            }
            \tablehead{}
            \tabletail {
              \multicolumn{2}{l}{\textbf{16 Zeilen ausgewählt}} \\
            }
            \tablelasttail {
              \multicolumn{2}{l}{\textbf{16 Zeilen ausgewählt}} \\
            }
            \begin{msoraclesql}
              \begin{supertabular}{ll}
                Sebastian & Schröder \\
                Udo & Schumacher \\
                Mia & Huber \\
                Simon & Witte \\
                Max & Bunzel \\
                Finn & Fischer \\
                Lara & Meierhöfer \\
                Jannis & Meier \\
              \end{supertabular}
            \end{msoraclesql}
          \end{small}
        \end{center}
        \item Schreiben Sie eine Abfrage, die zu jedem Mitarbeiter (Vorname,
        Nachname) den Vor- und den Nachnamen seines Vorgesetzten anzeigt.
        \begin{center}
          \begin{small}
            \changefont{pcr}{m}{n}
            \tablefirsthead {
              \multicolumn{1}{l}{\textbf{VORNAME\_M}} &
              \multicolumn{1}{l}{\textbf{NACHNAME\_M}} &
              \multicolumn{1}{l}{\textbf{VORNAME\_V}} &
              \multicolumn{1}{l}{\textbf{NACHNAME\_V}} \\
              \cmidrule(l){1-1}\cmidrule(l){2-2}\cmidrule(l){3-3}\cmidrule(l){4-4}
            }
            \tablehead{}
            \tabletail {
            }
            \tablelasttail {
              \multicolumn{4}{l}{\textbf{99 Zeilen ausgewählt}} \\
            }
            \begin{msoraclesql}
              \begin{supertabular}{llll}
                Finn & Seifert & Max & Winter \\
                Sarah & Werner & Max & Winter \\
                Tim & Sindermann & Sarah & Werner \\
                Sebastian & Schwarz & Sarah & Werner \\
                Emily & Meier & Finn & Seifert \\
                Peter & Möller & Finn & Seifert \\
              \end{supertabular}
            \end{msoraclesql}
          \end{small}
        \end{center}
\clearpage
        \item Verändern Sie die Abfrage aus der vorangegangenen Aufgabe so,
        dass alle Mitarbeiter, einschließlich des Mitarbeiters
        \enquote{Winter}, der keinen Vorgesetzten hat, angezeigt werden.
        Sortieren Sie das Ergebnis aufsteigend nach der Vorgesetzten\_ID. Der
        Mitarbeiter \enquote{Winter} soll ganz oben auf der Liste stehen.
        \begin{center}
          \begin{small}
            \changefont{pcr}{m}{n}
            \tablefirsthead {
              \multicolumn{1}{l}{\textbf{VORNAME}} &
              \multicolumn{1}{l}{\textbf{NACHNAME}} &
              \multicolumn{1}{l}{\textbf{VORNAME}} &
              \multicolumn{1}{l}{\textbf{NACHNAME}} \\
              \cmidrule(l){1-1}\cmidrule(l){2-2}\cmidrule(l){3-3}\cmidrule(l){4-4}
            }
            \tablehead{}
            \tabletail {
              \multicolumn{4}{l}{\textbf{100 Zeilen ausgewählt}} \\
            }
            \tablelasttail {
              \multicolumn{4}{l}{\textbf{100 Zeilen ausgewählt}} \\
            }
            \begin{msoraclesql}
              \begin{supertabular}{llll}
                Max & Winter &  &  \\
                Finn & Seifert & Max & Winter \\
                Sarah & Werner & Max & Winter \\
                Tim & Sindermann & Sarah & Werner \\
                Sebastian & Schwarz & Sarah & Werner \\
                Emily & Meier & Finn & Seifert \\
              \end{supertabular}
            \end{msoraclesql}
          \end{small}
        \end{center}
        \item Erstellen Sie eine Abfrage, die ermittelt, ob es Mitarbeiter gibt,
        die keine Kundenberatung durchführen. Ausgenommen sind leitende
        Mitarbeiter (Mitarbeiter die in keiner Bankfiliale arbeiten).
        \begin{center}
          \begin{small}
            \changefont{pcr}{m}{n}
            \tablefirsthead {
              \multicolumn{1}{l}{\textbf{VORNAME}} &
              \multicolumn{1}{l}{\textbf{NACHNAME}} \\
              \cmidrule(l){1-1}\cmidrule(l){2-2}
            }
            \tablehead{}
            \tabletail {
              \multicolumn{2}{l}{\textbf{60 Zeilen ausgewählt}} \\
            }
            \tablelasttail {
              \multicolumn{2}{l}{\textbf{60 Zeilen ausgewählt}} \\
            }
            \begin{msoraclesql}
              \begin{supertabular}{ll}
                Finn & Bauer \\
                Stefan & Beck \\
                Lina & Becker \\
                Emma & Berger \\
                Udo & Bosse \\
                Georg & Dühning \\
                Tom & Fischer \\
              \end{supertabular}
            \end{msoraclesql}
          \end{small}
        \end{center}
        \item Schreiben Sie eine Abfrage, die für alle Mitarbeiter, die
        höchstens 3 Jahre älter, aber keinesfalls jünger sind als ihr
        Vorgesetzter, den Vornamen, den Nachnamen, das Geburtsdatum und das
        Geburtsdatum des Vorgesetzten anzeigt.
        \begin{center}
          \begin{small}
            \changefont{pcr}{m}{n}
            \tablefirsthead {
              \multicolumn{1}{l}{\textbf{VORNAME}} &
              \multicolumn{1}{l}{\textbf{NACHNAME}} &
              \multicolumn{1}{l}{\textbf{GEBURTSDATUM}} &
              \multicolumn{1}{l}{\textbf{Geburtstag Chef}} \\
              \cmidrule(l){1-1}\cmidrule(l){2-2}\cmidrule(l){3-3}\cmidrule(l){4-4}
            }
            \tablehead{}
            \tabletail {
%               \multicolumn{4}{l}{\textbf{20 Zeilen ausgewählt}} \\
            }
            \tablelasttail {
              \multicolumn{4}{l}{\textbf{20 Zeilen ausgewählt}} \\
            }
            \begin{msoraclesql}
              \begin{supertabular}{llll}
                Finn & Seifert & 17.10.85 & 31.08.88 \\
                Jessica & Weber & 10.06.92 & 27.06.92 \\
                Dirk & Peters & 16.09.91 & 27.06.92 \\
                Chris & Lang & 08.10.86 & 30.01.89 \\
                Marie & Kipp & 27.09.90 & 16.09.91 \\
              \end{supertabular}
            \end{msoraclesql}
          \end{small}
        \end{center}
\clearpage
        \item Schreiben Sie eine Abfrage, die für alle Mitarbeiter, die am
        gleichen Ort arbeiten, an dem sie auch wohnen, deren Vorname, Nachname
        den Wohnort und den Arbeitsort anzeigt. Beschriften Sie die Spalten, wie
        es in der Lösung zu sehen ist. Sortieren Sie die Abfragen in
        absteigender Reihenfolge nach dem Wohnort.
        \begin{center}
          \begin{small}
            \changefont{pcr}{m}{n}
            \tablefirsthead {
              \multicolumn{1}{l}{\textbf{VORNAME}} &
              \multicolumn{1}{l}{\textbf{NACHNAME}} &
              \multicolumn{1}{l}{\textbf{Wohnort}} &
              \multicolumn{1}{l}{\textbf{Arbeitsort}} \\
              \cmidrule(l){1-1}\cmidrule(l){2-2}\cmidrule(l){3-3}\cmidrule(l){4-4}
            }
            \tablehead{}
            \tabletail {
              \multicolumn{4}{l}{\textbf{3 Zeilen ausgewählt}} \\
            }
            \tablelasttail {
              \multicolumn{4}{l}{\textbf{3 Zeilen ausgewählt}} \\
            }
            \begin{msoraclesql}
              \begin{supertabular}{llll}
                Emily & Günther & Plötzkau & Plötzkau \\
                Jannis & Friedrich & Güsten & Güsten \\
                Tim & Zimmermann & Egeln & Egeln \\
              \end{supertabular}
            \end{msoraclesql}
          \end{small}
        \end{center}
        \item Erstellen Sie eine Abfrage, die ermittelt, ob es Mitarbeiter gibt
        (Vorname und Nachname), die keine Kundenberatung durchführen.
        Ausgenommen sind leitende Mitarbeiter (Mitarbeiter die in keiner
        Bankfiliale arbeiten) und Filialleiter.
        \begin{center}
          \begin{small}
            \changefont{pcr}{m}{n}
            \tablefirsthead {
              \multicolumn{1}{l}{\textbf{VORNAME}} &
              \multicolumn{1}{l}{\textbf{NACHNAME}} \\
              \cmidrule(l){1-1}\cmidrule(l){2-2}
            }
            \tablehead{}
            \tabletail {
              \multicolumn{2}{l}{\textbf{40 Zeilen ausgewählt}} \\
            }
            \tablelasttail {
              \multicolumn{2}{l}{\textbf{40 Zeilen ausgewählt}} \\
            }
            \begin{msoraclesql}
              \begin{supertabular}{ll}
                Amelie & Krüger \\
                Anna & Schneider \\
                Chris & Simon \\
                Christian & Haas \\
                Elias & Sindermann \\
                Emilia & Köhler \\
                Emma & Krüger \\
              \end{supertabular}
            \end{msoraclesql}
          \end{small}
        \end{center}
        \item Erstellen Sie eine Abfrage, die alle Eigenkunden anzeigt, die nur
        Girokonten aber keine anderen Konten besitzen.
        \begin{center}
          \begin{small}
            \changefont{pcr}{m}{n}
            \tablefirsthead {
              \multicolumn{1}{l}{\textbf{VORNAME}} &
              \multicolumn{1}{l}{\textbf{NACHNAME}} \\
              \cmidrule(l){1-1}\cmidrule(l){2-2}
            }
            \tablehead{}
            \tabletail {
              \multicolumn{2}{l}{\textbf{21 Zeilen ausgewählt}} \\
            }
            \tablelasttail {
              \multicolumn{2}{l}{\textbf{21 Zeilen ausgewählt}} \\
            }
            \begin{msoraclesql}
              \begin{supertabular}{ll}
                Amelie & Becker \\
                Amelie & Richter \\
                Chris & Walther \\
                Emilia & Keller \\
                Georg & Keller \\
                Johanna & Schäfer \\
              \end{supertabular}
            \end{msoraclesql}
          \end{small}
        \end{center}
\clearpage
        \item Erstellen Sie mit Hilfe einer Abfrage eine Liste, die den Vor- und
        den Nachnamen aller Kunden enthält, die sowohl ein Sparbuch, als auch
        ein Depot besitzten. Ob die Kunden ein Girokonto haben oder nicht ist
        irrelevant.
        \begin{center}
          \begin{small}
            \changefont{pcr}{m}{n}
            \tablefirsthead {
              \multicolumn{1}{l}{\textbf{VORNAME}} &
              \multicolumn{1}{l}{\textbf{NACHNAME}} \\
              \cmidrule(l){1-1}\cmidrule(l){2-2}
            }
            \tablehead{}
            \tabletail {
            }
            \tablelasttail {
              \multicolumn{2}{l}{\textbf{176 Zeilen ausgewählt}} \\
            }
            \begin{msoraclesql}
              \begin{supertabular}{ll}
                Alexander & Lorenz \\
                Alina & Baumann \\
                Alina & Huber \\
                Alina & Peters \\
                Alina & Schumacher \\
                Alina & Schütz \\
                Amelie & Fuchs \\
                Amelie & Günther \\
              \end{supertabular}
            \end{msoraclesql}
          \end{small}
        \end{center}
        \item Schreiben Sie eine Abfrage, die eine Liste aller Eigenkunden
        ausgibt, die ein Girokonto und ein Sparbuch besitzten, aber kein Depot.
        \begin{center}
          \begin{small}
            \changefont{pcr}{m}{n}
            \tablefirsthead {
              \multicolumn{1}{l}{\textbf{VORNAME}} &
              \multicolumn{1}{l}{\textbf{NACHNAME}} \\
              \cmidrule(l){1-1}\cmidrule(l){2-2}
            }
            \tablehead{}
            \tabletail {
              \multicolumn{2}{l}{\textbf{134 Zeilen ausgewählt}} \\
            }
            \tablelasttail {
              \multicolumn{2}{l}{\textbf{134 Zeilen ausgewählt}} \\
            }
            \begin{msoraclesql}
              \begin{supertabular}{ll}
                Alina & Braun \\
                Andy & Klingner \\
                Anna & Schubert \\
                Anna & Sindermann \\
                Anna & Wagner \\
                Bea & Witte \\
                Ben & Lehmann \\
                Chris & Beck \\
                Chris & Weber \\
              \end{supertabular}
            \end{msoraclesql}
          \end{small}
        \end{center}
      \end{enumerate}
