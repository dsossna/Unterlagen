\clearpage
    \section{Übungen - Unterabfragen}
      \begin{enumerate}
        \item Schreiben Sie eine Abfrage, die für alle Eigenkunden, die keinen
        Berater haben (die nicht in der Tabelle
        \identifier{EigenkundeMitarbeiter} enthalten sind), den Vor- und den
        Nachnamen anzeigt.
        \begin{itemize}
          \item Lösen Sie die Aufgabe mit Hilfe des \languageorasql{EXISTS}-Operators!
          \item Lösen Sie die Aufgabe mit Hilfe des \languageorasql{IN}-Operators!
        \end{itemize}
        \begin{center}
          \begin{small}
            \changefont{pcr}{m}{n}
            \tablefirsthead {
              \multicolumn{1}{l}{\textbf{VORNAME}} &
              \multicolumn{1}{l}{\textbf{NACHNAME}} \\
              \cmidrule(l){1-1}\cmidrule(l){2-2}
            }
            \tablehead{}
            \tabletail {
              \multicolumn{2}{l}{\textbf{16 Zeilen ausgewählt}} \\
            }
            \tablelasttail {
              \multicolumn{2}{l}{\textbf{16 Zeilen ausgewählt}} \\
            }
            \begin{msoraclesql}
              \begin{supertabular}{ll}
                Sebastian & Schröder \\
                Udo & Schumacher \\
                Mia & Huber \\
                Simon & Witte \\
                Max & Bunzel \\
                Finn & Fischer \\
                Lara & Meierhöfer \\
                Jannis & Meier \\
              \end{supertabular}
            \end{msoraclesql}
          \end{small}
        \end{center}
        \item Erstellen Sie eine Abfrage, die ermittelt, ob es Mitarbeiter gibt
        (Vorname und Nachname), die keine Kundenberatung durchführen.
        Ausgenommen sind leitende Mitarbeiter (Mitarbeiter die in keiner
        Bankfiliale arbeiten) und Filialleiter.
        \begin{itemize}
          \item Lösen Sie die Aufgabe mit Hilfe des \languageorasql{EXISTS}-Operators!
          \item Lösen Sie die Aufgabe mit Hilfe des \languageorasql{IN}-Operators!
        \end{itemize}
        \begin{center}
          \begin{small}
            \changefont{pcr}{m}{n}
            \tablefirsthead {
              \multicolumn{1}{l}{\textbf{VORNAME}} &
              \multicolumn{1}{l}{\textbf{NACHNAME}} \\
              \cmidrule(l){1-1}\cmidrule(l){2-2}
            }
            \tablehead{}
            \tabletail {
              \multicolumn{2}{l}{\textbf{40 Zeilen ausgewählt}} \\
            }
            \tablelasttail {
              \multicolumn{2}{l}{\textbf{40 Zeilen ausgewählt}} \\
            }
            \begin{msoraclesql}
              \begin{supertabular}{ll}
                Amelie & Krüger \\
                Anna & Schneider \\
                Chris & Simon \\
                Christian & Haas \\
                Elias & Sindermann \\
                Emilia & Köhler \\
                Emma & Krüger \\
              \end{supertabular}
            \end{msoraclesql}
          \end{small}
        \end{center}
\clearpage
        \item Schreiben Sie eine Abfrage, die den häufigsten Vornamen der
        Bankmitarbeiter anzeigt und wie oft dieser in der Tabelle
        \identifier{Mitarbeiter} vorkommt.
        \begin{center}
          \begin{small}
            \changefont{pcr}{m}{n}
            \tablefirsthead {
              \multicolumn{1}{l}{\textbf{VORNAME}} &
              \multicolumn{1}{r}{\textbf{ANZAHL}} \\
              \cmidrule(l){1-1}\cmidrule(r){2-2}
            }
            \tablehead{}
            \tabletail {
              \multicolumn{2}{l}{\textbf{1 Zeile ausgewählt}} \\
            }
            \tablelasttail {
              \multicolumn{2}{l}{\textbf{1 Zeile ausgewählt}} \\
            }
            \begin{msoraclesql}
              \begin{supertabular}{lr}
                Chris & 5 \\
              \end{supertabular}
            \end{msoraclesql}
          \end{small}
        \end{center}
        \item Schreiben Sie eine Abfrage, welche die drei Eigenkunden mit den
        niedrigsten Guthaben auf den Girokonten anzeigt.
        \begin{center}
          \begin{small}
            \changefont{pcr}{m}{n}
            \tablefirsthead {
              \multicolumn{1}{l}{\textbf{VORNAME}} &
              \multicolumn{1}{l}{\textbf{NACHNAME}} &
              \multicolumn{1}{r}{\textbf{GUTHABEN}} \\
              \cmidrule(l){1-1}\cmidrule(l){2-2}\cmidrule(r){3-3}
            }
            \tablehead{}
            \tabletail {
              \multicolumn{3}{l}{\textbf{3 Zeilen ausgewählt}} \\
            }
            \tablelasttail {
              \multicolumn{3}{l}{\textbf{3 Zeilen ausgewählt}} \\
            }
            \begin{msoraclesql}
              \begin{supertabular}{llr}
                Franz & Walther & -140505,1 \\
                Jan & Simon & -98218,6 \\
                Philipp & Hartmann & -69705,6 \\
              \end{supertabular}
            \end{msoraclesql}
          \end{small}
        \end{center}
        \item Verändern Sie die Abfrage aus der vorangegangenen Aufgabe so,
        dass die drei Eigenkunden mit dem niedrigsten Guthaben (Girokonto +
        Sparbuch) angezeigt werden. Es müssen auch diejenigen Kunden angezeigt
        werden, die nur ein Girokonto oder nur ein Sparbuch haben!
        \begin{center}
          \begin{small}
            \changefont{pcr}{m}{n}
            \tablefirsthead {
              \multicolumn{1}{l}{\textbf{VORNAME}} &
              \multicolumn{1}{l}{\textbf{NACHNAME}} &
              \multicolumn{1}{r}{\textbf{SUM(GUTHABEN)}} \\
              \cmidrule(l){1-1}\cmidrule(l){2-2}\cmidrule(r){3-3}
            }
            \tablehead{}
            \tabletail {
              \multicolumn{3}{l}{\textbf{3 Zeilen ausgewählt}} \\
            }
            \tablelasttail {
              \multicolumn{3}{l}{\textbf{3 Zeilen ausgewählt}} \\
            }
            \begin{msoraclesql}
              \begin{supertabular}{llr}
                Franz & Walther & -139154,4 \\
                Jan & Simon & -98218,6 \\
                Philipp & Hartmann & -69065,9 \\
              \end{supertabular}
            \end{msoraclesql}
          \end{small}
        \end{center}
        \item Schreiben Sie eine Abfrage, die alle Eigenkunden anzeigt, welche
        im Jahr 1985 keine Buchungen verursacht haben.
        \begin{center}
          \begin{small}
            \changefont{pcr}{m}{n}
            \tablefirsthead {
              \multicolumn{1}{l}{\textbf{VORNAME}} &
              \multicolumn{1}{l}{\textbf{NACHNAME}} \\
              \cmidrule(l){1-1}\cmidrule(l){2-2}
            }
            \tablehead{}
            \tabletail {
              \multicolumn{2}{l}{\textbf{285 Zeilen ausgewählt}} \\
            }
            \tablelasttail {
              \multicolumn{2}{l}{\textbf{285 Zeilen ausgewählt}} \\
            }
            \begin{msoraclesql}
              \begin{supertabular}{ll}
                Sarah & Bauer \\
                Sofia & Bauer \\
                Tom & Bauer \\
                Alina & Baumann \\
              \end{supertabular}
            \end{msoraclesql}
          \end{small}
        \end{center}
\clearpage
        \item Schreiben Sie eine Abfrage, die für jede Bankfiliale den
        Mitarbeiter mit dem höchsten Gehalt ausgibt.
        \begin{center}
          \begin{small}
            \changefont{pcr}{m}{n}
            \tablefirsthead {
              \multicolumn{1}{l}{\textbf{BANKFILIALE}} &
              \multicolumn{1}{l}{\textbf{VORNAME}} &
              \multicolumn{1}{l}{\textbf{NACHNAME}} &
              \multicolumn{1}{r}{\textbf{GEHALT}} \\
              \cmidrule(l){1-1}\cmidrule(l){2-2}\cmidrule(l){3-3}\cmidrule(r){4-4}
            }
            \tablehead{}
            \tabletail {
              \multicolumn{4}{l}{\textbf{20 Zeilen ausgewählt}} \\
            }
            \tablelasttail {
              \multicolumn{4}{l}{\textbf{20 Zeilen ausgewählt}} \\
            }
            \begin{msoraclesql}
              \begin{supertabular}{lllr}
                Poststraße 1 06449 Aschersleben & Dirk & Peters & 12000 \\
                Kirchstraße 8 39444 Hecklingen & Leonie & Kaiser & 12000 \\
                Schmiedestraße 3 39240 Staßfurt & Finn & Köhler & 12000 \\
                Am Dom 11 06449 Giersleben & Lena & Große & 12000 \\
              \end{supertabular}
            \end{msoraclesql}
          \end{small}
        \end{center}
        \item Schreiben Sie eine Abfrage, die für jeden Wohnort
        (\identifier{Eigenkunde.Ort}) den Kunden anzeigt, der im Jahr 1987 das
        höchste Einkommen hatte (Das Einkommen ist die Summe aller Beträge
        eines Kunden, in der Tabelle \identifier{Buchung}). Sortieren Sie die
        Abfrage nach den Wohnorten.
        \begin{center}
          \begin{small}
            \changefont{pcr}{m}{n}
            \tablefirsthead {
              \multicolumn{1}{l}{\textbf{ORT}} &
              \multicolumn{1}{l}{\textbf{VORNAME}} &
              \multicolumn{1}{l}{\textbf{NACHNAME}} &
              \multicolumn{1}{r}{\textbf{BETRAG}} \\
              \cmidrule(l){1-1}\cmidrule(l){2-2}\cmidrule(l){3-3}\cmidrule(r){4-4}
            }
            \tablehead{}
            \tabletail {
              \multicolumn{4}{l}{\textbf{30 Zeilen ausgewählt}} \\
            }
            \tablelasttail {
              \multicolumn{4}{l}{\textbf{30 Zeilen ausgewählt}} \\
            }
            \begin{msoraclesql}
              \begin{supertabular}{lllr}
                Alsleben & Peter & Koch & 57855,4 \\
                Aschersleben & Lara & Dühning & 2395,7 \\
                Barby & Chris & Beck & -6817,8 \\
              \end{supertabular}
            \end{msoraclesql}
          \end{small}
        \end{center}
        \item Erstellen Sie eine Abfrage, die die Umsätze der Bank
        (SUM(Buchung.Betrag)) für die Jahre 1985 bis einschließlich 1989
        als Pivottabelle anzeigt.
        \begin{center}
          \begin{small}
            \changefont{pcr}{m}{n}
            \tablefirsthead {
              \multicolumn{1}{r}{\textbf{'1985'}} &
              \multicolumn{1}{r}{\textbf{'1986'}} &
              \multicolumn{1}{r}{\textbf{'1987'}} &
              \multicolumn{1}{r}{\textbf{'1988'}} &
              \multicolumn{1}{r}{\textbf{'1989'}} \\
              \cmidrule(r){1-1}\cmidrule(r){2-2}\cmidrule(r){3-3}\cmidrule(r){4-4}\cmidrule(r){5-5}
            }
            \tablehead{}
            \tabletail {
              \multicolumn{5}{l}{\textbf{1 Zeile ausgewählt}} \\
            }
            \tablelasttail {
              \multicolumn{5}{l}{\textbf{1 Zeile ausgewählt}} \\
            }
            \begin{msoraclesql}
              \begin{supertabular}{rrrrr}
                559132,5 & 539497,2 & -2036841,3 & 1081361 & 1027003,1 \\
              \end{supertabular}
            \end{msoraclesql}
          \end{small}
        \end{center}
\clearpage
        \item Verändern Sie die Abfrage aus der vorangegangenen Aufgabe so,
        dass die Beträge innerhalb der einzelnen Jahre nach Quartalen
        aufgeteilt werden.
        \begin{center}
          \begin{small}
            \changefont{pcr}{m}{n}
            \tablefirsthead {
              \multicolumn{1}{l}{\textbf{QUARTAL}} &
              \multicolumn{1}{r}{\textbf{'1985'}} &
              \multicolumn{1}{r}{\textbf{'1986'}} &
              \multicolumn{1}{r}{\textbf{'1987'}} &
              \multicolumn{1}{r}{\textbf{'1988'}} &
              \multicolumn{1}{r}{\textbf{'1989'}} \\
              \cmidrule(l){1-1}\cmidrule(r){2-2}\cmidrule(r){3-3}\cmidrule(r){4-4}\cmidrule(r){5-5}\cmidrule(r){6-6}
            }
            \tablehead{}
            \tabletail {
              \multicolumn{6}{l}{\textbf{4 Zeilen ausgewählt}} \\
            }
            \tablelasttail {
              \multicolumn{6}{l}{\textbf{4 Zeilen ausgewählt}} \\
            }
            \begin{msoraclesql}
              \begin{supertabular}{lrrrrr}
                1 & 32204,8 & 985,2 & 2981,1 & 176852 & 9777,1 \\
                3 & -11792,8 & -71935,3 & 191697,3 & 282848 & 681185,9 \\
                2 & 151841,1 & 53654,8 & -2174503,9 & 430097,2 & 223402,7 \\
                4 & 386879,4 & 556792,5 & -57015,8 & 191563,8 & 112637,4 \\
              \end{supertabular}
            \end{msoraclesql}
          \end{small}
        \end{center}
        \item Verändern Sie die Abfrage aus der vorangegangenen Aufgabe so,
        dass eine Summenzeile, unterhalb der Pivottabelle angezeigt wird.
        \begin{center}
          \begin{small}
            \changefont{pcr}{m}{n}
            \tablefirsthead {
              \multicolumn{1}{l}{\textbf{QUARTAL}} &
              \multicolumn{1}{r}{\textbf{'1985'}} &
              \multicolumn{1}{r}{\textbf{'1986'}} &
              \multicolumn{1}{r}{\textbf{'1987'}} &
              \multicolumn{1}{r}{\textbf{'1988'}} &
              \multicolumn{1}{r}{\textbf{'1989'}} \\
              \cmidrule(l){1-1}\cmidrule(r){2-2}\cmidrule(r){3-3}\cmidrule(r){4-4}\cmidrule(r){5-5}\cmidrule(r){6-6}
            }
            \tablehead{}
            \tabletail {
              \multicolumn{6}{l}{\textbf{5 Zeilen ausgewählt}} \\
            }
            \tablelasttail {
              \multicolumn{6}{l}{\textbf{5 Zeilen ausgewählt}} \\
            }

            \begin{msoraclesql}
              \begin{supertabular}{lrrrrr}
                1 & 32204,8 & 985,2 & 2981,1 & 176852 & 9777,1 \\
                2 & 151841,1 & 53654,8 & -2174503,9 & 430097,2 & 223402,7 \\
                3 & -11792,8 & -71935,3 & 191697,3 & 282848 & 681185,9 \\
                4 & 386879,4 & 556792,5 & -57015,8 & 191563,8 & 112637,4 \\
                Summe & 559132,5 & 539497,2 & -2036841,3 & 1081361 & 1027003,1 \\
              \end{supertabular}
            \end{msoraclesql}
          \end{small}
        \end{center}
      \end{enumerate}
