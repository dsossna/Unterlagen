  \chapter{Filestream}
  \chaptertoc{}
  \cleardoubleevenpage

    \section{Filestream}
      Mit dem Filestream-Feature ist es möglich, nicht strukturierte Daten
      (PDFs, Videos, Audiodateien, Bilder), sogenannte BLOB-Daten, auf dem
      Dateisystemabzulegen und diese gleichzeitig mit der Datenbank zu
      verknüpfen. Der Zugriff auf die Daten kann dann mittels TSQL, aber auch
      über das Dateisystem erfolgen.
      \subsection{Wie funktioniert Filestream}
        \subsubsection{Implementierung im Dateisystem}
          Die Realisierung von Filestream geschieht mit Hilfe der in Microsoft
          Windows integrierten \enquote{Win32-Dateisystemschnittstelle}
          (API\footnote{API = Application Programming Interface}).
          Diese API ermöglicht es dem SQL Server einen
          Win32-Namespace\footnote{Namespaces sind UNC-Pfade, die den Zugriff
          auf Betriebssystemresourcen (Netzwerkfreigaben, Geräte, WMI, u. ä
          ermöglichen)} zu überwachen um so alle Änderungen, die dort
          stattgefunden haben, zu registrieren.
        
        \subsubsection{Implementierung in SQL Server}
          Für den Zugriff mittels TSQL muss in einer Datenbank eine
          Filestream-Dateigruppe erstellt werden, in der dann, seit
          der Version 2012 des SQL Servers, eine sogenannte \enquote{FileTable}
          plaziert werden kann. FileTables sehen aus wie normale Tabellen,
          greifen aber, unter Zuhilfenahme von SQL Server-eigenen Routinen, auf
          das Dateisystem zu. So kann mit \SELECT{} der Verzeichnisinhalt
          ausgelesen und mittels \INSERT, \UPDATE{} und \DELETE{} geändert werden.
        
          \bild{Filestream}{filestream}{1.3}
        \subsubsection{Der Neue - VARBINARY(max)}
          Um die Speicherung von BLOBs in der FileTable zu ermöglichen bringt
          der SQL Server einen neuen Datentyp mit, den
          \identifier{varbinary(max)}-Typ. Dieser kann unstrukturierte
          Binärdaten in der Datenbank und im Dateisystem ablegen. Bei der Ablage
          der Daten im Filestream gibt es keine Größenbeschränkung für diesen
          Datentyp.
          \begin{merke}
            Wird eine \identifier{varbinary(max)}-Spalte nicht im Dateisystem,
            sondern in der Datenbank abgelegt, besitzt sie eine
            Größenbeschränkung von 2 GB.
          \end{merke}
      \subsection{Filestream konfigurieren}
        Das Filestream-Feature ist nicht standardmäßig aktivert. Wird es
        benötigt, muss der Administrator sowohl im SQL Server Configurations
        Manager, als auch im Management Studio einige Klicks vornehmen.
        \subsubsection{Filestream im SQL Server Configurations Manager}
          In einem ersten Schritt muss das Filestream-Feature im SQL Server
          Configurations Manager für den Windows-Dienst MSSQLSERVER aktiviert
          werden.
          \begin{enumerate}
            \item Öffnen Sie den SQL Server Configurations Manager
            \item Wählen Sie die Rubrik \enquote{SQL Server-Dienste}
            \item Öffnen Sie die Eigenschaften des SQL Server-Dienstes
            (MSSQLSERVER)
            \item Klicken Sie die Registerkarte \enquote{FILESTREAM} an.
            \item Aktivieren Sie die beiden Optionen \enquote{FILESTREAM für
            Transact-SQL-Zugriff aktivieren} und \enquote{FILESTREAM für
            E/A-Dateizugriff aktivieren}.
            \bild{Die Registerkarte FILESTREAM}{filestream_register_card}{2}
            \item Starten Sie den SQL Server-Dienst neu!
          \end{enumerate}
        \subsubsection{Filestream im Management Studio}
          Der zweite Schritt ist das Aktivieren des Filestream-Features auf
          Instanzebene. Obwohl beide Schritte identisch und somit redundand
          erscheinen, ist die Trennung zwischen dem Windows-Dienst und der
          Instan durchaus sinnvoll. Auf den Windows-Dienst sollten im
          Normalfall nur Betriebssystemadministratoren Zugriff haben, nicht
          jedoch die Datenbankadministratoren\footnote{Anmerkung des Autors: In
          der Praxis herscht hier meist Personalunion bei diesen beiden
          Zuständigkeiten!}. Umgekehr sollte ein Betriebssystemadministrator
          keinen Zugriff auf die SQL Server-Instanz haben. 
          \begin{enumerate}
            \item Öffnen Sie die Eigenschaften Ihrer SQL Server-Instanz
            \item Klicken Sie auf die Rubrik \enquote{Erweitert}
            \bild{Filestream im Management Studio}{filestream_in_ssms}{1.5}
            \item Geben Sie für die Datenbankoption
            \enquote{FILESTREAM-Zugriffsebene} den Wert \enquote{Vollzugriff
            aktiviert} an.
          \end{enumerate}
          \begin{literaturinternet}
            \item \cite{llobelwp20101212ss2fp2o4}
          \end{literaturinternet}
      \subsection{Eine Filestream-Dateigruppe anlegen}
        Filestream-Dateigruppen sind spezielle Dateigruppen. Sie stellen die
        Schnittstelle zum Dateisystem her, die dann von den FileTables genutzt
        werden kann.
        \subsubsection{Anlegen der Dateigruppe}
          Eine Filestream-Dateigruppe wird auf die gleiche Art und Weise
          angelegt, wie eine normale Dateigruppe, lediglich das Schlüsselwort
          \languagemssql{CONTAINS FILESTREAM} kommt hinzu.
          \begin{lstlisting}[language=ms_sql, caption={Eine
          Filestream-Dateigruppe anlegen}, label=admin04_01]
USE [Bank_2014]
GO

ALTER DATABASE Bank_2014
ADD FILEGROUP FSstorage CONTAINS FILESTREAM;
GO
          \end{lstlisting}
          Mit dem Kommando aus \beispiel{admin04_01} wird der Datenbank
          \identifier{bank\_2014} eine Filestream-Dateigruppe namens
          \identifier{fsstorage} hinzugefügt. Genau wie bei normalen
          Dateigruppen wird diese als Standarddateigruppe markiert, da sie die
          erste ist. Sollten noch weitere Filestream-Dateigruppen existieren,
          kann diese Markierung, unabhängig von den normalen  Dateigruppen,
          verschoben werden.
        \subsubsection{Hinzufügen eines Data Containers}
          Die Dateigruppe \identifier{fsstorage} existiert zwar, aber noch
          können dort keine Daten abgelegt werden, da noch kein
          \enquote{Speicherort} definiert wurde. Mit Hilfe der
          \languagemssql{ADD FILE}-Klausel des \languagemssql{ALTER
          DATABASE}-Statements kann ein Verzeichnis angegeben werden, in dem die
          Filestream-Daten abgelegt werden. Dieses Verzeichnis wird als
          \enquote{Data Container} bezeichnet.
          \begin{merke}
            Damit die Definition des Data Containers erfolgreich sein kann, muss der
            Pfad \oscommand{D:\textbackslash u01\textbackslash
            bank\_2014\textbackslash filestream} schon vorher existieren. Das
            Verzeichnis \oscommand{fsstorage} darf jedoch noch nicht existieren.
          \end{merke}
\clearpage
          \begin{lstlisting}[language=ms_sql, caption={Einen Data-Container
          hinzufügen}, label=admin04_02]
USE [Bank_2014]
GO

ALTER DATABASE Bank_2014
ADD FILE (
  NAME     = 'fsstorage',
  FILENAME = 'D:\u01\bank_2014\filestream\fsstorage'
) TO FILEGROUP FSstorage;
GO
          \end{lstlisting}
          Die Auswirkungen von \beispiel{admin04_02} sind, dass das Verzeichnis
          \oscommand{fsstorage} angelegt wird. Darin befinden sich nun die Datei
          \oscommand{filestream.hdr} und das Verzeichnis \oscommand{\$FSLOG}.
          \bild{Ein Data Container}{filestream_hdr_and_fslog}{2.25}
          \begin{merke}
            Die Datei \oscommand{filestream.hdr} ist eine wichtige Headerdatei,
            ohne die die Filestream-Dateigruppe nicht funktionieren kann.
          \end{merke}
    \section{FileTables}
      In der soeben fertiggestellten Filestream-Dateigruppe können nun
      FileTables angelegt werden. Jede FileTable erstellt einen eigenen
      Verzeichnisbaum innerhalb der Dateigruppe. Das hat den Vorteil, dass
      mehrere FileTables, für unterschiedliche Zwecke in der gleichen
      Filestream-Dateigruppe, angelegt werden können und das trotzdem die Daten
      sauber von einander getrennt gespeichert werden.
      \subsection{Konfigurieren des Features FileTable}
        Für den transaktionsbasierten Zugriff mit TSQL sind keine weiteren
        vorbereitenden Schritte notwendig, da sich FileTables wie ganz normale
        Tabellen verhalten. Lediglich der externe Zugriff über das
        Dateisystemmuss noch konfiguriert werden.
        \subsubsection{Konfigurationstätigkeiten auf Datenbankebene}
          Nachdem das Filestream-Feature auf Instanz-Ebene bereits aktiviert und
          konfiguriert wurde, müssen auf Datenbankebene noch zwei weitere
          Eigenschaften eingestellt werden.
          \begin{itemize}
              \item \textbf{Nicht transaktionaler Zugriff}: Der externe Zugriff
              auf die Filestream-Daten einer Datenbank kann in drei Stufen
              geregelt werden:
              \begin{itemize}
                  \item \textbf{Off}: Kein externe Zugriff
                  \item \textbf{ReadOnly}: Nur lesender Zugriff
                  \item \textbf{Full}: Lese- und Schreibzugriff
              \end{itemize}
              \item \textbf{Database-level directory}: Für jede Datenbank
              muss ein eigener Verzeichnisknotenpunkt angegeben werden. Dies
              geschieht in den Eigenschaften der Datenbank.
              \bild{Der Datenbankpfad in der
              Filestream-Freigabe}{database_level_filestream_directory}{1.3}
          \end{itemize}
          \begin{merke}
            Wird eine neue Datenbank angelegt, enthält diese kein Database-level
            Directory. Dieses muss im Nachhinein durch eine \languagemssql{ALTER
            DATABASE}-Anweisung oder mittels der grafischen Oberfläche angegeben
            werden.
          \end{merke}
          \beispiel{admin04_03} zeigt, wie die Filestream-Einstellungen einer
          Datenbank mit Hilfe von SQL geändert werden können.
          \begin{lstlisting}[language=ms_sql,
          caption={Filestream-Einstellungen einer Datenbank ändern},
          label=admin04_03]
USE [Bank_2014]
GO

ALTER DATABASE Bank_2014
SET FILESTREAM ( NON_TRANSACTED_ACCESS = FULL,
                 DIRECTORY_NAME = N'directory_name' );
GO
          \end{lstlisting}
          \begin{literaturinternet}
            \item \cite{gg509097}
          \end{literaturinternet}
\clearpage
          Für die Einrichtung eines Database-Level Directory gelten die
          folgenden Anforderungen:
          \begin{itemize}
              \item Das Verzeichnis kann solange geändert werden, wie die
              Datenbank noch keine FileTable enthält.
              \item Jede Datenbank muss ihr eigenes Verzeichnis haben.
              \item Wird eine vorhandene Datenbank auf SQL Server 2014
              aktualisiert, wird der Wert für \identifier{directoy\_name} auf
              NULL zurückgesetzt
              \item Beim Löschen der Datenbank werden alle Inhalte im
              Database-Level Directory gelöscht.
          \end{itemize}
        \subsubsection{Der Aufbau des externen Zugriffspfades}  
          Der externe E/A-Zugriff auf eine FileTable wird
          mit Hilfe einer virtuellen Netzwerkfreigabe geregelt. Die gesamte
          Verzeichnisstruktur der Freigabe wird innerhalb der Datenbank
          gespeichert und über das Betriebssystem bereitgestellt. Der UNC-Pfad der
          Freigabe baut sich wie folgt auf:
        \begin{itemize}
          \item \oscommand{\textbackslash\textbackslash servername}: Der
          DNS-Name des Datenbankservers
          \item \oscommand{Instanz-Level-Freigabename}: Der Name der
          Freigabe auf Instanz-Ebene. Dieser wird im SQL Server
          Configuration Manager angegeben. Standardmäßig lautet er
          \oscommand{MSSQLSERVER}. Um diesen Wert ändern zu können, muss der
          SQL Server-Dienst neugestartet werden.
          \bild{Die Registerkarte FILESTREAM}{filestream_register_card}{1.5}
          \item \oscommand{Database-Level directory}
          \item \oscommand{FileTable directory}: Ein Verzeichnis, das beim
          anlegen der FileTable angegeben werden muss.
        \end{itemize}
        Der vollständige UNC-Pfad (Namespace) lautet somit:
          
        \begin{small}
        \oscommand{\textbackslash\textbackslash servername\textbackslash
        Instanz-Level-Freigabename\textbackslash Database-level
        directory\textbackslash FileTable-directory}
        \end{small}
        \begin{literaturinternet}
          \item \cite{gg492087}
        \end{literaturinternet}
      \subsection{Anlegen und Ändern von FileTables}
        \subsubsection{Anlegen einer FileTable}
          Eine FileTable ist nicht nur wegen ihrer Funktionalität eine
          besondere Tabelle, sondern auch aufgrund der Tatsache, dass sie ein
          festes Spaltenschema besitzt. Das bedeutet, dass beim Anlegen einer
          FileTable keine Spaltendefinition angegeben werden muss bzw. angegeben
          werden kann. Lediglich drei Angaben sind erlaubt:
          \begin{itemize}
            \item \textbf{FILETABLE\_DIRECTORY}: Ein Verzeichnis, unter dem
            die Dateien der FileTable abgelegt werden. Dieses Verzeichnis ist
            Teil des externen Zugriffspfades für die FileTable.
            \item \textbf{FILETABLE\_COLLATE\_FILENAME}: Eine Sortierung für
            die Dateinamen. Die benutzte Sortierung darf nicht nach Groß- und
            Kleinschreibung unterscheiden, damit die
            Windows-Dateisystemsemantik nicht verletzt wird.
            \item Die Bezeichner für die drei festgelegten Constraints einer
            FileTable:
            \begin{itemize}
              \item \textbf{FILETABLE\_PRIMARY\_KEY\_CONSTRAINT\_NAME}
              \item \textbf{FILETABLE\_STREAMID\_UNIQUE\_CONSTRAINT\_NAME}
              \item \textbf{FILETABLE\_FULLPATH\_UNIQUE\_CONSTRAINT\_NAME}
            \end{itemize} 
          \end{itemize}
        \subsubsection{Beschränkungen für Pfad- und Dateinamen}             
          Beim Anlegen von Dateien und Verzeichnissen innerhalb einer FileTable
          muss folgendes beachtet werden:
          \begin{itemize}
              \item Der Verzeichnisname kann maximal 255 Zeichen aufweisen. 
              \item Er muss den Anforderungen an einen gültigen
              Dateisystem-Verzeichnisnamen genügen.
              \item Er muss für jede FileTable eindeutig sein.
              \item Wird kein Verzeichnisname angegeben, wird der Name der
              FileTable als Verzeichnisname genutzt.
              \item Eine FileTable kann maximal 15 Verzeichnisebenen enthalten.
              Wird diese Maximalzahl erreicht, kann die unterste Ebene keine
              Dateien mehr enthalten, da diese als weitere Ebene gelten würden.
              \item Der Windows Explorer und viele andere Anwendungsprogramme
              geben eine maximale Dateinamenslänge von 260 Zeichen vor. Mit
              Hilfe einer FileTable ist es möglich, Dateinamen zu erzeugen, die
              wesentlich länger sind. Solche Dateien können weder durch den
              Explorer noch durch andere Anwendungsprogramm angezeigt werden.
          \end{itemize}
          \begin{literaturinternet}
            \item \cite{gg492087}
          \end{literaturinternet}
          \beispiel{admin04_04} zeigt, wie mit Hilfe des \languagemssql{CREATE
          TABLE AS FileTable}-Statements eine FileTable angelegt wird.
          \begin{lstlisting}[language=ms_sql, caption={Anlegen einer Filetable},
          label=admin04_04]
USE [Bank_2014]
GO

CREATE TABLE ExternalResources AS FileTable 
WITH (
  FileTable_Directory = 'ExternalResources',
  FileTable_Collate_Filename = Latin1_General_CI_AS
);
GO

          \end{lstlisting}
          Um eine Liste aller in der Datenbank vorhandenen FileTables zu
          erhalten, kann die View \identifier{sys.filetables} abgefragt werden.
          \begin{literaturinternet}
            \item \cite{gg509088}
          \end{literaturinternet}
        \subsubsection{Das Schema einer FileTable}
          Eine FileTable ist eine ganz gewöhnliche SQL Server-Tabelle mit der
          Einschränkung, dass ihr Schema fest vorgegeben ist. Die vier
          wichtigsten Spalten aus diesem Schema sind in
          \tabelle{filetableschema} zu sehen.
          \begin{center}
          \label{filetableschema}

            \begin{small}
            \tablefirsthead {
              \hline
              \multicolumn{1}{|l}{\textbf{Bezeichner}} &
              \multicolumn{1}{|l}{\textbf{Datentype}} &
              \multicolumn{1}{|l|}{\textbf{Beschreibung}}\\
              \hline
            }
            \tablehead{
            }
            \tabletail {
              \hline
            }
            \tablelasttail {
              \hline
            }
              \begin{supertabular}{|l|l|p{6.5cm}|}
                stream\_id & uniqueidentifier ROWGUIDCOL & Eindeutiger
                Schlüsselwert \\
                \hline
                file\_stream & VARBINARY(max) FILESTREAM & Der binäre Content
                der FileTable (BLOB) \\
                \hline
                name & NVARCHAR(255) & Der Name der Datei \\
                \hline
                path\_locator & hierarchyid & Logische Position der Datei
                innerhalb des SQL Server Namespaces \\
              \end{supertabular}
            \end{small}
          \end{center}
          Die \identifier{path\_locator}-Spalte ist der Primärschlüssel einer
          FileTable. Sie besitzt den spezielle Datentyp
          \identifier{hierarchyid}, der mit dem SQL Server 2008 eingeführt
          wurde. Mit der Hilfe dieser Spalte ist es möglich, die Position des
          BLOBs im Dateisystem zu ermitteln, um dann auf die Datei zuzugreifen.
          
          Zusätzlich zu diesen vier Spalten existieren noch 13 weitere, deren
          Werte automatisch berechnet werden.
          \begin{literaturinternet}
            \item \cite{gg492084}
            \item \cite{tallan20120101iafsiad}
          \end{literaturinternet}
        \subsubsection{Ändern und Löschen einer FileTable}
          Da FileTables ein festgelegtes Schema haben, kann der Administrator
          auch keine Änderung daran vornehmen. Lediglich die Erstellung von
          Indizes, Triggern und Constraints, sowie einige andere, weniger
          erwähnenswerte Dinge sind möglich.
          
          Gelöscht wird eine FileTable mit Hilfe des \languagemssql{DROP
          TABLE}-Statements. Zu beachten ist, dass der gesamten
          Dateisysteminhalt der FileTable mitgelöscht wird.            
      \subsection{Benutzen von FileTables}
        \subsubsection{Befüllen mit Daten}
          Eine FileTable kann auf unterschiedliche Art und Weise mit Daten
          befüllt werden. Im einfachsten Falle werden Dateien einfach mit Hilfe
          des externen Zugriffspfades in die FileTable hineinkopiert. Ein
          solcher Zugriff ist jedoch nicht durch das Transaktionskonzept der
          Datenbank abgesichert. Soll ein transaktionaler Zugriff erfolgen,
          bieten sich Tools wie z. B. \oscommand{bcp} an. Sollte der
          Zugriffspfad einer FileTable nicht bekannt sein, kann dieser mit der
          Funktion \identifier{filetablerootpath} ermittelt werden.
          \begin{lstlisting}[language=ms_sql, caption={Zugriffspfad zu einer
          FileTable ermitteln}, label=admin04_05a]
USE [Bank_2014]
GO

SELECT FileTableRootPath('ExternalResources') AS Rootpath
          \end{lstlisting}
          \begin{center}
            \begin{small}
              \changefont{pcr}{m}{n}
              \tablefirsthead{
                \multicolumn{1}{l}{\textbf{Rootpath}} \\
                \cmidrule(l){1-1}
              }
              \tablehead{}
              \tabletail{
                \multicolumn{1}{l}{\textbf{1 Zeile}} \\
              }
              \tablelasttail{
                \multicolumn{1}{l}{\textbf{1 Zeile}} \\
              }
              \begin{mssql}
                \begin{supertabular}{l}
                 \textbackslash\textbackslash
                 FEA11-119SRV12\textbackslash
                 MSSQLSERVER\textbackslash bank\_2014\textbackslash
                 ExternalResources
                 \\
                \end{supertabular}
              \end{mssql}
            \end{small}
          \end{center}
          
          Müssen Dateien aus einer normalen Tabelle (gespeicherte Pfade und
          Dateinamen) in eine FileTable migriert werden, können SQL-Anweisungen
          wie das \languagemssql{OPENROWSET} oder das \languagemssql{INSERT
          INTO ... SELECT * FROM} genutzt weren.
           \begin{lstlisting}[language=ms_sql, caption={Mit TSQL eine Datei in
           eine FileTable einfügen}, label=admin04_05b]
USE [Bank_2014]
GO

INSERT INTO ExternalResources (name, File_Stream)
SELECT 'bank_sql_server.bak', *
FROM   OPENROWSET(BULK N'H:\bank_sql_server.bak', SINGLE_BLOB) AS Content
          \end{lstlisting}
          \begin{literaturinternet}
            \item \cite{sqlarticlesagwwf}
          \end{literaturinternet}
        \subsubsection{Den transaktionalen Zugriff aktivieren/deaktivieren}
          Im Falle dessen, dass Wartungsarbeiten an einer Datenbank mit aktiven
          FileTables durchgeführt werden müssen, kann es notwendig sein, den
          nicht transaktionalen Zugriff auf die FileTables vorübergehend zu
          deaktiven, da der Administrator anderenfalls keinen exklusiven Zugriff
          auf die Datenbank erhalten kann.
          
          \beispiel{admin04_05} zeigt, wie der nicht transaktionale Zugriff auf
          alle FileTables einer Datenbank deaktiviert bzw. reaktiviert wird.
          \begin{lstlisting}[language=ms_sql, caption={Deaktivieren und
          reaktivieren des Zugriffs auf FileTables}, label=admin04_05]
USE [Bank_2014]
GO

-- Deactivate Access
ALTER DATABASE bank_2014
SET FILESTREAM ( NON_TRANSACTED_ACCESS = OFF );
GO

-- Reactivate Access
ALTER DATABSE bank_2014
SET FILESTREAM ( NON_TRANSACTED_ACCESS = FULL );
GO
          \end{lstlisting}
          Das Kommando in \beispiel{admin04_05} hat den Nachteil, dass
          es blockiert, falls noch offene Dateihandles in einer der FileTables
          sind. Entweder müssen die Benutzer alle Dateien schließen
          oder der Administrator muss den Zugriff der Benutzer auf die Dateien
          unterbrechen.
          \begin{merke}
            Beim schließen von offenen Dateihandles durch den Administrator kann
            es passieren, dass Benutzer nicht gespeicherte Informationen
            verlieren.
          \end{merke}
          Wird das \languagemssql{ALTER DATABASE}-Komanndo um den Zusatz
          \languagemssql{WITH ROLLBACK IMMEDIATE} ergänzt, werden automatisch
          alle offenen Dateihandles geschlossen.
          \begin{lstlisting}[language=ms_sql, caption={Deaktivieren und
          des Zugriffs auf FileTables und gleichzeitiges Schließen aller
          offenen Dateihandles}, label=admin04_06]
USE [Bank_2014]
GO

-- Deactivate Access
ALTER DATABASE bank_2014
SET FILESTREAM ( NON_TRANSACTED_ACCESS = OFF )
WITH ROLLBACK IMMEDIATE;
GO
          \end{lstlisting}
        \subsubsection{Offene Dateihandles beenden}
          Manchmal kann es notwendig werden, ein offenes Dateihandle zu
          schließen, z. B. dann, wenn dies aufgrund eines Fehlers nicht
          automatisch geschieht. Mit Hilfe der Dynamic Management View
          \identifier{sys.dm\_filestream\_non\_transacted\_handles} können alle
          offenen handles abgefragt werden, um sie dann unter Zuhilfenahme der
          Stored Procedure \identifier{sp\_kill\_filestream\_non\_transacted\_handle} zu
          schließen.
          \begin{lstlisting}[language=ms_sql, caption={Abfragen der offenen
          Filestream-Dateihandles}, label=admin04_07]
USE [Bank_2014]
GO

SELECT database_id, object_id, handle_id
FROM   sys.dm_filestream_non_transacted_handles;
GO 
          \end{lstlisting}
          \begin{center}
            \begin{small}
              \changefont{pcr}{m}{n}
              \tablefirsthead{
                \multicolumn{1}{l}{\textbf{database\_id}} &
                \multicolumn{1}{l}{\textbf{object\_id}} &
                \multicolumn{1}{l}{\textbf{handle\_id}} \\
                \cmidrule(l){1-1}\cmidrule(l){2-2}\cmidrule(l){3-3}
              }
              \tablehead{}
              \tabletail{
                \multicolumn{2}{l}{\textbf{3 Zeilen ausgew\"ahlt}} \\
              }
              \tablelasttail{
                \multicolumn{2}{l}{\textbf{3 Zeilen ausgew\"ahlt}} \\
              }
              \begin{msoraclesql}
                \begin{supertabular}{lll}
                  0 & 0         & 235 \\
                  8 & 581577110 & 525 \\
                  8 & 581577110 & 491 \\
                \end{supertabular}
              \end{msoraclesql}
            \end{small}
          \end{center}
          Zum schließen des Handles wird der Wert der Spalte
          \identifier{handle\_id} benötigt.
          \begin{lstlisting}[language=ms_sql, caption={Schließen eines offenen
          Filestream-Dateihandles}, label=admin04_08]
USE [Bank_2014]
GO

EXEC §sp_kill_filestream_non_transacted_handles§ @handle_id = 525;
GO
          \end{lstlisting}
          \begin{literaturinternet}
            \item \cite{ff929106}
            \item \cite{ff929168}
          \end{literaturinternet}