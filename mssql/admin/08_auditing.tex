  \chapter{Auditing}
    \chaptertoc{}
    \cleardoubleevenpage
      Auditing ist das \"Uberwachen und Aufzeichnen von ausgew\"ahlten Aktionen,
      die innerhalb der Datenbank stattfinden. Es kann auf Einzelnen oder auf
      einer Kombination von Faktoren (z. B. Nutzername, Anwendung, Anmeldezeit,
      usw.) basieren. Auditing \"ublicherweise f\"ur die folgenden Zwecke
      genutzt:
      \begin{itemize}
        \item Abschreckung von Nutzern, vor unerlaubten Zugriffen auf Objekte,,
        au\ss erhalb ihres Verantwortungsbereiches.
        \item Daten \"uber bestimmte Aktivit\"aten in der Datenbank sammeln
        \item Verd\"achtige Nutzeraktivit\"aten aufdecken
        \item Aufdecken von Problemen mit Autorisations- und Zugriffskontrollmechanismen
      \end{itemize}
      In Microsoft SQL Server kann das Auditing auf zwei Ebenen stattfinden, als
      Serverüberwachung oder als Datenbanküberwachung.
      \begin{merke}
        Datenbanküberwachungen sind nicht in allen SQL Server Editionen
        verfügbar. Sie existieren lediglich in der Enterprise, der Developer
        oder der Eval-Edition.
      \end{merke}
    \section{Bestandteile des SQL Server Auditings}
      \subsection{Überwachungen}
        Überwachungen sind Objekte die auf Server-Instanzebene erstellt
        werden. Sie regeln verschiedene Dinge, wie z. B.:
        \begin{itemize}
          \item Was passiert, wenn das Auditing nicht durchgeführt werden
          kann?
          \item Wo werden die Auditing-Daten gespeichert?
          \item Maximale Größe/Anzahl der Auditing-Dateien?
        \end{itemize}
        Überwachungen stellen die Grundlage für das Auditing dar. Alle
        weiteren Komponenten basieren auf Ihnen und können deshalb auch nur
        dann erstellt werden, wenn eine Überwachung existiert.
      \subsection{Serverüberwachungsspezifikationen}
        In einer Serverüberwachungsspezifikation kann der Administrator
        eine Liste von Überwachungsaktionsgruppen erstellen, um so
        verschiedene Ereignisse/Aktionen auf Server-Instanzebene zu
        überwachen.
        
        Nach Aktivierung der Serverüberwachungsspezifikation werden die
        gesammelten Überwachungsdaten an eine Überwachung gesendet, die diese
        wiederum in einem Überwachungsziel speichert.
        \begin{merke}
          Überwachungen und Serverüberwachungsspezifikationen stehen in einer
          1:1-Beziehung zueinander. Jede Überwachung kann nur mit genau einer
          Serverüberwachungsspezifikation verbunden werden und eine
          Serverüberwachungsspezifikationen ist ohne Überwachung nicht
          funktionsfähig.
        \end{merke}
      \subsection{Datenbanküberwachugnsspezifikationen}
        Datenbanküberwachugnsspezifikationen sind in ihrer Arbeits- und
        Funktionsweise mit den Serverüberwachungsspezifikationen identisch.
        Sie zeichnen Überwachungsaktionen auf und leiten diese an eine
        Überwachung weiter.
        \begin{merke}
          Genau wie bei den Serverüberwachungsspezifikationen gilt auch hier,
          dass zwischen Datenbanküberwachugnsspezifikation und Überwachung
          eine 1:1-Beziehung existiert.
        \end{merke}
      \subsection{Überwachungsaktionen und Überwachungsaktionsgruppen}
        Unter Überwachungsaktionen bzw. Aktionsgruppen versteht man einzelne
        Ereignisse und Gruppen von Ereignissen, die im SQL Server auf treten
        und überwacht werden können. Sie existieren auf Server- und
        Datenbankebe, außerdem gibt es Überwachungsereignisse, die zur
        Überwachung des Auditing notwendig sind.
        \begin{merke}
          Diese Ereignisse bzw. Gruppen basieren auf den \enquote{Extended
          Events}, welche noch an späterer Stelle in dieser Unterlage behandelt
          werden.
        \end{merke}
        \begin{literaturinternet}
          \item \cite{cc280663}
        \end{literaturinternet}
      \subsection{Überwachungsziele}
        Unter dem Begriff \enquote{Überwachungsziel} versteht der SQL Server
        einen Speicherort für die Auditing-Datensätze. Hierfür stehen drei
        Möglichkeiten zur Verfügung:
        \begin{itemize}
          \item \textbf{Datei}: Die Auditing-Datensätze werden in einer Datei
          bzw. eine Gruppe von Dateien auf einem Datenträger abgelegt.
          \item \textbf{Anwendungsprotokoll}: SQL Server kann seine
          Auditing-Daten in das zentrale Auditing-Protokoll des
          Betriebssystems, das Anwendungsprotokoll schreiben.
          \item \textbf{Sicherheitsprotokoll}: SQL Server kann seine
          Auditing-Daten in das Sicherheitsprotokoll, eine gut geschützte
          Alternative zum Anwendungsprotokoll, schreiben.
        \end{itemize}
    \section{Erstellen und verwalten von Überwachungen}
      \subsection{Konfiguration und Benutzung von Überwachungszielen}
        Für die Erstellung einer Überwachung muss ein Überwachungsziel
        ausgewählt werden. Dieses Ziel bestimmt, wo die Überwachungsdaten
        gespeichert werden.
        \subsubsection{Ein Dateipfad als Überwachungsziel}
        
        \subsubsection{Datensätze in das Anwendungsprotokoll schreiben}
          Das Windows-Anwendungsprotokoll ist eine zentrale Sammelstelle für
          Ereignisse des Betriebssystems und anderer Softwareprodukte.
          Grundsätzlich kann jede Software auf unkomplizierte Weise ihre
          Einträge in das Anwendungsprotokoll schreiben und jeder Benutzer hat
          lesenden Zugriff darauf. Es ist kein zusätzlicher
          Konfigurationsaufwand notwendig, um dieses Protokoll zu
          benutzen. Daraus resultiert, dass das Anwendungsprotokoll zwar einfach
          und praktisch zu Handhaben, jedoch auch sehr unsicher ist.
          \begin{merke}
            Dieses Überwachungsziel ist ungeeignet, wenn der Administrator des
            SQL Servers und der \enquote{Sicherheitsbeauftragte} zwei
            unterschiedliche Personen sind, da der SQL Server Administrator
            nicht am Zugriff auf das Anwendungsprotokoll gehindert werden kann.
          \end{merke}
        \subsubsection{Benutzung des Sicherheitsprotokolls vorbereiten}
      \subsection{Erstellen von Überwachungen mit dem SSMS}
        \subsubsection{Erstellen der Überwachung}
          \begin{enumerate}
            \item Wählen Sie im Objekt-Explorer das Menü \enquote{Sicherheit} auf
            Instanzebene und öffnen Sie dort das Kontextmenü des Punktes
            \enquote{Überwachungen}.
            \bild{Erstel-len einer Über-wachung}{create_audit_spec_1}{1.3}
            \item Klicken Sie auf \enquote{Neue Überwachung\ldots}
            \bild{Erstel-len einer Über-wachung}{create_audit_spec_2}{1.3}
            \item Es öffnet sich der Dialog \enquote{Überwachung erstellen}
            \bild{Erstel-len einer Über-wachung}{create_audit_spec_3}{1.1}
            \item Tragen Sie die folgenden Angaben ein:
            \begin{itemize}
              \item \textbf{Überwachungsname}: Serverüberwachungen
              \item \textbf{Überwachungsziel}: Datei
              \item \textbf{Dateipfad}: \oscommand{D:\textbackslash
              audit\textbackslash mssqlserver} (Pfad vorher anlegen)
              \item \textbf{Überwachungsdateien}: Option \enquote{Unbegrenzt}
              entfernen
              \item \textbf{Anzahl der Dateien}: 10
              \item \textbf{Maximale Dateigröße}: 10 MB
            \end{itemize}
            \item Klicken Sie auf OK.
          \end{enumerate}
        \subsubsection{Bedeutung der Optionen für das Überwachungsziel}
          \begin{itemize}
            \item \textbf{Dateipfad}: In diesem Verzeichnis wird ein Ordner
            angelegt, welcher die Überwachungsdaten aufnimmt.
            \item \textbf{Anzahl der Dateien}: Durch diese Angabe wird
            festgelegt, nach wie vielen Dateien der SQL Server anfängt, die
            erste Datei wieder zu überschreiben (Rollover). Dadurch wird eine
            zyklische Wiederverwendung der Audit-Dateien erreicht.
            \item \textbf{Maximale Dateigröße}: Gibt die maximale Größe einer
            Audit-Datei an.
          \end{itemize}
        \subsubsection{Bedeutung der Optionen für Überwachungsprotokollfehler}
          Es kann vorkommen, dass der SQL Server aus einem unbekannten Grund
          nicht mehr in der lage ist die Daten des Auditings zu speichern. Da
          Auditing aber meist dazu benutzt wird, um sicherheitskritische
          Systeme zu überwachen, ist der Ausfall des Überwachungszieles selbst
          auch ein als kritisch zu bewertender Vorfall. Der Administrator hat
          drei verschiedene Optionen zur Verfügung, um das Verhalten des SQL
          Servers in einem solchen Fall zu beeinflussen und angemessen auf den
          Ausfall zu reagieren.
          \begin{itemize}
            \item \textbf{Weiter}: Das Auditing läuft weiter und der SQL Server
            versucht auch weiterhin Daten in das Überwachungsziel zu schreiben.
            Diese Option ermöglich es unter Umständen, dass Benutzer Aktivitäten
            auf dem Server ausführen, welche durch Ihre Sicherheitsrichtlinien
            verboten sind, ohne das dies Protokolliert werden kann.
            \begin{merke}
              Wählen Sie diese Option nur dann aus, wenn die Verfügbarkeit des
              Datenbankmoduls wichtiger ist als die Sicherheit des Systems.
            \end{merke}
            \item \textbf{Server herunterfahren}: Ist diese Option ausgewählt,
            wird der Server automatisch beim Ausfall des Überwachungszieles
            heruntergefahren.
            \begin{merke}
              Das Herunterfahren des Servers bietet zwar ein hohes Maß an
              Sicherheit, es kann aber auch für Denial-Of-Service Attacken
              genutzt werden. Verwenden Sie diese Option nur dann, wenn die
              Sicherheit des Servers höher bewertet wird, als die Verfügbarkeit.
            \end{merke}
            \item \textbf{Fehlerhafter Vorgang}: Diese Option stellt einen
            Kompromiss aus \enquote{Weiter} und \enquote{Server herunterfahren}
            dar. Der Server blockiert alle Aktionen, welche eine Überwachung
            nach sich ziehen würden. Andere nicht überwachte Aktivitäten sind
            problemlos möglich. Der Server belibt verfügbar.
          \end{itemize}
          \begin{literaturinternet}
            \item \cite{cc280525}
          \end{literaturinternet}
      \subsection{Erstellen von Überwachungen mit SQL}
        Wahlweise kann eine Überwachung auch mittels SQL-Statement erstellt
        werden.
        \begin{lstlisting}[language=ms_sql,caption={Erstellen
        einer Überwachung},label=sql21_01]
USE [master]
GO

CREATE SERVER AUDIT [Server\"uberwachungen]
TO FILE 
( FILEPATH = N'D:\audit\mssqlserver',
  MAXSIZE = 10 MB,
  MAX_ROLLOVER_FILES = 10,
  RESERVE_DISK_SPACE = OFF
)
WITH
( QUEUE_DELAY = 1000,
  ON_FAILURE = CONTINUE
);

GO          
        \end{lstlisting}
        \beispiel{sql21_01} zeigt, wie die Überwachung
        \enquote{Serverüberwachung}, die zuvor mit Hilfe des SSMS erstellt wurde
        mittels SQL-Kommand kreiert wird.
      \subsection{Überwachungsziel Datei}
        \subsubsection{NTFS-Berechtigungen}
      
      \subsection{Überwachungsziel Anwendungsprotokoll}
      
      \subsection{Überwachungsziel Sicherheitskprotokoll}
      
    \section{Auditing auf Serverebene}
      
      \subsection{Serverüberwachungsspezifikationen} 
    \section{Auditing auf Datenbankebene}
    
