  \chapter{Installation des SQL Server 2014}
  \label{installation_of_sql_server_2014}
  \chaptertoc{}
  \cleardoubleevenpage
    Mit dem Betrieb eines Windows Dienstes ergeben sich in der Praxis verschiedene
    administrative Probleme. Jeder Dienst benötigt ein Dienstkonto, welches in
    manchen Fällen weitreichende Berechtigungen haben muss. Um den Server sicher
    zu gestalten, muss sich der Administrator unter anderem darum kümmern, dass
    das Konto ein sicheres Passwort besitzt, welches in regelmäßigen Zeitabständen
    gewechselt werden sollte.
    
    In der Vergangenheit wurde dieses Problem häufig dadurch gelöst, dass für alle
    Dienste eines Servers die lokalen Konten \identifier{Lokaler Dienst},
    \identifier{Netzwerkdienst} oder \identifier{Lokales System} genutzt wurden. Ein
    anderer Lösungsansatz bestand darin, normale Domänenkonten mit den passenden
    Berechtigungen auszustatten, um so eine Verwaltbarkeit der Dienstkonten in der
    Domäne zu erreichen.
    
    Mit Windows Server 2008 R2 / Windows 7 führte Microsoft zwei neue Kontoarten
    ein, die zur Entlastung der Administratoren und zur Verbesserung der
    Sicherheit dienen sollten. Es handelt sich hierbei um die \enquote{Managed
    Service Accounts} (MSA) und die \enquote{Virtual Accounts}.
    \section{Managed Service Accounts}
      Managed Service Accounts sind genau das, was ihr Name verspricht:
      Automatisch verwaltete Domänenaccounts, speziell für den Betrieb von
      Windows Diensten.
      \subsection{Fähigkeiten}
        Um die gesetzten Ziele für Verwaltbarkeit und Sicherheit erreichen zu
        können, sind dieser Kontoart eine ganze Reihe neuer Fähigkeiten
        verliehen worden. Im Einzelnen sind dies:
        \begin{itemize}
            \item Automatische Passwortverwaltung
            \item Vereinfachte Verwaltung von SPNs
            \item Direkte Verknüpfung mit genau einem Computerkonto
            \item Statische Bindung an genau einen Zielserver
            \item Delegation der SPN-Verwaltung an andere Administratoren
        \end{itemize}
        Gerade die ersten beiden Punkte stellen eine wesentliche
        Arbeitserleichterung für den Administrator dar. An seiner statt
        übernimmt das Active Directory die Kennwortverwaltung für alle MSAs. Es
        gelten dabei folgende Regeln:
\clearpage
        \begin{itemize}
            \item Einem MSA wird ein stark verschlüsselts Passwort mit einer Länge
            von 120 Zeichen zugewiesen. Diese Passwörter bestehen aus Buchstaben,
            Ziffern und Sonderzeichen.
            \item Das Kennwort wird automatisch alle 30 Tage geändert. Dieses
            Intervall kann durch den Administrator beeinflusst werden.
        \end{itemize}
        Der Administrator könnte zwar eine manuelle Passwortänderung
        veranlassen, jedoch ist ein Eingreifen seiner seits, im Normalfall,
        nicht erforderlich. Durch die Verknüpfung eines MSA mit einem
        Computerkonto wird erreicht, dass das MSA als Zwidder aus Benutzer- und
        Computerkonto erscheint. Dadurch vereinte es die Vorteile beider
        Kontoarten.
      \subsection{Nachteile}
        Nach einem kurzen Blick hinter die Kulissen fallen aber sofort zwei ganz
        bedeutende Nachteile auf:
        \begin{itemize}
          \item Statische Bindung an genau einen Zielserver, was eine Nutzung auf
          einem geclusterten System unmöglich macht und
          \item Dienste wie Exchange und SQL Server werden nicht unterstützt.
        \end{itemize}
        \begin{merke}
          Ein Nachteil, der sich aus der statischen Bindung zwischen MSA und
          Zielserver ergibt ist der, dass MSAs auf geclusterten Systemen nicht
          eingesetzt werden können, da dort ein Dienst wechselweise auf mehreren
          Clusterknoten betrieben werden kann.
        \end{merke}
        \bild{MSA - Computer\-konto - Client}{msa_computeraccount_client}{0.15}
        Mit Windows Server 2012 kommt jedoch die Erlösung für alle Admins. Die
        erweiterte Fassung der Managed Service Accounts, die \enquote{Group
        Managed Service Accounts}, kurz \enquote{gMSA}. Diese bieten nun alle
        Vorteile der MSAs, inklusive der Tatsache, dass die gerade genannten
        Nachteile überwunden sind.
        \begin{merke}
          Um im Folgenden MSAs und gMSAs besser unterscheiden zu können, werden
          MSAs als \enquote{standalone Managed Service Accounts}, kurz sMSA,
          bezeichnet!
        \end{merke}
    \section{Group Managed Service Accounts}
      \subsection{Grundvoraussetzungen}
        Um gMSAs nutzen zu können, müssen Clients mit Windows XP oder höher
        betrieben werden, und es muss sich mindestens ein Domänencontroller mit
        Windows Server 2012 in der Domänengesamtstruktur befinden. Außerdem
        wird ein Windows Server 2012 oder Windows 8 Rechner mit dem
        \enquote{Active Directory-Modul für Windows Powershell} benötigt.
        \begin{merke}
          Wird Windows 7 als Clientbetriebssystem eingesetzt, sollte das
          Hotfix KB2494158 \enquote{Managed service account authentication fails
          after its password is changend} installiert werden!
        \end{merke}
        Sollte die Domäne noch auf dem Windows Server 2003 / 2008 Level betrieben
        werden, muss der Administrator einige zusätzliche Konfigurationsschritte
        ausführen.
        \begin{enumerate}
            \item Ausführen von \texttt{adprep /forestprep} auf Forest-Ebene.
            \item Ausführen von \texttt{adprep /domainprep} in jeder Domäne,
            welche mit MSAs umgehen können muss.
            \item Ausrollen eines Domaincontrollers mit Windows Server 2012.
        \end{enumerate}
        \begin{merke}
          In dieser Unterrichtsunterlage wird davon ausgegangen, dass als
          Betriebssystem \enquote{Windows Server 2012 R2} zum Einsatz kommt!
        \end{merke}
        Um die Verwaltung von gMSAs zu ermöglichen, müssen drei Voraussetzungen
        auf dem betreffenden Admin-PC bzw. in der Domäne getroffen werden:
        \begin{itemize}
            \item Installation des .NET Framework 3.5.1
            \item Installation des \enquote{Active Directory-Modul für Windows
            Powershell}
            \item Erstellung eines \enquote{KDS\footnote{KDS = Key Distribution
            Service (neu in Windows Server 2012)} Root Key} in der Domäne
        \end{itemize}
        \subsubsection{Installation des .NET Framework 3.5.1}
          Installieren Sie das .NET Framework 3.5.1 und das Active
          Directory-Modul für Windows Powershell wie folgt:
          \begin{enumerate}
              \item Legen Sie die Windows Server 2012 - DVD ins Laufwerk ein!
              \item Starten Sie den Server Manager!
              \item Klicken Sie im Menü \enquote{Verwalten} auf
              \enquote{Rollen und Features hinzufügen}! Es öffnet sich der
              \enquote{Assistent zum Hinzufügen von Rollen und Features}.
              \item Wählen Sie im Fenster \enquote{Installationstyp auswählen} die
              Option \enquote{Rollenbasierte oder featurebasierte Installation}!
              \item Im Dialogfenster \enquote{Zielserver auswählen} wählen Sie den
              betreffenden Server aus!
              \item Überspringen Sie den Dialog \enquote{Serverrollen auswählen}!
              \item Öffnen Sie im Dialogfenster \enquote{Features auswählen} die
              Option \enquote{.NET Framework 3.5-Funktionen} und wählen Sie das \enquote{.NET
              Framework 3.5} aus!
              \bild{Assistent zum Hinzufügen von Rollen und Features -
              1}{sql_server_install_dotnetfx35_1}{0.42}
              \item Öffnen Sie die Option \enquote{Remoteserver-Verwaltungstools},
              \enquote{Rollenverwaltungstools}, \enquote{AD DS- und AD LDS-Tools}
              und wählen Sie \enquote{Active Directory-Modul für Windows
              Powershell}. Klicken Sie auf \enquote{Weiter}!
              \bild{Assistent zum Hinzufügen von Rollen und Features -
              2}{sql_server_AD_Module_for_windows_powershell_1}{0.42}
              \item Sie werden nun aufgefordert, einen alternativen Quellspfad für
              das .NET Framework anzugeben, da sich das .NET-Installationspaket noch
              nicht auf Ihrem Rechner befindet. Klicken Sie unten links auf den
              Link \enquote{Alternativen Quellpfad angeben}.
              \bild{Assistent zum Hinzufügen von Rollen und Features -
              3}{sql_server_install_dotnetfx35_2}{0.42}
\clearpage
              \item Geben Sie folgenden Pfad im soeben erscheinenden Dialogfenster
              an:
              \\
              \texttt{<DVD-Laufwerk>\textbackslash sources\textbackslash sxs}
              \bild{Assistent zum Hinzufügen von Rollen und Features -
              4}{sql_server_install_dotnetfx35_3}{0.41}
              \item Klicken Sie auf \enquote{OK}!
              \item Klicken Sie auf \enquote{Installieren} um den
              Installationsvorgang zu starten
              \bild{Assistent zum Hinzufügen von Rollen und Features -
              5}{sql_server_install_dotnetfx35_4}{0.41}
          \end{enumerate}
        \subsubsection{Bereitstellung des KDS Root Key}
          Die Bereitstellung des KDS Root Key erfolgt nun mittels der Powershell.
          Vorausetzung für die folgende Prozedur ist die Mitgliedschaft in der
          Gruppe der Domänen-Admins bzw. Enterprise-Admins.
          \begin{enumerate}
              \item Starten Sie die Powershell ISE, indem Sie auf den Windows-Button
              klicken und als Suchbegriff \enquote{ISE} eingeben! Es erscheint am
              rechten Bildschirmrand eine Auflistung der Suchergebnisse.
              \bild{Starten der Powershell ISE}{start_ise}{0.21}
              \item Erstellen Sie den KDS Root Key.
              \begin{lstlisting}[language=powershell, caption={Erstellen eines KDS
              Root Keys}, label=create_kds_root_key]
Add-KDSRootKey |-EffectiveImmediately|
              \end{lstlisting}
            \end{enumerate}
          \begin{merke}
            Wurde der KDS Root key auf diese Art und Weise erzeugt, muss erst eine
            Sicherheitsfrist von 10 Stunden verstreichen, bevor ein gMSA genutzt
            werden kann.
          \end{merke}
          Diese Frist existiert, damit in einer Umgebung mit mehreren
          Domänen-Controllern sichergestellt wird, dass wirklich alle DCs den KDS
          Root Key repliziert bekommen, bevor ein erster gMSA genutzt wird.
          Andernfalls könnte es passieren, dass ein Domänen-Controller die
          Passwortanfrage eines gMSA nicht beantworten und somit ein Dienst nicht
          starten kann.
  
          Um diese Frist in einer Domäne mit nur einem DC zu umgehen, muss das
          \languagepowershell{Add-KDSRootKey}-Commandlet mit einer zusätzlichen
          Angabe ausgeführt werden:
            \begin{lstlisting}[language=powershell, caption={Erstellen eines KDS
              Root Keys}, label=create_kds_root_key_2]
  Add-KDSRootKey |-EffectiveTime| ((get-date)§.addhours§(-10))
          \end{lstlisting}
          \begin{literaturinternet}
            \item \cite{jj128430}
          \end{literaturinternet}
      \subsection{Kurzer Ausflug: AGDLP}
        \enquote{AGDLP} - Account, Global Group, Domain local Group, Permission - 
        bezeichnet ein Verfahren für die rollenbasierte Zuteilung von
        Berechtigungen (RBAC\footnote{RBAC = Role-Based Access Controls}).
        Seine Aufgabe ist es, für ein strukturiertes Rollen-Rechte-Konzept in
        einer Active Directory Domäne oder sogar in einem Forest zu sorgen.
  
        Hinter der Abkürzung AGDLP steckt im Wesentlichen die Idee, dass
        Berechtigungen nicht einem einzelnen Objekt, z. B. einem Nutzer
        zugeteilt werden, sondern einer Gruppe, deren Mitglieder dann die
        Berechtigungen erben. In einer Active Directory Domäne existieren für
        diesen Zweck zwei unterschiedliche Gruppenarten:
          \begin{itemize}
              \item \textbf{Globale Gruppen}: Mit Hilfe dieser Gruppen werden
              Rollen oder Tätigkeiten abgebildet (z. B. DB-Admin, AccountManager
              usw.). In diesen Gruppen dürfen Nutzer und andere globale Gruppen
              der gleichen Domäne Mitglied sein und sie können von Resourcen
              anderer Domänen genutzt werden.
              \item \textbf{Domänen-lokale Gruppen}: Diese Gruppenart ist es, die die
              Berechtigungen zugewiesen bekommt. Es können universelle, globale
              und lokale Gruppen Mitglied sein. Auch Nutzerkonten sind als
              Mitglieder erlaubt. Domänen-lokale Gruppen können nur innerhalb der
              eigenen Domäne genutzt werden.
          \end{itemize}
          Gemäß der Empfehlung von Microsoft sollte AGDLP wie folgt umgesetzt
          werden:
          \begin{itemize}
              \item Zuerst wird eine globale Gruppe erstellt.
              \item Alle betreffenden Accounts werden Mitglieder der globalen
              Gruppe.
              \item Es wird eine domänen-lokale Gruppe erstellt. Diese erhält alle
              vorgesehenen Berechtigungen.
              \item Die globale Gruppe wird Mitglied der domänen-lokalen Gruppe.
          \end{itemize}
          \begin{literaturinternet}
            \item \cite{AGDLP}
          \end{literaturinternet}
          \bild{AGDLP}{agdlp}{0.15}
      \subsection{Einen Managed Service Account erstellen}
        \begin{merke}
          Um einen MSA erstellen zu können muss der Benutzer
          \begin{itemize}
              \item der Domänen-Administrator sein oder
              \item ein Mitglied der Gruppe der Domänen-Admins sein oder
              \item ein Mitglied der Gruppe der Konten-Operatoren sein oder
              \item er muss die \enquote{Create/DeletemsDS-ManagedServiceAccount}
              Berechtigung haben.
          \end{itemize}
        \end{merke}
        \begin{enumerate}
            \item Starten Sie die Powershell ISE
            \item Nachdem die ISE gestartet hat, importieren Sie zuerst das
            Active Directory-Modul mit dem Kommando:
            \begin{lstlisting}[language=powershell, caption={Importieren des
            Active Directory-Moduls für Windows Powershell}, label=create_msa_1]
Import-Module ActiveDirectory
            \end{lstlisting}
            \item Erstellen Sie eine neue Sicherheitsgruppe im Active Directory
            mit Hifle des Commandlets \languagepowershell{New-ADGroup}. Dieser
            Schritt kann auch mittels der MMC durchgeführt werden.
            \begin{lstlisting}[language=powershell, caption={Eine globale
            Sicherheitsgruppe erstellen}, label=create_msa_2]
New-ADGroup |-Name| SQLServerSVC |-GroupScope| Global `
|-GroupCategory| Security |-SamAccountName| SQLServerSVC$ `
|-Path| "OU=Groups, OU=MS-C-IX-04-20, DC=MS-C-IX-04, DC=FUS"
            \end{lstlisting}
            \bild{Die Gruppe \enquote{SQLServerSVC}}{sqlserversvc}{0.4}
            \item Fügen Sie Ihr Computerkonto der Gruppe \identifier{SQLServerSVC}
            als Mitglied hinzu. Dieser Schritt kann mittels Powershell oder mit
            Hilfe der Microsoft Management Konsole geschehen.
            \begin{lstlisting}[language=powershell, caption={Einer Gruppe ein
            Mitglied hinzufügen}, label=create_msa_3]
Add-ADGroupMember `
|-Identity| `
"CN=SQLServerSVC,OU=Groups,OU=MS-C-IX-04-20,DC=MS-C-IX-04,DC=FUS"
|-Members| `
"CN=FEA11-119SRV20,OU=Computers,OU=MS-C-IX-04-20,DC=MS-C-IX-04,DC=FUS"
            \end{lstlisting}
            \bild{Gruppen\-mitglieder von
            SQLServerSVC}{sqlserversvc_group_members}{0.4}
            \item Benutzen Sie das Commandlet \languagepowershell{New-ADServiceAccount} um
            einen neuen MSA zu erstellen.
            \begin{lstlisting}[language=powershell, caption={Erstellen des MSA},
            label=create_msa_4]
New-ADServiceAccount |-Name| MSSQLServer20 |-SamAccountName| MSSQLServer20$ `
|-DNSHostname| FEA11-119SRVAD.MS-C-IX-04.FUS `
|-Path| "OU=MSA,OU=MS-C-IX-04-20,DC=MS-C-IX-04,DC=FUS" `
|-PrincipalsAllowedToRetrieveManagedPassword| `
"CN=SQLServerSVC,OU=Groups,OU=MS-C-IX-04-20,DC=MS-C-IX-04,DC=FUS" `
|-Enabled| ~$true~
            \end{lstlisting}
        \end{enumerate}
        Das Commandlet \languagepowershell{New-ADServiceAccount} nimmt aktuell
        ca. 20 verschiedene Parameter entgegen. In \beispiel{create_msa_4} wurden
        nur die wichtigsten benutzt. Welche Bedeutung sie haben wird nachfolgend
        erläutert:
        \begin{itemize}
            \item \languagepowershell{|-Name|}: Setzt das \identifier{Name}-Attribute
            des Accounts im Active Directory
            \item \languagepowershell{|-SamAccountName|}: Gibt den Security
            Account Managed Name des gMSA an. Dieser darf nicht länger als 20
            Zeichen sein, um die Kompatibilität zu älteren Systemen zu wahren.
            Wird am Ende des SamAccountName kein \$ angegeben, hängt Windows
            automatisch eines an.
            \item \languagepowershell{|-DNSHostname|}: Gibt den Namen eines
            DNS-Servers an. Dieser Parameter wird benötigt, um den
            Domänen-Controller auffinden zu können.
            \item \languagepowershell{|-Path|}: Nimmt einen X.500 konformen Pfad
            entgegen, mit dessen Hilfe angegeben wird, wo im AD der gMSA erstellt
            werden soll.
            \item
            \languagepowershell{|-PrincipalsAllowedToRetrieveManagedPassword|}:
            Setzt das \enquote{msDS-GroupMSAMembership}-Attribute und gibt damit
            an, welche Computer den gMSA nutzen können.
            \item \languagepowershell{|-Enabled|}: Gibt an, ob der gMSA aktiviert
            oder deaktivert ist. Es sind nur die beiden Werte
            \languagepowershell{\$true} und \languagepowershell{\$false}
            zulässig.
        \end{itemize}
        In\beispiel{create_msa_2} wird für den Parameter
        \languagepowershell{|-PrincipalsAllowedToRetrieveManagedPassword|} die
        Gruppe \identifier{SQLServerSVC} benutzt. Alle Computerkonten, die dieser
        Gruppe als Mitglieder zugewiesen werden haben dadurch zugriff auf den gMSA
        \identifier{MSSQLServer20}.
        Sollte aus irgend einem Grund nicht mehr bekannt sein, welche
        Benutzer/Gruppen berechtigt sind, dass gMSA zu nutzen
        (\languagepowershell{|-PrincipalsAllowedToRetrieveManagedPassword|}), kann
        dies mittels des Commandlets \languagepowershell{Get-ADServiceAccount}
        nachgeschlagen werden.
        \begin{lstlisting}[language=powershell, caption={Einer Gruppe ein
        Mitglied hinzufügen}, label=create_msa_5]
Get-ADServiceAccount `
|-Identity| "CN=MSSQLServer20,OU=MSA,OU=MS-C-IX-04-20,DC=MS-C-IX-04,DC=FUS" `
|-properties| principalsallowedtoretrievemanagedpassword
        \end{lstlisting}
        \begin{literaturinternet}
          \item \cite{hh852236}
          \item \cite{ee617258}
          \item \cite{askpfeplat20121217}
        \end{literaturinternet}
      \subsection{(De-)Installation eines gMSA auf einem Zielserver}
        \subsubsection{Installation eines gMSA auf dem Zielserver}
          Die Installation eines gMSA auf dem Zielrechner ist kein zwingend
          notwendiger Schritt. Warum sollte sie dennoch erfolgen? Durch die
          Installation wird der betroffene Server ermächtigt, die periodische
          Passwortänderung für den gMSA durchzuführen. Ohne Installation müsste
          dieser entscheidende Punkt manuell durchgeführt werden.
          \begin{merke}
            Für die Installation des MSA auf einem Clientcomputer muss der
            Benutzer ein Mitglied der Gruppe der lokalen Administratoren sein.
          \end{merke}
          Installiert wird der gMSA mit dem Commandlet
          \languagepowershell{Install-ADServiceAccount}.
          \begin{enumerate}
              \item Booten Sie Ihren Windows Server neu! 
              \item Starten Sie die Powershell ISE im Administratormodus (Als
              Administrator ausführen)!
              \item Führen Sie das genannte Commandlet aus!
                \begin{lstlisting}[language=powershell, caption={Einen MSA auf dem
                Zielserver installieren}, label=install_msa_1]
Install-ADServiceAccount `
|-Identity| "CN=MSSQLServer20,OU=MSA,OU=MS-C-IX-04-20,DC=MS-C-IX-04,DC=FUS"
                \end{lstlisting}
              \item Testen Sie, ob die Installation erfolgreich war!
                \begin{lstlisting}[language=powershell, caption={Prüfen des
                Installationserfolges}, label=install_msa_2]
Test-ADServiceAccount `
|-Identity| "CN=MSSQLServer20,OU=MSA,OU=MS-C-IX-04-20,DC=MS-C-IX-04,DC=FUS"
                \end{lstlisting}
          \end{enumerate}
          Sollte die Installation erfolgreich gewesen sein, liefert das Commandlet
          \languagepowershell{Test-AdServiceAccount} den Wert
          \languagepowershell{\$true} zurück. Anderenfalls wird eine detailierte
          Fehlermeldung ausgegeben.
          \begin{merke}
            Damit dieser Vorgang erfolgreich sein kann, muss die ISE zwingend im
            Administratormodus ausgeführt werden!
          \end{merke}
          \begin{literaturinternet}
            \item \cite{ee617223}
          \end{literaturinternet}
  \clearpage        
        \subsubsection{Deinstallation eines gMSA vom Zielserver}
          Deinstalliert wird ein gMSA mit dem
          \languagepowershell{Uninstall-ADServiceAccount} Commandlet.
          \begin{merke}
            Auch hier muss die ISE zwingend im Administratormodus (Als
            Administrator ausführen) gestartet werden!
          \end{merke}
          \begin{enumerate}
              \item Starten Sie die Powershell ISE im Administratormodus (Als
              Administrator ausführen)!
              \item Führen Sie das genannte Commandlet aus!
                \begin{lstlisting}[language=powershell, caption={Einen MSA vom
                Zielserver deinstallieren}, label=uninstall_msa_1]
  Uninstall-ADServiceAccount `
  |-Identity| "CN=MSSQLServer20,OU=MSA,OU=MS-C-IX-04-20,DC=MS-C-IX-04,DC=FUS"
          \end{lstlisting}
              \item Testen Sie, ob die Deinstallation erfolgreich war!
                \begin{lstlisting}[language=powershell, caption={Prüfen des
                Deinstallationserfolges}, label=uninstall_msa_2]
  Test-ADServiceAccount `
  |-Identity| "CN=MSSQLServer20,OU=MSA,OU=MS-C-IX-04-20,DC=MS-C-IX-04,DC=FUS"
                \end{lstlisting}
          \end{enumerate}
          \begin{literaturinternet}
            \item \cite{ee617202}
          \end{literaturinternet}
        \subsubsection{Zurücksetzen des Passworts}
          Das Passwort eines gMSA kann mit Hilfe des Commandlets
          \languagepowershell{Reset-ADServiceAccountPassword} geschehen.
          \begin{lstlisting}[language=powershell, caption={Das Passwort eines
          MSA mauell zurücksetzen}, label=reset_msa_password]
  Reset-ADServiceAccountPassword `
  "CN=MSSQLServer20,OU=MSA,OU=MS-C-IX-04-20,DC=MS-C-IX-04,DC=FUS"
          \end{lstlisting}    
        \begin{literaturinternet}
          \item \cite{dd391923}
          \item \cite{jj128431}
          \item \cite{sqlosteam20140219}
          \item \cite{ee617202}
        \end{literaturinternet}
    \section{Virtual Accounts}
      Virtual Accounts sind das lokale Gegenstück zu den MSAs. Sie existieren nur
      auf dem lokalen Rechner und erfordern wenig/keinen administrativen Aufwand.
      Für diese Kontoart gelten die folgenden Punkte:
      \begin{itemize}
          \item Der Name eines virtuellen Kontos beginnt immer mit \enquote{NT
          Service\textbackslash}
          \item Es existiert, ohne das es vorher angelegt werden muss. 
          \item Die Kennwortverwaltung läuft automatisch ab, wie bei den MSAs
          \item Es erscheint nicht in der MMC \enquote{Lokale Benutzer und
          Gruppen}
          \item Es kann in jeder ACL mit aufgeführt werden
          \item Benötigt es Zugriff auf externe Ressourcen im Netzwerk, geschieht
          dies mit den Berechtigungen des Computerkonto des eigenen Servers.
      \end{itemize}
      Virtual Accounts sind somit ideal für einfache Dienste, die hauptsächlich
      Zugriff auf lokale Ressourcen benötigen. Erstellen kann man Sie, indem man
      in den Eigenschaften eines Dienstes einfach einen neuen Benutzernamen, z. B.
      \identifier{NT Service\textbackslash SQLServer} einträgt. Der Benutzer muss
      nicht vorher angelegt werden.
      \bild{Ein virtueller Account für Google Update}{virtual_account}{0.42}
      \begin{literaturinternet}
        \item \cite{vafduw78as2008r2}
        \item \cite{msavvaiws2008r2}
      \end{literaturinternet}
    \section{Installation des SQL Server 2014}
      \subsection{Systemvoraussetzungen}
        Eine vollständige Liste aller Hardware- und Softwareanforderungen für
        SQL Server 2014 kann dem Microsoft Artikel \parencite{ms143506}
        entnommen werden. An dieser Stelle erfolgt lediglich eine kurze Auflistung
        der notwendigsten Informationen.
        \subsubsection{Softwareanforderungen}
          Die in der folgenden Auflistung gezeigten Softwareprodukte müssen
          zwingend vor der Installation des SQL Server 2014 auf dem System
          vorhanden sein, andernfalls kann der Installationsvorgang nicht begonnen
          werden:
          \begin{itemize}
              \item \textbf{.NET Framework 3.5.1}: Wird der SQL Server auf einem
              Computer mit Windows Vista SP2 oder Windows Server 2008 SP2
              installiert, muss das .NET Framework 3.5.1 aus dem Internet
              heruntergeladen und installiert werden. Bei Windows Server 2008 R2
              SP1 ist das .NET Framework bereits vorhanden und muss nur noch aktiviert
              werden.
              \item \textbf{.NET Framework 4.0}: Die Version 4.0 des .NET
              Frameworks wird automatisch während des SQL Server-Setup Vorganges
              installiert. Ein Eingreifen des Administrators ist nicht notwendig.
              \item \textbf{Windows Powershell}: Die Windows Powershell 2.0 wird
              von der Database Engine und verschieden anderen Komponenten
              benötigt, weshalb auch sie im Vorfeld aktiviert werden muss.
          \end{itemize}
        \subsubsection{Hardwareanforderungen}
          Die folgende Liste zeigt die Hardwareanforderungen, die der Server, auf
          dem Microsoft SQL Server 2014 installiert wird, erfüllen sollte. Es
          werden immer die Mindestanforderungen sowie die optimalen Bedingungen
          angegeben.
          \begin{itemize}
              \item \textbf{CPU}: Die Mindesttaktrate des Prozessors beträgt 1.4
              GHz. Microsoft empfiehlt aber eine Taktrate von 2.0 GHz oder höher,
              da die performante Ausführung von SQL-Anweisungen sehr viel Leistung
              beansprucht.
              \item \textbf{Prozessortyp}: Idealerweise hat der SQL Server
              einen oder mehrere Prozessoren der Typen AMD Opteron, AMD Athlon 64,
              Intel Xeon mit Intel EM64T Unterstützung oder Intel Pentium IV mit
              EM64T Unterstützung zur Verfügung.
              \item \textbf{Speicher}: Es sollte ein Minimum von 1 GB vorhanden
              sein, empfohlen werden jedoch 4 GB oder mehr
          \end{itemize}
          \begin{merke}
            Zu Beachten ist, dass sich alle hier angegebenen Hardwareanforderungen
            auf 64-Bit Systeme beziehen. Bei der Installation auf einem 32-Bit
            System können die Hardwareanforderungen stark abweichen!
          \end{merke}
        \subsubsection{Anforderungen an den Datenträger}
          Während der Installation legt der SQL Server 2014 ca. 6 GB an temporären
          Daten an. Des Weiteren benötigen die einzelnen SQL Server Komponenten
          Speicherplatz, der in der folgenden Liste aufgeführt ist:
          \begin{itemize}
              \item Database Engine, Replikation, Volltextsuche und Data Quality
              Services: \textbf{811 MB}
              \item Analysis Services: \textbf{345 MB}
              \item Reporting Services und Berichts- Manager: \textbf{304 MB}
              \item Integration Services: \textbf{591 MB}
              \item Master Data Services: \textbf{243 MB}
              \item Clientkomponenten ohne SQL Server Onlinedokumentation:
              \textbf{1823 MB}
              \item SQL Server Onlinedokumentationstool: \textbf{375 KB}
          \end{itemize}
          \begin{merke}
            Microsoft empfiehlt, dass die Installation des SQL Server niemals auf
            einem Domänen-Controller erfolgen sollte, da dies eine ganze Reihe von
            Einschränkungen mit sich bringt!
          \end{merke}
          \begin{literaturinternet}
            \item \cite{ms143506}
          \end{literaturinternet}
      \subsection{SQL Server-Instanzen}
        Unter einer SQL Server-Instanz wird einen Windows Dienst verstanden, der
        die Fähigkeiten des SQL Server zur Verfügung stellt. Es gibt zwei
        unterschiedliche Arten von SQL Server-Instanzen.
        \subsubsection{Standardinstanz}
          In den meisten Situationen wird nur eine Instanz des SQL Servers auf
          einem Computer benötigt. Diese Instanz wird als Standardinstanz
          bezeichnet und ist an ihrem festgelegten Namen \identifier{MSSQLSERVER}
          zu erkennen.
        \subsubsection{Benannte Instanz}
          Sobald mehrere Instanzen des SQL Servers auf einem Rechner gefordert
          sind, müssen diese alle einen eigenen Instanznamen aufweisen. Deshalb
          werden diese Zusätzlichen Instanzen als benannten Instanzen bezeichnet.
          Für die Namensvergabe bei einer benannten Instanz ist folgendes zu beachten:
          \begin{itemize}
              \item Der Instanzname ist nicht case-sensitiv
              (Groß-/Kleinschreibung wird nicht berücksichtigt)
              \item Es dürfen keine reservierten Worte in einem Instanznamen
              vorkommen, wie zum Beispiel \identifier{Default} oder
              \identifier{MSSQLServer}.
              Dies würde zu einem Fehler führen.
              \item Die maximale Namenslänge beträgt 16 Zeichen.
              \item Instanznamen müssen mit einem Buchstaben (a - z oder A - Z)
              bzw. einem Unterstrich beginnen.
              \item Ab dem zweiten Zeichen dürfen auch die Ziffern 0 - 9, das
              Dollarzeichen (\$) und der Unterstrich verwendet werden.
              \item Grundsätzlich unzulässige Zeichen sind :
              Leerzeichen, Backslash \textbackslash, Komma , , Semikolon ; ,
              Doppelpunkt : , Hochkomma ' , kaufmännisches Und-Zeichen \&, Hashtag
              \# und das At-Zeichen @.
          \end{itemize}
      \subsection{Die Verzeichnisse des SQL Servers}
        Bei der Installation eines Microsoft SQL Server 2014 werden eine ganze
        Reihe verschiedener Verzeichnisse angelegt. Im Wesentlichen hängt dies
        damit zusammen, dass jede SQL Server-Instanz eigene Komponenten mitbringt,
        die von den Komponenten anderer Instanzen getrennt gespeichert werden
        müssen.
        \subsubsection{Instanzspezifische Komponenten}
          Um die instanzspezifischen Komponenten in unterschiedliche Verzeichnisse
          zu trennen, werden unterschiedliche Instanz-IDs für alle Komponenten
          erzeugt. Zum Beispiel hat die Database Engine einer
          Standardinstanz eines SQL Server 2014 standardmässig die Instanz-ID
          \texttt{MSSQL12.MSSQLSERVER}. Dieser Wert setzt sich wie folgt zusammen:
          \begin{itemize}
              \item \textbf{MSSQL}: Fixe Zeichenfolge
              \item \textbf{12}: Die Versionsnummer des SQL Server 2014. Diese
              wird ggf. von einem Unterstrich und einer Nebenversionsnummer
              gefolgt.
              \item \textbf{MSSQLSERVER}: Der Instanzname, der getrennt durch
              einen Punkt, an die anderen Angaben angehängt wird.
          \end{itemize}
          Andere Beispiele für Instanz-IDs sind \texttt{MSAS12.MSSQLSERVER} für
          eine Analysis Services Instanz oder \texttt{MSSQL12.CRM} für eine
          benannte Instanz mit dem Namen CRM.
\clearpage
          Die Instanz-ID wird dann im Installationspfad verwendet, um die
          einzelnen Komponenten zu trennen, z. B. wird die Database Engine einer
          SQL Server 2014 Instanz in das Verzeichis \texttt{C:\textbackslash Program Files\textbackslash Microsoft SQL
          Server\textbackslash MSSQL12.MSSQLSERVER} installiert.
          \begin{merke}
            \begin{itemize}
                \item Das Stammverzeichnis \texttt{C:\textbackslash Program Files\textbackslash Microsoft SQL
                Server} darf nicht auf einem Wechseldatenträger, einem
                komprimierten Dateisystem oder einem Verzeichnis mit Systemdateien
                liegen. Des Weiteren ist die Ablage in einem freigegebenen
                Verzeichnis eines Failoverclusters ebenfalls nicht erlaubt.
                \item Seit SQL Server 2012 darf das Stammverzeichnis auf einer
                Windows Freigabe liegen.
                \item Während einer Installation kann eine benutzerdefinierte
                Instanz-ID angegeben werden. Diese sollte jedoch keine
                Sonderzeichen oder reservierte Wörter enthalten.
            \end{itemize}
          \end{merke}
          Eine vollständige Liste alle Verzeichnisse, die während der Installation
          eines Microsoft SQL Server angelegt werden, kann im den folgenden
          Technet Artikel abgerufen werden.
          \begin{literaturinternet}
            \item \cite{ms143547}
          \end{literaturinternet}
        \subsubsection{Gemeinsam genutzte SQL Server Komponenten}
        Unter gemeinsam genutzten Komponenten versteht man Teile der SQL Server
        Software, die von allen auf einem Server installierten Instanzen genutzt
        werden können und dabei nur einmal vorhanden sein müssen.
        Dazu zählen beispielsweise die Verwaltungstools, die Master Data
        Services oder die SQL Server Integration Services. Installiert werden
        diese Komponenten in das Verzeichnis \texttt{C:\textbackslash Program Files\textbackslash Microsoft SQL
        Server\textbackslash 120}.
  
        Der Laufwerksbuchstabe kann in diesem Pfad kann variieren.
        \begin{literaturinternet}
          \item \cite{ms143547}
        \end{literaturinternet}
      \subsection{SQL Server installieren}
        \begin{enumerate}
          \item Nach dem einlegen der DVD erscheint das Aktionsfenster. Wählen Sie
          hier die Aktion \enquote{Setup.exe ausführen}.
          \item Der Setupvorgang wird nun gestartet.
          \item Es öffnet sich der \enquote{SQL Server-Installationscenter}.
          Wählen Sie links außen die Option \enquote{Installation} und dann im
          rechten Fenster \enquote{Neue eigenständige SQL Server-Installation
          oder Hinzufügen von Funktionen zu einer vorhandenen Installation}
          \bild{Das SQL-Server-Installations\-center}{sql_server_setup_2}{0.3}
          \item Der erste Schritt bei der Installation des SQL Server 2014
          ist die Abfrage eines Product Keys, um die Echtheit der Softwarelizenz
          zu verifizieren.
          \bild{Eingabe des Product Keys}{sql_server_setup_3}{0.3}
          \item Nach dem ein gülter Product Key eingegeben wurde, muss der
          Benutzer die Zurkenntnisnahme und die Einhalt der Lizenzbedingungen
          bestätigen, bevor das Setup vorgesetzt werden kann. Einer Übertragung
          der Funktionsverwendungsdaten muss nicht zugestimmt werden.
          \bild{Die Lizenz\-bedingungen}{sql_server_setup_4}{0.3}
          \item Im dritten Schritt des SQL Server 2014-Setup wird eine
          Überprüfung der Globalen Regeln durchgeführt. Hierbei handelt es sich
          um Voraussetzungen, die gegeben sein müssen, damit die Installation
          des SQL Server 2014 erfolgreich sein kann.
          \bild{Überprüfung der Globalen Regeln}{sql_server_setup_5}{0.33}
          \item Eine Überprüfung von Produktupdates ist fest in den
          Installationsablauf integriert. Dieser Vorgang kann bei bedarf
          übersprungen werden, z. B. wenn keine direkte Internetverbindung
          besteht.
          \bild{Überprüfung der Produkt\-updates}{sql_server_setup_6}{0.33}
\clearpage
          \item Jetzt werden die heruntergeladenen Produktupdates angezeigt.
          Sollte keine Internetverbindung zur Verfügung stehen, wird das Setup
          die Fehlermeldung \enquote{0x8024402C} anzeigen.
          \bild{Installation der Produktupdates}{sql_server_setup_7}{0.33}
          \item SQL Server 2014-Setup installiert nun die SQL Server
          Setupdateien. Im Zuge dieses Installationsschrittes werden auch
          heruntergeladene Updates mitinstalliert.
          \bild{Installation der Setupdateien}{sql_server_setup_8}{0.31}
\clearpage
          \item In diesem Schritt werden weitere Setupregeln geprüft, um die
          Installierbarkeit des SQL Servers sicher zustellen.
          \bild{Überprüfung der Setupregeln}{sql_server_setup_9}{0.31}
          
          \item Bei der Auswahl der SQL Server Setuprolle gibt es drei Optionen:
          \begin{itemize}
              \item \textbf{SQL Server-Funktionsinstallation}
              
              Mit dieser Option wird eine Standalone-Installation des SQL Server
              mit vollem Funktionsumfang gestartet, unabhängig von Microsoft
              SharePoint.
              \item \textbf{SQL Server PowerPivot für SharePoint}
              
              Installiert PowerPivot für Microsoft SharePoint.
              \item \textbf{Alle Funktionen mit Standardwerten}
              
              Führt eine SQL Server-Funktionsinstallation mit Standardwerten
              durch.
          \end{itemize}
          \bild{Auswahl der Setuprolle}{sql_server_setup_10}{0.33}
          \item Bei der SQL Server-Funktionsauswahl können die zu installierenden
          Komponenten des SQL Server, sowie die Installationsverzeichnisse
          gewählt werden.
          \bild{Die SQL Server-Funktion\-sauswahl}{sql_server_setup_11}{0.28}
          \item Die Prüfung der Installationsregeln ist vergleichbar mit der
          Überprüfung der Setupunterstützungsregeln. Es werden Probleme
          aufgespürt, die zu einer Blockade des Installationsvorganges führen
          würden.
          \bild{Prüfung der Funktionsregeln}{sql_server_setup_12}{0.33}
          \item Bei der Instanzkonfiguration werden verschiedene Dinge erledigt.
          Zum einen wird der zu installierende Instanztyp, standard oder benannt,
          ausgewählt und zum anderen wird die Instanz-ID festgelegt. Die
          Instanz-ID gibt, zusammen mit dem Instanzstammverzeichnis, das
          Verzeichnis an, in welches die Softwareinstallation erfolgen soll. Bei
          einer Standardinstanz sind Instanzname und Instanz-ID meist gleich,
          nämlich \identifier{MSSQLSERVER}. Bei einer benannten Instanz weichen beide
          häufig von einander ab.
\clearpage
          \bild{Instanz-konfiguration}{sql_server_setup_13}{0.33}
          \item Während der \enquote{Serverkonfiguration} müssen zwei Schritte
          ausgeführt werden. Zuerst sollten die Dienstkonten für die verschiedenen
          SQL-Server-Dienste ausgewählt werden. Hierzu sollten schon im Vorfeld
          gMSAs erstellt werden. Der einzige Dienst, der standardmässig
          deaktiviert ist und für den auch an dieser Stelle kein Dienstkonto
          ausgewählt werden kann, ist der \enquote{SQL Server-Browser}.
          \bild{Auswahl der Dienst\-konten}{sql_server_setup_14}{0.3}
          \item Auf der Registerkarte \enquote{Sortierung} kann eine
          Sortierreihenfolge, engl. \enquote{collation order} ausgewählt werden.
          Diese legt verschiedene Dinge fest, wie z. B.:
          \begin{itemize}
              \item Welcher Zeichensatz wird genutzt (hier Latin1)?
              \item In welcher Reihenfolge stehen die Buchstaben im Alphabet?
              \item Wo werden die Umlaute ä, ö und ü bzw. das sz ß einsortiert?
              \item Wo werden Ziffern in die Sortierreihenfolge eingereiht?
              \item Was geschieht mit diakritischen Zeichen (\'{a}, \`{o}
              \^{u})?
              \item Ist Groß-/Kleinschreibung für die Sortierung relevant?
          \end{itemize}
          Der Name der Standardsortierreihenfolge
          \enquote{Latin1\_General\_CI\_AS} enthält die
          beiden Buchstabenkombinationen \enquote{CI} und \enquote{AS}. CI steht
          für case-insensitive, was bedeutet, dass Groß-/Kleinschreibung irrelevant ist. AS heißt        
          akzent-sensitiv und meint, dass diakritische Zeichen und Umlaute von
          normalen Buchstaben unterschieden werden (ü ist nicht gleich u und \`{o}
          ist ungleich o).
          \bild{Auswahl der Sortier-reihenfolge}{sql_server_setup_14a}{0.3}
          Mit einem Klick auf die Schaltfläche \enquote{Anpassen} kann eine eigene
          Sortierreihenfolge zusammengestellt werden.
          \bild{Erstellen einer eigenen
          Sortierreihenfolge}{sql_server_setup_14b}{0.3}
          \item Auf der Registerkarte \enquote{Serverkonfiguration} stehen zwei
          mögliche Authentifizierungsmodi zur Verfügung: Die
          Windows-Authentifizierung und der Gemische Modus. Gemäß Empfehlung von
          Microsoft sollte nur noch die Windows-Authentifizierung genutzt werden,
          da die gesamte Nutzerwaltung so in das Active Directory übertragen wird,
          ohne das der SQL Server noch darin eingebunden wäre. Über die
          Schaltflächen \enquote{Aktuellen Benutzer hinzufügen} bzw.
          \enquote{Hinzufügen} können Windows-Benutzer hinzugefügt werden, die
          dann als SQL Server-Administratoren berechtigt werden.
\clearpage
          \begin{merke}
            Es muss mindestens ein Windows-Account ausgewählt werden! Dieser muss
            nicht zwingend lokaler Administrator oder Domänen-Admin sein.
          \end{merke}
          \bild{Datenbank\-modul\-konfiguration -
          Serverkonfiguration}{sql_server_setup_15}{0.27}
          \item Auf der Registerkarte \enquote{Datenverzeichnisse} können die
          Standardverzeichnisse für die verschiedenen Dateiarten des SQL Servers
          gewählt werden.
          \bild{Datenbank\-modul\-konfiguration
          -Daten\-verzeich\-nisse}{sql_server_setup_15a}{0.27}
\clearpage
          \item Die Registerkarte \enquote{FILESTREAM} bietet die Möglichkeit, das
          Feature Filestream des SQL Server 2014 zu aktivieren.
          \bild{Datenbank\-modul\-konfiguration -
          Filestream}{sql_server_setup_15b}{0.33}
          \item Da bei der Funktionsauswahl die SQL Server Reporting Services
          mit ausgewählt wurden, muss nun hier entschieden werden, ob diese nur
          installiert oder auch konfiguriert werden sollen.
          \bild{Installation / Konfiguration der
          Reporting-services}{sql_server_setup_16}{0.33}
          \item Die Prüfung der Funktionskonfigurationsregeln für die
          Installation ist vergleichbar mit der Überprüfung der Setupsregeln. Es werden Probleme
          aufgespürt, die zu einer Blockade des Installationsvorganges führen
          würden.
          \bild{Prüfung der Funktions\-konfigurations\-regeln für
          die Installation}{sql_server_setup_17}{0.33}
          \item In diesem vorletzten Schritt zeigt der Setup-Assistent eine
          Zusammenfassung aller Installationsoptionen. Mit einem Klick auf
          \enquote{Installieren} startet die Installation der SQL Server-Software.
          \bild{Zusammen\-fassung der
          Installations\-optionen}{sql_server_setup_18}{0.33}
\clearpage
          \item Die Installation der SQL Server-Software läuft.
          \bild{Installation der SQL
          Server\-Software}{sql_server_setup_19}{0.33}
          \item Am Ende der Installation zeigt das Setup noch eine
          Zusammenfassung der ausgeführten Tätigkeiten an.
          \bild{Installations-zusammen-fassung}{sql_server_setup_20}{0.33}
        \end{enumerate}
    \section{Installationsnachbereitung}
      Nach erfolgreicher Installation des SQL Servers sollte bzw. muss der
      Administrator noch einige Einstellungen am Server vornehmen.
      \subsection{Der SQL Server Configuration Manager}
        Der SQL Server Configuration Manager erlaubt es dem Administrator
        Änderungen an den SQL Server-Diensten und den SQL
        Server-Netzwerkeinstellungen vorzunehmen.
        \begin{merke}
          Microsoft empfiehlt, alle Einstellungen an den SQL Server-Diensten nur
          mittels des SQL Server Configuration Manager vorzunehmen!
        \end{merke}
        \subsubsection{Den Configuration Manager starten}
          Gestartet wird der Configuration Manager aus dem Windows Startmenü.
          \bild{Starten des Configuration
          Manager}{start_sql_server_configuration_manager}{0.6}
        \subsubsection{Netzwerkkonfiguration - TCP/IP}
          Der SQL Server kennt drei Netzwerkprotokolle. Dies sind:
          \enquote{Shared Memory}, \enquote{Named Pipes} und \enquote{TCP/IP}. Für
          jedes dieser Protokolle können die Einstellungen im SQL Server
          Configuration Manager vorgenommen werden. An dieser Stelle wird aber nur
          auf die Konfigurationsmöglichkeiten für das Protokoll TCP/IP
          eingegangen, da die anderen beiden eine eher untergeordnete Rolle
          spielen. \bild{Die
          TCP/IP-Eigen\-schaften
          1}{sql_server_configuration_manager_tcp_ip_properties_1}{0.55}
          Nach einem Klick auf \enquote{Eigenschaften} zeigt sich der Dialog aus
          \abbildung{sql_server_configuration_manager_tcp_ip_properties_2}, mit
          seinen beiden Registerkarten \enquote{Protokoll} und
          \enquote{IP-Adressen}.
          Auf der Registerkarte \enquote{Protokoll} können die folgenden Optionen
          geändert werden:
          \begin{itemize}
              \item \textbf{Aktiviert}: [ Ja | Nein ] Wird hier der Wert
              \enquote{Nein} eingetragen, steht das TCP/IP-Protokoll für den SQL Server nicht
              mehr zur Verfügung. Mögliche Werte sind Ja und Nein
              \item \textbf{Alle überwachen}: [ Ja | Nein ] Diese Option steht
              standardmässig auf Ja, was bedeutet, dass der SQL Server auf allen im zur
              Verfügungstehenden IP-Adressen lauscht. Wird der Wert auf Nein
              geändert, müssen im Folgenden alle IP-Adressen einzeln konfiguriert
              werden.
              \item \textbf{Erhalten}: [ number ] Gibt das
              Verbindungserhaltungsinterval in Milisekunden wieder. Ein Wert von 30.000 bedeutet, dass ein
              Verbindung für mindestens 30 Sekunden offengehalte wird, auch wenn
              sie inaktiv ist. Bei einem Wert von 0 würde eine inaktive Verbindung
              sofort geschlossen werden.
          \end{itemize}
          \bild{Die TCP/IP-Eigen\-schaften
          2}{sql_server_configuration_manager_tcp_ip_properties_2}{0.58}
          Wechselt man auf die Registerkarte \enquote{IP-Adressen} erhält man
          dort die Möglichkeit, alle IP-Adressen des Servers einzeln zu
          konfigurieren.
          
          Hier können folgende Einstellungen pro IP-Adresse geändert werden:
          \begin{itemize}
              \item \textbf{Aktiv}: [ Ja | Nein ] Gibt an, ob die IP-Adresse im
              Betriebssystem aktiviert oder deaktivert ist.
              \item \textbf{Aktiviert}: [ Ja | Nein ] Gibt an, ob der SQL Server
              auf dieser IP-Adresse Verbindungsanforderungen entgegen nimmt oder
              nicht. Diese Option wird nur dann wirksam, wenn auf der
              Registerkarte \enquote{Protkoll} der Parameter \enquote{Alle
              überwachen} den Wert Ja hat, andernfalls wird sie ignoriert.
              \item \textbf{Dynamische TCP-Ports}: [ NULL | 0 ] Durch das
              Eintragen des Wertes 0 wird beim nächsten Start des SQL
              Server-Dienstes ein freier TCP-Port dynamisch vergeben. Um einen
              Port statisch zu vergeben, muss das Feld leer gelassen werden.
              \item \textbf{IP-Adresse}: [ w.x.y.z ] Hier wird die IP-Adresse
              eingetragen. Dies kann sowohl eine IPv4- als auch eine IPv6-Adresse
              sein.
              \item \textbf{TCP-Port}: [ 0 - 65535 ] Hier wird der TCP-Port zur
              IP-Adresse gewählt. Es kann auch eine Liste von Ports,
              getrennt durch Koma-Zeichen, angegeben werden, wenn auf der
              Registerkarte \enquote{Protokoll} der Parameter \enquote{Alle
              überwachen} den Wert Ja hat.
          \end{itemize}
          \bild{Die Register\-karte IP-Adressen der
          TCP/IP-Eigen\-schaften}{sql_server_configuration_manager_tcp_ip_properties_3}{0.53}
        \subsubsection{Netzwerkkonfiguration - Dynamische Ports}
          Der SQL Server kann dynamische Ports benutzen. Das heißt, dass bei jedem
          Neustart des SQL Server-Dienstes automatisch ein freier TCP-Port gewählt
          wird. Alle Clients müssen dann diesen Port zur Verbindung mit dem SQL
          Server benutzen. Um dies realisieren zu können, wird der SQL Server
          Browser dienst zur Verfügung gestellt.
          \begin{merke}
            Dynamische Ports sind nicht ganz so dynamsich, wie es auf den ersten
            Blick scheint. SQL Server versucht nämlich den bereits
            benutzten Port zu reservierten, es sei denn, dass dieser schon
            anders gebunden wurde. Zudem kann die Nutzung dynamischer Ports,
            in Verbindung mit einer Firewall zu Verbindungsproblemen bei den
            Clients führen.
          \end{merke}
          Der SQL Server Browser-Dienst lauscht auf dem Port 1434 und 
          veröffentlicht für die Clients die Verbindungsinformationen zu den SQL
          Server-Instanzen. Ein Client muss lediglich ein UDP-Paket an den Port
          1434 schicken, um alle notwendigen Informationen zu erhalten. 
          \begin{merke}
            Microsoft empfiehlt im Technet Artikel \enquote{Configure a Windows
            Firewall for Database Engine Acces} \parencite{ms175043}, den
            SQL Server Browser-Dienst möglichst nicht zu benutzen, da er sehr
            viele Informationen preisgibt.
          \end{merke}
        \subsubsection{Netzwerkkonfiguration - Netwerkaliase}
        Ein Netzwerkalias ist ein Name für eine Verbindungszeichenfolge zu einer
        SQL Server-Instanz. Der Aliasname ist vom Benutzer frei wählbar. Für die
        zugehörige Verbindungszeichenfolge gibt es drei unterschiedliche
        Varianten:
        \begin{itemize}
            \item tcp:<servername>[\textbackslash<instancename>],<port>  
            \item tcp:<IPAddress>[\textbackslash<instancename>],<port>
            \item local 
        \end{itemize}
        \begin{merke}
          In einer Verbindungszeichenfolge werden die beiden Addresstypen IPv4
          und IPv6 akzeptiert.
        \end{merke}
        Netzwerkaliasnamen kommen immer dann zum Einsatz, wenn:
        \begin{itemize}
            \item Ein anderer Port, als der Standardport 1433 genutzt werden soll,
            \item oder wenn eine Instanz auf einen anderen Serverumgezogen
            werden muss.
        \end{itemize}
        Sie lösen das Problem, dass in jeder Clientsoftware eine
        Verbindungszeichenfolge angegeben werden muss, die auf einen ganz
        bestimmten Server verweist. Wenn z. B. die benannte Instanz
        \identifier{CRM} bisher auf dem Server \identifier{SRV01} war und nun
        auf \identifier{SRV02} umgezogen werden muss, lautete die alte
        Verbindungszeichenfolge SRV01\textbackslash CRM. Wurde kein Alias
        verwendet, müsste in jeder Clientsoftware die neue
        Verbindungszeichenfolge SRV02\textbackslash CRM eingerichtet werden. Ist
        die Zeichenfolge fest in der Anwendung codiert, ist eine solche
        Umstellung nur mittels des Anwendungsentwicklers möglich, der die
        Software erstellt hat. Ist die Software konfigurierbar, muss auf jedem
        Computer, auf dem die Software benutzt wird, die Konfiguration geändert
        werden, was einen sehr hohen Arbeitsaufwand bedeuten kann.
        
        Mit Hilfe eines Netzwerkaliases wird das Problem deutlich vereinfacht.
        Statt an jeder Software eine Änderung zu vollziehen, wird lediglich die
        Verbindungszeichenfolge des Alias geändert und die Software selbst
        bleibt unverändert.
        \bild{Anlegen eines
        Netzwerk\-aliasnamens}{sql_server_configuration_manager_aliases}{0.53}
      \subsection{Instant File Initialization}
        Erstellt der Microsoft SQL Server eine neue Datei, wird diese immer zu
        erst mit Nullen auf ihre volle Größe aufgefüllt, bevor sie benutztbar ist.
        Da dieser Vorgang sehr kostenintensiv ist, sollte er umgangen werden.
        \subsubsection{Instant File Initialization in der lokalen
        Sicherheitsrichtlinie}
        In der lokalen Sicherheitsrichtlinie kann dem SQL Server-Dienstkonto
        unter:
        \begin{itemize}
            \item Lokale Richtlinien
            \item Zuweisen von Benutzerrechten
        \end{itemize}
        das Recht: \enquote{Performe Volume Maintenance Tasks} bzw.
        \enquote{Durchführen von Volumenwartungsaufgaben} zugewiesen werden.
        \bild{Die lokale
        Sicherheitsrichtlinie}{performe_volume_maintenance_tasks}{0.53}
        \subsubsection{Instant File Initialization als Gruppenrichtlinie}
        \label{instant_file_initialization}
          Befindet sich der SQL Server in einer Windows-Domäne, was fast immer der
          Fall ist, dann sollten Berechtigungen, wie das Durchführen von
          Volumenwartungsaufgaben als Gruppenrichtlinie im Active Directory
          hinterlegt werden. Gehen Sie dazu wie folgt vor:
          \begin{enumerate}
              \item Starten Sie den Gruppenrichtlinieneditor
              \bild{Starten
              des
              Gruppenrichtlinieneditors}{gruppenrichtlinieneditor_starten}{0.4}
              \item Klicken Sie im linken Teilfenster, mit der rechten Maustaste,
              auf \enquote{Gruppenrichtlinienverwaltung}. Wählen Sie im
              Kontextmenü \enquote{Gesamtstruktur hinzufügen}.
              \bild{Hinzufügen einer
              Gesamt\-struktur 1}{gesamtstruktur_hinzufuegen}{0.4}
              \item Geben Sie im Dialogfenster \enquote{Gesamtstruktur hinzufügen}
              den Namen der Domäne ein. Dies ist hier: \identifier{MS-C-IX-04.fus}.
              \bild{Hinzufügen einer
              Gesamt\-struktur 2}{dialog_gesamtstruktur_hinzufuegen}{0.4}
              \item Klicken Sie im linken Teilfenster auf
              \enquote{Gesamtstruktur: ms-c-ix-04.fus}, \enquote{Domänen},
              \enquote{ms-c-ix-04.fus}, Ihre OU, z. B.
              \enquote{MS-C-IX-04-20} und dann auf die OU \enquote{Computers}.
              \item Klicken Sie mit der rechten Maustaste auf die OU
              \enquote{Computers} und wählen Sie aus dem Kontextmenü den Punkt
              \enquote{Gruppenrichtlinienobjekt hier erstellen und verknüpfen}.
              \bild{Hinzufügen einer
              Gruppenrichtlinie}{gruppenrichtlinie_anlegen}{0.4}
              \item Geben Sie der Gruppenrichtlinie im Dialogfenster
              \enquote{Neues Gruppenrichtlinienobjekt} einen Namen, z. B.
              \enquote{MS-C-IX-04-20\_Perfome\_Volume\_Maintenance\_Tasks}
              \bild{Hinzufügen einer
              Gruppenrichtlinie}{neues_gruppenrichtlinienobjekt}{0.35}
              \item Es erscheint eine Warnmeldung. Klicken Sie auf OK!
              \bild{Warnung}{gpo_warning}{0.35}
\clearpage
              \item Klicken Sie mit der rechten Maustaste auf die
              Gruppenrichtlinie und wählen Sie aus dem Kontextmenü den Punkt 
              \enquote{Bearbeiten}
              \bild{Die Gruppenrichtlinie bearbeiten}{edit_gpo}{0.4}
              \item Im Gruppenrichtlinienverwaltungs-Editor wählen Sie im linken
              Teilfenster: \enquote{Computerkonfiguration}, \enquote{Richtlinien},
              \enquote{Windows-Einstellungen}, \enquote{Sicherheitseinstellungen},
              \enquote{Lokale Richtlinien}, \enquote{Zuweisen von Benutzerrechten}
              \bild{Die Gruppenrichtlinie
              bearbeiten}{gruppenrichtlinienverwaltungseditor_1}{0.33}
              \item Wählen Sie die Richtlinie \enquote{Durchführen von
              Volumewartungsaufgaben} mit einem Doppelklick aus.
\clearpage
              \item Konfigurieren Sie die Richtlinie, in dem Sie Ihr gMSA-Konto,
              welches von Ihrem SQL Server-Dienst genutzt wird in die Richtlinie
              aufnehmen.
              \bild{Die Gruppenrichtlinie
              bearbeiten}{gruppenrichtlinienverwaltungseditor_2}{0.33}
              \item Schließen Sie die Gruppenrichtlinienverwaltung
          \end{enumerate}
      \subsection{Lock Pages in Memory (Sperren von Seiten im Speicher)}
        Um ein möglichst performantes Arbeiten mit den Nutzdaten zu ermöglichen,
        verfügt der SQL Server über eine ganze Reihe von Buffers und Caches, die
        im Arbeitsspeicher des Servers liegen. Um die Performance des SQL Servers
        nicht zu schädigen ist es wichtig, dass diese Buffers und Caches keines
        Falls auf den Datenträger ausgelagert werden (pagefile.sys). Da Microsoft
        Windows aber standardmässig die Inhalte des Arbeitsspeichers bei bedarf
        auslagert, könnte genau dieser unerwünschte Fall eintreten.
        \subsubsection{Lock Pages in Memory in der lokalen Sicherheitsrichtlinie}
          Um dies zu verhindern kann dem SQL Server-Dienstkonto in der
          lokalen Sicherheitsrichtlinie unter:
        \begin{itemize}
            \item Lokale Richtlinien
            \item Zuweisen von Benutzerrechten
        \end{itemize}
        das Recht: \enquote{Sperren von Seiten im Speicher} bzw. \enquote{Lock
        pages in memory} zugewiesen werden. Diese Richtlinie erlaubt es dem SQL
        Server, dass Auslagern seiner Speicherseiten durch das Betriebssystem zu
        verhindern.
        \bild{Die lokale
        Sicherheitsrichtlinie}{lock_pages_in_memory}{0.33}
        \subsubsection{Lock Pages in Memory als Gruppenrichtlinie}
          Die Richtlinie \enquote{Lock Pages in Memory} wird auf die gleiche Art
          und Weise als Gruppenrichtlinie angelegt, wie es unter
          \ref{instant_file_initialization} beschrieben wurde.
        \begin{literaturinternet}
          \item \cite{ms190730}
        \end{literaturinternet}