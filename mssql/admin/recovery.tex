      \subsection{Crashrecovery}
        x Auch als Restart Recovery bezeichnet
        x Kann parallel an mehreren DBs durchgeführt werden
        x Muss zwei Situationen bereinigen
          Bestätigte Änderungen sind im Log aber nicht in den Datendateien,
          Änderungen sind in den Datendateien, aber im Log nicht als bestätigt
          gekennz.

                    Jedes Virtual Log File besitzt genau einen Status, der ausdrückt,
          wofür es gebraucht wird:
          \begin{itemize}
              \item \textbf{Active}: VLFs die gerade mit Informationen befüllt
              werden haben diesen Status. Er kann sich über mehrere VLFs
              erstrecken und beginnt dort, wo sich die \enquote{Minimum LSN}
              befindet und endet, wo die \enquote{Last Checkpoint LSN} zu
              finden ist.
              \item \textbf{Reusable}: Ein VLF mit diesem Status enthält keine
              für ein Recovery relevanten Informationen mehr und kann deshalb
              überschrieben werden.
              \item \textbf{Recoverable}: Enthält ein VLF Informationen die zum
              Recovery der Datenbank nach einem Instanz-Crash benötigt werden,
              erhält es den Status \enquote{Recoverable} und darf noch nicht
              überschrieben werden.
              \item \textbf{Unused}: Dieser Status wird für VLFs vergeben, die
              noch nie genutzt wurden
                    \end{itemize}