  \chapter{Partially contained databases}
  \chaptertoc{}
  \cleardoubleevenpage
    \section{Partially contained databases - Teilweise eigenständige
    Datenbanken} 
      Die Architektur des Microsoft SQL Server sieht vor, dass Einstellungen,
      die zum Betrieb einer Datenbank notwendig sind, an verschiedenen Orten
      gespeichert werden. Beispielsweise werden Logins in der
      \identifier{master}-Datenbank gespeichert, um instanzweit zur Verfügung zu
      stehen, während Benutzerkonten direkt in der betreffenden Datenbank
      abgelegt werden. Ein anderes Beispiel sind Jobs des SQL Server Agent.
      Diese werden in der \identifier{msdb}-Datenbank abgelegt, können sich aber
      auf Datenbanken auswirken.
      
      Die durch diese Umstände bewirkte Verflechtung von Instanz und Datenbanken
      macht es schwierig eine Datenbank auf einen anderen Server / in eine
      andere Instanz zu migrieren. Mit Hilfe des Konzeptes der
      \enquote{Teilweise eigenständigen Datenbank} (engl. Partially contained
      databases) versucht Microsoft diesem Problem entgegen zu treten.

      In einer Partially contained database werden verschiedene Metadaten
      in der Datenbank selbst und nicht in der \identifier{master}-Datenbank
      gespeichert. Der Fachbegriff dafür, dass bestimmte Objekte ohne
      Verbindung zur Instanz oder zu einer anderen Datenbank existieren können
      lautet \enquote{Kapselung}
      \begin{merke}
        Von großer Bedeutung ist das Wort \enquote{Partially}, da nicht alle
        Metadaten in der Datenbank selbst abgelegt werden. Diese Art von
        Datenbank ist tatsächlich nur zu einem gewissen Teil unabhängig.
      \end{merke}
      \subsection{Was genau bedeutet Kapselung?}
        Bei der Kapselung geht es darum, dass Objekte in einer Datenbank
        keinerleie Bezüge/Verknüpfungen nach außen haben dürfen. Ein Beispiel
        hierfür ist die Systemview \identifier{sys.tables}. Die Basisobjekte,
        auf welche sich diese View bezieht, befinden sich alle in der
        betroffenen Datenbank selbst. Ein Gegenbeispiel wäre ein Objekt, wie
        z. B. ein SQL-Benutzer mit Anmeldename. Ein solches Benutzerkonto
        bezieht sich auf ein Windows-Login und dieses Login liegt in der
        \identifier{master}-Datenbank. Somit existiert ein Bezug nach außen.
        
        Man kann somit zwei Arten von Objekten unterscheiden:
        \begin{itemize}
          \item \textbf{Contained entities}: Ein Objekt wird dann als
          \enquote{vollständig enthalten} bezeichnet, wenn es die
          Datenbankbegrenzung nie überschreitet, sich also nur auf Funktionen
          stützt, welche in der Datenbank enthalten sind.
          \item \textbf{Uncontained entities}: Ein Objekt, welches die
          Datenbankgrenze überschreitet wird als \enquote{nicht enthalten}
          bezeichnet, da es Verknüpfungen zu Objekten außerhalb seiner eigenen
          Datenbank hat.
        \end{itemize}
        \begin{merke}
          Als \enquote{Datenbankbegrenzung} wird die Grenze zwischen einer
          Datenbank und der Instanz bzw. die Grenze zwischen zwei Datenbanken
          bezeichnet. Sie ist eine eindeutig definierbare Trennlinie zwischen
          den Funktionen der Datenbank (dem Datenmodell) und den Funktionen
          der Instanz (dem Verwaltungsmodell).
        \end{merke}
        \subsection{Welche Probleme werden gelöst?}
          \begin{itemize}
            \item Die Datenbank übernimmt die Authentifizierung von Benutzern
            komplett. Es sind keine Logins mehr von Nöten.
            \item Weitere Metadaten, z. B. eine Liste der Datenbanken, von denen
            die Partially contained database abhängig ist und weitere
            Systemeinstellungen, die zum Betrieb der Datenbank notwendig
            sind, werden in der Datenbank selbst gespeichert.
            \item Temporäre Tabellen werden direkt im Kontext der Partially
            contained database gespeichert, so dass es keine Konflikte mehr bei
            Sortierreihenfolgen zwischen der \identifier{tempdb} und der
            Datenbank selbst gibt.
          \end{itemize}
        \subsection{Einschränkungen für Partially contained databases}
          Das Konzept der teilweise eigenständigen Datenbanken bringt nicht nur
          Vorteile mit sich, es existieren auch Einschränkungen. Beispielsweise
          unterstützen diese Datenbank weder das Feature der Replikation noch
          das Change-Data-Capture. Eine vollständige Liste mit allen
          Einschränkungen ist in der MSDN enthalten.
          \begin{literaturinternet}
            \item \cite{ff929071}
          \end{literaturinternet}
      \section{Partially contained databases erzeugen}
        \subsection{Konfigurieren der Instanz}
          Bevor Partially contained databases erstellt werden können muss zuerst
          die Authentifizierung für eigenständige Datenbanken eingeschaltet
          werden. Dies wird durch eine einfache Instanzeinstellung vorgenommen.
          \bild{Eigen\-stän\-dige Daten\-banken aktivieren}{pcd_activate}{1.2}
          \begin{lstlisting}[language=ms_sql,caption={Eigen\-stän\-dige Daten\-banken
          aktivieren},label=sql20_01]
USE [master]
GO

EXEC §sp_configure§ 'contained database authentication', 1
RECONFIGURE
GO          
          \end{lstlisting}
        \subsection{Erstellen der Datenbank}
          \subsubsection{Eine neue Partially contained database erstellen}
            Eine teilweise eigenständige Datenbank wird mit Hilfe des
            \languagemssql{CREATE DATABASE}-Statements und dem Zusatz
            \languagemssql{CONTAINMENT = PARTIAL} erzeugt. Wahlweise kann dies
            aber auch im SSMS geschehen.
            \bild{Erzeugen einer teilweise eigen\-stän\-digen Daten\-bank}{create_pcd}{1.2}
            \begin{lstlisting}[language=ms_sql,caption={Erstellen einer teilweise eigen\-stän\-digen Daten\-bank},label=sql20_02]
USE [master]
GO

CREATE DATABASE [Bank_PCD]
  CONTAINMENT = PARTIAL
  ON  PRIMARY ( NAME = N'Bank_PCD',
  FILENAME = N'D:\120\bank_pcd\data\Bank_pcd.mdf',
  SIZE = 1GB, MAXSIZE = 2GB,
  FILEGROWTH = 512MB)
  LOG ON ( NAME = N'Bank_PCD_Log',
  FILENAME = N'D:\120\bank_pcd\log\Bank_pcd.ldf',
  SIZE = 100MB, MAXSIZE = 2GB,
  FILEGROWTH = 10%)
GO
            \end{lstlisting}
          \subsubsection{Den Containmenttype einer Datenbank ändern}
            Soll eines bestehende Datenbank in eine Partially contained
            database migriert werden, so geschieht dies mit dem
            \languagemssql{ALTER DATABASE}-Statement.
            \begin{lstlisting}[language=ms_sql,caption={Migration einer Datenbank in eine teilweise eigen\-stän\-dige Daten\-bank},label=sql20_03]
USE [master]
GO

ALTER DATABASE [Bank_PCD] SET CONTAINMENT = PARTIAL;
GO
            \end{lstlisting}
            \bild{Migration einer Datenbank in eine teilweise eigen\-stän\-dige
            Daten\-bank}{migrate_to_pcd}{1.3}
            \begin{literaturinternet}
              \item \cite{hh534404}
            \end{literaturinternet}
          \subsubsection{Identifizieren der Datenbankkapselung}
            Wurde eine Datenbank in eine Partially contained database migriert,
            stellt sich trotz Migration die Frage, ob wirklich alle Objekte
            innerhalb der datenbank die Datenbankbegrenzung einhalten. Um dies
            herauszufinden, gibt es zwei Mechanismen:
            \begin{itemize}
              \item Die View \identifier{sys.dm\_db\_uncontained\_entities}
              zeigt alle Objekte an, die möglicherweise die Datenbankbegrenzung
              überschreiten.
              \item Das Extended Event \identifier{database\_uncontained\_usage}
              wird immer dann ausgelöst, wenn zur Laufzeit eines Programms eine
              uncontained Entity gefunden wird.
            \end{itemize}
            \begin{merke}
              Die View \identifier{sys.dm\_db\_uncontained\_entities} kann
              Entitäten, welche dynamisches SQL enthalten nicht korrekt
              auswerten und deshalb auch nicht feststellen, ob solche Objekte
              die Datenbankbegrenzung verletzen oder nicht.
            \end{merke}
            \begin{literaturinternet}
              \item \cite{ff929071}
            \end{literaturinternet}
        \subsection{Contained users verwalten}
          Grundsätzlich sind die \enquote{enthaltenen Benutzerkonten} bzw.
          Contained Users mit den normalen Benutzerkonten identisch, bis auf
          eine Ausnahme: Es wird kein Login für einen Contained user erstellt.
          Die Metadaten eines Contained users werden in der datenbank selbst und
          nicht in der \identifier{master}-Datenbank abgelegt.
          
          Es gibt zwei Arten Contained users:
          \begin{itemize}
            \item \textbf{SQL Benutzer mit Kennwort}: Diese Art von
            Benutzerkonto ist das pont dont zur Kombination \enquote{SQL
            Benutzer mit Anmeldename und Login mit SQL
            Server-Authentifizierung}. Der Unterschied besteht darin, dass kein
            Login für ihn angelegt wird. Das Passwort für ein solches
            Benutzerkonto wird in der Datenbank selbst verwaltet.
            \item \textbf{Windows-Benutzer}: Der Windows Benutzer ist mit
            der Kombination \enquote{SQL Benutzer mit Anmeldename und Login mit
            Windows-Authentifizierung} vergleichbar.
          \end{itemize}          
          Zu beachten ist, dass das Authentifizierungsverfahren bei beiden
          Benutzertypen unterschiedlich abläuft. Meldet sich ein Benutzer mit
          einem Windows-Account an, wird zuerst nach einem Login mit
          Windows-Authentifzierung gesucht und erst wenn kein passendes Login
          gefunden wird, wird nach einem Windows-Benutzer gesucht. D. h. die
          Authentifizierung wird zuerst auf Instanzebene und dann auf
          Datenbankebene versucht.
          
          Bei einer Anmeldung mit einem SQL Server-Account, wird zuerst nach
          einem SQL Benutzer mit Kennwort und dann nach einem Login mit SQL
          Server-Authentifizierung gesucht. Hier wird also zuerst auf
          Datenbankebene und dann auf Instanzebene authentifiziert.
          \subsubsection{Contained users erstellen}
            Contained Users können wie bereits in
            \ref{administer_database_users} gezeigt, mit Hilfe des SSMS oder mit
            SQL-Kommandos erstellt werden. Die beiden folgenden Statements
            zeigen wie ein SQL Benutzer ohne Anmeldename und ein
            Windows-Benutzer erstellt werden.
            \begin{lstlisting}[language=ms_sql,caption={Erstellen
            eines SQL Benutzers mit Kennwort},label=sql20_04]
USE [Bank_PCD]
GO

CREATE USER [sql_user_without_login]
WITH PASSWORD = 'P@ssw0rd'
GO
            \end{lstlisting}
            
            \begin{lstlisting}[language=ms_sql,caption={Erstellen
            eines Windows-Benutzers},label=sql20_05]
USE [Bank_PCD]
GO

CREATE USER [ORA-B-IX-62\sqluser]
GO
            \end{lstlisting}
            Contained users können nur innerhalb einer eigenständigen Datenbank
            erstellt werden. Versucht man einen contained user in einer nicht
            eigenständigen Datenbank zu erstellen antwortet SQL Server mit den
            folgenden Fehlermeldung.
            \begin{lstlisting}[language=ms_sql,caption={Fehler beim Erstellen
            eines SQL Benutzers mit Kennwort in einer nicht
            eigenständigen Datenbank},label=sql20_06]
USE [Bank]
GO

CREATE USER [sql_user_without_login]
WITH PASSWORD = 'P@ssw0rd'
GO

Meldung 33233, Ebene 16, Status 1, Zeile 3
Sie k\"onnen nur einen Benutzer mit einem Kennwort in einer eigenst\"andigen 
Datenbank erstellen.
            \end{lstlisting}
          \subsubsection{Login an einer Partially contained database}
            Für die Anmeldung an einer Partially contained database muss in den
            Verbindungseigenschaften der Name der gewünschten Datenbank
            angegeben werden.
            \bild{Login an einer Partially contained database}{connection_properties_login_to_pcd}{1.3}
            Alternativ zur Anmeldung mittels SSMS kann auch mit Hilfe des
            SQLCMD-Tools eine Verbindung zu einer Contained database hergestellt
            werden.
            \begin{lstlisting}[language=terminal,caption={Login an einer
            Partially contained database mit dem SQLCMD-Tool},label=sql20_07]
sqlcmd -S FEA11-119SRV12\MSSQLSERVER -U sql_user_without_login -P P@ssw0rd
-d Bank_PCD
            \end{lstlisting}
            \begin{literaturinternet}
              \item \cite{bmmcspb202ss2pcdp1}
              \item \cite{hh534404}
            \end{literaturinternet}
          \subsubsection{Sicherheitsrelevante Einstellungen}  
            Bei der Benutzung von eigenständigen Benutzerkonten gibt es einige
            kritische Punkte, die man in seine Überlegungen mit einbeziehen
            sollte. Darunter fallen die Datenbankeigenschaft
            \identifier{auto\_close} und das Systemprivileg \privileg{alter any
            user}.
            
            Durch das Setzen der Datenbankeigenschaft \identifier{auto\_close}
            wird eine Datenbank geschlossen, sobald sich der letzte Benutzer von
            ihr abgemeldet hat. Da bei einer eigenständigen Datenbank die
            Authentifizierung nicht durch die Instanz, sondern durch die
            Datenbank selbst gemacht wird, bedeutet dies, dass nach dem
            Schließen der Datenbank zusätzliche Resourcen aufgewandt werden
            müssen, um sie wieder zu öffnen. Wenn sich nun ein einziger Benutzer
            an der Datenbank wiederholt an- und abmeldet, kann dem Server daraus
            eine große Belastung entstehen, was sich letzten endes wie eine
            Denial of Service Attacke auswirkt.
            
            Das Systemprivileg \privileg{alter any user} stellt in einer nicht
            eigenständigen Datenbank kein allzu großes Risiko dar, weil die
            Authentifizierung auf Instanzebene abläuft. In einer eigenständigen
            Datenbank jedoch, kann mit Hilfe dieses Privileges einem Benutzer
            Zugang zu einer Datenbank verschafft bzw. ein neuer Benutzer
            erstellt werden. Daher sollten dieses Privileg und die beiden Rollen
            \identifier{db\_owner} und \identifier{db\_securityadmin}, welche in
            Besitz dieses Privileges sind, mit größter Vorsicht gehandhabt werden.
            \begin{merke}
              Wenn in einer Datenbank das Gast-Konto aktiviert ist, besteht
              zusätzlich die Gefahr, dass sich die \enquote{illegal erstellten}
              Benutzerkonto Zugang zu dieser Datenbank verschaffen.
            \end{merke}
            Eine ähnliche Gefahrt droht von den beiden Systemprivilegien
            \privileg{alter any database} (fixed server role
            \privileg{dbcreateor}) und \privileg{control database} (fixed
            server role \privileg{db\_owner}). Wenn ein Benutzer eines der
            beiden Privilegien besitzt, kann er den status einer Datenbank auf
            \enquote{teilweise eigenständig} setzen. Daraus resultiert wiederum
            das soeben beschriebene Problem mit eigenständigen Benutzerkonten,
            die sich illegal Zugang zu anderen Datenbanken verschaffen können.
            \begin{literaturinternet}
              \item \cite{ff929055}
            \end{literaturinternet}
          \subsubsection{Das Passwort eines eigenständigen Benutzerkontos ändern}
            \begin{lstlisting}[language=ms_sql,caption={Ändern des
            Passwortes eines eigenständigen Benutzers},label=sql20_08]
USE [Bank_PCD]
GO

ALTER USER [sql_user_without_login] 
WITH 
  PASSWORD = 'Pa$$w0rd'
  OLD_PASSWORD = 'P@ssw0rd';
GO
            \end{lstlisting}
          \subsubsection{Benutzerkonten migrieren}
            Um alle Benutzerkonten des Types \enquote{SQL Benutzer mit
            Anmeldename}, die auf einem Login mit Passwort-Authentifizierung
            basieren in einen \enquote{SQL Benutzer mit Passwort} zu
            konvertieren, stellt Microsoft eigens die Stored Procedure
            \languagemssql{sp_migrate_user_to_contained} zur Verfügung. Das
            folgende Beispiel ist der MSDN (\parencite{ff929139}) entnommen und
            zeigt, wie ein solcher Vorgang für eine Partially contained database
            abläuft.
\clearpage
            \begin{lstlisting}[language=ms_sql,caption={Benutzerkonten
            migrieren},label=sql20_09]
DECLARE @username sysname;  
DECLARE user_cursor CURSOR  
    FOR   
        SELECT dp.name   
        FROM sys.database_principals AS dp  
        JOIN sys.server_principals AS sp  
        ON dp.sid = sp.sid  
        WHERE dp.authentication_type = 1 AND sp.is_disabled = 0;  
OPEN user_cursor  
FETCH NEXT FROM user_cursor INTO @username  
    WHILE @@FETCH_STATUS = 0  
    BEGIN  
        EXECUTE sp_migrate_user_to_contained   
        @username = @username,  
        @rename = N'keep_name',
        @disablelogin = N'disable_login';  
    FETCH NEXT FROM user_cursor INTO @username  
    END  
CLOSE user_cursor;  
DEALLOCATE user_cursor;              
            \end{lstlisting}
            \begin{literaturinternet}
              \item \cite{ff929139}
            \end{literaturinternet}
