  \chapter{Backup und Recovery}
    \chaptertoc{}
    \cleardoubleevenpage
    
    \section{Datenbank-Snaphots}
  DB-Snapshots (S. 155)
          \subsubsection{Verschieben von Systemdatenbanken nach einem Hardwarefehler}
            Nach einem Hardwarefehler kann es sein, dass Datendateien von
            Systemdatenbanken beschädigt wurde oder verloren gegangen sind. In
            so einem Fall kann es erforderlich sein, den Speicherort einer
            Systemdatenbank auf einen anderen Datenträger zu verschieben und die
            Datenbank dann dort wiedherzustellen. Problematisch dabei ist, dass
            die SQL Server-Instanz nicht gestartet werden kann, wenn nicht alle
            Systemdatenbanken verfügbar und voll funktionsfähig sind. Hier hilft
            dem Administrator das Traceflag 3608 weiter.
            \begin{enumerate}
                \item Starten des SQL Server-Dienstes mit minimaler
                Konfiguration und mit dem Traceflag 3608.
                \begin{lstlisting}[language=terminal,caption={Starten des SQL
                Server unter Umgehung des automatischen Recoveries},
                label=admin03_24]
NET START MSSQLSERVER /f /T3608
                \end{lstlisting}
                \begin{merke}
                  Das Traceflag 3608 sorgt dafür, dass der SQL Server beim Start
                  kein Recovery an seinen Datenbanken durchführt. Einzige
                  Ausnahme ist die \identifier{master}-Datenbank.
                \end{merke}
              \item Führen Sie ein \languagemssql{ALTER DATABASE ... MODIFY
              FILE}-Kommando für alle Daten- und Log-Dateien der
              \identifier{msdb}-Datenbank durch und geben Sie dabei den neuen
              Pfad der Dateien an.
              \item Stop Sie die SQL Server-Instanz
              \item Verschieben Sie die Dateien an ihren neuen Speicherort
              \item Starten Sie die SQL Server-Instanz erneut (ohne /f und /T)
              \item Benutzen Sie Katalogsicht \identifier{sys.master\_files}, um
              das Ergebnis des Verschiebevorganges zu verifizieren.
              
              \begin{lstlisting}[language=ms_sql,caption={Abfragen
                der View \identifier{sys.master\_files}},label=admin03_23]
                USE
                master GO

                SELECT 
                name AS logical_name
                , physical_name AS new_location
                , state_desc as state
                FROM   sys.master_files
                WHERE  database_id = DB_ID('msdb')
              \end{lstlisting}
          \end{enumerate}
            \begin{literaturinternet}
            \item \cite{ms188396}
          \end{literaturinternet}
  
  
  https://technet.microsoft.com/en-us/library/jj853293.aspx