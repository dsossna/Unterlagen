\clearpage
    \section{Projekt - Benutzen von Filestream und FileTables}
      \begin{enumerate}
        %Führen Sie das Skript \identifier{add_foto_column.sql} aus. Dieses
        % Skript fügt der Tabelle \identifier{mitarbeiter} in der Datenbank
        %\identifier{bank\_2014} eine Spalte namens \identifier{foto} hinzu.
        \item Bereiten Sie Ihre SQL Server-Instanz auf die Benutzung von FileStream vor.
Der Bezeichner für die Filestream-Freigabe lautet: \identifier{mssqlserver}. Der
Bezeichner für das Verzeichnis der Datenbank \identifier{bank\_2014 }lautet:
\identifier{bank\_2014}. Aktivieren Sie den nichttransaktionalen Zugriff auf den
Filestream.

        %Führen Sie das Skript \identifier{add\_foto\_column.sql} aus. Dieses
        %Skript fügt der Tabelle \identifier{mitarbeiter} in der Datenbank
        %\identifier{bank\_2014} eine Spalte namens \identifier{foto} hinzu.
            \item Registrieren Sie Ihre Datenbank in Ihrem neuen Recovery Katalog!

        
        %Legen Sie die Filestream-Dateigruppe \identifier{externalresources} an. Das Filestream-Verzeichnis
        % soll in \oscommand{D:\textbackslash u01\textbackslash bank\_2014\textbackslash
        %filestream\textbackslash fsstorage} liegen. Der logische Name für das
        %Filestream-Verzeichnis soll \identifier{bank\_2014\_externalresources}
        % sein.
            \item Wechseln Sie den Undo-Tablespace zurück zum Original Undo-Tablespace und löschen Sie \identifier{undotbs02} und \identifier{undotbs03}.

        
        %Erstellen Sie die FileTable \identifier{fotos}. Benutzen Sie
        %\identifier{mitarbeiter\_fotos} als FileTable-Verzeichnis.
            \item Erstellen Sie ein unkomprimiertes Backup Set des Tablespaces
    \identifier{bank}! Das Backup Set soll in der FRA (Fast Recovery Area)
    abgelegt werden!

        
        %Benutzen Sie die die \enquote{OpenRowset}-Funktionalität des SQL
        % Servers, um die Fotos der Mitarbeiter in die FileTable \identifier{fotos} zu importieren.
        %Verknüpfen Sie die Inhalt der Tabelle \identifier{mitarbeiter} mit den
        %Fotos in der Tabelle \identifier{fotos}. Welches Attribut der FileTable
        % eignet sich dafür?
        \item Legen Sie die Tabelle \identifier{buchung} gemäß Ihrem ER-Modell als SOT
an und fügen Sie zwischen die beiden Spalten \identifier{buchungsdatum} und
\identifier{konto\_id} die Spalte \identifier{buchungstext} VARCHAR(200) ein!
Legen Sie auf die Spalte \identifier{buchungsdatum} einen Index. Legen Sie auch
auf die Spalte \identifier{buchungstext} einen Index.
      \end{enumerate}