\clearpage
    \section{Projekt - Benutzen von Filestream und FileTables}
      \begin{enumerate}
        %Führen Sie das Skript \identifier{add_foto_column.sql} aus. Dieses
        % Skript fügt der Tabelle \identifier{mitarbeiter} in der Datenbank
        %\identifier{bank\_2014} eine Spalte namens \identifier{foto} hinzu.
            \item Führen Sie ein Recovery bei Verlust einer Redo Log Datei durch!
      \begin{enumerate}
        \item Starten Sie das Skript \oscommand{lab\_delete\_redolog\_member.sql}! Es löscht einen beliebigen Member einer Ihrer Redo Log Gruppen.
          \begin{lstlisting}[language=terminal]
SQL> @/home/oracle/labs/lab_delete_redolog_member.sql
          \end{lstlisting}
        \item Die Datenbank läuft weiterhin normal und es liegen keinerlei Probleme vor. Öffnen Sie die Alert Log Datei, um herauszufinden welcher Redo Log Member gelöscht wurde!
        \item Führen Sie geeignete Maßnahmen zur Problembehebung durch!
      \end{enumerate}


        %Führen Sie das Skript \identifier{add\_foto\_column.sql} aus. Dieses
        %Skript fügt der Tabelle \identifier{mitarbeiter} in der Datenbank
        %\identifier{bank\_2014} eine Spalte namens \identifier{foto} hinzu.
        \item Nehmen Sie Ihren Übungsserver in die Domäne
\identifier{MS-C-IX-04.FUS} auf. Achten Sie darauf, dass das
Computerkonto Ihres Datenbankservers in der 
Organisationseinheit \oscommand{IT-SysAdmin\textbackslash
EXER\textbackslash Serverxx\textbackslash Computers} abgelegt wird, bzw. verschieben Sie es dorthin.

        
        %Legen Sie die Filestream-Dateigruppe \identifier{externalresources} an. Das Filestream-Verzeichnis
        % soll in \oscommand{D:\textbackslash u01\textbackslash bank\_2014\textbackslash
        %filestream\textbackslash fsstorage} liegen. Der logische Name für das
        %Filestream-Verzeichnis soll \identifier{bank\_2014\_externalresources}
        % sein.
              \item Bringen Sie Ihre Datenbank in die Mount-Phase.

        
        %Erstellen Sie die FileTable \identifier{fotos}. Benutzen Sie
        %\identifier{mitarbeiter\_fotos} als FileTable-Verzeichnis.
            \item Welchen Default Tablespace benutzt \identifier{bob} und aus welcher View können Sie diese Angabe entnehmen?

        
        %Benutzen Sie die die \enquote{OpenRowset}-Funktionalität des SQL
        % Servers, um die Fotos der Mitarbeiter in die FileTable \identifier{fotos} zu importieren.
        %Verknüpfen Sie die Inhalt der Tabelle \identifier{mitarbeiter} mit den
        %Fotos in der Tabelle \identifier{fotos}. Welches Attribut der FileTable
        % eignet sich dafür?
            \item Prüfen Sie den Audittrail auf neue Informationen! Welche Tätigkeiten wurden an der auditierten Tabelle ausgeführt?

      \end{enumerate}