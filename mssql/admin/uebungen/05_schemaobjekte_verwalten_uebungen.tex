\clearpage
    \section{\"Ubungen - Schemaobjekte verwalten}
      Benutzen Sie für die Durchführung der Übungen die Datenbank
      \identifier{Bank\_2014}.
      \begin{enumerate}
        %Erstellen Sie die Tabelle \identifier{bank} gemäß Ihrem ER-Modell. 
        %Machen Sie das Attribut \identifier{bank\_id} zum Primärschlüssel und
        % achten sie darauf, dass die Tabelle als Heap erstellt wird.
            \item Führen Sie ein Recovery bei Verlust einer Redo Log Datei durch!
      \begin{enumerate}
        \item Starten Sie das Skript \oscommand{lab\_delete\_redolog\_member.sql}! Es löscht einen beliebigen Member einer Ihrer Redo Log Gruppen.
          \begin{lstlisting}[language=terminal]
SQL> @/home/oracle/labs/lab_delete_redolog_member.sql
          \end{lstlisting}
        \item Die Datenbank läuft weiterhin normal und es liegen keinerlei Probleme vor. Öffnen Sie die Alert Log Datei, um herauszufinden welcher Redo Log Member gelöscht wurde!
        \item Führen Sie geeignete Maßnahmen zur Problembehebung durch!
      \end{enumerate}

        
        %Ist es sinnvoll, die Tabelle \identifier{bank} als Heap anzulegen?
        %Begründen Sie Ihre Aussage!
        \item Nehmen Sie Ihren Übungsserver in die Domäne
\identifier{MS-C-IX-04.FUS} auf. Achten Sie darauf, dass das
Computerkonto Ihres Datenbankservers in der 
Organisationseinheit \oscommand{IT-SysAdmin\textbackslash
EXER\textbackslash Serverxx\textbackslash Computers} abgelegt wird, bzw. verschieben Sie es dorthin.

        
        %Legen Sie ein \UNIQUE-Constraint auf die Spalte \identifier{bic} der
        %Tabelle \identifier{bank}. Die Tabelle muss weiterhin in Form eines
        %Heaps gespeichert bleiben.
              \item Bringen Sie Ihre Datenbank in die Mount-Phase.

        
        %Fügen Sie Ihrer Datenbank die Dateigruppe \identifier{in\_memory} hinzu
        %und bereiten Sie alles vor, um eine SOT darin anzulegen.
            \item Welchen Default Tablespace benutzt \identifier{bob} und aus welcher View können Sie diese Angabe entnehmen?

        
        %Legen Sie die Tabelle \identifier{buchung} gemäß Ihrem ER-Modell als SOT
        %an und fügen Sie zwischen die beiden Spalten
        %\identifier{buchungsdatum} und \identifier{konto\_id} die Spalte
        % \identifier{buchungstext} VARCHAR(200) ein! Legen Sie auf die
        %Spalte \identifier{buchungsdatum} einen Index. Legen Sie auch
        %auf die Spalte \identifier{buchungstext} einen Index.
            \item Prüfen Sie den Audittrail auf neue Informationen! Welche Tätigkeiten wurden an der auditierten Tabelle ausgeführt?

        
        %Erstellen Sie die Tabelle \identifier{eigenkunde} gemäß Ihrem ER-Modell
        %und legen Sie ein \UNIQUE-Constraint auf die Spalte
        %\identifier{personalausweisnr}. Die Sortierung des Indexes, der zu
        % diesem Constraint gehört soll absteigend sein.
        \item Erstellen Sie die Tabelle \identifier{eigenkunde}, gemäß Ihrem ER-Modell,
in der Dateigruppe \identifier{kam} und legen Sie ein \UNIQUE-Constraint auf
die Spalte \identifier{personalausweisnr}. Die Sortierung des Indexes, der zu diesem
Constraint gehört soll absteigend sein.
        
        %Erstellen Sie die Tabelle \identifier{mitarbeiter} gemäß Ihrem
        % ER-Modell und legen Sie ein \UNIQUE-Constraint auf die Spalte
        %\identifier{sozversnr}. Schließen Sie die Spalten \identifier{nachname}
        % und \identifier{vorname} in den Index als Nichtschlüsselspalten in.
            \item Welche Tablespaces in Ihrer Datenbank sind Bigfile Tablespaces?


        %Legen Sie einen nicht gruppierten Index auf die Tabelle
        %\identifier{eigenkunde} (Spalte \identifier{ablaufdatum}). Der Index
        %soll nur die Zeilen enthalten, die einen Personalausweis zeigen, der vor dem 01.01.2016
        %abgelaufen ist. Schließen Sie die Spalte \identifier{personalausweisnr}
        % als Nichtschlüsselspalte in den Index ein.
        \item Schreiben Sie eine SQL-Abfrage, die anzeigen, ob die Einstellungen
aus der vorangegangenen Aufgabe bereits wirksam geworden sind oder
nicht!


        %Ermitteln Sie den Prozentsatz der logischen Indexfragmentierung der
        % beiden Indizes \identifier{mitarbeiter\_pk} und \identifier{girokonto\_pk}.
            \item Führen Sie das gleiche Statement nochmals aus und ermitteln Sie erneut die verursachten Kosten und die benötigte Ausführungszeit! Haben sich die Kosten verändert?


        %Ergreifen Sie für jeden der beiden Indizes die geeignete Maßnahme, um
        %deren Fragmentierung zu reduzieren.
            \item Verändern Sie die Datendatei \oscommand{uebungs\_ts02.dbf} so, dass sie auf bis zu 1 G Größe anwachsen kann.


        %Erstellen Sie die Tabelle \identifier{girokonto} gemäß Ihrem ER-Modell.
        %Legen Sie den Primärschlüssel dieser Tabelle mit einem clustered Index
        % an.
            \item Führen Sie das Skript \oscommand{/home/oracle/labs/lab\_fullts.sql}
    aus. Nach der Ausführung des Skripts existiert ein neuer Tablespace
    \identifier{fullts} und ein neuer Nutzer \identifier{alice}.


        %Erstellen Sie einen nicht gruppierten Index aufl der Spalte
        %\identifier{guthaben} der Tabelle \identifier{girokonto}.
            \item Fügen Sie dem Profil \identifier{p\_clerk} den Parameter \parameter{PASSWORD\_GRACE\_TIME} mit dem Wert 5 Tage hinzu!


        %Deaktivieren Sie den gruppierten Index der Tabelle
        % \identifier{girokonto} und beantworten Sie die folgenden Fragen:
        \item Nehmen Sie die im Folgenden beschriebenen Verändernungen an den
Datendateien Ihrer Datenbank \identifier{Bank 2014} vor.
\begin{center}
  \begin{small}
    \changefont{pcr}{m}{n}
    \tablefirsthead {
      \multicolumn{1}{c}{\textbf{Name}} &
      \multicolumn{1}{c}{\textbf{Größe}} &
      \multicolumn{1}{c}{\textbf{Wachstum}} &
      \multicolumn{1}{c}{\textbf{G Max.}} &
      \multicolumn{1}{c}{\textbf{Pfad}} \\
    }
    \tablehead{
    }
    \tabletail {
    }
    \tablelasttail {
    }
    \begin{supertabular}{lrrrl}
        bank\_2014\_primary\_01 &  10 M &  + 10 M & \color{red}{768 M} &
        D:\textbackslash u01\textbackslash bank\_2014\textbackslash data \\
        \hline
        bank\_2014\_hr\_01     & 100 M & \color{red}{+ 50 M}&
        \color{red}{1 G} & D:\textbackslash u01\textbackslash
        bank\_2014\textbackslash data \\
        bank\_2014\_hr\_02     & 100 M & \color{red}{+ 50 M}&
        \color{red}{1 G} & D:\textbackslash u01\textbackslash
        bank\_2014\textbackslash data \\
        \hline
        bank\_2014\_kam\_01    & \color{red}{500 M} & + 100 M & \color{red}{2 G} &
        E:\textbackslash u02\textbackslash bank\_2014\textbackslash data \\
    \end{supertabular}
  \end{small}
\end{center}


        %Suchen Sie einen Weg, den gruppierten und den nicht gruppierten Index
        % der Tabelle \identifier{girokonto} in einem Zug zu reaktivieren.
        \item Schreiben Sie eine Abfrage, um zu überprüfen, ob die Datendatei
ordnungsgemäß angefügt wurde.


        %Lesen Sie den Artikel \parencite{utilUDPaC} und notieren Sie sich
        mindestens drei sinnvolle Fragen!
              \item Erstellen Sie aus den aktuellen Einstellungen Ihrer Instanz ein neues SPFile names \oscommand{/home/oracle/spfileorcl.ora}.

      \end{enumerate}