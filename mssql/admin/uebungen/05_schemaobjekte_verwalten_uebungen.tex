\clearpage
    \section{\"Ubungen - Schemaobjekte verwalten}
      Benutzen Sie für die Durchführung der Übungen die Datenbank
      \identifier{Bank\_2014}.
      \begin{enumerate}
        %Erstellen Sie die Tabelle \identifier{bank} gemäß Ihrem ER-Modell. 
        %Machen Sie das Attribut \identifier{bank\_id} zum Primärschlüssel und
        % achten sie darauf, dass die Tabelle als Heap erstellt wird.
        \item Bereiten Sie Ihre SQL Server-Instanz auf die Benutzung von FileStream vor.
Der Bezeichner für die Filestream-Freigabe lautet: \identifier{mssqlserver}. Der
Bezeichner für das Verzeichnis der Datenbank \identifier{bank\_2014 }lautet:
\identifier{bank\_2014}. Aktivieren Sie den nichttransaktionalen Zugriff auf den
Filestream.
        
        %Ist es sinnvoll, die Tabelle \identifier{bank} als Heap anzulegen?
        %Begründen Sie Ihre Aussage!
            \item Registrieren Sie Ihre Datenbank in Ihrem neuen Recovery Katalog!

        
        %Legen Sie ein \UNIQUE-Constraint auf die Spalte \identifier{bic} der
        %Tabelle \identifier{bank}. Die Tabelle muss weiterhin in Form eines
        %Heaps gespeichert bleiben.
            \item Wechseln Sie den Undo-Tablespace zurück zum Original Undo-Tablespace und löschen Sie \identifier{undotbs02} und \identifier{undotbs03}.

        
        %Fügen Sie Ihrer Datenbank die Dateigruppe \identifier{in\_memory} hinzu
        %und bereiten Sie alles vor, um eine SOT darin anzulegen.
            \item Erstellen Sie ein unkomprimiertes Backup Set des Tablespaces
    \identifier{bank}! Das Backup Set soll in der FRA (Fast Recovery Area)
    abgelegt werden!

        
        %Legen Sie die Tabelle \identifier{buchung} gemäß Ihrem ER-Modell als SOT
        %an und fügen Sie zwischen die beiden Spalten
        %\identifier{buchungsdatum} und \identifier{konto\_id} die Spalte
        % \identifier{buchungstext} VARCHAR(200) ein! Legen Sie auf die
        %Spalte \identifier{buchungsdatum} einen Index. Legen Sie auch
        %auf die Spalte \identifier{buchungstext} einen Index.
        \item Legen Sie die Tabelle \identifier{buchung} gemäß Ihrem ER-Modell als SOT
an und fügen Sie zwischen die beiden Spalten \identifier{buchungsdatum} und
\identifier{konto\_id} die Spalte \identifier{buchungstext} VARCHAR(200) ein!
Legen Sie auf die Spalte \identifier{buchungsdatum} einen Index. Legen Sie auch
auf die Spalte \identifier{buchungstext} einen Index.
        
        %Erstellen Sie die Tabelle \identifier{eigenkunde} gemäß Ihrem ER-Modell
        %und legen Sie ein \UNIQUE-Constraint auf die Spalte
        %\identifier{personalausweisnr}. Die Sortierung des Indexes, der zu
        % diesem Constraint gehört soll absteigend sein.
              \item Starten und öffnen Sie Ihre Datenbank.

        
        %Erstellen Sie die Tabelle \identifier{mitarbeiter} gemäß Ihrem
        % ER-Modell und legen Sie ein \UNIQUE-Constraint auf die Spalte
        %\identifier{sozversnr}. Schließen Sie die Spalten \identifier{nachname}
        % und \identifier{vorname} in den Index als Nichtschlüsselspalten in.
            \item Konfigurieren Sie die Benennungsmethode Directory Naming für Ihre
    Datenbank. Der Directory Server ist ein OID und wird unter der IP-Adresse
    192.168.111.250 zufinden sein (Ports: 389 und 636). Der LDAP-Context
    lautet: \enquote{cn=OracleContext}. Nutzen Sie für diese Aufgabe den
    Oracle Net Configuration Assistant (\oscommand{netca} - Konfigurieren von
    Directoy-Verwendung).


        %Legen Sie einen nicht gruppierten Index auf die Tabelle
        %\identifier{eigenkunde} (Spalte \identifier{ablaufdatum}). Der Index
        %soll nur die Zeilen enthalten, die einen Personalausweis zeigen, der vor dem 01.01.2016
        %abgelaufen ist. Schließen Sie die Spalte \identifier{personalausweisnr}
        % als Nichtschlüsselspalte in den Index ein.
            \item Welche Kosten hat das Statement verursacht und wie viel Zeit wurde für die Ausführung benötigt?


        %Ermitteln Sie den Prozentsatz der logischen Indexfragmentierung der
        % beiden Indizes \identifier{mitarbeiter\_pk} und \identifier{girokonto\_pk}.
            \item Verschieben Sie jetzt das Backup Set des \identifier{bank}-Tablespaces
    auf das SBT-Gerät!


        %Ergreifen Sie für jeden der beiden Indizes die geeignete Maßnahme, um
        %deren Fragmentierung zu reduzieren.
            \item Löschen Sie das Backup Set des \identifier{bank}-Tablespaces von der Festplatte, so dass es nur noch auf SBT verfügbar ist!


        %Erstellen Sie die Tabelle \identifier{girokonto} gemäß Ihrem ER-Modell.
        %Legen Sie den Primärschlüssel dieser Tabelle mit einem clustered Index
        % an.
            \item Der Tablespace \identifier{bank} liegt derzeit in unverschlüsselter Form vor. Dies muss zukünftig anders sein. Bereiten Sie einen Tablespace \identifier{bank\_encrypted} vor, der mittels \identifier{AES256} verschlüsselung gesichert ist und der die gleichen Dimensionen hat, wie der Originaltablespace \identifier{bank}! Rufen Sie alle notwendigen Informationen über den Tablespace \identifier{bank} aus dem Data Dictionary ab! Welche Views helfen Ihnen dabei?


        %Erstellen Sie einen nicht gruppierten Index aufl der Spalte
        %\identifier{guthaben} der Tabelle \identifier{girokonto}.
           \item Führen Sie mit Hilfe von RMAN ein Block-Media-Recovery durch, um die be\-schä\-dig\-ten Blöcke zu reparieren.


        %Deaktivieren Sie den gruppierten Index der Tabelle
        % \identifier{girokonto} und beantworten Sie die folgenden Fragen:
        \item Deaktivieren Sie den gruppierten Index der Tabelle \identifier{girokonto}
und beantworten Sie die folgenden Fragen:

Können die Daten der Tabelle \identifier{girokonto} noch selektiert werden?

  \rule{0.94\textwidth}{0.5pt}

Welche Auswirkungen hatte das Deaktivieren des clustered Index auf den nicht
gruppierten Index der Spalte \identifier{guthaben}

  \rule{0.94\textwidth}{0.5pt}

  \rule{0.94\textwidth}{0.5pt}



        %Suchen Sie einen Weg, den gruppierten und den nicht gruppierten Index
        % der Tabelle \identifier{girokonto} in einem Zug zu reaktivieren.
        \item Schreiben Sie eine Abfrage, um zu überprüfen, ob die Datendatei
ordnungsgemäß angefügt wurde.


        %Lesen Sie den Artikel \parencite{utilUDPaC} und notieren Sie sich
        mindestens drei sinnvolle Fragen!
              \item Erstellen Sie aus den aktuellen Einstellungen Ihrer Instanz ein neues SPFile names \newline \oscommand{/home/oracle/spfileorcl.ora}.

      \end{enumerate}