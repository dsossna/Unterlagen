\clearpage
    \section{\"Ubungen - Verwalten einer SQL Server-Instanz}
      \begin{enumerate}
        %Richten Sie auf Ihrem Übungsserver alle notwendigen Werkzeuge ein, um
        % mit der Domäne \identifier{MS-C-IX-04.FUS} arbeiten zu können.
            \item Führen Sie ein Recovery bei Verlust einer Redo Log Datei durch!
      \begin{enumerate}
        \item Starten Sie das Skript \oscommand{lab\_delete\_redolog\_member.sql}! Es löscht einen beliebigen Member einer Ihrer Redo Log Gruppen.
          \begin{lstlisting}[language=terminal]
SQL> @/home/oracle/labs/lab_delete_redolog_member.sql
          \end{lstlisting}
        \item Die Datenbank läuft weiterhin normal und es liegen keinerlei Probleme vor. Öffnen Sie die Alert Log Datei, um herauszufinden welcher Redo Log Member gelöscht wurde!
        \item Führen Sie geeignete Maßnahmen zur Problembehebung durch!
      \end{enumerate}


        %Nehmen Sie Ihren Übungsserver in die Domäne
        %\identifier{MS-C-IX-04.FUS} auf. Achten Sie darauf, dass das
        %Computerkonto Ihres Datenbankservers in der 
        %Organisationseinheit \oscommand{IT-SysAdmin\textbackslash
        %EXER\textbackslash Serverxx\textbackslash Computers} abgelegt wird,
        % bzw. verschieben Sie es dorthin.
        \item Nehmen Sie Ihren Übungsserver in die Domäne
\identifier{MS-C-IX-04.FUS} auf. Achten Sie darauf, dass das
Computerkonto Ihres Datenbankservers in der 
Organisationseinheit \oscommand{IT-SysAdmin\textbackslash
EXER\textbackslash Serverxx\textbackslash Computers} abgelegt wird, bzw. verschieben Sie es dorthin.

        
        %Deaktivieren Sie den Zugriff mittels Shared Memory auf Ihre SQL
        %Server-Instanz!
              \item Bringen Sie Ihre Datenbank in die Mount-Phase.

        
        %Der Zugriff auf Ihre SQL Server-Instanz soll nur noch mittels der 
        %öffentlichen IP-Adresse 192.168.111.xx erfolgen können! Alle anderen
        % IPs müssen deaktiviert werden!
            \item Welchen Default Tablespace benutzt \identifier{bob} und aus welcher View können Sie diese Angabe entnehmen?

        
        %Richten Sie die im Folgenden aufgeführten gMSAs in der
        %Organisationseinheit \oscommand{IT-SysAdmin\textbackslash
        %EXER\textbackslash Serverxx\textbackslash MSAs} für die Dienste Ihrer
        % SQL Server-Instanz ein. Ersetzen Sie dabei \enquote{xx} durch Ihre Platznummer!
            \item Prüfen Sie den Audittrail auf neue Informationen! Welche Tätigkeiten wurden an der auditierten Tabelle ausgeführt?


        %Stoppen Sie die beiden Dienste \identifier{SQL Server Reporting
        %Services} und \identifier{SQL Server Integration Services 12.0}.
        %Konfigurieren Sie beide so, dass Sie manuell gestartet werden müssen.
        \item Erstellen Sie die Tabelle \identifier{eigenkunde}, gemäß Ihrem ER-Modell,
in der Dateigruppe \identifier{kam} und legen Sie ein \UNIQUE-Constraint auf
die Spalte \identifier{personalausweisnr}. Die Sortierung des Indexes, der zu diesem
Constraint gehört soll absteigend sein.

        %Konfigurieren Sie Ihre Instanz MSSQLSERVER so, dass Sie immer
        %mindestens 200M Arbeitsspeichern zur Verfügung hat und nie mehr als 2G
        %benutzt!
            \item Welche Tablespaces in Ihrer Datenbank sind Bigfile Tablespaces?


        %Schreiben Sie eine SQL-Abfrage, die anzeigen, ob die Einstellungen
        %aus der vorangegangenen Aufgabe bereits wirksam geworden sind oder
        %nicht!
        \item Schreiben Sie eine SQL-Abfrage, die anzeigen, ob die Einstellungen
aus der vorangegangenen Aufgabe bereits wirksam geworden sind oder
nicht!

        
        %Deaktivieren Sie die Überwachung von fehlerhaften Anmeldungen. Es
        %soll keine Anmeldeüberwachung geben!
            \item Führen Sie das gleiche Statement nochmals aus und ermitteln Sie erneut die verursachten Kosten und die benötigte Ausführungszeit! Haben sich die Kosten verändert?


        %Fügen Sie Ihre Instanz MSSQLSERVER die Datenbank \identifier{Bank
        %2014} hinzu. Der Aufbau der Datenbank kann der folgenden Tabelle
        %entnommen werden. Die Namen der Dateigruppen sind in den logischen
        % Namen der Datendateien enthalten (Primary, HR, CRM, KAM und MAIL).
            \item Verändern Sie die Datendatei \oscommand{uebungs\_ts02.dbf} so, dass sie auf bis zu 1 G Größe anwachsen kann.

        
        %Fügen Sie der \identifier{Bank 2014} die Dateigruppe
        %\identifier{staging} hinzu. Benutzen Sie dazu die folgenden Angaben:
            \item Führen Sie das Skript \oscommand{/home/oracle/labs/lab\_fullts.sql}
    aus. Nach der Ausführung des Skripts existiert ein neuer Tablespace
    \identifier{fullts} und ein neuer Nutzer \identifier{alice}.

        
        %Fügen Sie der Dateigruppe \identifier{crm} eine weitere Datendatei
        %hinzu. Verwenden Sie dazu die folgenden Angaben:
            \item Fügen Sie dem Profil \identifier{p\_clerk} den Parameter \parameter{PASSWORD\_GRACE\_TIME} mit dem Wert 5 Tage hinzu!

        
        %Nehmen Sie die im Folgenden beschriebenen Verändernungen an den
        %Datendateien Ihrer Datenbank \identifier{Bank 2014} vor.
        \item Nehmen Sie die im Folgenden beschriebenen Verändernungen an den
Datendateien Ihrer Datenbank \identifier{Bank 2014} vor.
\begin{center}
  \begin{small}
    \changefont{pcr}{m}{n}
    \tablefirsthead {
      \multicolumn{1}{c}{\textbf{Name}} &
      \multicolumn{1}{c}{\textbf{Größe}} &
      \multicolumn{1}{c}{\textbf{Wachstum}} &
      \multicolumn{1}{c}{\textbf{G Max.}} &
      \multicolumn{1}{c}{\textbf{Pfad}} \\
    }
    \tablehead{
    }
    \tabletail {
    }
    \tablelasttail {
    }
    \begin{supertabular}{lrrrl}
        bank\_2014\_primary\_01 &  10 M &  + 10 M & \color{red}{768 M} &
        D:\textbackslash u01\textbackslash bank\_2014\textbackslash data \\
        \hline
        bank\_2014\_hr\_01     & 100 M & \color{red}{+ 50 M}&
        \color{red}{1 G} & D:\textbackslash u01\textbackslash
        bank\_2014\textbackslash data \\
        bank\_2014\_hr\_02     & 100 M & \color{red}{+ 50 M}&
        \color{red}{1 G} & D:\textbackslash u01\textbackslash
        bank\_2014\textbackslash data \\
        \hline
        bank\_2014\_kam\_01    & \color{red}{500 M} & + 100 M & \color{red}{2 G} &
        E:\textbackslash u02\textbackslash bank\_2014\textbackslash data \\
    \end{supertabular}
  \end{small}
\end{center}

      
        %Schreiben Sie eine Abfrage, um zu überprüfen, ob die Datendatei
        %ordnungsgemäß angefügt wurde.
        \item Schreiben Sie eine Abfrage, um zu überprüfen, ob die Datendatei
ordnungsgemäß angefügt wurde.

        
        %Verschieben Sie die Datendatei mit dem Namen
        %\identifier{bank\_2014\_kam\_01} nach \oscommand{D:\textbackslash
        %u01\textbackslash bank\_2014\textbackslash data}.
              \item Erstellen Sie aus den aktuellen Einstellungen Ihrer Instanz ein neues SPFile names \oscommand{/home/oracle/spfileorcl.ora}.

        
        %Machen Sie die Dateigruppe \identifier{CRM} zur Standarddateigruppe
        %der Datenbank \identifier{Bank 2014}.
              \item Erstellen Sie eine weitere Kontrolldateikopie \oscommand{/u02/oradata/orcl/control03.ctl} und binden Sie sie in Ihre Instanz ein.

        
        %Konfigurieren Sie für die Datenbankoption
        %\identifier{quoted\_identifier} den Wert \identifier{true}.
            \item Vergrößern Sie Ihre FRA auf 6 Gigabyte!

        
        %Ermitteln Sie welchen Wert die Datenbankoption
        %\identifier{auto\_close} hat und recherchieren Sie, welche Bedeutung
        % diese Option hat bzw. was sie bewirkt.
        \item Ermitteln Sie welchen Wert die Datenbankoption \identifier{auto\_close}
hat und recherchieren Sie, welche Bedeutung diese Option hat bzw. was sie
bewirkt.


\clearpage
        %\item Verschieben Sie Ihre Datenbank \identifier{tempdb} gemäß den
        %folgenden Angaben:
        \item Verschieben Sie Ihre Datenbank \identifier{tempdb} gemäß den
folgenden Angaben:
\begin{center}
  \begin{small}
    \changefont{pcr}{m}{n}
    \tablefirsthead {
      \multicolumn{1}{c}{\textbf{Name}} &
      \multicolumn{1}{c}{\textbf{Größe}} &
      \multicolumn{1}{c}{\textbf{Wachstum}} &
      \multicolumn{1}{c}{\textbf{G Max.}} &
      \multicolumn{1}{c}{\textbf{Pfad}} \\
    }
    \tablehead{
    }
    \tabletail {
    }
    \tablelasttail {
    }
    \begin{supertabular}{lrrrl}
      tempdev &   1 G & + 200 M & 2 G &
      G:\textbackslash u04\textbackslash tempdb\textbackslash data \\
      templog & 200 M & + 50 M & 500 M & F:\textbackslash
      u03\textbackslash tempdb\textbackslash log \\
    \end{supertabular}
  \end{small}
\end{center}

      
        %Verschieben Sie das Fehlerprotokoll Ihrer Instanz nach
        %\oscommand{D:\textbackslash u01\textbackslash errorlog}.
            \item Prüfen Sie im RMAN welche Backups nun nicht mehr zur Verfügung stehen.

        
        %Beim Anlegen der Datenbank ist Ihnen ein Fehler unterlaufen. Statt
        %die Datenbank \identifier{Bank\_2014} zu nennen, haben Sie sie
        %\identifier{Bank 2014} genannt. Recherchieren Sie, wie eine
        %Datenbank umbenannt werden kann und ersetzten Sie das Leerzeichen im
        %Datenbanknamen durch einen Unterstrich!
            \item Löschen Sie die Einträge für alle nicht mehr verfügbaren Backups.

        
    \end{enumerate}