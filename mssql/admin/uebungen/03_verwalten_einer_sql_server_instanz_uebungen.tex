\clearpage
    \section{\"Ubungen - Verwalten einer SQL Server-Instanz}
      \begin{enumerate}
        %Richten Sie auf Ihrem Übungsserver alle notwendigen Werkzeuge ein, um
        % mit der Domäne \identifier{MS-C-IX-04.FUS} arbeiten zu können.
        \item Bereiten Sie Ihre SQL Server-Instanz auf die Benutzung von FileStream vor.
Der Bezeichner für die Filestream-Freigabe lautet: \identifier{mssqlserver}. Der
Bezeichner für das Verzeichnis der Datenbank \identifier{bank\_2014 }lautet:
\identifier{bank\_2014}. Aktivieren Sie den nichttransaktionalen Zugriff auf den
Filestream.

        %Nehmen Sie Ihren Übungsserver in die Domäne
        %\identifier{MS-C-IX-04.FUS} auf. Achten Sie darauf, dass das
        %Computerkonto Ihres Datenbankservers in der 
        %Organisationseinheit \oscommand{IT-SysAdmin\textbackslash
        %EXER\textbackslash Serverxx\textbackslash Computers} abgelegt wird,
        % bzw. verschieben Sie es dorthin.
            \item Registrieren Sie Ihre Datenbank in Ihrem neuen Recovery Katalog!

        
        %Deaktivieren Sie den Zugriff mittels Shared Memory auf Ihre SQL
        %Server-Instanz!
            \item Wechseln Sie den Undo-Tablespace zurück zum Original Undo-Tablespace und löschen Sie \identifier{undotbs02} und \identifier{undotbs03}.

        
        %Der Zugriff auf Ihre SQL Server-Instanz soll nur noch mittels der 
        %öffentlichen IP-Adresse 192.168.111.xx erfolgen können! Alle anderen
        % IPs müssen deaktiviert werden!
            \item Erstellen Sie ein unkomprimiertes Backup Set des Tablespaces
    \identifier{bank}! Das Backup Set soll in der FRA (Fast Recovery Area)
    abgelegt werden!

        
        %Richten Sie die im Folgenden aufgeführten gMSAs in der
        %Organisationseinheit \oscommand{IT-SysAdmin\textbackslash
        %EXER\textbackslash Serverxx\textbackslash MSAs} für die Dienste Ihrer
        % SQL Server-Instanz ein. Ersetzen Sie dabei \enquote{xx} durch Ihre Platznummer!
        \item Legen Sie die Tabelle \identifier{buchung} gemäß Ihrem ER-Modell als SOT
an und fügen Sie zwischen die beiden Spalten \identifier{buchungsdatum} und
\identifier{konto\_id} die Spalte \identifier{buchungstext} VARCHAR(200) ein!
Legen Sie auf die Spalte \identifier{buchungsdatum} einen Index. Legen Sie auch
auf die Spalte \identifier{buchungstext} einen Index.

        %Stoppen Sie die beiden Dienste \identifier{SQL Server Reporting
        %Services} und \identifier{SQL Server Integration Services 12.0}.
        %Konfigurieren Sie beide so, dass Sie manuell gestartet werden müssen.
              \item Starten und öffnen Sie Ihre Datenbank.


        %Konfigurieren Sie Ihre Instanz MSSQLSERVER so, dass Sie immer
        %mindestens 200M Arbeitsspeichern zur Verfügung hat und nie mehr als 2G
        %benutzt!
            \item Konfigurieren Sie die Benennungsmethode Directory Naming für Ihre
    Datenbank. Der Directory Server ist ein OID und wird unter der IP-Adresse
    192.168.111.250 zufinden sein (Ports: 389 und 636). Der LDAP-Context
    lautet: \enquote{cn=OracleContext}. Nutzen Sie für diese Aufgabe den
    Oracle Net Configuration Assistant (\oscommand{netca} - Konfigurieren von
    Directoy-Verwendung).


        %Schreiben Sie eine SQL-Abfrage, die anzeigen, ob die Einstellungen
        %aus der vorangegangenen Aufgabe bereits wirksam geworden sind oder
        %nicht!
            \item Welche Kosten hat das Statement verursacht und wie viel Zeit wurde für die Ausführung benötigt?

        
        %Deaktivieren Sie die Überwachung von fehlerhaften Anmeldungen. Es
        %soll keine Anmeldeüberwachung geben!
            \item Verschieben Sie jetzt das Backup Set des \identifier{bank}-Tablespaces
    auf das SBT-Gerät!


        %Fügen Sie Ihre Instanz MSSQLSERVER die Datenbank \identifier{Bank
        %2014} hinzu. Der Aufbau der Datenbank kann der folgenden Tabelle
        %entnommen werden. Die Namen der Dateigruppen sind in den logischen
        % Namen der Datendateien enthalten (Primary, HR, CRM, KAM und MAIL).
            \item Löschen Sie das Backup Set des \identifier{bank}-Tablespaces von der Festplatte, so dass es nur noch auf SBT verfügbar ist!

        
        %Fügen Sie der \identifier{Bank 2014} die Dateigruppe
        %\identifier{staging} hinzu. Benutzen Sie dazu die folgenden Angaben:
            \item Der Tablespace \identifier{bank} liegt derzeit in unverschlüsselter Form vor. Dies muss zukünftig anders sein. Bereiten Sie einen Tablespace \identifier{bank\_encrypted} vor, der mittels \identifier{AES256} verschlüsselung gesichert ist und der die gleichen Dimensionen hat, wie der Originaltablespace \identifier{bank}! Rufen Sie alle notwendigen Informationen über den Tablespace \identifier{bank} aus dem Data Dictionary ab! Welche Views helfen Ihnen dabei?

        
        %Fügen Sie der Dateigruppe \identifier{crm} eine weitere Datendatei
        %hinzu. Verwenden Sie dazu die folgenden Angaben:
           \item Führen Sie mit Hilfe von RMAN ein Block-Media-Recovery durch, um die be\-schä\-dig\-ten Blöcke zu reparieren.

        
        %Nehmen Sie die im Folgenden beschriebenen Verändernungen an den
        %Datendateien Ihrer Datenbank \identifier{Bank 2014} vor.
        \item Deaktivieren Sie den gruppierten Index der Tabelle \identifier{girokonto}
und beantworten Sie die folgenden Fragen:

Können die Daten der Tabelle \identifier{girokonto} noch selektiert werden?

  \rule{0.94\textwidth}{0.5pt}

Welche Auswirkungen hatte das Deaktivieren des clustered Index auf den nicht
gruppierten Index der Spalte \identifier{guthaben}

  \rule{0.94\textwidth}{0.5pt}

  \rule{0.94\textwidth}{0.5pt}


      
        %Schreiben Sie eine Abfrage, um zu überprüfen, ob die Datendatei
        %ordnungsgemäß angefügt wurde.
        \item Schreiben Sie eine Abfrage, um zu überprüfen, ob die Datendatei
ordnungsgemäß angefügt wurde.

        
        %Verschieben Sie die Datendatei mit dem Namen
        %\identifier{bank\_2014\_kam\_01} nach \oscommand{D:\textbackslash
        %u01\textbackslash bank\_2014\textbackslash data}.
              \item Erstellen Sie aus den aktuellen Einstellungen Ihrer Instanz ein neues SPFile names \newline \oscommand{/home/oracle/spfileorcl.ora}.

        
        %Machen Sie die Dateigruppe \identifier{CRM} zur Standarddateigruppe
        %der Datenbank \identifier{Bank 2014}.
        \item Machen Sie die Dateigruppe \identifier{CRM} zur Standarddateigruppe
der Datenbank \identifier{Bank 2014}.

        
        %Konfigurieren Sie für die Datenbankoption
        %\identifier{quoted\_identifier} den Wert \identifier{true}.
        \item Konfigurieren Sie für die Datenbankoption \identifier{quoted\_identifier}
den Wert \identifier{true}.

        
        %Ermitteln Sie welchen Wert die Datenbankoption
        %\identifier{auto\_close} hat und recherchieren Sie, welche Bedeutung
        % diese Option hat bzw. was sie bewirkt.
        \item Ermitteln Sie welchen Wert die Datenbankoption \identifier{auto\_close}
hat und recherchieren Sie, welche Bedeutung diese Option hat bzw. was sie
bewirkt.


\clearpage
        %\item Verschieben Sie Ihre Datenbank \identifier{tempdb} gemäß den
        %folgenden Angaben:
            \item Führen Sie das Skript labs/lab\_delete\_backups.sql aus. Dieses Skript wird eine will\-kür\-liche Anzahl Ihrer Backups löschen.

      
        %Verschieben Sie das Fehlerprotokoll Ihrer Instanz nach
        %\oscommand{D:\textbackslash u01\textbackslash errorlog}.
            \item Prüfen Sie im RMAN welche Backups nun nicht mehr zur Verfügung stehen.

        
        %Beim Anlegen der Datenbank ist Ihnen ein Fehler unterlaufen. Statt
        %die Datenbank \identifier{Bank\_2014} zu nennen, haben Sie sie
        %\identifier{Bank 2014} genannt. Recherchieren Sie, wie eine
        %Datenbank umbenannt werden kann und ersetzten Sie das Leerzeichen im
        %Datenbanknamen durch einen Unterstrich!
        \item Beim Anlegen der Datenbank ist Ihnen ein Fehler unterlaufen. Statt
die Datenbank \identifier{Bank\_2014} zu nennen, haben Sie sie
\identifier{Bank 2014} genannt. Recherchieren Sie, wie eine
Datenbank umbenannt werden kann und ersetzten Sie das Leerzeichen im
Datenbanknamen durch einen Unterstrich!

        
    \end{enumerate}