\clearpage
    \section{Lösungen - Verwalten einer SQL Server-Instanz}
      \begin{enumerate}
              \item Führen Sie ein Recovery bei Verlust einer Redo Log Datei durch!
      \begin{enumerate}
        \item Starten Sie das Skript \oscommand{lab\_delete\_redolog\_member.sql}! Es löscht einen beliebigen Member einer Ihrer Redo Log Gruppen.
          \begin{lstlisting}[language=terminal]
SQL> @/home/oracle/labs/lab_delete_redolog_member.sql
          \end{lstlisting}
        \item Die Datenbank läuft weiterhin normal und es liegen keinerlei Probleme vor. Öffnen Sie die Alert Log Datei, um herauszufinden welcher Redo Log Member gelöscht wurde!
        \item Führen Sie geeignete Maßnahmen zur Problembehebung durch!
      \end{enumerate}

          
          Fügen Sie Ihrem Computer die \enquote{Richtlinienverwaltung} und die
          \enquote{Remoteverwaltungstools} hinzu!
          
          \item Nehmen Sie Ihren Übungsserver in die Domäne
\identifier{MS-C-IX-04.FUS} auf. Achten Sie darauf, dass das
Computerkonto Ihres Datenbankservers in der 
Organisationseinheit \oscommand{IT-SysAdmin\textbackslash
EXER\textbackslash Serverxx\textbackslash Computers} abgelegt wird, bzw. verschieben Sie es dorthin.

          \begin{lstlisting}[language=powershell, caption={Aufnehmen eines
          Computers in eine Windows Domäne}, label=admin_03_loesung_01]
~$ou~ = "OU=Computers,OU=Serverxx,OU=EXER,OU=IT-SysAdmin,DC=MS-C-IX-04-DC=FUS"
Add-Computer |-ComputerName| FEA11-119SRVxx `
|-LocalCredential| FEA11-119SRVxx\Administrator `
|-DomainName| MS-C-IX-04.FUS `
|-Credential| MS-C-IX_04\Administrator `
|-OUPath| ~$ou~
|-Restart| |-Force|
          \end{lstlisting}
          
                \item Bringen Sie Ihre Datenbank in die Mount-Phase.

          \bild{Den Shared Memory-Zugriff
          deaktivieren}{deactivate_shared_memory}{2}
\clearpage          
              \item Welchen Default Tablespace benutzt \identifier{bob} und aus welcher View können Sie diese Angabe entnehmen?

          \bild{IP-Adressen deaktivieren}{deactivate_ip_addresses}{2}
          
              \item Prüfen Sie den Audittrail auf neue Informationen! Welche Tätigkeiten wurden an der auditierten Tabelle ausgeführt?

          \begin{lstlisting}[language=powershell, caption={Die notwendigen
          gMSAs erstellen}, label=admin_03_loesung_02]
~$gou~ = "OU=GROUPS,OU=Serverxx,OU=EXER,OU=IT-SysAdmin,DC=MS-C-IX-04-DC=FUS"
New-ADGroup |-Name| SQLServerSVCxx |-GroupScope| Global `
|-GroupCategory| Security `
|-Path| ~$gou~

~$cou~ = "OU=Computers,OU=Serverxx,OU=EXER,OU=IT-SysAdmin,DC=MS-C-IX-04-DC=FUS"
~$identity~ = "CN=SQLServerSVCxx," + ~$cou~
~$member~ = "CN=FEA11-119SRVxx," + ~$cou~

Add-ADGroupMember |-Identity| ~$identity~ |-Members| $member
          \end{lstlisting}
\clearpage
          \begin{lstlisting}[language=powershell]           
~$mou~ = "OU=gMSAs,OU=Serverxx,OU=EXER,OU=IT-SysAdmin,DC=MS-C-IX-04-DC=FUS"

New-ADServiceAccount |-Name| MSSQLSRVRxx `
|-DNSHostname| FEA11-119SRVAD.MS-C-IX-04.FUS `
|-Path| ~$mou~ `
|-PrincipalsAllowedToRetrieveManagedPassword| ~$identity~ `
|-Enabled| ~$true~

New-ADServiceAccount |-Name| MSSQLAGENTxx `
|-DNSHostname| FEA11-119SRVAD.MS-C-IX-04.FUS `
|-Path| ~$mou~ `
|-PrincipalsAllowedToRetrieveManagedPassword| ~$identity~ `
|-Enabled| ~$true~

New-ADServiceAccount |-Name| MSDTSSRVRxx `
|-DNSHostname| FEA11-119SRVAD.MS-C-IX-04.FUS `
|-Path| ~$mou~ `
|-PrincipalsAllowedToRetrieveManagedPassword| ~$identity~ `
|-Enabled| ~$true~

New-ADServiceAccount |-Name| MSRPTSRVRxx `
|-DNSHostname| FEA11-119SRVAD.MS-C-IX-04.FUS `
|-Path| ~$mou~ `
|-PrincipalsAllowedToRetrieveManagedPassword| ~$identity~ `
|-Enabled| ~$true~

New-ADServiceAccount |-Name| MSSQLBRWSRxx `
|-DNSHostname| FEA11-119SRVAD.MS-C-IX-04.FUS `
|-Path| ~$mou~ `
|-PrincipalsAllowedToRetrieveManagedPassword| ~$identity~ `
|-Enabled| ~$true~

#Execute these steps as local admin on your computer

~$identity~ = "CN=MSSQLSRVRxx," + ~$mou~
Install-ADServiceAccount |-Identity| ~$identity~

~$identity~ = "CN=MSSQLAGENTxx," + ~$mou~
Install-ADServiceAccount |-Identity| ~$identity~

~$identity~ = "CN=MSDTSSRVRxx," + ~$mou~
Install-ADServiceAccount |-Identity| ~$identity~

~$identity~ = "CN=MSRPTSRVRxx," + ~$mou~
Install-ADServiceAccount |-Identity| ~$identity~

~$identity~ = "CN=MSSQLBRWSRxx," + ~$mou~
Install-ADServiceAccount |-Identity| ~$identity~
          \end{lstlisting}        
        \item Erstellen Sie die Tabelle \identifier{eigenkunde}, gemäß Ihrem ER-Modell,
in der Dateigruppe \identifier{kam} und legen Sie ein \UNIQUE-Constraint auf
die Spalte \identifier{personalausweisnr}. Die Sortierung des Indexes, der zu diesem
Constraint gehört soll absteigend sein.
        \begin{lstlisting}[language=powershell, caption={Stoppen und
        umkonfigurieren von Diensten}, label=admin_03_loesung_03]
Set-Service |-DisplayName| 'SQL Server Integration Services 12.0' `
|-StartupType| Manual

Set-Service |-DisplayName| 'SQL Server Reporting Services (MSSQLSERVER)' `
|-StartupType| Manual

Stop-Service |-DisplayName| 'SQL Server Integration Services 12.0'
Stop-Service |-DisplayName| 'SQL Server Reporting Services (MSSQLSERVER)'     
        \end{lstlisting}
            \item Welche Tablespaces in Ihrer Datenbank sind Bigfile Tablespaces?

        \begin{lstlisting}[language=ms_sql, caption={Ändern der
        Servereigenschaften}, label=admin_03_loesung_04]
EXEC sp_configure 'min server memory (MB)', '200'
EXEC sp_configure 'max server memory (MB)', '2048'
RECONFIGURE
        \end{lstlisting}
        \item Schreiben Sie eine SQL-Abfrage, die anzeigen, ob die Einstellungen
aus der vorangegangenen Aufgabe bereits wirksam geworden sind oder
nicht!

        \begin{lstlisting}[language=ms_sql, caption={Ändern der
        Servereigenschaften}, label=admin_03_loesung_05]
SELECT name, value, value_in_use
FROM   sys.configurations
WHERE  configuration_id IN (1543, 1544)
GO
        \end{lstlisting}
        \begin{center}
          \begin{small}
            \changefont{pcr}{m}{n}
            \tablefirsthead {
              \multicolumn{1}{l}{\textbf{name}} &
              \multicolumn{1}{l}{\textbf{value}} &
              \multicolumn{1}{l}{\textbf{value\_in\_use}} \\
              \cmidrule(l){1-1}\cmidrule(l){2-2}\cmidrule(l){3-3}
            }
            \tablehead{}
            \tabletail {
              \multicolumn{1}{l}{\textbf{2 Zeilen ausgew\"ahlt}} \\
            }
            \tablelasttail {
              \multicolumn{1}{l}{\textbf{2 Zeilen ausgew\"ahlt}} \\
            }
            \begin{mssql}
              \begin{supertabular}{lll}
                min server memory (MB) & 200 & 200 \\
                max server memory (MB) & 2048 & 2048 \\
              \end{supertabular}
            \end{mssql}
          \end{small}
        \end{center}
            \item Führen Sie das gleiche Statement nochmals aus und ermitteln Sie erneut die verursachten Kosten und die benötigte Ausführungszeit! Haben sich die Kosten verändert?

        \begin{lstlisting}[language=ms_sql, caption={Ändern der
        Servereigenschaften}, label=admin_03_loesung_06]
USE [master]
GO

EXEC xp_instance_regwrite N'HKEY_LOCAL_MACHINE', 
     N'Software\Microsoft\MSSQLServer\MSSQLServer', 
     N'AuditLevel', REG_DWORD, 0
GO
        \end{lstlisting}
        
            \item Verändern Sie die Datendatei \oscommand{uebungs\_ts02.dbf} so, dass sie auf bis zu 1 G Größe anwachsen kann.

        \begin{lstlisting}[language=ms_sql, caption={Ändern der
        Servereigenschaften}, label=admin_03_loesung_07]
USE [master]
GO

CREATE DATABASE [Bank 2014]
  CONTAINMENT = NONE
  ON  PRIMARY 
  ( NAME = N'bank_2014_primary_01', 
    FILENAME = N'D:\u01\bank_2014\data\bank_2014_primary_01.mdf', 
    SIZE = 10 MB, MAXSIZE = 500 MB, FILEGROWTH = 10 MB ), 
  FILEGROUP [CRM] 
  ( NAME = N'bank_2014_crm_01', 
    FILENAME = N'E:\u02\bank_2014\data\bank_2014_crm_01.ndf', 
    SIZE = 50 MB, MAXSIZE = 100 MB, FILEGROWTH = 10 MB ), 
  FILEGROUP [HR] 
  ( NAME = N'bank_2014_hr_01', 
    FILENAME = N'D:\u01\bank_2014\data\bank_2014_hr_01.ndf', 
    SIZE = 100 MB, MAXSIZE = 500 MB, FILEGROWTH = 20 MB ),
  ( NAME = N'bank_2014_hr_02', 
    FILENAME = N'D:\u01\bank_2014\data\bank_2014_hr_02.ndf', 
    SIZE = 100 MB, MAXSIZE = 500 MB, FILEGROWTH = 20 MB ), 
  FILEGROUP [KAM] 
  ( NAME = N'bank_2014_kam_01', 
    FILENAME = N'E:\u02\bank_2014\data\bank_2014_kam_01.ndf', 
    SIZE = 100 MB, MAXSIZE = 500 MB, FILEGROWTH = 100 MB ), 
        \end{lstlisting}
\clearpage
        \begin{lstlisting}[language=ms_sql]        
  FILEGROUP [MAIL] 
  ( NAME = N'bank_2014_mail_01', 
    FILENAME = N'D:\u01\bank_2014\data\bank_2014_mail_01.ndf', 
    SIZE = 500 MB, MAXSIZE = 1 GB, FILEGROWTH = 100 MB ),
  ( NAME = N'bank_2014_mail_02', 
    FILENAME = N'D:\u01\bank_2014\data\bank_2014_mail_02.ndf', 
    SIZE = 500 MB, MAXSIZE = 1 GB, FILEGROWTH = 100 MB ),
  ( NAME = N'bank_2014_mail_03', 
    FILENAME = N'D:\u01\bank_2014\data\bank_2014_mail_03.ndf', 
    SIZE = 500 MB, MAXSIZE = 1 GB, FILEGROWTH = 100 MB )
  LOG ON 
  ( NAME = N'Bank 2014_log', 
    FILENAME = N'F:\u03\bank_2014\log\bank_2014_log.ldf', 
    SIZE = 768 MB, MAXSIZE = 3 GB, FILEGROWTH = 768 MB )
GO
        \end{lstlisting}
        
            \item Führen Sie das Skript \oscommand{/home/oracle/labs/lab\_fullts.sql}
    aus. Nach der Ausführung des Skripts existiert ein neuer Tablespace
    \identifier{fullts} und ein neuer Nutzer \identifier{alice}.

        \begin{lstlisting}[language=ms_sql, caption={Hinzufügen einer
        Dateigruppe mit Datendatei}, label=admin_03_loesung_08]
USE [master]
GO

ALTER DATABASE [Bank 2014] 
ADD FILEGROUP [STAGING]
GO

ALTER DATABASE [Bank 2014] 
ADD FILE (NAME = N'bank_2014_staging', 
          FILENAME = N'E:\u02\bank_2014\data\bank_2014_staging_01.ndf', 
          SIZE = 800 MB, MAXSIZE = 2 GB, FILEGROWTH = 40 MB) 
TO FILEGROUP [STAGING]
GO
        
        \end{lstlisting}
\clearpage        
            \item Fügen Sie dem Profil \identifier{p\_clerk} den Parameter \parameter{PASSWORD\_GRACE\_TIME} mit dem Wert 5 Tage hinzu!

        \begin{lstlisting}[language=ms_sql, caption={Hinzufügen einer
        Datendatei zu einer bestehenden Dateigruppe}, label=admin_03_loesung_09]
USE [master]
GO

ALTER DATABASE [Bank 2014] 
ADD FILE (NAME = N'bank_2014_crm_02', 
          FILENAME = N'E:\u02\bank_2014\data\bank_2014_crm_02.ndf', 
          SIZE = 50 MB, MAXSIZE = 500 MB, FILEGROWTH = 10 MB )
TO FILEGROUP [CRM]
GO        
        \end{lstlisting}
        
      \item Nehmen Sie die im Folgenden beschriebenen Verändernungen an den
Datendateien Ihrer Datenbank \identifier{Bank 2014} vor.
\begin{center}
  \begin{small}
    \changefont{pcr}{m}{n}
    \tablefirsthead {
      \multicolumn{1}{c}{\textbf{Name}} &
      \multicolumn{1}{c}{\textbf{Größe}} &
      \multicolumn{1}{c}{\textbf{Wachstum}} &
      \multicolumn{1}{c}{\textbf{G Max.}} &
      \multicolumn{1}{c}{\textbf{Pfad}} \\
    }
    \tablehead{
    }
    \tabletail {
    }
    \tablelasttail {
    }
    \begin{supertabular}{lrrrl}
        bank\_2014\_primary\_01 &  10 M &  + 10 M & \color{red}{768 M} &
        D:\textbackslash u01\textbackslash bank\_2014\textbackslash data \\
        \hline
        bank\_2014\_hr\_01     & 100 M & \color{red}{+ 50 M}&
        \color{red}{1 G} & D:\textbackslash u01\textbackslash
        bank\_2014\textbackslash data \\
        bank\_2014\_hr\_02     & 100 M & \color{red}{+ 50 M}&
        \color{red}{1 G} & D:\textbackslash u01\textbackslash
        bank\_2014\textbackslash data \\
        \hline
        bank\_2014\_kam\_01    & \color{red}{500 M} & + 100 M & \color{red}{2 G} &
        E:\textbackslash u02\textbackslash bank\_2014\textbackslash data \\
    \end{supertabular}
  \end{small}
\end{center}

      \begin{lstlisting}[language=ms_sql, caption={Ändern der
      Datendateieigenschaften}, label=admin_03_loesung_10]
USE [master]
GO

ALTER DATABASE [Bank 2014] 
MODIFY FILE ( NAME = N'bank_2014_hr_01', MAXSIZE = 1 GB )
GO

ALTER DATABASE [Bank 2014] 
MODIFY FILE ( NAME = N'bank_2014_hr_02', MAXSIZE = 1 GB )
GO

ALTER DATABASE [Bank 2014] 
MODIFY FILE ( NAME = N'bank_2014_kam_01', SIZE = 500 MB, MAXSIZE = 2 GB )
GO

ALTER DATABASE [Bank 2014]
MODIFY FILE ( NAME = N'bank_2014_primary_01', MAXSIZE = 768 MB )
GO
      \end{lstlisting}
\clearpage
      \item Schreiben Sie eine Abfrage, um zu überprüfen, ob die Datendatei
ordnungsgemäß angefügt wurde.

      \begin{lstlisting}[language=ms_sql, caption={Abfragen der
      Datendateieigenschaften}, label=admin_03_loesung_11]
USE [Bank 2014]
GO

SELECT name, size * 8 / 1024 AS Size, 
       growth * 8 / 1024 AS Growth, 
       max_size * 8 / 1024 AS Max_Size
FROM   sys.database_files
      \end{lstlisting}
      
            \item Erstellen Sie aus den aktuellen Einstellungen Ihrer Instanz ein neues SPFile names \oscommand{/home/oracle/spfileorcl.ora}.

      \begin{lstlisting}[language=ms_sql, caption={Verschieben einer
      Datendatei}, label=admin_03_loesung_12]
USE [master]
GO

ALTER DATABASE [Bank 2014]
SET OFFLINE;

-- Move Datafile on filesystem
-- Change filesystemrights

ALTER DATABASE [Bank 2014]
MODIFY FILE ( NAME = N'bank_2014_kam_01',
FILENAME = N'D:\u01\bank_2014\data\bank_2014_kam_01.ndf' )
GO

ALTER DATABASE [Bank 2014]
SET ONLINE;     
      \end{lstlisting}

            \item Erstellen Sie eine weitere Kontrolldateikopie \oscommand{/u02/oradata/orcl/control03.ctl} und binden Sie sie in Ihre Instanz ein.

      \begin{lstlisting}[language=ms_sql, caption={Ändern der
      Standarddateigruppe}, label=admin_03_loesung_12]
USE [Bank 2014]
GO

IF NOT EXISTS (
   SELECT name 
   FROM sys.filegroups 
   WHERE is_default=1 AND name = N'CRM')
     ALTER DATABASE [Bank 2014]
     MODIFY FILEGROUP [CRM] DEFAULT
GO
      \end{lstlisting}
            
          \item Vergrößern Sie Ihre FRA auf 6 Gigabyte!

      \begin{lstlisting}[language=ms_sql, caption={Die Datenbankoption
      quoted\_identifier ändern}, label=admin_03_loesung_13]
USE [master]
GO

ALTER DATABASE [Bank 2014] 
SET QUOTED_IDENTIFIER ON 
WITH NO_WAIT
GO 
      \end{lstlisting}
      
      \item Ermitteln Sie welchen Wert die Datenbankoption \identifier{auto\_close}
hat und recherchieren Sie, welche Bedeutung diese Option hat bzw. was sie
bewirkt.

      
      Die Datenbankoption \identifier{auto\_close} bewirkt, dass die Datenbank
      geschlossen wird, sobald sich der letzte noch aktive Nutzer abgemeldet
      hat. Dies hat zur Folge, dass alle im Arbeitspeicher befindlichen Teile
      der Datenbank von dort entfernt werden, wodurch die Performance dieser
      Datenbank sehr starkt negativ beeinflusst wird.
      
      \item Verschieben Sie Ihre Datenbank \identifier{tempdb} gemäß den
folgenden Angaben:
\begin{center}
  \begin{small}
    \changefont{pcr}{m}{n}
    \tablefirsthead {
      \multicolumn{1}{c}{\textbf{Name}} &
      \multicolumn{1}{c}{\textbf{Größe}} &
      \multicolumn{1}{c}{\textbf{Wachstum}} &
      \multicolumn{1}{c}{\textbf{G Max.}} &
      \multicolumn{1}{c}{\textbf{Pfad}} \\
    }
    \tablehead{
    }
    \tabletail {
    }
    \tablelasttail {
    }
    \begin{supertabular}{lrrrl}
      tempdev &   1 G & + 200 M & 2 G &
      G:\textbackslash u04\textbackslash tempdb\textbackslash data \\
      templog & 200 M & + 50 M & 500 M & F:\textbackslash
      u03\textbackslash tempdb\textbackslash log \\
    \end{supertabular}
  \end{small}
\end{center}

      \begin{lstlisting}[language=ms_sql, caption={Verschieben der
      Systemdatenbank tempdb}, label=admin_03_loesung_14]
USE [master]
GO

ALTER DATABASE [tempdb]
MODIFY FILE ( NAME = N'tempdev',
              FILENAME = N'G:\u04\tempdb\data\tempdb.mdf' )
GO

ALTER DATABASE [tempdb]
MODIFY FILE ( NAME = N'templog',
              FILENAME = N'F:\u03\tempdb\log\templog.ldf' )
GO

-- Shutdown the instance

-- Move file on filesystem to their new locations

-- Change filesystemrights

-- Startup the instance
      \end{lstlisting}
\clearpage      
          \item Prüfen Sie im RMAN welche Backups nun nicht mehr zur Verfügung stehen.

      \bild{Verschieben des errorlog}{move_error_log}{0.48}
      
          \item Löschen Sie die Einträge für alle nicht mehr verfügbaren Backups.

      \begin{lstlisting}[language=ms_sql, caption={Umbennen einer Datenbank},
      label=admin_03_loesung_15]
USE [master]
GO

ALTER DATABASE [Bank 2014]
Modify Name = Bank_2014
GO
      
      \end{lstlisting}
    \end{enumerate}
