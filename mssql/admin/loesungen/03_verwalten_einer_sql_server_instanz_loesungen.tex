\clearpage
    \section{Lösungen - Verwalten einer SQL Server-Instanz}
      \begin{enumerate}
          \item Bereiten Sie Ihre SQL Server-Instanz auf die Benutzung von FileStream vor.
Der Bezeichner für die Filestream-Freigabe lautet: \identifier{mssqlserver}. Der
Bezeichner für das Verzeichnis der Datenbank \identifier{bank\_2014 }lautet:
\identifier{bank\_2014}. Aktivieren Sie den nichttransaktionalen Zugriff auf den
Filestream.
          
          Fügen Sie Ihrem Computer die \enquote{Richtlinienverwaltung} und die
          \enquote{Remoteverwaltungstools} hinzu!
          
              \item Registrieren Sie Ihre Datenbank in Ihrem neuen Recovery Katalog!

          \begin{lstlisting}[language=powershell, caption={Aufnehmen eines
          Computers in eine Windows Domäne}, label=admin_03_loesung_01]
~$ou~ = "OU=Computers,OU=Serverxx,OU=EXER,OU=IT-SysAdmin,DC=MS-C-IX-04-DC=FUS"
Add-Computer |-ComputerName| FEA11-119SRVxx `
|-LocalCredential| FEA11-119SRVxx\Administrator `
|-DomainName| MS-C-IX-04.FUS `
|-Credential| MS-C-IX_04\Administrator `
|-OUPath| ~$ou~
|-Restart| |-Force|
          \end{lstlisting}
          
              \item Wechseln Sie den Undo-Tablespace zurück zum Original Undo-Tablespace und löschen Sie \identifier{undotbs02} und \identifier{undotbs03}.

          \bild{Den Shared Memory-Zugriff
          deaktivieren}{deactivate_shared_memory}{2}
\clearpage          
              \item Erstellen Sie ein unkomprimiertes Backup Set des Tablespaces
    \identifier{bank}! Das Backup Set soll in der FRA (Fast Recovery Area)
    abgelegt werden!

          \bild{IP-Adressen deaktivieren}{deactivate_ip_addresses}{2}
          
          \item Legen Sie die Tabelle \identifier{buchung} gemäß Ihrem ER-Modell als SOT
an und fügen Sie zwischen die beiden Spalten \identifier{buchungsdatum} und
\identifier{konto\_id} die Spalte \identifier{buchungstext} VARCHAR(200) ein!
Legen Sie auf die Spalte \identifier{buchungsdatum} einen Index. Legen Sie auch
auf die Spalte \identifier{buchungstext} einen Index.
          \begin{lstlisting}[language=powershell, caption={Die notwendigen
          gMSAs erstellen}, label=admin_03_loesung_02]
~$gou~ = "OU=GROUPS,OU=Serverxx,OU=EXER,OU=IT-SysAdmin,DC=MS-C-IX-04-DC=FUS"
New-ADGroup |-Name| SQLServerSVCxx |-GroupScope| Global `
|-GroupCategory| Security `
|-Path| ~$gou~

~$cou~ = "OU=Computers,OU=Serverxx,OU=EXER,OU=IT-SysAdmin,DC=MS-C-IX-04-DC=FUS"
~$identity~ = "CN=SQLServerSVCxx," + ~$cou~
~$member~ = "CN=FEA11-119SRVxx," + ~$cou~

Add-ADGroupMember |-Identity| ~$identity~ |-Members| $member
          \end{lstlisting}
\clearpage
          \begin{lstlisting}[language=powershell]           
~$mou~ = "OU=gMSAs,OU=Serverxx,OU=EXER,OU=IT-SysAdmin,DC=MS-C-IX-04-DC=FUS"

New-ADServiceAccount |-Name| MSSQLSRVRxx `
|-DNSHostname| FEA11-119SRVAD.MS-C-IX-04.FUS `
|-Path| ~$mou~ `
|-PrincipalsAllowedToRetrieveManagedPassword| ~$identity~ `
|-Enabled| ~$true~

New-ADServiceAccount |-Name| MSSQLAGENTxx `
|-DNSHostname| FEA11-119SRVAD.MS-C-IX-04.FUS `
|-Path| ~$mou~ `
|-PrincipalsAllowedToRetrieveManagedPassword| ~$identity~ `
|-Enabled| ~$true~

New-ADServiceAccount |-Name| MSDTSSRVRxx `
|-DNSHostname| FEA11-119SRVAD.MS-C-IX-04.FUS `
|-Path| ~$mou~ `
|-PrincipalsAllowedToRetrieveManagedPassword| ~$identity~ `
|-Enabled| ~$true~

New-ADServiceAccount |-Name| MSRPTSRVRxx `
|-DNSHostname| FEA11-119SRVAD.MS-C-IX-04.FUS `
|-Path| ~$mou~ `
|-PrincipalsAllowedToRetrieveManagedPassword| ~$identity~ `
|-Enabled| ~$true~

New-ADServiceAccount |-Name| MSSQLBRWSRxx `
|-DNSHostname| FEA11-119SRVAD.MS-C-IX-04.FUS `
|-Path| ~$mou~ `
|-PrincipalsAllowedToRetrieveManagedPassword| ~$identity~ `
|-Enabled| ~$true~

#Execute these steps as local admin on your computer

~$identity~ = "CN=MSSQLSRVRxx," + ~$mou~
Install-ADServiceAccount |-Identity| ~$identity~

~$identity~ = "CN=MSSQLAGENTxx," + ~$mou~
Install-ADServiceAccount |-Identity| ~$identity~

~$identity~ = "CN=MSDTSSRVRxx," + ~$mou~
Install-ADServiceAccount |-Identity| ~$identity~

~$identity~ = "CN=MSRPTSRVRxx," + ~$mou~
Install-ADServiceAccount |-Identity| ~$identity~

~$identity~ = "CN=MSSQLBRWSRxx," + ~$mou~
Install-ADServiceAccount |-Identity| ~$identity~
          \end{lstlisting}        
              \item Starten und öffnen Sie Ihre Datenbank.

        \begin{lstlisting}[language=powershell, caption={Stoppen und
        umkonfigurieren von Diensten}, label=admin_03_loesung_03]
Set-Service |-DisplayName| 'SQL Server Integration Services 12.0' `
|-StartupType| Manual

Set-Service |-DisplayName| 'SQL Server Reporting Services (MSSQLSERVER)' `
|-StartupType| Manual

Stop-Service |-DisplayName| 'SQL Server Integration Services 12.0'
Stop-Service |-DisplayName| 'SQL Server Reporting Services (MSSQLSERVER)'     
        \end{lstlisting}
            \item Konfigurieren Sie die Benennungsmethode Directory Naming für Ihre
    Datenbank. Der Directory Server ist ein OID und wird unter der IP-Adresse
    192.168.111.250 zufinden sein (Ports: 389 und 636). Der LDAP-Context
    lautet: \enquote{cn=OracleContext}. Nutzen Sie für diese Aufgabe den
    Oracle Net Configuration Assistant (\oscommand{netca} - Konfigurieren von
    Directoy-Verwendung).

        \begin{lstlisting}[language=ms_sql, caption={Ändern der
        Servereigenschaften}, label=admin_03_loesung_04]
EXEC sp_configure 'min server memory (MB)', '200'
EXEC sp_configure 'max server memory (MB)', '2048'
RECONFIGURE
        \end{lstlisting}
            \item Welche Kosten hat das Statement verursacht und wie viel Zeit wurde für die Ausführung benötigt?

        \begin{lstlisting}[language=ms_sql, caption={Ändern der
        Servereigenschaften}, label=admin_03_loesung_05]
SELECT name, value, value_in_use
FROM   sys.configurations
WHERE  configuration_id IN (1543, 1544)
GO
        \end{lstlisting}
        \begin{center}
          \begin{small}
            \changefont{pcr}{m}{n}
            \tablefirsthead {
              \multicolumn{1}{l}{\textbf{name}} &
              \multicolumn{1}{l}{\textbf{value}} &
              \multicolumn{1}{l}{\textbf{value\_in\_use}} \\
              \cmidrule(l){1-1}\cmidrule(l){2-2}\cmidrule(l){3-3}
            }
            \tablehead{}
            \tabletail {
              \multicolumn{1}{l}{\textbf{2 Zeilen ausgew\"ahlt}} \\
            }
            \tablelasttail {
              \multicolumn{1}{l}{\textbf{2 Zeilen ausgew\"ahlt}} \\
            }
            \begin{mssql}
              \begin{supertabular}{lll}
                min server memory (MB) & 200 & 200 \\
                max server memory (MB) & 2048 & 2048 \\
              \end{supertabular}
            \end{mssql}
          \end{small}
        \end{center}
            \item Verschieben Sie jetzt das Backup Set des \identifier{bank}-Tablespaces
    auf das SBT-Gerät!

        \begin{lstlisting}[language=ms_sql, caption={Ändern der
        Servereigenschaften}, label=admin_03_loesung_06]
USE [master]
GO

EXEC xp_instance_regwrite N'HKEY_LOCAL_MACHINE', 
     N'Software\Microsoft\MSSQLServer\MSSQLServer', 
     N'AuditLevel', REG_DWORD, 0
GO
        \end{lstlisting}
        
            \item Löschen Sie das Backup Set des \identifier{bank}-Tablespaces von der Festplatte, so dass es nur noch auf SBT verfügbar ist!

        \begin{lstlisting}[language=ms_sql, caption={Ändern der
        Servereigenschaften}, label=admin_03_loesung_07]
USE [master]
GO

CREATE DATABASE [Bank 2014]
  CONTAINMENT = NONE
  ON  PRIMARY 
  ( NAME = N'bank_2014_primary_01', 
    FILENAME = N'D:\u01\bank_2014\data\bank_2014_primary_01.mdf', 
    SIZE = 10 MB, MAXSIZE = 500 MB, FILEGROWTH = 10 MB ), 
  FILEGROUP [CRM] 
  ( NAME = N'bank_2014_crm_01', 
    FILENAME = N'E:\u02\bank_2014\data\bank_2014_crm_01.ndf', 
    SIZE = 50 MB, MAXSIZE = 100 MB, FILEGROWTH = 10 MB ), 
  FILEGROUP [HR] 
  ( NAME = N'bank_2014_hr_01', 
    FILENAME = N'D:\u01\bank_2014\data\bank_2014_hr_01.ndf', 
    SIZE = 100 MB, MAXSIZE = 500 MB, FILEGROWTH = 20 MB ),
  ( NAME = N'bank_2014_hr_02', 
    FILENAME = N'D:\u01\bank_2014\data\bank_2014_hr_02.ndf', 
    SIZE = 100 MB, MAXSIZE = 500 MB, FILEGROWTH = 20 MB ), 
  FILEGROUP [KAM] 
  ( NAME = N'bank_2014_kam_01', 
    FILENAME = N'E:\u02\bank_2014\data\bank_2014_kam_01.ndf', 
    SIZE = 100 MB, MAXSIZE = 500 MB, FILEGROWTH = 100 MB ), 
        \end{lstlisting}
\clearpage
        \begin{lstlisting}[language=ms_sql]        
  FILEGROUP [MAIL] 
  ( NAME = N'bank_2014_mail_01', 
    FILENAME = N'D:\u01\bank_2014\data\bank_2014_mail_01.ndf', 
    SIZE = 500 MB, MAXSIZE = 1 GB, FILEGROWTH = 100 MB ),
  ( NAME = N'bank_2014_mail_02', 
    FILENAME = N'D:\u01\bank_2014\data\bank_2014_mail_02.ndf', 
    SIZE = 500 MB, MAXSIZE = 1 GB, FILEGROWTH = 100 MB ),
  ( NAME = N'bank_2014_mail_03', 
    FILENAME = N'D:\u01\bank_2014\data\bank_2014_mail_03.ndf', 
    SIZE = 500 MB, MAXSIZE = 1 GB, FILEGROWTH = 100 MB )
  LOG ON 
  ( NAME = N'Bank 2014_log', 
    FILENAME = N'F:\u03\bank_2014\log\bank_2014_log.ldf', 
    SIZE = 768 MB, MAXSIZE = 3 GB, FILEGROWTH = 768 MB )
GO
        \end{lstlisting}
        
            \item Der Tablespace \identifier{bank} liegt derzeit in unverschlüsselter Form vor. Dies muss zukünftig anders sein. Bereiten Sie einen Tablespace \identifier{bank\_encrypted} vor, der mittels \identifier{AES256} verschlüsselung gesichert ist und der die gleichen Dimensionen hat, wie der Originaltablespace \identifier{bank}! Rufen Sie alle notwendigen Informationen über den Tablespace \identifier{bank} aus dem Data Dictionary ab! Welche Views helfen Ihnen dabei?

        \begin{lstlisting}[language=ms_sql, caption={Hinzufügen einer
        Dateigruppe mit Datendatei}, label=admin_03_loesung_08]
USE [master]
GO

ALTER DATABASE [Bank 2014] 
ADD FILEGROUP [STAGING]
GO

ALTER DATABASE [Bank 2014] 
ADD FILE (NAME = N'bank_2014_staging', 
          FILENAME = N'E:\u02\bank_2014\data\bank_2014_staging_01.ndf', 
          SIZE = 800 MB, MAXSIZE = 2 GB, FILEGROWTH = 40 MB) 
TO FILEGROUP [STAGING]
GO
        
        \end{lstlisting}
\clearpage        
           \item Führen Sie mit Hilfe von RMAN ein Block-Media-Recovery durch, um die be\-schä\-dig\-ten Blöcke zu reparieren.

        \begin{lstlisting}[language=ms_sql, caption={Hinzufügen einer
        Datendatei zu einer bestehenden Dateigruppe}, label=admin_03_loesung_09]
USE [master]
GO

ALTER DATABASE [Bank 2014] 
ADD FILE (NAME = N'bank_2014_crm_02', 
          FILENAME = N'E:\u02\bank_2014\data\bank_2014_crm_02.ndf', 
          SIZE = 50 MB, MAXSIZE = 500 MB, FILEGROWTH = 10 MB )
TO FILEGROUP [CRM]
GO        
        \end{lstlisting}
        
      \item Deaktivieren Sie den gruppierten Index der Tabelle \identifier{girokonto}
und beantworten Sie die folgenden Fragen:

Können die Daten der Tabelle \identifier{girokonto} noch selektiert werden?

  \rule{0.94\textwidth}{0.5pt}

Welche Auswirkungen hatte das Deaktivieren des clustered Index auf den nicht
gruppierten Index der Spalte \identifier{guthaben}

  \rule{0.94\textwidth}{0.5pt}

  \rule{0.94\textwidth}{0.5pt}


      \begin{lstlisting}[language=ms_sql, caption={Ändern der
      Datendateieigenschaften}, label=admin_03_loesung_10]
USE [master]
GO

ALTER DATABASE [Bank 2014] 
MODIFY FILE ( NAME = N'bank_2014_hr_01', MAXSIZE = 1 GB )
GO

ALTER DATABASE [Bank 2014] 
MODIFY FILE ( NAME = N'bank_2014_hr_02', MAXSIZE = 1 GB )
GO

ALTER DATABASE [Bank 2014] 
MODIFY FILE ( NAME = N'bank_2014_kam_01', SIZE = 500 MB, MAXSIZE = 2 GB )
GO

ALTER DATABASE [Bank 2014]
MODIFY FILE ( NAME = N'bank_2014_primary_01', MAXSIZE = 768 MB )
GO
      \end{lstlisting}
\clearpage
      \item Schreiben Sie eine Abfrage, um zu überprüfen, ob die Datendatei
ordnungsgemäß angefügt wurde.

      \begin{lstlisting}[language=ms_sql, caption={Abfragen der
      Datendateieigenschaften}, label=admin_03_loesung_11]
USE [Bank 2014]
GO

SELECT name, size * 8 / 1024 AS Size, 
       growth * 8 / 1024 AS Growth, 
       max_size * 8 / 1024 AS Max_Size
FROM   sys.database_files
      \end{lstlisting}
      
            \item Erstellen Sie aus den aktuellen Einstellungen Ihrer Instanz ein neues SPFile names \newline \oscommand{/home/oracle/spfileorcl.ora}.

      \begin{lstlisting}[language=ms_sql, caption={Verschieben einer
      Datendatei}, label=admin_03_loesung_12]
USE [master]
GO

ALTER DATABASE [Bank 2014]
SET OFFLINE;

-- Move Datafile on filesystem
-- Change filesystemrights

ALTER DATABASE [Bank 2014]
MODIFY FILE ( NAME = N'bank_2014_kam_01',
FILENAME = N'D:\u01\bank_2014\data\bank_2014_kam_01.ndf' )
GO

ALTER DATABASE [Bank 2014]
SET ONLINE;     
      \end{lstlisting}

      \item Machen Sie die Dateigruppe \identifier{CRM} zur Standarddateigruppe
der Datenbank \identifier{Bank 2014}.

      \begin{lstlisting}[language=ms_sql, caption={Ändern der
      Standarddateigruppe}, label=admin_03_loesung_12]
USE [Bank 2014]
GO

IF NOT EXISTS (
   SELECT name 
   FROM sys.filegroups 
   WHERE is_default=1 AND name = N'CRM')
     ALTER DATABASE [Bank 2014]
     MODIFY FILEGROUP [CRM] DEFAULT
GO
      \end{lstlisting}
            
      \item Konfigurieren Sie für die Datenbankoption \identifier{quoted\_identifier}
den Wert \identifier{true}.

      \begin{lstlisting}[language=ms_sql, caption={Die Datenbankoption
      quoted\_identifier ändern}, label=admin_03_loesung_13]
USE [master]
GO

ALTER DATABASE [Bank 2014] 
SET QUOTED_IDENTIFIER ON 
WITH NO_WAIT
GO 
      \end{lstlisting}
      
      \item Ermitteln Sie welchen Wert die Datenbankoption \identifier{auto\_close}
hat und recherchieren Sie, welche Bedeutung diese Option hat bzw. was sie
bewirkt.

      
      Die Datenbankoption \identifier{auto\_close} bewirkt, dass die Datenbank
      geschlossen wird, sobald sich der letzte noch aktive Nutzer abgemeldet
      hat. Dies hat zur Folge, dass alle im Arbeitspeicher befindlichen Teile
      der Datenbank von dort entfernt werden, wodurch die Performance dieser
      Datenbank sehr starkt negativ beeinflusst wird.
      
          \item Führen Sie das Skript labs/lab\_delete\_backups.sql aus. Dieses Skript wird eine will\-kür\-liche Anzahl Ihrer Backups löschen.

      \begin{lstlisting}[language=ms_sql, caption={Verschieben der
      Systemdatenbank tempdb}, label=admin_03_loesung_14]
USE [master]
GO

ALTER DATABASE [tempdb]
MODIFY FILE ( NAME = N'tempdev',
              FILENAME = N'G:\u04\tempdb\data\tempdb.mdf' )
GO

ALTER DATABASE [tempdb]
MODIFY FILE ( NAME = N'templog',
              FILENAME = N'F:\u03\tempdb\log\templog.ldf' )
GO

-- Shutdown the instance

-- Move file on filesystem to their new locations

-- Change filesystemrights

-- Startup the instance
      \end{lstlisting}
\clearpage      
          \item Prüfen Sie im RMAN welche Backups nun nicht mehr zur Verfügung stehen.

      \bild{Verschieben des errorlog}{move_error_log}{0.48}
      
      \item Beim Anlegen der Datenbank ist Ihnen ein Fehler unterlaufen. Statt
die Datenbank \identifier{Bank\_2014} zu nennen, haben Sie sie
\identifier{Bank 2014} genannt. Recherchieren Sie, wie eine
Datenbank umbenannt werden kann und ersetzten Sie das Leerzeichen im
Datenbanknamen durch einen Unterstrich!

      \begin{lstlisting}[language=ms_sql, caption={Umbennen einer Datenbank},
      label=admin_03_loesung_15]
USE [master]
GO

ALTER DATABASE [Bank 2014]
Modify Name = Bank_2014
GO
      
      \end{lstlisting}
    \end{enumerate}
