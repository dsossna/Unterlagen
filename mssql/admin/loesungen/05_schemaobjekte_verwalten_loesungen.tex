\clearpage
    \section{Lösungen - Schemaobjekte verwalten}
      Benutzen Sie für die Durchführung der Übungen die Datenbank
      \identifier{Bank\_2014}.
      \begin{enumerate}
        %Erstellen Sie die Tabelle \identifier{bank} gemäß Ihrem ER-Modell. 
        %Machen Sie das Attribut \identifier{bank\_id} zum Primärschlüssel und
        % achten sie darauf, dass die Tabelle als Heap erstellt wird.
        \item Bereiten Sie Ihre SQL Server-Instanz auf die Benutzung von FileStream vor.
Der Bezeichner für die Filestream-Freigabe lautet: \identifier{mssqlserver}. Der
Bezeichner für das Verzeichnis der Datenbank \identifier{bank\_2014 }lautet:
\identifier{bank\_2014}. Aktivieren Sie den nichttransaktionalen Zugriff auf den
Filestream.
        \begin{lstlisting}[language=ms_sql]
USE [Bank_2014]
GO

CREATE TABLE bank (
  Bank_ID         NUMERIC NOT NULL,
  BIC             VARCHAR(11) NOT NULL,
  Name            VARCHAR(50) NOT NULL,
  Strasse         VARCHAR(50),
  Hausnummer      VARCHAR(20),
  PLZ             CHAR(5),
  Ort             VARCHAR(20),
  Rating          VARCHAR(3),
  CONSTRAINT bank_pk PRIMARY KEY NONCLUSTERED (Bank_ID)
)
ON KAM;
GO
        \end{lstlisting}
        
        %Ist es sinnvoll, die Tabelle \identifier{bank} als Heap anzulegen?
        %Begründen Sie Ihre Aussage!
            \item Registrieren Sie Ihre Datenbank in Ihrem neuen Recovery Katalog!

        
        Die Nutzung einer Tabelle in Form eines Heaps, anstatt eines gruppierten
        Indizes, kann sinnvoll sein, wenn die Tabelle sehr klein ist und somit
        das Lesen der gesamten Tabelle kostengünstiger ist, als das Durchsuchen
        eines clustered Index. 
        
        In der Praxis kann es hierfür noch weitere Gründe geben, die allerdings
        den Rahmen dieser Unterrichtsunterlage deutlich sprengen würden.

\clearpage
        %Legen Sie ein \UNIQUE-Constraint auf die Spalte \identifier{bic} der
        %Tabelle \identifier{bank}. Die Tabelle muss weiterhin in Form eines
        %Heaps gespeichert bleiben.
            \item Wechseln Sie den Undo-Tablespace zurück zum Original Undo-Tablespace und löschen Sie \identifier{undotbs02} und \identifier{undotbs03}.

        \begin{lstlisting}[language=ms_sql]
USE [Bank_2014]
GO

ALTER TABLE bank
ADD CONSTRAINT bic_uk UNIQUE NONCLUSTERED (BIC);
GO
        \end{lstlisting}
        
        %Fügen Sie Ihrer Datenbank die Dateigruppe \identifier{in\_memory} hinzu
        %und bereiten Sie alles vor, um eine SOT darin anzulegen.
            \item Erstellen Sie ein unkomprimiertes Backup Set des Tablespaces
    \identifier{bank}! Das Backup Set soll in der FRA (Fast Recovery Area)
    abgelegt werden!

        \begin{lstlisting}[language=ms_sql]
USE [Bank_2014]
GO

-- Memory optimized filegroup anlegen
ALTER DATABASE [Bank_2014] 
ADD FILEGROUP [in_memory]
CONTAINS MEMORY_OPTIMIZED_DATA; 
GO

-- Es muss eine Filestream Datendatei geben,
-- um Tabellen anlegen zu koennen.
ALTER DATABASE [Bank_2014]
ADD FILE (
  NAME =     'bank_2014_in_memory_01',
  FILENAME = 'D:\u01\bank_2014\filestream\in_memory'
)
TO FILEGROUP [in_memory]
GO
        \end{lstlisting}
\clearpage        
        %Legen Sie die Tabelle \identifier{buchung} gemäß Ihrem ER-Modell als SOT
        %an und fügen Sie zwischen die beiden Spalten
        %\identifier{buchungsdatum} und \identifier{konto\_id} die Spalte
        % \identifier{buchungstext} VARCHAR(200) ein! Legen Sie auf die
        %Spalte \identifier{buchungsdatum} einen Index. Legen Sie auch
        %auf die Spalte \identifier{buchungstext} einen Index.
        \item Legen Sie die Tabelle \identifier{buchung} gemäß Ihrem ER-Modell als SOT
an und fügen Sie zwischen die beiden Spalten \identifier{buchungsdatum} und
\identifier{konto\_id} die Spalte \identifier{buchungstext} VARCHAR(200) ein!
Legen Sie auf die Spalte \identifier{buchungsdatum} einen Index. Legen Sie auch
auf die Spalte \identifier{buchungstext} einen Index.
        
        \begin{lstlisting}[language=ms_sql]
USE [Bank_2014]
GO

CREATE TABLE Buchung (
  Buchungs_ID       NUMERIC NOT NULL,
  Betrag            NUMERIC(12,2),
  Buchungsdatum     DATETIME2 NOT NULL,
  Buchungstext      VARCHAR(200) COLLATE LATIN1_GENERAL_BIN2 NOT NULL,
  Konto_ID          NUMERIC,
  Transaktions_ID   NUMERIC,
  CONSTRAINT Buchung_pk PRIMARY KEY NONCLUSTERED (Buchungs_ID),
  INDEX      idx_Buchungsdatum NONCLUSTERED (Buchungsdatum),
  INDEX      idx_Buchungstext NONCLUSTERED (Buchungstext)
)
WITH (
  MEMORY_OPTIMIZED = ON
)
GO
        \end{lstlisting}
        %Erstellen Sie die Tabelle \identifier{mitarbeiter} gemäß Ihrem
        % ER-Modell und legen Sie ein \UNIQUE-Constraint auf die Spalte \identifier{sozversnr}. Die
        %Sortierung des Indexes, der zu diesem Constraint gehört soll absteigend
        %sein.
              \item Starten und öffnen Sie Ihre Datenbank.

        \begin{lstlisting}[language=ms_sql]
USE [Bank_2014]
GO

CREATE TABLE Eigenkunde (
  Kunden_ID            NUMERIC NOT NULL,
  Geburtsdatum         DATETIME2,
  Strasse              VARCHAR(50),
  Hausnummer           VARCHAR(15),
  PLZ                  CHAR(5),
  Ort                  VARCHAR(20),
  PersonalausweisNr    VARCHAR(9),
  Ausstellungsdatum    DATETIME2,
  Ablaufdatum          DATETIME2,
  Staatsangehoerigkeit VARCHAR(20),
  Geburtsort           VARCHAR(50),
  Telefon              VARCHAR(15),
  Emailadresse         VARCHAR(25),
  CONSTRAINT eigenkunde_pk PRIMARY KEY (Kunden_ID),
  CONSTRAINT personalausweisNr_uk UNIQUE (PersonalausweisNr DESC)
)
ON KAM;
GO
        \end{lstlisting}
      
        %Erstellen Sie die Tabelle \identifier{mitarbeiter} gemäß Ihrem
        % ER-Modell und legen Sie ein \UNIQUE-Constraint auf die Spalte
        %\identifier{sozversnr}. Schließen Sie die Spalten \identifier{nachname}
        % und \identifier{vorname} in den Index als Nichtschlüsselspalten in.
            \item Konfigurieren Sie die Benennungsmethode Directory Naming für Ihre
    Datenbank. Der Directory Server ist ein OID und wird unter der IP-Adresse
    192.168.111.250 zufinden sein (Ports: 389 und 636). Der LDAP-Context
    lautet: \enquote{cn=OracleContext}. Nutzen Sie für diese Aufgabe den
    Oracle Net Configuration Assistant (\oscommand{netca} - Konfigurieren von
    Directoy-Verwendung).

        \begin{lstlisting}[language=ms_sql]
USE [Bank_2014]
GO

CREATE TABLE Mitarbeiter (
  Mitarbeiter_ID   NUMERIC NOT NULL,
  Vorname          VARCHAR(30),
  Nachname         VARCHAR(35),
  Vorgesetzter_ID  NUMERIC,
  Bankfiliale_ID   NUMERIC,
  Geburtsdatum     DATETIME2,
  SozVersNr        VARCHAR(20),
  Gehalt           NUMERIC(12,2),
  Strasse          VARCHAR(50),
  Hausnummer       VARCHAR(20),
  PLZ              CHAR(5),
  Ort              VARCHAR(20),
  Provision        NUMERIC,
  CONSTRAINT Mitarbeiter_pk PRIMARY KEY (Mitarbeiter_ID)
)
ON KAM;
GO

CREATE UNIQUE NONCLUSTERED INDEX idx_SozVersNr
ON dbo.Mitarbeiter (SozVersNr)
INCLUDE (Vorname, Nachname);
GO
        \end{lstlisting}

        %Legen Sie einen nicht gruppierten Index auf die Tabelle
        %\identifier{eigenkunde} (Spalte \identifier{ablaufdatum}). Der Index
        %soll nur die Zeilen enthalten, die einen Personalausweis zeigen, der vor dem 01.01.2016
        %abgelaufen ist. Schließen Sie die Spalte \identifier{personalausweisnr}
        % als Nichtschlüsselspalte in den Index ein.
            \item Welche Kosten hat das Statement verursacht und wie viel Zeit wurde für die Ausführung benötigt?

        \begin{lstlisting}[language=ms_sql]
USE [Bank_2014]
GO

CREATE NONCLUSTERED INDEX idx_ablaufdatum
ON Eigenkunde (Ablaufdatum)
INCLUDE (PersonalausweisNr)
WHERE  Ablaufdatum < CONVERT(DATETIME2, '2016-01-01');
GO
        \end{lstlisting}
\clearpage
        %Ermitteln Sie den Prozentsatz der logischen Indexfragmentierung der
        % beiden Indizes \identifier{mitarbeiter\_pk} und \identifier{girokonto\_pk}.
            \item Verschieben Sie jetzt das Backup Set des \identifier{bank}-Tablespaces
    auf das SBT-Gerät!

        \begin{lstlisting}[language=ms_sql]
USE [Bank_2014]
GO

SELECT a.index_id, name, avg_fragmentation_in_percent
FROM   sys.dm_db_index_physical_stats(DB_ID(N'Bank'), 
       OBJECT_ID('dbo.mitarbeiter'), NULL, NULL, NULL) AS a
       JOIN sys.indexes AS b
     ON (a.object_id = b.object_id AND a.index_id = b.index_id);
GO

/* 1 mitarbeiter_pk 80 -> REBUILD */ 

SELECT a.index_id, name, avg_fragmentation_in_percent
FROM   sys.dm_db_index_physical_stats(DB_ID(N'Bank'), 
       OBJECT_ID('dbo.girokonto'), NULL, NULL, NULL) AS a
       JOIN sys.indexes AS b
     ON (a.object_id = b.object_id AND a.index_id = b.index_id);
GO

/* 1 girokonto_pk 25 -> REORGANIZE */
        \end{lstlisting}

        %Ergreifen Sie für jeden der beiden Indizes die geeignete Maßnahme, um
        %deren Fragmentierung zu reduzieren.
            \item Löschen Sie das Backup Set des \identifier{bank}-Tablespaces von der Festplatte, so dass es nur noch auf SBT verfügbar ist!

        \begin{lstlisting}[language=ms_sql]
USE [Bank_2014]
GO

/* 1 mitarbeiter_pk 80 -> REBUILD */ 

ALTER INDEX mitarbeiter_pk 
ON Mitarbeiter
REBUILD;

/* 1 girokonto_pk 25 -> REORGANIZE */

ALTER INDEX girokonto_pk
ON Girokonto
REORGANIZE;
        \end{lstlisting}
\clearpage
        %Erstellen Sie die Tabelle \identifier{girokonto} gemäß Ihrem ER-Modell.
        %Legen Sie den Primärschlüssel dieser Tabelle mit einem clustered Index
        % an.
            \item Der Tablespace \identifier{bank} liegt derzeit in unverschlüsselter Form vor. Dies muss zukünftig anders sein. Bereiten Sie einen Tablespace \identifier{bank\_encrypted} vor, der mittels \identifier{AES256} verschlüsselung gesichert ist und der die gleichen Dimensionen hat, wie der Originaltablespace \identifier{bank}! Rufen Sie alle notwendigen Informationen über den Tablespace \identifier{bank} aus dem Data Dictionary ab! Welche Views helfen Ihnen dabei?

        \begin{lstlisting}[language=ms_sql]
USE [Bank_2014]
GO

CREATE TABLE Girokonto (
  Konto_ID       NUMERIC NOT NULL,
  Gebuehr        NUMERIC(12,2),
  Guthaben       NUMERIC(12,2),
  Habenzins      NUMERIC(5,2),
  Kreditrahmen   NUMERIC(12,2),
  CONSTRAINT girokonto_pk PRIMARY KEY CLUSTERED (Konto_ID)
)
ON KAM;
        \end{lstlisting}

        %Erstellen Sie einen nicht gruppierten Index aufl der Spalte
        %\identifier{guthaben} der Tabelle \identifier{girokonto}.
           \item Führen Sie mit Hilfe von RMAN ein Block-Media-Recovery durch, um die be\-schä\-dig\-ten Blöcke zu reparieren.

        \begin{lstlisting}[language=ms_sql]
USE [Bank_2014]
GO

CREATE NONCLUSTERED INDEX idx_guthaben
ON Girokonto (Guthaben);
GO
        \end{lstlisting}

        %Deaktivieren Sie den gruppierten Index der Tabelle
        % \identifier{girokonto} und beantworten Sie die folgenden Fragen:
        \item Deaktivieren Sie den gruppierten Index der Tabelle \identifier{girokonto}
und beantworten Sie die folgenden Fragen:

Können die Daten der Tabelle \identifier{girokonto} noch selektiert werden?

  \rule{0.94\textwidth}{0.5pt}

Welche Auswirkungen hatte das Deaktivieren des clustered Index auf den nicht
gruppierten Index der Spalte \identifier{guthaben}

  \rule{0.94\textwidth}{0.5pt}

  \rule{0.94\textwidth}{0.5pt}


        \begin{lstlisting}[language=ms_sql]
USE [Bank_2014]
GO

ALTER INDEX girokonto_pk
ON Girokonto
DISABLE;
        \end{lstlisting}
\clearpage
        %Suchen Sie einen Weg, den gruppierten und den nicht gruppierten Index
        % der Tabelle \identifier{girokonto} in einem Zug zu reaktivieren.
        \item Schreiben Sie eine Abfrage, um zu überprüfen, ob die Datendatei
ordnungsgemäß angefügt wurde.

        \begin{lstlisting}[language=ms_sql]
USE [Bank_2014]
GO

ALTER INDEX ALL
ON Girokonto
REBUILD;
        \end{lstlisting}

        %Lesen Sie den Artikel \parencite{utilUDPaC} und notieren Sie sich
        %mindestens drei sinnvolle Fragen!
              \item Erstellen Sie aus den aktuellen Einstellungen Ihrer Instanz ein neues SPFile names \newline \oscommand{/home/oracle/spfileorcl.ora}.

      \end{enumerate}