\clearpage
    \section{Lösungen - Schemaobjekte verwalten}
      Benutzen Sie für die Durchführung der Übungen die Datenbank
      \identifier{Bank\_2014}.
      \begin{enumerate}
        %Erstellen Sie die Tabelle \identifier{bank} gemäß Ihrem ER-Modell. 
        %Machen Sie das Attribut \identifier{bank\_id} zum Primärschlüssel und
        % achten sie darauf, dass die Tabelle als Heap erstellt wird.
            \item Führen Sie ein Recovery bei Verlust einer Redo Log Datei durch!
      \begin{enumerate}
        \item Starten Sie das Skript \oscommand{lab\_delete\_redolog\_member.sql}! Es löscht einen beliebigen Member einer Ihrer Redo Log Gruppen.
          \begin{lstlisting}[language=terminal]
SQL> @/home/oracle/labs/lab_delete_redolog_member.sql
          \end{lstlisting}
        \item Die Datenbank läuft weiterhin normal und es liegen keinerlei Probleme vor. Öffnen Sie die Alert Log Datei, um herauszufinden welcher Redo Log Member gelöscht wurde!
        \item Führen Sie geeignete Maßnahmen zur Problembehebung durch!
      \end{enumerate}

        \begin{lstlisting}[language=ms_sql]
USE [Bank_2014]
GO

CREATE TABLE bank (
  Bank_ID         NUMERIC NOT NULL,
  BIC             VARCHAR(11) NOT NULL,
  Name            VARCHAR(50) NOT NULL,
  Strasse         VARCHAR(50),
  Hausnummer      VARCHAR(20),
  PLZ             CHAR(5),
  Ort             VARCHAR(20),
  Rating          VARCHAR(3),
  CONSTRAINT bank_pk PRIMARY KEY NONCLUSTERED (Bank_ID)
)
ON KAM;
GO
        \end{lstlisting}
        
        %Ist es sinnvoll, die Tabelle \identifier{bank} als Heap anzulegen?
        %Begründen Sie Ihre Aussage!
        \item Nehmen Sie Ihren Übungsserver in die Domäne
\identifier{MS-C-IX-04.FUS} auf. Achten Sie darauf, dass das
Computerkonto Ihres Datenbankservers in der 
Organisationseinheit \oscommand{IT-SysAdmin\textbackslash
EXER\textbackslash Serverxx\textbackslash Computers} abgelegt wird, bzw. verschieben Sie es dorthin.

        
        Die Nutzung einer Tabelle in Form eines Heaps, anstatt eines gruppierten
        Indizes, kann sinnvoll sein, wenn die Tabelle sehr klein ist und somit
        das Lesen der gesamten Tabelle kostengünstiger ist, als das Durchsuchen
        eines clustered Index. 
        
        In der Praxis kann es hierfür noch weitere Gründe geben, die allerdings
        den Rahmen dieser Unterrichtsunterlage deutlich sprengen würden.

\clearpage
        %Legen Sie ein \UNIQUE-Constraint auf die Spalte \identifier{bic} der
        %Tabelle \identifier{bank}. Die Tabelle muss weiterhin in Form eines
        %Heaps gespeichert bleiben.
              \item Bringen Sie Ihre Datenbank in die Mount-Phase.

        \begin{lstlisting}[language=ms_sql]
USE [Bank_2014]
GO

ALTER TABLE bank
ADD CONSTRAINT bic_uk UNIQUE NONCLUSTERED (BIC);
GO
        \end{lstlisting}
        
        %Fügen Sie Ihrer Datenbank die Dateigruppe \identifier{in\_memory} hinzu
        %und bereiten Sie alles vor, um eine SOT darin anzulegen.
            \item Welchen Default Tablespace benutzt \identifier{bob} und aus welcher View können Sie diese Angabe entnehmen?

        \begin{lstlisting}[language=ms_sql]
USE [Bank_2014]
GO

-- Memory optimized filegroup anlegen
ALTER DATABASE [Bank_2014] 
ADD FILEGROUP [in_memory]
CONTAINS MEMORY_OPTIMIZED_DATA; 
GO

-- Es muss eine Filestream Datendatei geben,
-- um Tabellen anlegen zu koennen.
ALTER DATABASE [Bank_2014]
ADD FILE (
  NAME =     'bank_2014_in_memory_01',
  FILENAME = 'D:\u01\bank_2014\filestream\in_memory'
)
TO FILEGROUP [in_memory]
GO
        \end{lstlisting}
\clearpage        
        %Legen Sie die Tabelle \identifier{buchung} gemäß Ihrem ER-Modell als SOT
        %an und fügen Sie zwischen die beiden Spalten
        %\identifier{buchungsdatum} und \identifier{konto\_id} die Spalte
        % \identifier{buchungstext} VARCHAR(200) ein! Legen Sie auf die
        %Spalte \identifier{buchungsdatum} einen Index. Legen Sie auch
        %auf die Spalte \identifier{buchungstext} einen Index.
            \item Prüfen Sie den Audittrail auf neue Informationen! Welche Tätigkeiten wurden an der auditierten Tabelle ausgeführt?
        
        \begin{lstlisting}[language=ms_sql]
USE [Bank_2014]
GO

CREATE TABLE Buchung (
  Buchungs_ID       NUMERIC NOT NULL,
  Betrag            NUMERIC(12,2),
  Buchungsdatum     DATETIME2 NOT NULL,
  Buchungstext      VARCHAR(200) COLLATE LATIN1_GENERAL_BIN2 NOT NULL,
  Konto_ID          NUMERIC,
  Transaktions_ID   NUMERIC,
  CONSTRAINT Buchung_pk PRIMARY KEY NONCLUSTERED (Buchungs_ID),
  INDEX      idx_Buchungsdatum NONCLUSTERED (Buchungsdatum),
  INDEX      idx_Buchungstext NONCLUSTERED (Buchungstext)
)
WITH (
  MEMORY_OPTIMIZED = ON
)
GO
        \end{lstlisting}
        %Erstellen Sie die Tabelle \identifier{mitarbeiter} gemäß Ihrem
        % ER-Modell und legen Sie ein \UNIQUE-Constraint auf die Spalte \identifier{sozversnr}. Die
        %Sortierung des Indexes, der zu diesem Constraint gehört soll absteigend
        %sein.
        \item Erstellen Sie die Tabelle \identifier{eigenkunde}, gemäß Ihrem ER-Modell,
in der Dateigruppe \identifier{kam} und legen Sie ein \UNIQUE-Constraint auf
die Spalte \identifier{personalausweisnr}. Die Sortierung des Indexes, der zu diesem
Constraint gehört soll absteigend sein.
        \begin{lstlisting}[language=ms_sql]
USE [Bank_2014]
GO

CREATE TABLE Eigenkunde (
  Kunden_ID            NUMERIC NOT NULL,
  Geburtsdatum         DATETIME2,
  Strasse              VARCHAR(50),
  Hausnummer           VARCHAR(15),
  PLZ                  CHAR(5),
  Ort                  VARCHAR(20),
  PersonalausweisNr    VARCHAR(9),
  Ausstellungsdatum    DATETIME2,
  Ablaufdatum          DATETIME2,
  Staatsangehoerigkeit VARCHAR(20),
  Geburtsort           VARCHAR(50),
  Telefon              VARCHAR(15),
  Emailadresse         VARCHAR(25),
  CONSTRAINT eigenkunde_pk PRIMARY KEY (Kunden_ID),
  CONSTRAINT personalausweisNr_uk UNIQUE (PersonalausweisNr DESC)
)
ON KAM;
GO
        \end{lstlisting}
      
        %Erstellen Sie die Tabelle \identifier{mitarbeiter} gemäß Ihrem
        % ER-Modell und legen Sie ein \UNIQUE-Constraint auf die Spalte
        %\identifier{sozversnr}. Schließen Sie die Spalten \identifier{nachname}
        % und \identifier{vorname} in den Index als Nichtschlüsselspalten in.
            \item Welche Tablespaces in Ihrer Datenbank sind Bigfile Tablespaces?

        \begin{lstlisting}[language=ms_sql]
USE [Bank_2014]
GO

CREATE TABLE Mitarbeiter (
  Mitarbeiter_ID   NUMERIC NOT NULL,
  Vorname          VARCHAR(30),
  Nachname         VARCHAR(35),
  Vorgesetzter_ID  NUMERIC,
  Bankfiliale_ID   NUMERIC,
  Geburtsdatum     DATETIME2,
  SozVersNr        VARCHAR(20),
  Gehalt           NUMERIC(12,2),
  Strasse          VARCHAR(50),
  Hausnummer       VARCHAR(20),
  PLZ              CHAR(5),
  Ort              VARCHAR(20),
  Provision        NUMERIC,
  CONSTRAINT Mitarbeiter_pk PRIMARY KEY (Mitarbeiter_ID)
)
ON KAM;
GO

CREATE UNIQUE NONCLUSTERED INDEX idx_SozVersNr
ON dbo.Mitarbeiter (SozVersNr)
INCLUDE (Vorname, Nachname);
GO
        \end{lstlisting}

        %Legen Sie einen nicht gruppierten Index auf die Tabelle
        %\identifier{eigenkunde} (Spalte \identifier{ablaufdatum}). Der Index
        %soll nur die Zeilen enthalten, die einen Personalausweis zeigen, der vor dem 01.01.2016
        %abgelaufen ist. Schließen Sie die Spalte \identifier{personalausweisnr}
        % als Nichtschlüsselspalte in den Index ein.
        \item Schreiben Sie eine SQL-Abfrage, die anzeigen, ob die Einstellungen
aus der vorangegangenen Aufgabe bereits wirksam geworden sind oder
nicht!

        \begin{lstlisting}[language=ms_sql]
USE [Bank_2014]
GO

CREATE NONCLUSTERED INDEX idx_ablaufdatum
ON Eigenkunde (Ablaufdatum)
INCLUDE (PersonalausweisNr)
WHERE  Ablaufdatum < CONVERT(DATETIME2, '2016-01-01');
GO
        \end{lstlisting}
\clearpage
        %Ermitteln Sie den Prozentsatz der logischen Indexfragmentierung der
        % beiden Indizes \identifier{mitarbeiter\_pk} und \identifier{girokonto\_pk}.
            \item Führen Sie das gleiche Statement nochmals aus und ermitteln Sie erneut die verursachten Kosten und die benötigte Ausführungszeit! Haben sich die Kosten verändert?

        \begin{lstlisting}[language=ms_sql]
USE [Bank_2014]
GO

SELECT a.index_id, name, avg_fragmentation_in_percent
FROM   sys.dm_db_index_physical_stats(DB_ID(N'Bank'), 
       OBJECT_ID('dbo.mitarbeiter'), NULL, NULL, NULL) AS a
       JOIN sys.indexes AS b
     ON (a.object_id = b.object_id AND a.index_id = b.index_id);
GO

/* 1 mitarbeiter_pk 80 -> REBUILD */ 

SELECT a.index_id, name, avg_fragmentation_in_percent
FROM   sys.dm_db_index_physical_stats(DB_ID(N'Bank'), 
       OBJECT_ID('dbo.girokonto'), NULL, NULL, NULL) AS a
       JOIN sys.indexes AS b
     ON (a.object_id = b.object_id AND a.index_id = b.index_id);
GO

/* 1 girokonto_pk 25 -> REORGANIZE */
        \end{lstlisting}

        %Ergreifen Sie für jeden der beiden Indizes die geeignete Maßnahme, um
        %deren Fragmentierung zu reduzieren.
            \item Verändern Sie die Datendatei \oscommand{uebungs\_ts02.dbf} so, dass sie auf bis zu 1 G Größe anwachsen kann.

        \begin{lstlisting}[language=ms_sql]
USE [Bank_2014]
GO

/* 1 mitarbeiter_pk 80 -> REBUILD */ 

ALTER INDEX mitarbeiter_pk 
ON Mitarbeiter
REBUILD;

/* 1 girokonto_pk 25 -> REORGANIZE */

ALTER INDEX girokonto_pk
ON Girokonto
REORGANIZE;
        \end{lstlisting}
\clearpage
        %Erstellen Sie die Tabelle \identifier{girokonto} gemäß Ihrem ER-Modell.
        %Legen Sie den Primärschlüssel dieser Tabelle mit einem clustered Index
        % an.
            \item Führen Sie das Skript \oscommand{/home/oracle/labs/lab\_fullts.sql}
    aus. Nach der Ausführung des Skripts existiert ein neuer Tablespace
    \identifier{fullts} und ein neuer Nutzer \identifier{alice}.

        \begin{lstlisting}[language=ms_sql]
USE [Bank_2014]
GO

CREATE TABLE Girokonto (
  Konto_ID       NUMERIC NOT NULL,
  Gebuehr        NUMERIC(12,2),
  Guthaben       NUMERIC(12,2),
  Habenzins      NUMERIC(5,2),
  Kreditrahmen   NUMERIC(12,2),
  CONSTRAINT girokonto_pk PRIMARY KEY CLUSTERED (Konto_ID)
)
ON KAM;
        \end{lstlisting}

        %Erstellen Sie einen nicht gruppierten Index aufl der Spalte
        %\identifier{guthaben} der Tabelle \identifier{girokonto}.
            \item Fügen Sie dem Profil \identifier{p\_clerk} den Parameter \parameter{PASSWORD\_GRACE\_TIME} mit dem Wert 5 Tage hinzu!

        \begin{lstlisting}[language=ms_sql]
USE [Bank_2014]
GO

CREATE NONCLUSTERED INDEX idx_guthaben
ON Girokonto (Guthaben);
GO
        \end{lstlisting}

        %Deaktivieren Sie den gruppierten Index der Tabelle
        % \identifier{girokonto} und beantworten Sie die folgenden Fragen:
        \item Nehmen Sie die im Folgenden beschriebenen Verändernungen an den
Datendateien Ihrer Datenbank \identifier{Bank 2014} vor.
\begin{center}
  \begin{small}
    \changefont{pcr}{m}{n}
    \tablefirsthead {
      \multicolumn{1}{c}{\textbf{Name}} &
      \multicolumn{1}{c}{\textbf{Größe}} &
      \multicolumn{1}{c}{\textbf{Wachstum}} &
      \multicolumn{1}{c}{\textbf{G Max.}} &
      \multicolumn{1}{c}{\textbf{Pfad}} \\
    }
    \tablehead{
    }
    \tabletail {
    }
    \tablelasttail {
    }
    \begin{supertabular}{lrrrl}
        bank\_2014\_primary\_01 &  10 M &  + 10 M & \color{red}{768 M} &
        D:\textbackslash u01\textbackslash bank\_2014\textbackslash data \\
        \hline
        bank\_2014\_hr\_01     & 100 M & \color{red}{+ 50 M}&
        \color{red}{1 G} & D:\textbackslash u01\textbackslash
        bank\_2014\textbackslash data \\
        bank\_2014\_hr\_02     & 100 M & \color{red}{+ 50 M}&
        \color{red}{1 G} & D:\textbackslash u01\textbackslash
        bank\_2014\textbackslash data \\
        \hline
        bank\_2014\_kam\_01    & \color{red}{500 M} & + 100 M & \color{red}{2 G} &
        E:\textbackslash u02\textbackslash bank\_2014\textbackslash data \\
    \end{supertabular}
  \end{small}
\end{center}

        \begin{lstlisting}[language=ms_sql]
USE [Bank_2014]
GO

ALTER INDEX girokonto_pk
ON Girokonto
DISABLE;
        \end{lstlisting}
\clearpage
        %Suchen Sie einen Weg, den gruppierten und den nicht gruppierten Index
        % der Tabelle \identifier{girokonto} in einem Zug zu reaktivieren.
        \item Schreiben Sie eine Abfrage, um zu überprüfen, ob die Datendatei
ordnungsgemäß angefügt wurde.

        \begin{lstlisting}[language=ms_sql]
USE [Bank_2014]
GO

ALTER INDEX ALL
ON Girokonto
REBUILD;
        \end{lstlisting}

        %Lesen Sie den Artikel \parencite{utilUDPaC} und notieren Sie sich
        %mindestens drei sinnvolle Fragen!
              \item Erstellen Sie aus den aktuellen Einstellungen Ihrer Instanz ein neues SPFile names \oscommand{/home/oracle/spfileorcl.ora}.

      \end{enumerate}