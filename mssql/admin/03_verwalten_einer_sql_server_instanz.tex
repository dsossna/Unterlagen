  \chapter{Verwalten einer SQL Server Instanz}
  \chaptertoc{}
  \cleardoubleevenpage

    \section{Die Windows Dienste des SQL Servers}
      Der Microsoft SQL Server ist ein Produkt, dass aus einer Vielzahl
      verschiedener Windowsdienste besteht. Zur Kontrolle und Konfiguration dieser
      Dienste wird der \enquote{SQL Server Configuration Manager} mitgeliefert.
      \begin{merke}
        Die Dienste des SQL Servers sollten immer nur mit Hilfe des SQL Server
        Configuration Managers verwaltet werden!
      \end{merke}
      \bild{Die Windows\-dienste des SQL Server 2014}{sql_server_services}{0.33}
      \subsection{SQL Server}
        Der \enquote{SQL Server}-Dienst ist der wichtigste von allen. Von ihm
        wird die sogenannte \enquote{Database Engine}, der
        Datenbankdienst, zur Verfügunggestellt. Mit seiner Hilfe werden alle
        Dateiarten verwaltet, die eine SQL Server Datenbank besitzt, er führt
        Transakt-SQL-Statements aus, überwacht die Einhaltung der Referentiellen
        Integrität und bewahrt somit die Konsistenz der Daten.
        \subsubsection{Namensgebung}
          Wenn es nur eine SQL Server Instanz auf einem Server gibt, trägt dieser
          Dienst den Namen \identifier{SQL Server (MSSQLSERVER)}. Im Falle einer
          benannten Instanz wird der Instanzname mitangegeben: \identifier{SQL Server
          (MSSQLSERVER\$CRM)}.
        \subsubsection{Startparameter}
          Hinter dem SQL Server-Dienst verbirgt sich die ausführbare Datei
          \oscommand{sqlserv.exe}. Diese befindet sich, wie die meisten
          \oscommand{.EXE} Dateien im Ordner \oscommand{BINN}. Zu diesem
          Programm gibt es eine Reihe von Startparametern. Die folgende Tabelle zeigt nur die
          wichtigsten Parameter. Eine vollständige Auflistung kann der MSDN
          entnommen werden.
          \begin{literaturinternet}
            \item \cite{ms162819}
          \end{literaturinternet}
          \begin{center}
            \begin{small}
            \tablefirsthead {
              \hline
            }
            \tablehead{
            }
            \tabletail {
              \hline
            }
            \tablelasttail {
              \hline
            }
              \begin{supertabular}{|c|p{10cm}|}
                \textbf{-d} & Gibt den vollständigen Dateipfad zur
                Master-Datenbank an. Es dürfen sich keine Leerzeichen zwischen
                dem Paramter und dem Dateipfad befinden. Sollte dieser Parameter
                nicht angegeben werden, benutzt SQL Server die Angaben aus der
                Windows-Registry. \\
                \hline
                \textbf{-l} & Gibt den vollständigen Dateipfad zur Log-Datei der
                Master-Datenbank an. Es darf sich kein Leerzeichen zwischen dem
                Paramter und dem Dateipfad befinden. \\
                \hline
                \textbf{-e} & Dieser Parameter gibt den vollständigen Dateipfad
                f\"ur das Fehler\-protokoll der SQL Server-Instanz an. Sollte
                dieser Parameter nicht angegeben worden sein, so wird f\"ur das Fehlerprotokoll der
                Standardinstanz der Dateiname \oscommand{C:\textbackslash
                Programme\textbackslash Microsoft SQL Server\textbackslash
                MSSQL\textbackslash Log\textbackslash Errorlog} benutzt. Bei
                einer benannten Instanz wird das Protokoll im Ordner \oscommand{C:\textbackslash
                Programme\textbackslash Microsoft SQL Server\textbackslash
                MSSQL\$\textit{Instanzname}\textbackslash Log\textbackslash
                Errorlog} gespeichert.\\
                \hline
                \textbf{-f} & Ermöglicht den Start des SQL Servers mit
                Minimalkonfiguration. Dieser Parameter sollte nur im
                Fehlerfall zum Einsatz kommen, falls der SQL Server aufgrund
                einer fehlerhaften Einstellung nicht mehr startet.\\
                \hline
                \textbf{-m} & Startet den SQL Server im Einzelbenutzermodus.
                Verwenden Sie diesen Parameter, wenn es Probleme mit einer der
                Systemdatenbanken gibt, z. B. die Master-Datenbank muss
                wiederhergestellt werden.\\
                \hline
                \textbf{-T} & Aktiviert die Ablaufverfolgung mittels
                Trace-Flags. Trace-Flags verändern das Verhalten des SQL
                Servers und dienen zu sehr unterschiedlichen Zwecken, meist
                aber zur Gewinnung zusätzlicher Informationen.\\
              \end{supertabular}
            \end{small}
          \end{center}
        \subsubsection{Startmodus und Recovery-Startmodus auswählen}
          f\"ur den SQL Server-Dienst werden drei verschiedene Startmodi
          bereitgestellt.
          \begin{itemize}
              \item \textbf{Automatisch}: Der Dienst wird mit dem Betriebssystem
              gestartet
              \item \textbf{Manuell}: Der Dienst muss manuell, durch den
              Administrator, gestartet werden
              \item \textbf{Deaktiviert}: Der Dienst wird nicht gestartet. Ein
              manuelles Starten ist ebenfalls nicht möglich.
          \end{itemize}
          Nach der Installation der SQL Server-Software ist der
          SQL Server-Dienst so konfiguriert, dass er automatisch mit dem
          Betriebssystem startet. Soll der Dienst beispielsweise deaktiviert
          oder aus irgendeinem Grund nicht automatisch gestartet werden, kann
          man den Startmodus im SQL Server Configuration Manager verändern.
\clearpage
          \bild{Den Start\-modus des SQL Server-Dienstes
          verändern}{changing_startmode_of_sql_server}{0.33}
          Sollte der SQL Server-Dienst einmal ausfallen/abstürzen, kann über die
          Windows Dienstekonsole eine Notfallaktion geplant werden. Auf der
          Registerkarte \enquote{Wiederherstellung} können f\"ur jeden Dienst
          Aktionen angegeben werden, die dann ausgeführt werden, wenn der Dienst
          das erste mal bzw. das zweite mal oder ein wiederholtes mal abgestürzt
          ist.
          
          Es empfiehlt sich, die Option \identifier{Erster Fehler} auf den
          Wert \enquote{Dienst neu starten} einzustellen, so dass der SQL Server
          Dienst neugestartet wird. Zusätzlich dazu sollten die beiden Optionen
          \identifier{Fehlerzähler nach n Tagen zurücksetzen} und
          \identifier{Dienst nach n Minute(n) neu starten} auf den Wert 1
          eingestellt werden.
          \bild{Eine Wieder\-herstellungs\-aktion
          planen}{recovery_startmode}{0.42}
      \subsection{SQL Server-Agent}
        Der SQL Server Agent wird durch die Datei \oscommand{SQLAGENT.EXE}
        bereitgestellt. Er ist ein Dienst, der dem Administrator die
        Automatisierung von Aufgaben innerhalb der Datenbank erlaubt. Er kann z.
        B. Backups zeitgesteuert ablaufen lassen, auf Warnungen des SQL Servers
        reagieren, Nachrichten an den Admin versenden und ähnliches.
        
        Dieser Dienst ist immer 1:1 mit einer SQL Server-Instanz verbunden,
        die ihm mittels des Initialisierungsparameters \oscommand{-i
        \textit{Instanzname}} mitgeteilt wird. Im Falle einer Standardinstanz
        trägt er den Namen \identifier{SQL Server-Agent}.
        Bei einer benannten Instanz ist der Instanzname mit angehängt:
        \identifier{SQL Server-Agent\$CRM}.
      \subsection{SQL Server-Browser}
        Der SQL Server-Browser hat die Aufgabe, nach eingehenden
        Ressourcenanforderungen von Datenbank-Clients zu lauschen. Er stellt den
        Clients Informationen zu allen installierten SQL Server- und SQL
        Server Analysis Services-Instanzen zur Verfügung. Außerdem ermöglicht er
        die Verbindung zu einer benannten SQL Server-Instanz oder den
        Verbindungsaufbau mittels der DAC (Dedicated Administrator Connection).
  
        Dieser Dienst wurde mit dem SQL Server 2005, als Ersatz f\"ur das
        \enquote{SQL Server Resolution Protocol} eingeführt. Er arbeitet auf Port
        1434 und wird durch die \oscommand{.EXE}-Datei
        \oscommand{sqlbrowser.exe} zur Verfügung gestellt.
      \subsection{SQL Server Integration Services 12.0}
        Die SQL Server Integration Services stellen eine komplexe Plattform zur
        Transformation und Integration von Daten dar. Mit Ihnen können Daten
        kopiert, per FTP heruntergeladen oder per Mail versandt werden. Die
        Integration Services sind in der Lage, Daten aus einer großen Zahl von
        Quellen (XML, relationale Datenbanken, Textdateien, Excel-Tabellen,
        usw.) zu lesen und zu verarbeiten.
        
        Der Integration Services-Dienst wird durch die ausführbare Datei
        \oscommand{MsDtsSrvr.exe} bereitgestellt. Diese benötigt keine
        Startparameter.
        \begin{merke}
          Da dieser Dienst optional ist, sollte er nur bei bedarf installiert
          werden!
        \end{merke}
      \subsection{Der SQL Server VSS Writer}
        Der SQL Server VSS Writer ermöglicht es Backups von Datendateien mittels
        des Windows Volume Shadow Copy Service zu erzeugen. Wenn er aktiviert ist,
        kann jede Backupsoftware, die den VSS-Dienst nutzt, Online-Backups der
        Datendateien machen. Ist der Dienst deaktiviert, können Online-Backups
        nur mit Hilfe des SQL Servers selbst durchgeführt werden, da der SQL
        Server-Dienst eine Sperre auf seine Dateien legt.
        
        Sofern Backups nur mit Hilfe des SQL Server \languagemssql{BACKUP}-Kommandos
        gemacht werden, kann dieser Dienst deaktiviert werden.
        \begin{merke}
          Der SQL Server VSS Writer muss immer mit dem lokalen System-Konto
          betrieben werden! Er ist der einzige SQL Server-Dienst, der nur über
          die Windows Dienste Konsole konfiguriert werden kann. Er ist nicht im SQL Server
          Configuration Manager abgebildet.
        \end{merke}
        \begin{literaturinternet}
          \item \cite{ms175536}
        \end{literaturinternet}       
    \section{Das SQL Server Management Studio}
      Das SQL Server Management Studio (SSMS) ist das zentrale Werkzeug zur
      Verwaltung des SQL Server 2014. Es wurde mit der Version 2005 des SQL Server
      eingeführt und ersetzte damals eine Vielzahl von eigenständigen
      Tools, wie z. B. den Enterprise Manager, den Query Analyzer oder den
      Analysis-Manager.
      \bild{Das SQL Server 2014 Management Studio}{ssms_2014}{0.4}
      \subsection{Login mit dem Management Studio}
        \subsubsection{Servertypes und Servername}
          Nach dem Starten des SSMS erscheint zuerst der Dialog
          \enquote{Verbindung mit Server herstellen}. Dieser dient dazu, sich bei
          einem der verschiedenen Servertypen zu authentifizieren. Dabei stehen
          zur Auswahl:
          \begin{itemize}
            \item Datenbankmodul
            \item Analysis Services
            \item Reporting Services
            \item Integration Services
          \end{itemize}
          \bild{Die verschiedenen Server\-typen im
          Login-Dialog}{ssms_login_dialog_servertypes}{0.4}
          Unter \enquote{Servername} muss der DNS-Name des Datenbankservers
          eingetragen werden, an dem die Anmeldung erfolgen soll. Wenn das Ziel
          eine benannte Instanz eines SQL Servers ist, muss der Instanzname
          zusätzlich zum Rechnernamen mit angegeben werden, z. B.
          \identifier{FEA11-119SRV12\$CRM}.
        \subsubsection{Authentifizierung}
          f\"ur die Authentifizierung stehen dem Benutzer zwei verschiedene
          Verfahren offen:
          \begin{itemize}
            \item \textbf{Windows-Authentifizierung}: Diese Form der
            Authentifizierung wird auch als \enquote{integrierte Sicherheit}
            bezeichnet und stellt den Standardauthentifizierungsmechanismus dar.
            Es werden bestimmte Windows Benutzer oder Gruppen im SQL Server als
            vertrauenswürdig eingetragen, so dass sich diese, dann ohne die
            Angabe eines Kennwortes, am SQL Server Anmelden dürfen.
            \item \textbf{SQL Server-Authentifizierung}: Bei diesem Verfahren
            wird die komplette Authentifizierung der Benutzer durch den SQL
            Server selbst durchgeführt. Die Sicherheit dieses Verfahrens ist
            längst nicht so hoch, wie bei der Windows-Authentifizierung.
          \end{itemize}
          \bild{Die verschiedenen Authen\-tifizierungs\-typen im
          Login-Dialog}{ssms_authentification_types}{0.4}
          \begin{merke}
            Weitere Informationen zu diesen beiden Authentifizierungsverfahren
            finden Sie im Kapitel \enquote{Sicherheit im SQL Server}.
          \end{merke}
      \subsection{Der Aufbau des SSMS - Objekt-Explorer}
        Nach einer erfolgreichen Authentifizierung erscheint dann das SQL Server
        Management Studio in seiner Grundeinstellung.
        \bild{Das SQL Server Management Studio}{ssms_1}{0.2}
        Zu Beginn ist nur ein einziger Bereich sichtbar, der
        \enquote{Objekt-Explorer}. Mit ihm ist es möglich:
        \begin{itemize}
            \item Server zu verwalten
            \item Objekte zu erstellen und zu verändern
            \item Log-Dateien auszuwerten
        \end{itemize}
        und noch vieles Anderes. 
        
        Die Daten werden im Objekt-Explorer immer hierarchisch aufgebaut. Zu
        jeder Instanz, werden die vorhandenen Objekte als Baum angezeigt. Da
        während des Logins eine Verbindung zu einem Datenbankmodul hergestellt
        wurde, wird im Objekt-Explorer die betreffende Datenbankinstanz
        angezeigt.
        \bild{Die Datenbankinstanz
        FEA11-119SRV12}{object_explorer_instance}{0.4} 
        Abhängig davon, mit welcher Art von Dienst man verbunden ist,
        variiert der Inhalt des Objekt-Explorers.
        \bild{Zwei unterschiedliche Instanzen im
        Objekt-Explorer}{object_explorer_two_instances}{0.4}
        \subsubsection{Wenn der Objekt-Explorer einmal nicht da ist}
          Sollte der Objekt-Explorer aus irgendeinem Grund nicht zu sehen sein,
          kann er über das Menü \identifier{Ansicht} -
          \identifier{Objekt-Explorer} oder durch Drücken der Taste
          \identifier{F8} angezeigt werden.
          \bild{Das Menü Ansicht}{ssms_menu_view_objekt-explorer}{0.5}
        \subsubsection{Verbindung mit einer Instanz herstellen}
          Wie in \abbildung{object_explorer_two_instances} zu sehen ist, kann
          der Objekt-Explorer Verbindungen zu mehreren Instanzen aufbauen. Um
          sich mit einer Instanz zu verbinden, bzw. um eine zusätzliche
          Verbindung herzustellen, existiert der \enquote{Verbinden}-Button,
          oben links im Objekt-Explorer.
          \bild{Die Schaltfläche
          \enquote{Verbinden}}{objekt_explorer_connect_button}{0.5} Nach der
          Auswahl des Server-Typs erscheint der Login-Dialog, um den Benutzer zu authentifizieren.
        \subsubsection{Verbindungen trennen}
          Wenn die Verbindung zu einem Server Typ getrennt werden soll,
          geschieht dies über den \enquote{Trennen}-Button im Objekt-Explorer.
          Der betroffene Server Typ wird aus dem Objekt-Explorer entfernt.
          \bild{Die Schaltfläche
          \enquote{Trennen}}{objekt_explorer_disconnect_button}{0.5}
      \subsection{Der Aufbau des SSMS - Der Abfrageeditor}
        Eines der wichtigsten Tools im SSMS ist der Abfrageeditor. Er dient zur
        Eingabe und Ausführung jeglicher T-SQL Statements. Aufgerufen wird er
        durch einen Klick auf die Schaltfläche \enquote{Neue Abfrage} in der
        Symbolleiste.
        \bild{Die Schaltfläche
        \enquote{Neue Abfrage}}{ssms_neue_abfrage}{0.5}
        Der Abfrageeditor bietet verschiedene Funktionen an, die dem Anwender
        das Arbeiten sehr erleichtern können. Dies sind beispielsweise 
        IntelliSense oder Syntax Highlighting.
        \bild{IntelliSense und Syntax
        highlighting}{ssms_abfrage_editor_intellisense}{0.48}
        Wird eine Abfrage durch das Drücken der Taste \identifier{F5} oder durch
        das Anklicken der Schaltfläche \identifier{Ausführen} ausgeführt, so
        werden die Ergebnisse unten, im Fenster \enquote{Ergebnisse}, angezeigt.
\clearpage
        \begin{merke}
          Wenn der Abfrageeditor mehrere SQL-Statements enthält, führt das
          Drücken der Taste \identifier{F5} dazu, dass alle Statements
          ausgeführt werden. Soll nur ein einzelnes Statement ausgeführt werden,
          muss dieses vorher markiert werden.
        \end{merke}
        \bild{Die Ergebnissansicht}{ssms_abfrage_editor_ergebnisse}{0.4}
        Wie in \abbildung{ssms_abfrage_editor_ergebnisse} zu sehen ist, können
        die Ergebnisse einer Abfrage in unterschiedliche Art und Weise angezeigt
        werden. Es existieren drei Möglichkeiten:
        \begin{itemize}
          \item Anzeige im Gitternetz
          \item Anzeige als Text
          \item Abspeichern in einer Textdatei
        \end{itemize}
        Gesteuert wird die Anzeigeart durch die drei Schaltflächen in der mitte
        der Symbolleiste. Vorausgewählt ist die Anzeige im Gitternetz.
        \bild{IntelliSense und Syntax
        highlighting}{ssms_abfrage_editor_anzeige_art}{0.5}
      \subsection{Aufbau des SSMS - Der Projektmappen-Explorer}
        \subsubsection{Den Projektmappen-Explorer einblenden}
          Der Projektmappen-Explorer ist ein zusätzliches Tool, welches nicht
          standardmässig eingeblendet ist. Um ihn anzuzeigen kann der Shortcut
          \identifier{STRG + ALT + L} oder das Menü \identifier{Ansicht} genutzt
          werden.
          \begin{merke}
            Die Funktion des Projektmappen-Explorers wird, laut Aussage von
            Microsoft ab dem SQL Server 2014, nicht mehr unterstützt werden!
          \end{merke}
        \subsubsection{Aufbau des Projektmappen-Explorers}
          Der Projektmappen-Explorer untergliedert sich in Projektmappen. Eine
          Projektmappen stellt einen Kontainer f\"ur Projekte dar.
          Projekte können wiederum SQL-Skripte, Abfragen, Datenbankverbindungen
          oder Textdateien enthalten.
          \bild{Der Aufbau des
          Projekt\-mappen-Explorers}{ssms_projekt_mappen_explorer}{0.5}
        \subsubsection{Einsatzzweck des Projektmappen-Explorers}
          Die Verwendung des Projektmappen-Explorers bringt folgende Vorteile
          mit sich:
          \begin{itemize}
            \item Zusamenngehörende Elemente können in einem gemeinsamen
            Kontainer gespeichert werden (z. B. SQL-Skripte und
            Datenbankverbindungen)
            \item Die enthaltenen Dateien werden automatisch in einer sinnvollen
            Verzeichnisstruktur abgelegt.
            \item Projektmappen und Projekte können auch in anderen
            Microsoftprodukten (z. B. Microsoft Visual Studio) verwaltet und
            bearbeitet werden.
            \item Projektmappen können in ein Quellcodeverwaltungssystem
            eingefügt werden (z. B. Microsoft Source Safe)
          \end{itemize}
    \section{Die SQL Server Systemdatenbanken}
      \subsection{Die MASTER-Datenbank}
        Die Datenbank \identifier{master} ist die wichtigste, innerhalb des SQL
        Servers. Sie ist der zentrale Speicherort f\"ur instanzweite Metadaten,
        wie z. B. Anmeldeinformationen oder Verbindungsserver. Seit SQL Server
        2008 werden dort nur noch Metadaten und keine Objekte mehr gespeichert.
        Systemviews beispielsweise, werden in der Resource-Datenbank
        \identifier{mssqlsystemresource} abgelegt.
        \begin{merke}
          Ohne seine \identifier{master}-Datenbank kann der SQL Server nicht
          gestartet werden.
        \end{merke}
        \subsubsection{Physikalische Eingeschaften}
          Die \identifier{master}-Datenbank besteht aus zwei Dateien: Der
          Datendatei \oscommand{master.mdf} (5 MB), die sich in der primären
          Dateigruppe befindet und der Log-Datei \oscommand{master.ldf} (2 MB).
          f\"ur beide Dateien ist ein automatisches Wachstum konfiguriert.
        \subsubsection{Einschränkungen bei der Administration}
          f\"ur die \identifier{master}-Datenbank gelten verschiedene
          administrative Einschränkugen. Eine vollständige Liste kann dem
          Microsoft Technet entnommen werden.
          \begin{itemize}
            \item Es können keine Dateien oder Dateigruppen hinzugefügt werden
            \item Der Datenbankbesitzer kann nicht geändert werden (Besizter
            ist der Nutzer \identifier{sa})
            \item Die Datenbank kann nicht gelöscht werden
            \item Die Datenbank kann nicht umbenannt werden
            \item Die primäre Dateigruppe kann nur mit einem
            speziellen Verfahren wiederhergestellt werden.
          \end{itemize}
        \subsubsection{Empfehlungen}
          Da die \identifier{master}-Datenbank eine enorm wichtige Rolle spielt,
          empfiehlt Microsoft, nach den folgenden Operationen eine
          Sicherheitskopie dieser Datenbank anzulegen.
          \begin{itemize}
            \item Hinzufügen, Ändern oder Löschen von Datenbanken
            \item Hinzufügen, Ändern oder Löschen von Anmeldungen (Logins)
            \item Ändern von Server- oder Datenbankkonfigurationswerten
          \end{itemize}
          \begin{literaturinternet}
            \item \cite{ms187837}
          \end{literaturinternet}
      \subsection{Die MSDB-Datenbank}
        Die \identifier{msdb}-Datenbank spielt im Vergleich zur
        \identifier{master}-Datenbank nur eine untergeordnete Rolle. Sie sammelt
        Informationen über Datenbank-Backups, die sogenannten
        \enquote{Sicherungsverläufe}.
        
        Das SQL Server Management Studio verwendet diese Informationen dann, um
        im Falle dessen, dass ein Recovery notwendig wird, dem Nutzer einen
        Vorschlag f\"ur den Recovery-Vorgang unterbreiten zu können.
        \subsubsection{Physikalische Eingeschaften}
          Die \identifier{msdb}-Datenbank besteht aus zwei Dateien: Der
          Datendatei \oscommand{MSDBData.mdf} (17 MB), in der sich die
          primäre Dateigruppe befindet und der Log-Datei
          \oscommand{MSDBLog.ldf} (20 MB). f\"ur beide Dateien ist ein
          automatisches Wachstum konfiguriert.
        \subsubsection{Einschränkungen bei der Administration}
          f\"ur die \identifier{msdb}-Datenbank gelten eine Reihe von
          Einschränkugen bei der Administration. Eine vollständige Liste dieser
          Einschränkungen kann dem Microsoft Technet entnommen werden. Hier
          sollen nur die wichtigsten Einschränkungen aufgelistet werden.
          \begin{itemize}
            \item Die Datenbank kann nicht gelöscht werden
            \item Die Datenbank kann nicht umbenannt werden
          \end{itemize}
          \begin{literaturinternet}
            \item \cite{ms187112}
          \end{literaturinternet} 
      \subsection{Die MODEL-Datenbank}
        Die \identifier{model}-Datenbank dient als Vorlage f\"ur die Erstellung
        von neuen Datenbanken. Jedesmal, wenn eine Datenbank erstellt wird, wird
        zuerst die \identifier{model}-Datenbank kopiert und anschließend werden
        die vom Benutzer angegebenen Parameter/Datenbankoptionen geändert.
        \begin{merke}
          Da der SQL Server die \identifier{tempdb} bei jedem Neustart neu
          anlegt, muss die \identifier{model}-Datenbank, die auch f\"ur die
          \identifier{tempdb}-Datenbank als Vorlage dient, immer vorhanden sein.
        \end{merke}
        Es ist möglich, die \identifier{model}-Datenbank an die
        eigenen Erfordernisse anzupassen, so dass zukünftig zu erstellende
        Datenbanken diese Änderungen reflektieren. Beispielsweise
        könnten direkt verschiedene Objekte, wie z. B. Tabellen oder Benutzer in
        die \identifier{model}-Datenbank eingefügt werden, wenn diese immer in
        allen Datenbanken vorhanden sein sollen.
        \begin{merke}
          Eine Veränderung an der \identifier{model}-Datenbank hat keine
          nachträglichen Auswirkungen auf bereits bestehende Datenbanken!
        \end{merke}
        \subsubsection{Physikalische Eingeschaften}
          Die \identifier{model}-Datenbank besteht aus zwei Dateien: Der
          Datendatei \oscommand{model.mdf} (5 MB) und der Log-Datei
          \oscommand{modellog.ldf} (1 MB). f\"ur beide Dateien ist ein
          automatisches Wachstum konfiguriert.
        \subsubsection{Einschränkungen bei der Administration}
          f\"ur die \identifier{model}-Datenbank gelten eine Reihe von
          Einschränkugen bei der Administration. Eine vollständige Liste dieser
          Einschränkungen kann dem Microsoft Technet entnommen werden. Hier
          sollen nur die wichtigsten Einschränkungen aufgelistet werden.
          \begin{itemize}
            \item Es können keine Dateien oder Dateigruppen hinzugefügt werden
            \item Der Datenbankbesitzer kann nicht geändert werden (Besizter
            ist der Nutzer \identifier{dbo})
            \item Die Datenbank kann nicht gelöscht werden
            \item Die Datenbank kann nicht umbenannt werden
            \item Die Primäre Dateigruppe kann nur mit einem
            speziellen Verfahren recovered werden.
            \item Erstellen von verschlüsselten Objekten, da der
            Verschlüsselungsschlüssel an die Datenbank gebunden ist. Somit
            könnte nur \identifier{model} diese Objekte verwenden.
          \end{itemize}
          \begin{literaturinternet}
            \item \cite{ms186388}
          \end{literaturinternet}          
      \subsection{Die TEMP-Datenbank}
        Die \identifier{tempdb}-Datenbank ist, wie ihr Name schon sagt, ein
        Ablageort f\"ur temporäre Objekte. Sie steht allen Benutzer zur Verfügung
        und enthält sowohl Benutzer- als auch Systemobjekte. Des Weiteren wird
        sie seit SQL Server 2005 auch f\"ur die Versionierung von Tabellenzeilen
        herangezogen. Wie bereits erwähnt, wird die
        \identifier{tempdb}-Datenbank bei jedem Neustart des SQL Server neu
        erstellt, so dass immer eine bereinigte Version von ihr zur Verfügung
        steht.
        \begin{merke}
          f\"ur die \identifier{tempdb}-Datenbank sind keine Sicherungsvorgänge
          zulässig, da temporäre Objekte nicht gesichert werden können.
        \end{merke}
        \subsubsection{Physikalische Eingeschaften}
          Die \identifier{tempdb}-Datenbank besteht aus zwei Dateien: Der
          Datendatei \oscommand{tempdb.mdf} (8 MB) und der Log-Datei
          \oscommand{templog.ldf} (1 MB). f\"ur beide Dateien ist ein
          automatisches Wachstum konfiguriert.
        \subsubsection{Einschränkungen bei Administration und Nutzung}
          f\"ur die \identifier{tempdb}-Datenbank gelten eine Reihe von
          Einschränkugen bei der Administration. Eine vollständige Liste dieser
          Einschränkungen kann dem Microsoft Technet entnommen werden. Hier
          sollen nur die wichtigsten Einschränkungen aufgelistet werden.
          \begin{itemize}
            \item Es können keine Dateien oder Dateigruppen hinzugefügt werden
            \item Der Datenbankbesitzer kann nicht geändert werden (Besizter
            ist der Nutzer \identifier{dbo})
            \item Die Datenbank kann nicht gesichert werden.
            \item Die Datenbank kann nicht gelöscht werden
            \item Die Datenbank kann nicht umbenannt werden
            \item Die primäre Dateigruppe kann nicht, wie bei einer anderen
            Datenbank wiederhergestellt, werden. Hierf\"ur ist ein spezielles
            Verfahren notwendig.
            \item Erstellen von verschlüsselten Objekten, da der
            Verschlüsselungsschlüssel an die Datenbank gebunden ist. Somit
            könnte nur \identifier{model} diese Objekte verwenden.
          \end{itemize}
          Jeder Benutzer kann nur die temporären Objekte benutzen, die er selbst
          angelegt hat oder f\"ur die er vom System zusätzliche Berechtigungen
          erhalten hat. Es ist theoretisch sogar möglich, einem Benutzer den
          Zugriff auf die \identifier{tempdb}-Datenbank zu verweigern, was aber
          in der Praxis dazu führt, das verschiedene Standardroutinen des SQL
          Server dann f\"ur diesen Benutzer nicht mehr funktionieren.
          \begin{literaturinternet}
            \item \cite{ms190768}
          \end{literaturinternet}          
      \subsection{Die DISTRIBUTION-Datenbank}
        Die \identifier{distribution}-Datenbank spielt eine sehr spezielle Rolle
        unter den Systemdatenbanken. Sie speichert Informationen zu allen arten
        von Replikationen, in der SQL Server als Verteiler
        (Distributor) eingebunden ist.
        
        Die \identifier{distribution} ist die einzige Systemdatenbank, deren
        Name frei wählbar ist. \identifier{distribution} ist lediglich der
        Standardname, sofern nichts Anderes angegeben wird.
          \begin{literaturinternet}
            \item \cite{ms183524}
          \end{literaturinternet}          
      \subsection{Die MSSQLSYSTEMRESOURCE-Datenbank}
        Hierbei handelt es sich um eine schreibgeschützte Datenbank, die seit
        den Tagen von SQL Server 2005 existiert. Sie enthält alle Systemobjekte
        (Views und Tabellen), die ursprünglich einmal in der
        \identifier{master}-Datenbank abgelegt waren. Durch diese Auslagerung
        wurden Updates am SQL Server deutlich vereinfacht, da bei Änderungen an
        Systemobjekten einfach die komplette
        \identifier{mssqlsystemresource}-Datenbank ausgetauscht werden kann. Die
        \identifier{master}-Datenbank bleibt unberührt.
        
        Eine weitere Besonderheit der Resourcendatenbank ist, dass ihre
        Systemobjekte zwar physikalisch in ihr gespeichert sind, dass aber
        logische Verknüpfungen in alle anderen Datenbanken existieren. Das
        heißt, das z. B. die View \identifier{sys.databases} in der
        \identifier{mssqlsystemresource}-Datenbank gespeichert, aber auch in
        jeder anderen Datenbank verfügbar ist.
        \begin{merke}
          Die \identifier{mssqlsystemresource}-Datenbank darf keinerlei
          Benutzerobjekte enthalten und wird deshalb auch nicht im SSMS
          angezeigt.
        \end{merke}
        \subsubsection{Physikalische Eingeschaften}
          Die \identifier{mssqlsystemresource}-Datenbank besteht aus zwei
          Dateien: Aus der Datendatei
          \oscommand{mssqlsystemresource.mdf}, und aus der
          Log-Datei \oscommand{mssqlsystemresource.ldf}. Da es pro SQL
          Server-Instanz nur eine Resourcendatenbank gibt,liegt sie zentral im
          Verzeichnis \oscommand{C:\textbackslash Programme\textbackslash
          Microsoft SQL Server\textbackslash
          MSSQL12.\textit{<instance\_name>}\textbackslash MSSQL\textbackslash
          Binn}.
          \begin{literaturinternet}
            \item \cite{ms190940}
          \end{literaturinternet}          
      \subsection{Der Integration Services Katalog - SSISDB}
        Die SQL Server Integration Services, kurz SSIS, stellen ein mächtiges
        Werkzeug f\"ur sogenannte ETL-Prozesse - die Abkürzung ETL steht 
        f\"ur \enquote{Extract Transform and Load} - dar. SSIS kann als eine Art
        Programmiersprache betrachtet werden. Die in SSIS erstellten Programme
        werden als \enquote{Pakete} bezeichnet und nach ihrer Erstellung auf
        alle SQL Server verteilt, wo sie zum Einsatz kommen sollen.
        
        Um die Konfiguration und Bereitstellung von SSIS-Paketen zu
        vereinfachen, führte der SQL Server 2012 die \identifier{SSISDB}, auch
        \enquote{SSIS-Katalog} genannt, ein. Mit ihm kommt das \enquote{Project
        Deployment Model}, eine neue Form der Bereitstellung von
        SSIS-Programmen. Währen Pakete im Dateisystem abgelegt und durch eine
        Konfigurationsdatei konfiguriert wurden, werden Projekte im SSIS-Katalog
        gespeichert und auch deren Konfiguration wird dort abgelegt. Somit
        entsteht ein zentraler Punkt f\"ur SSIS-Programme und deren Konfiguration.
        
        Weitere Vorteile der Nutzung des SSIS-Kataloges sind:
        \begin{itemize}
            \item Durch die Möglichkeit den Katalog in Ordner zu unterteilen,
            kann eine Berechtigungsstruktur aufgebaut werden. Projekte
            werden in Ordnern gespeichert und Benutzer können Rechte auf Ordner
            erteilt werden.
            \item Mehrere Pakete können in einem Projekt zusammenarbeiten, was
            eine bessere Unterteilung großer SSIS-Programme ermöglicht.
            \item Projekte und Pakete können Parametrisiert werden um Projekt-
            und Paketeigenschaften während der Ausführung zu verändern.
            \item \enquote{Serverumgebungen}, \enquote{Servervariablen} und
            \enquote{Serverumgebungsverweise} ermöglichen es, dass ein Projekt
            auf den Zustand des SQL Servers, auf dem es ausgeführt wird,
            reagiert.
            \item Zentrales Logging vereinfacht das Debugging von Paketen.
        \end{itemize}
        \begin{literaturinternet}
          \item \cite{hh479588}
        \end{literaturinternet}          
    \section{Konfigurieren der Servereigenschaften}
      Die Konfiguration der Eigenschaften einer SQL Server-Instanz ist leider
      nicht an einer zentralen Stelle möglich. Bisher wurde der SQL Server
      Configuration Manager genutzt, um z. B. die Eigenschaften der SQL
      Server-Dienste zu konfigurieren, oder um die Netzwerkeinstellungen
      der Instanz zu verändern. Es gibt jedoch noch weitere Einstellungen, die
      nicht über den Configuration Manager geändert werden können, sondern
      über das SSMS.
      
      Alle Einstellungen, die nicht über den Configuration Manager geändert
      werden können, werden im Management Studio über den
      \enquote{Servereigenschaften-Dialog} beeinflußt. Geöffnet wird dieser,
      durch das Kontextmenü der betreffenden SQL Server-Instanz.
      \bild{Aufruf der
      Server\-eigen\-schaften einer
      Instanz}{server_properties_context_menu}{0.6}
      \subsection{Die Grundeinstellungen}
        Die Seite \enquote{Allgemein} der Servereigenschaften liefert einen
        Überblick über verschiedene, unveränderliche Werte der SQL
        Server-Instanz.
        \bild{Seite \enquote{Allgemein} der
        Server\-eigen\-schaften}{server_properties_general}{0.33}
      \subsection{Konfiguration des Arbeitsspeichers}
        Auf dieser Seite des Servereigenschaften-Dialogs kann der
        Arbeitsspeicherkonsum der SQL Server Instanz beeinflußt werden. Die
        beiden Werte \enquote{Minimaler Serverarbeitsspeicher (in MB)} und
        \enquote{Maximaler Serverarbeitsspeicher (in MB)} geben an, wie viel
        Arbeitsspeicher f\"ur die Instanz reserviert wurde bzw. wie viel sie
        höchstens verbrauchen kann.
        \begin{merke}
          Wird f\"ur die Option \enquote{Minimaler Serverarbeitsspeicher} ein
          Wert konfiguriert, bedeutet dies nicht, dass sich die Instanz von
          Anfang an so viel Arbeitsspeicher nimmt. Es handelt sich vielmehr um
          eine Zusicherung, dass die Instanz so viel Arbeitsspeicher
          im Minimum bekommen kann.
        \end{merke}
        Gemäß einer Empfehlung von Microsoft sollten die beiden Optionen
        minimaler und maximaler Serverarbeitsspeicher immer den gleichen Wert
        haben, um Verwaltungsarbeiten im RAM zu minimieren.
        \begin{merke}
          Ein Wert von 2147483648 beim \enquote{Maximalen
          Serverarbeitsspeicher} bedeutet, dass der Instanz keine Obergrenze
          gesetzt wurde.
        \end{merke} 
        \bild{Seite
        \enquote{Arbeitsspeicher} der
        Server\-eigen\-schaften}{server_properties_ram}{0.35}
        Zu beachten sind in diesem Fenster noch der \enquote{Skript}-Button,
        oben links und die beiden Optionsfelder am unteren Rand.
        \enquote{Konfigurierte Werte} bedeutet, dass im Fenster die Werte zu
        sehen sind, die der Nutzer eingestellt hat. Nach einem Umschalten auf
        \enquote{Ausgeführte Werte} werden die aktuell gültigen Verbrauchswerte
        der Instanz angezeigt.
        
        Mit Hilfe des \enquote{Skript}-Buttons können die im Dialog
        vorgenommenen Einstellungen in ein SQL-Skript verwandelt werden. Dies
        ermöglicht das Erlernen der T-SQL-Syntax und auch die Speicherung der
        Konfiguration einer Instanz in Form eines SQL-Skripts.
      \subsection{Die Prozessoreinstellungen}
        Die Seite \enquote{Prozessoren} bietet die Möglichkeit, einer SQL
        Server Instanz bestimmte Prozessoren zu zuweisen (Prozessoraffinität).
        Dies kann aus zweierlei Gründen nützlich sein:
        \begin{itemize}
            \item Wenn mehrere Instanzen des SQL Server auf einem Computer
            laufen, kann jede ihre eigenen Prozessoren erhalten, so dass kein
            \enquote{Streit um die Resourcen} entsteht.
            \item Weiterhin kann so verhindert werden, dass das Betriebssystem
            einen Thread von einem Prozessor auf einen anderen verschiebt,
            um zusätzlicher Aufwand beim Laden und Entladen der Prozessorcaches
            zu vermeiden.
        \end{itemize} 
        \begin{merke}
          Da in Windows Server 2008 die Threadverwaltung grundlegend geändert
          wurde, soll die Prozessoraffinitätseinstellung in künftigen SQL
          Server-Versionen nicht mehr existieren.
        \end{merke}
        Da die Einstellung der Prozessoraffinität dynamisch erfolgt, muss der
        SQL Server nicht neugestartet werden.
        \begin{literaturinternet}
          \item \cite{ms187104}
        \end{literaturinternet}          
        
        Eine zweite Einstellungsmöglichkeit ist die \enquote{E/A-Affinität}.
        Mit dieser Option werden die I/O-Prozesse der Instanz an bestimmte
        Prozessoren gebunden. Der Hintergrund dieser Option ist, wie auch schon
        bei der Prozessoraffinität, dass die Verschiebung von Threads durch das
        Betriebssystem von einem Prozessor auf einen anderen verhindert werden
        soll. In den meisten Fällen, muss die E/A-Affinität nicht konfiguriert
        werden, weil durch die Standardeinstellung bereits optimale Ergebnisse
        erzielt werden.
        \begin{merke}
          Laut Microsoft sollten niemals die Prozessoraffinität und die
          E/A-Affinität auf einem Prozessor gleichzeit aktiviert sein
        \end{merke}
        Die Option \enquote{Maximale Arbeitsthreadanzahl} ist f\"ur Systeme mit
        einer sehr hohen Nutzerzahl gedacht. Wird hier ein Wert ungleich 0
        eingestellt, kann die Anzahl der Workerthreads f\"ur die Instanz
        begrenzt werden, um eine Überlastung des Servers zu verhinden.
        
        Wird ein Haken bei \enquote{SQL Server-Priorität höher stufen}
        gesetzt, behandelt das Betriebssystem die Threads des SQL Server
        vorrangig, was eine Leistungssteigerung beim SQL Server erzielen kann.
        Hierdurch kann es jedoch auch geschehen, dass anderen Prozessen so
        viel Leistung entzogen wird, dass diese kaum mehr arbeitsfähig sind.
        \bild{Seite
        \enquote{Prozessoren} der Server\-eigen\-schaften}{server_properties_processor}{0.33}
        \begin{literaturinternet}
          \item \cite{ms189629}
        \end{literaturinternet}          
      \subsection{Sicherheitseinstellungen}
        Die Seite \enquote{Sicherheit} besitzt vier verschiedene
        Optionsgruppen, wobei die Gruppe \enquote{Serverauthentifizierung} die
        wichtigste ist. Bei der Installation des SQL Servers musste
        eine Auswahl getroffen werden, welche Form der Authentifizierung von
        der Instanz zugelassen wird. Es standen die Windows-Authentifizierung
        und die gemischte Authentifizierung (Windows- + SQL-Authentifizierung)
        zur Auswahl. Dies kann hier geändert werden.
        \begin{merke}
          Das Ändern der Serverauthentifizierung erfordert einen Neustart des
          SQL Server-Dienstes!
        \end{merke}
        Die beiden Optionsgruppen \enquote{Anmeldungsüberwachung} und
        \enquote{Serverproxykonto} sind nur noch aus Kompatibilitätsgründen
        vorhanden. Mit dem SQL Server 2008 wurden diese beiden Dinge durch
        völlig neue Technologien ersetzt.
        \bild{Seite
        \enquote{Sicherheit} der
        Server\-eigen\-schaften}{server_properties_security}{0.33}
        \begin{literaturinternet}
          \item \cite{ms188470}
        \end{literaturinternet}          
      \subsection{Verbingungseinstellungen}
        Die erste Option auf dieser Seite, \enquote{Maximale Anzahl von
        gleichzeitigen Verbindungen}, legt fest, wie viele Benutzer sich mit
        einer SQL Server-Instanz verbinden können. Sie ist beispielsweise dann
        interessant, wenn mit Client-Access-Licenses (CAL) gearbeitet wird und
        die Anzahl der gleichzeitigen Anmeldungen die Anzahl der CALs nicht
        überschreiten darf.
        
        Wird die Option \enquote{Abfragekontrolle verwenden, um Abfragen mit
        langer Ausführungszeit zu verhinden} aktiviert, überwacht der SQL
        Server f\"ur alle Abfragen die Berechnungsdauer ihres Ausführungsplanes.
        Übersteigt diese Dauer die hier angegebene Zeitspanne,
        wird die Berechnung abgebrochen. Besonders kostenintensive Abfragen
        werden so unterbunden.
        \begin{merke}
          Zu beachten ist, dass hier nur die Zeitspanne f\"ur das Berechnen der
          Ergebnismenge gilt, nicht aber der Zeitraum, der f\"ur das Anzeigen
          der Ergebnismenge benötigt wird!
        \end{merke}
        
        Mit den Standardverbindungsoptionen kann festgelegt werden, welche
        Besonderheiten f\"ur alle angemeldeten Clients gelten. Eine vollständige
        Liste mit den Bedeutungen der einzelnen Optionen, kann in der MSDN,
        unter dem Link \parencite{ms180124}, nachgeschlagen werden.          
        \bild{Seite
        \enquote{Verbindungen} der
        Server\-eigen\-schaften}{server_properties_connections}{0.33}
        \begin{literaturinternet}
          \item \cite{ms180124}
        \end{literaturinternet}          
      \subsection{Datenbankeinstellungen}
        Fast alle Einstellungen auf dieser Seite werden nur selten geändert.
        Beispielsweise wird der \enquote{Standardfüllfaktor f\"ur Indizes}, oft
        nur f\"ur bestimmte Indizes angepasst, da der Standardwert 0 meist
        optimal ist. Auch die Einstellungen der Gruppe \enquote{Sichern und
        Wiederherstellen} werden so gut wie nie allgemein geändert, sondern f\"ur
        alle Sicherungen getrennt.
        \bild{Seite
        \enquote{Datenbank\-einstellungen} der
        Server\-eigen\-schaften}{server_properties_databasesettings}{0.33}
        \begin{literaturinternet}
          \item \cite{ms178521}
        \end{literaturinternet}          
      \subsection{Erweiterte Einstellungen}
        Unter den erweiterten Einstellungen sind zwei, die eine
        besondere Erwähnung verdienen. \enquote{Eigenständige Datenbank
        aktivieren} und \enquote{FILESTREAM-Zugriffsebene}. Beide werden bei den
        Themen \enquote{Filestream} bzw. \enquote{Partially contained databases}
        näher besprochen.
        \bild{Seite
        \enquote{Erweitert} der
        Server\-eigen\-schaften}{server_properties_advanced}{0.33}
        Unter der Rubrik \enquote{Netzwerk} kann die Netzwerkpaketgröße und
        der Remotetimeout angepasst werden. Eine Veränderung der
        Netzwerkpaketgröße kann, abhängig von der Art und Weise, wie auf den
        SQL Server zugegriffen wird, sinnvoll sein. Wenn z. B. hauptsächlich
        Datenimports getätigt werden, kann eine höhere Paketgröße die
        Importvorgänge beschleunigen. Bei normalen Lese-/Schreibzugriffen kann
        sich eine Vergrößerung der Netzwerkpakete aber auch negativ auswirken,
        da die zu transportierenden Datenmengen viel zu klein, f\"ur die großen
        Pakete sind.
        \begin{literaturinternet}
          \item \cite{ms189357}
        \end{literaturinternet}          
      \subsection{Berechtigungen}
        Hier können den existierenden Logins Berechtigungen auf Server-Ebene
        erteilt werden. Nähere Informationen zu Logins und Berechtigungen
        werden im Kapitel \enquote{Sicherheit im SQL Server} bereitgestellt.
        
        \bild{Seite
        \enquote{Datenbank\-einstellungen} der
        Server\-eigen\-schaften}{server_properties_permissions}{0.33}
    \section{Die gespeicherte Systemprozedur sp\_configure}
      \identifier{sp\_configure} ist eine standardmässig vorhandene
      gespeicherte Systemprozedur, die es ermöglicht, die
      Servereigenschaften mit Hilfe von T-SQL anzuzeigen oder zuverändern.
      \subsection{Basisoptionen und erweiterte Optionen}
        SQL Server unterteilt seine Konfigurationsoptionen in zwei Gruppen:
        \begin{itemize}
            \item \textbf{Basisparameter}: Hierbei handelt es sich um eine
            Liste von 18 Parametern, die sehr häufig Änderungen erfahren.
            \item \textbf{Erweiterte parameter}: Dies sind 52 weitere
            Parameter, die lauf Empfehlung von Microsoft, nur von erfahrenen
            Administratoren oder ausgebildeten SQL Server Technikern geändert
            werden sollten.
        \end{itemize}
        Standardmässig zeigt \identifier{\identifier{sp\_configure}igure} nur die
        Basisparameter an. Um einen erweiterten Parameter zu sehen bzw. zu
        ändern, muss der SQL Server erst dementsprechend konfiguriert werden.
      \subsection{Ändern der Konfiguration mit sp\_configure}
        Die allgemeine Syntax f\"ur die Anwendung der Systemprozedur
        \identifier{sp\_configure} lautet folgendermassen:
        \begin{lstlisting}[language=ms_sql,caption={Die Syntax von
sp\_configure},label=admin03_01]
sp_configure [[@configname = ] 'option_name'] , [@configvalue = ] 'value' ]]
          \end{lstlisting}
        \beispiel{admin03_02} zeigt die Benutzung von
        \languagemssql{sp_configure} zur Änderung der
        Serverkonfigurationsoption \languagemssql{backup compression default}.
        \begin{lstlisting}[language=ms_sql,caption={Ein erstes Beispiel zur
          Nutzung von sp\_configure},label=admin03_02]
EXEC §sp_configure§ @configname = 'backup compression default', 
                  @configvalue = '1'
          \end{lstlisting}
        Das gleiche Kommando kann auch in einer Kurzform ausgeführt werden,
        indem man die Parameterbezeichner \languagemssql{@configname} und
        \languagemssql{@configvalue} nicht mit angibt.
        \begin{lstlisting}[language=ms_sql,caption={Die Kurzform der Syntax
          f\"ur sp\_configure},label=admin03_03]
EXEC §sp_configure§ 'backup compression default', '0'
          \end{lstlisting}
        Die Tatsache, dass der SQL Server auf diese beiden Statements mit einer
        Fehlermeldung antwortet zeigt, dass die Konfigurationseinstellung
        bisher noch nicht wirksam geworden ist.
        \begin{small}
          \changefont{pcr}{m}{n}
          \begin{mssql}
Die Konfigurationsoption 'backup compression default' wurde von 1 in
0 geändert. Führen Sie zum Installieren die RECONFIGURE-Anweisung
aus.
            \end{mssql}
        \end{small}
        f\"ur jede Konfigurationseinstellung gibt es zwei Werte:
        \begin{itemize}
            \item Den konfigurierten Wert. Dieser wird mit Hilfe von
            \identifier{sp\_configure} geändert.
            \item Den aktiven Wert. Dieser wird entweder durch die Anweisung
            \languagemssql{RECONFIGURE} oder durch einen Neustart der Instanz
            geändert.
        \end{itemize}
        Um die Änderung aus \beispiel{admin03_02} bzw. \beispiel{admin03_03}
        zu aktivieren, muss also noch das Kommando \languagemssql{RECONFIGURE}
        ausgeführt werden.
        \begin{lstlisting}[language=ms_sql,caption={Ein erstes Beispiel zur
          Nutzung von sp\_configure},label=admin03_04]
EXEC §sp_configure§ 'backup compression default', '1'
RECONFIGURE
          \end{lstlisting}
        \begin{merke}
          Vor der Ausführung des \languagemssql{RECONFIGURE}-Kommandos wird
          der SQL Server Plan cache komplett geleert, d. h. alle
          Ausführungspläne gehen verloren und müssen neu erarbeitet werden.
          Dies kann in einer Produktionsumgebung zu einem massiven
          Performanceeinbruch führen.
        \end{merke}
      \subsection{Basiskonfigurationseinstellungen anzeigen}
        Wird die Prozedur \identifier{sp\_configure} ohne Parameter
        aufgerufen, zeigt sie die aktuellen
        Basisserverkonfigurationseinstellungen an.
      \begin{lstlisting}[language=ms_sql,caption={sp\_configure zur Anzeige
        der Basiskonfigurationswerte nutzen},label=admin03_05]
EXEC §sp_configure§
          \end{lstlisting}
        \begin{center}
          \begin{scriptsize}
            \changefont{pcr}{m}{n}
            \tablefirsthead {
              \multicolumn{1}{l}{\textbf{name}} &
              \multicolumn{1}{l}{\textbf{minimum}} &
              \multicolumn{1}{l}{\textbf{maximum}} &
              \multicolumn{1}{l}{\textbf{config\_value}} &
              \multicolumn{1}{l}{\textbf{run\_value}} \\
              \cmidrule(l){1-1}\cmidrule(l){2-2}\cmidrule(l){3-3}\cmidrule(l){4-4}\cmidrule(l){5-5}
              }
            \tablehead{
            }
            \tabletail {
            }
            \tablelasttail {
            }
            \begin{mssql}
              \begin{supertabular}{lllll}
allow updates                     & 0  & 1          & 0     & 0 \\
backup compression default        & 0  & 1          & 0     & 0 \\
clr enabled                       & 0  & 1          & 0     & 0 \\
contained database authentication & 0  & 1          & 0     & 0 \\
cross db ownership chaining       & 0  & 1          & 0     & 0 \\
default language                  & 0  & 9999       & 1     & 1 \\
filestream access level           & 0  & 2          & 0     & 0 \\
max text repl size (B)            & -1 & 2147483647 & 65536 & 65536 \\
nested triggers                   & 0  & 1          & 1     & 1 \\
remote access                     & 0  & 1          & 1     & 1 \\
remote admin connections          & 0  & 1          & 0     & 0 \\ 
remote login timeout (s)          & 0  & 2147483647 & 10    & 10 \\ 
remote proc trans                 & 0  & 1          & 0     & 0 \\
remote query timeout (s)          & 0  & 2147483647 & 600   & 600 \\
server trigger recursion          & 0  & 1          & 1     & 1 \\
show advanced options             & 0  & 1          & 0     & 0 \\
user options                      & 0  & 32767      & 0     & 0 \\
                \end{supertabular}
            \end{mssql}
          \end{scriptsize}
        \end{center}
      \subsection{Erweiterte Konfigurationseinstellungen abfragen}
        Um erweiterte Konfigurationsoptionen sehen bzw. ändern zu können, muss
        die Einstellung \languagemssql{show advanced options} auf den Wert 1
        eingestellt werden.
        \begin{lstlisting}[language=ms_sql,caption={Einblenden
          der erweiterten Konfigurationsoptionen},label=admin03_06]
EXEC §sp_configure§ 'show advanced options', '1'
RECONFIGURE
          \end{lstlisting}
        Ein Aufruf von \identifier{sp\_configure} zeigt nun eine Liste mit
        insgesamt 70 Parametern an. Um die erweiterten Parameter wieder
        auszublenden, muss \languagemssql{show advanced options} auf den Wert
        0 zurückgesetzt werden.
        \begin{literaturinternet}
          \item \cite{ms176069}
          \item \cite{ms188787}
        \end{literaturinternet}
    \section{Datengewinnung - Systemsichten}
      Jedes Datenbank Management System bietet dem Administrator einen Weg, um
      Daten über den Server, die Datenbanken, die Benutzer und viele andere
      Dinge zu sammeln. In Microsoft SQL Server werden dem Admin Views, die
      sogenannten Systemsichten, zur Verfügung gestellt, um die gewünschten
      Daten abfragen zu können.
      \subsection{Katalogsichten}
        Katalogsichten existieren seit SQL Server 2005 und liefern         
        Informationen, die aus den verschiedenen Teilen der Software stammen
        (Caches, Pools, SQLOS, usw.). Durch sie kann sich der
        Administrator ein umfassendes Bild über die Instanz machen, an der er
        arbeitet. Microsoft baut mit den Katalogsichten ein \enquote{Data
        Dictionary}, gemäß dem SQL-Standard, auf. Mit jeder Version von SQL
        Server wächst die Anzahl der Katalogsichten. Außerdem verändern sich
        manche Sichten auch von Version zu Version.
        
        Die Bezeichner der Katalogsichten sind nach einem einfachen Schema
        aufgebaut.
        \begin{itemize}
            \item Alle Katalogsichten beginnen im Namen mit der Zeichenfolge
            \identifier{sys.}
            \item Der Name lässt immer auf die enthaltenen Informationen
            schließen.
        \end{itemize}
        Beispielsweise bietet die Katalogsicht \identifier{sys.databases}
        Infos zu allen Datenbanken innerhalb einer SQL Server-Instanz. Im
        Verlauf dieser Unterrichtshilfe werden immer wieder die wichtigsten
        Katalogsichten zum jeweiligen Thema angezeigt.
        
        \begin{merke}
          Die Katalogsichten sind in der Resourcen-Datenbank gespeichert und
          werden in alle anderen Datenbanken projiziert.
        \end{merke}
        
        \begin{literaturinternet}
          \item \cite{ms174365}
        \end{literaturinternet}
      \subsection{Informationsschemasichten}
        Diese Views haben den gleichen Zweck wie Katalogsichten.
        Sie zeigen wichtige Informationen über die Instanz und ihre Datenbanken
        an. Der Unterschied zu den Katalogsichten liegt zum einen darin, dass
        der Aufbau der Informationsschemasichten im ANSI SQL-Standard genau definiert
        ist, während die Katalogsichten von den Microsoftentwicklern frei
        gestaltet werden können. Zum anderen sind die
        Informationsschemasichten nicht in der Resourcen Datenbank
        gespeichert, sondern in den Datenbanken selbst.
        
        \begin{literaturinternet}
          \item \cite{ms186778}
        \end{literaturinternet}
      \subsection{Kompatibilitätssichten}
        Kompatibilitätssichten dienen, wie ihr Name schon sagt, nur zur
        Aufrechterhaltung der Kompatibilität. In SQL Server 2000 wurden dem
        Administrator \enquote{Systemtabellen} zur Verfügung gestellt, deren
        Aufgabe heute Katalog- und Informationsschemasichten übernommen
        haben. Damit aber auch noch alte SQL Server 2000-Anwendungen
        funktionieren, bietet Microsoft mit den Kompatibilitätassichten einen
        temporären Ausweg.
        
        Sie sind daran zu erkennen, dass ihr Name immer mit
        \identifier{sys} beginnt. Zu jeder Kompatibilitätssicht existiert
        auch eine aktuelle Katalogsicht, beispielsweise gibt es zur View
        \identifier{sysusers} die Katalogsicht
        \identifier{sys.database\_principals}.
        
        Um veraltete Anwendungen auf den neuesten Stand zu bringen, bietet
        Microsoft in der MSDN eine Tabelle zur Zuordnung von Systemtabellen zu
        Katalogsichten an (\parencite{ms187997}).
        
        \begin{literaturinternet}
          \item \cite{ms187376}
        \end{literaturinternet}
      \subsection{Dynamische Verwaltungssichten und -funktionen}
        Dynamische Verwaltungssichten und -funktionen unterscheiden sich von
        Katalog- und Informationsschemasichten darin, dass sie ihre Daten
        nicht aus Tabellen generieren, sonderen dass sie sie direkt aus dem
        SQL Server oder dem Betriebssystem abgreifen. Aus diesem Grund werden
        sie als dynamische Views bezeichnet, weil ihre Informationen immer
        ohne Verzögerung direkt aus dem System abgerufen werden. Mit ihrer
        Hilfe ist es möglich, interne, versionsspezifische Statusinformationen
        über den SQL Server abzufragen, um so nach Problemen zu suchen.
        
        Die Namenskonvention f\"ur dynamsiche Verwaltungssichten und -funktionen
        sieht vor, dass die Bezeichner immer mit \identifier{sys.dm\_}
        beginnen. Auch f\"ur diese Gruppe von Views werden im Verlauf dieser
        Unterrichtsunterlage immer wieder die wichtigsten Vertreter zum
        jeweiligen Thema genannt.
        
        \begin{literaturinternet}
          \item \cite{ms188754}
        \end{literaturinternet}
    \section{Datenbanken erstellen}
      Microsoft SQL Server ermöglicht das Erstellen von Datenbanken auf zwei
      unterschiedlichen Wegen: Mit Hilfe des SQL Server Management Studios,
      oder mit dem SQL-Kommando \languagemssql{CREATE DATABASE}. Um eine neue
      Datenbank anlegen zu können, muss der Benutzer:
      \begin{itemize}
        \item ein Mitglied der festen Serverrole \identifier{sysadmin} sein,
        \item oder eines der beiden Serverrechte \identifier{control} bzw.
        \identifier{alter},
        \item oder das Serverrecht \identifier{create database} haben. 
      \end{itemize}
      Um eine neue Datenbank anzulegen kopiert SQL Server einfach die
      \identifier{model}-Datenbank, weshalb die primäre Datendatei der neuen
      Datenbank mindestens genauso groß sein muss, wie die der
      \identifier{model}-Datenbank. Insgesamt darf eine Datenbank unter SQL
      Server 2014 nicht kleiner als 6 MB sein (primäre Datendatei 5 MB + Log
      Datei 1 MB). f\"ur sekundäre Datendateien wird eine Mindestgröße von 1 MB
      angenommen, sofern der Benutzer keine Angabe macht.
      \subsection{Anlegen einer Datenbank mit dem SSMS}
        \begin{enumerate}
          \item Klicken Sie im Objekt-Explorer mit der rechten Maustaste
          auf den Knotenpunkt \enquote{Datenbanken}. Es erscheint ein
          Kontextmenü. Dort wählen sie den Punkt \enquote{Neue
          Datenbank\ldots}.
          \bild{Anlegen einer neuen Datenbank}{create_db_1}{0.33}
          \item Es öffnet sich der Dialog \enquote{Neue Datenbank}.
          \bild{Der Dialog \enquote{Neue Datenbank}}{create_db_2}{0.33}
          Auf der Seite \enquote{Allgemein} dieses Dialogfensters können die
          folgenden Eigenschaften der neuen Datenbank geändert werden:
          \begin{itemize}
            \item \textbf{Datenbankname}: Ein innerhalb der SQL Server-Instanz
            eindeutiger Bezeichner f\"ur die neue Datenbank, der maximal 128
            Zeichen lang sein darf. Der Datenbankname muss den Regeln f\"ur
            Bezeichner folgen.
            \item \textbf{Besitzer}: Eine SQL Server-Anmeldung, die als
            Besitzter der Datenbank eingetragen werden soll. Der
            Datenbankeigentümer hat innerhalb einer Datenbank uneingeschränkte
            Rechte. Dieser Wert sollte immer auf \oscommand{<Standard>}
            belassen werden.
            \item \textbf{Logischer Name}: Der logische Name Datei ist der
            Bezeichner, mit dessen Hilfe in T-SQL auf die Datei zugegriffen
            wird. Er muss innerhalb der Datenbank eindeutig sein und, wie der
            Datenbankname auch, den Regeln f\"ur Bezeichner folgen.
            \item \textbf{Dateityp}: Gibt an, welche Art von Datei angelegt
            wird. Dies kann sein: Zeilendaten, Protokoll, Filestream.
            \item \textbf{Dateigruppe}: Der Bezeichner der Dateigruppe, zu der
            die angegebene Datendatei gehören soll. Die erste Dateigruppe hat
            immer den Bezeichner \identifier{primary}. Innerhalb dieser
            Dateigruppe befindet sich auch immer die primäre Datendatei.
            \item \textbf{Anfangsgröße}: Legt die anfängliche Größe der Daten-
            und Log-Dateien fest. Es können die Größenangaben KB, MB, GB und
            TB benutzt werden. Ohne Größenangabe wird automatsich MB
            angenommen. Der maximalwert in diesem Feld ist 2.147.483.648.
            \item \textbf{Automatische Vergrößerung/Maximale Größe}:
            Ermöglicht die Konfiguration eines automatischen Wachstums f\"ur
            Datendateien (Wachstumsrate und Größenbeschränkung). Die
            Standardwachstumsrate beträgt bei Datendateien 1 MB und bei
            Log-Dateien 10 \% (der Prozentsatz bezieht sich immer auf die
            aktuelle Dateigröße, zum Zeitpunkt des Anwachsens). Die hier
            angegebene Größe wird immer auf den nächsten, durch 64 KB
            teilbaren Wert aufgerundet. Durch die Angabe des Wertes 0 wird
            angezeigt, das kein automatisches Wachstums stattfinden darf. Wird
            f\"ur die Größenbeschränkung kein Wert oder der Wert
            \languagemssql{UNLIMITED} angegeben, kann die Datei solange
            wachsen, bis die physikalischen Grenzen des Datenträger, auf dem
            sie sich befindet, erreicht sind.
            \item \textbf{Pfad + Dateiname}: Gibt den physikalischen
            Dateinamen vor. Die Datendatei darf nur auf einem lokalen
            Datenträger, auf einer SAN oder auf einem iSCSI-Laufwerk abgelegt
            werden. Der angegebene Pfad muss vor dem Ausführen der
            \languagemssql{CREATE DATABASE}-Anweisung bereits vorhanden sein.
            \begin{merke}
              Es sollte niemals ein komprimiertes Dateisystem f\"ur die Ablage von
              Daten- oder Log-Dateien genutzt werden.
            \end{merke}
          \end{itemize}
          \item Wechseln Sie in das Fenster \enquote{Optionen}.
          \bild{Verändern der Datenbankoptionen}{create_db_3}{0.33}
          Hier können alle Datenbankoptionen einzeln gesetzt werden. Viele
          dieser Einstellungen werden noch im weiteren Verlauf dieser
          Unterrichtsunterlage behandelt. f\"ur eine vollständige Liste der
          Datenbankoptionen beachten Sie bitte den nächsten Literaturhinweis.
          \item Wechseln Sie in das Fenster \enquote{Dateigruppen}. Auf dieser
          Seite können der Datenbank weitere Dateigruppen hinzugefügt werden.
          Der Haken in der Spalte \enquote{Standard} sagt aus, dass alle
          Objekte, wie z. B. Tabellen in einer Datei, dieser Dateigruppe
          angelegt werden, sofern der Benutzer keine andere Angabe macht.
          Klicken Sie auf OK. Die Datenbank wird erstellt.
          \bild{Anpassen der Dateigruppen}{create_db_4}{0.33}
        \end{enumerate}
        
        \begin{literaturinternet}
          \item \cite{ms187997}
        \end{literaturinternet}
        \subsection{Anlegen einer Datenbank mittels T-SQL (CREATE DATABASE)}
        Da der SQL Server beim Anlegen einer neuen Bank lediglich eine Kopie
        der \identifier{model}-Datenbank erstellt, ist die einfachste Form des
        \languagemssql{CREATE DATABASE}-Statements ohne Parameter.
        \begin{lstlisting}[language=ms_sql,caption={Die
        einfachste Form von \languagemssql{CREATE DATABASE}},label=admin03_07]
CREATE DATABASE demo_sql;
        \end{lstlisting}
        Der einzige Wert, der zwingend angegeben werden muss, ist der Name der
        neuen Datenbank. Die primäre Dateigruppe und die Log-Datei sind
        1:1-Kopien der \identifier{model}-Datenbank. Um die primäre
        Dateigruppe zu beeinflussen, müssen die beiden Schlüsselwörter
        \languagemssql{ON PRIMARY}, zusammen mit weiteren Angaben angefügt
        werden.
        \begin{lstlisting}[language=ms_sql,caption={Die
        Dateigruppe PRIMARY verändern},label=admin03_08]
IF EXISTS (SELECT * FROM sys.databases WHERE name='demo_sql')
DROP DATABASE demo_sql

CREATE DATABASE demo_sql
ON PRIMARY (
  NAME       = 'SALES',
  FILENAME   = 'D:\data\demo_sql\sales01.mdf',
  SIZE       = 100MB,
  MAXSIZE    = 2GB,
  FILEGROWTH = 100MB
)
        \end{lstlisting}
        In \beispiel{admin03_08} wird eine Datenbank mit zwei Dateien
        angelegt:
        \begin{itemize}
            \item \oscommand{sales01.mdf}: Die primäre Datendatei, mit einer
            Größe von 100 MB, einer maximalen Größe von 2 GB und einer
            Wachstumsrate von je 100 MB.
            \item \oscommand{demo\_sql\_log.ldf}: die Log-Datei der Datenbank.
            Diese wird mit einer Größe von 25 MB, einer maximalen Größe von 2
            TB und einer Wachstumsrate von 10 \%. Der Name der Datei wurde aus
            dem Datenbanknamen und dem suffix \identifier{\_log}
            zusammengesetzt.
        \end{itemize}
        \begin{merke}
          Wenn kein logischer Name f\"ur die Log-Datei angegeben wird, darf der
          Datenbankname max. 123 Zeichen Lang sein, damit das suffix
          \identifier{\_log} an den Log-Dateinamen angehängt werden kann.
        \end{merke}
        \begin{lstlisting}[language=ms_sql,caption={Eine
        Dateigruppe mit mehreren Dateien ausstatten und die Log-Datei
        verändern},label=admin03_09]
IF EXISTS (SELECT * FROM sys.databases WHERE name='demo_sql')
DROP DATABASE demo_sql

CREATE DATABASE demo_sql
ON PRIMARY (
  NAME       = 'SALES01',
  FILENAME   = 'D:\data\demo_sql\sales01.mdf',
  SIZE       = 100MB,
  MAXSIZE    = 2GB,
  FILEGROWTH = 100MB
),
(
  NAME       = 'SALES02',
  FILENAME   = 'D:\data\demo_sql\sales02.ndf',
  SIZE       = 100MB,
  MAXSIZE    = 2GB,
  FILEGROWTH = 100MB
)
LOG ON (
  NAME       = 'DEMO_SQL_LOG',
  FILENAME   = 'D:\data\demo_sql\demo_sql_log.ldf',
  SIZE       = 64MB,
  MAXSIZE    = 1GB,
  FILEGROWTH = 16MB
)
      \end{lstlisting}
      \beispiel{admin03_09} zeigt eine nochmals erweitere Syntax, mit deren
      Hilfe die Dateigruppe Primary mit zwei Datenbankdatei,
      \oscommand{sales01.mdf} und \oscommand{sales02.ndf} ausgestatt wird.
      Des Weiteren werden die Größe, die Wachstumsrate und die Maximalgröße der
      Log-Datei verändert.
      
      Im nächsten Beispiel wird gezeigt, wie eine Datenbank mit mehreren Dateigruppen
      erstellt werden kann. Zusätzlich zur immer existenten Gruppe
      \identifier{primary} wird die Dateigruppe \identifier{CRM} mit einer
      Datendatei angelegt.
\clearpage      
      \begin{lstlisting}[language=ms_sql,caption={Eine
        Dateigruppe mit mehreren Dateigruppen ausstatten und die Log-Datei
        verändern},label=admin03_10]
IF EXISTS (SELECT * FROM sys.databases WHERE name='demo_sql')
DROP DATABASE demo_sql

CREATE DATABASE demo_sql
ON PRIMARY (
  NAME       = 'SALES01',
  FILENAME   = 'D:\data\demo_sql\sales01.mdf',
  SIZE       = 100MB,
  MAXSIZE    = 2GB,
  FILEGROWTH = 100MB
),
(
  NAME       = 'SALES02',
  FILENAME   = 'D:\data\demo_sql\sales02.ndf',
  SIZE       = 100MB,
  MAXSIZE    = 2GB,
  FILEGROWTH = 100MB
),
FILEGROUP CRM (
  NAME       = 'CRM01',
  FILENAME   = 'D:\data\demo_sql\crm01.ndf',
  SIZE       = 500MB,
  MAXSIZE    = 16GB,
  FILEGROWTH = 500MB
)
LOG ON (
  NAME       = 'DEMO_SQL_LOG',
  FILENAME   = 'D:\data\demo_sql\demo_sql_log.ldf',
  SIZE       = 64MB,
  MAXSIZE    = 1GB,
  FILEGROWTH = 16MB
)
      \end{lstlisting}
      \begin{literaturinternet}
        \item \cite{ms176061}
      \end{literaturinternet}
    \section{Datenbanken bearbeiten und löschen}
      \subsection{Die Datenbank demo\_grafisch}
        In den folgenden Abschnitten und Kapiteln wird f\"ur alle weiteren
        Beispiele die SQL Server-Datenbank \identifier{demo\_grafisch} benutzt.
        Sie hat folgenden Aufbau:
        \begin{center}
          \begin{scriptsize}
            \changefont{pcr}{m}{n}
            \tablefirsthead {
              \multicolumn{1}{>{\centering}p{1.6cm}}{\textbf{Logischer Name}}&
              \multicolumn{1}{c}{\textbf{Dateityp}} &
              \multicolumn{1}{>{\centering}p{1.35cm}}{\textbf{Datei- gruppe}}&
              \multicolumn{1}{>{\centering}p{1.35cm}}{\textbf{Anfangs- größe
              (MB)}} & 
              \multicolumn{1}{>{\centering}p{2.5cm}}{\textbf{Automatische
              Vergrößerung}} & \multicolumn{1}{c}{\textbf{Pfad + Dateiname}} \\
              \hline
              }
            \tablehead{}
            \tabletail {
            }
            \tablelasttail {
            }
            \begin{supertabular}{llll>{\centering}p{2.5cm}p{3.75cm}}
              \identifier{primary} & Zeilendaten & \identifier{primary} & 100 & Um 100 MB,
              auf 2048 MB & \oscommand{D:\textbackslash u01\textbackslash
              demo\_grafisch\textbackslash data} \oscommand{primary\_01.mdf}
              \\
              \hline
              \identifier{crm} & Zeilendaten & \identifier{crm} & 500 & Um 100 MB, auf 4096
              MB & \oscommand{D:\textbackslash u01\textbackslash
              demo\_grafisch\textbackslash data} \oscommand{crm01.ndf} \\
              \hline
              \identifier{sales01} & Zeilendaten & \identifier{sales} & 500 &
              Um 100 MB, auf 4096 MB & \oscommand{D:\textbackslash u01\textbackslash
              demo\_grafisch\textbackslash data} \oscommand{sales01.ndf} \\
              \hline
              \identifier{sales02} & Zeilendaten & \identifier{sales} & 500 &
              Um 100 MB, auf 4096 MB & \oscommand{D:\textbackslash u01\textbackslash
              demo\_grafisch\textbackslash data} \oscommand{sales02.ndf} \\
              \hline
              \identifier{logfile} & Protokoll & --- & 1 & Um 10 \%, auf
              2097152 MB & \oscommand{D:\textbackslash u01\textbackslash
              demo\_grafisch\textbackslash data} \oscommand{demo\_grafisch\_log.ldf} \\
            \end{supertabular}
          \end{scriptsize}
        \end{center}
      \subsection{Die Eigenschaften einer Datenbank verändern}
        Die Eigenschaften einer Datenbank können entweder mit Hilfe des SSMS
        oder durch das SQL-Kommando \languagemssql{ALTER DATABASE} geändert
        werden. Die folgende Liste zeigt ausschnittsweise, was an einer
        Datenbank, nach ihrer Erstellung, geändert werden kann.
        \begin{itemize}
            \item Ändern des Datenbanknames
            \item Hinzufügen von Datendateien und Dateigruppen
            \item Hinzufügen von Log-Dateien
            \item Ändern von Dateieigenschaften (Größe, Wachstumsrate,
            Maximalgröße, usw.)
            \item Verändern der physikalischen Position einer Datendatei auf
            dem Datenträger
        \end{itemize}
        Eine vollständige Liste der Möglichkeiten, die das
        \languagemssql{ALTER DATABASE}-Kommando bietet ist im Microsoft
        Developer Network abgebildet.
        \begin{literaturinternet}
          \item \cite{ms174269}
        \end{literaturinternet}
        \subsubsection{Umbenennen einer Datendatei}  
          In der \identifier{demo\_grafisch}-Datenbank wurde der
          logische Name der primären Datendatei \oscommand{primary01.mdf}
          fälschlicherweise nur mit \identifier{primary}, statt mit
          \identifier{primary01} angegeben. Dies soll nun geändert werden.
          \begin{lstlisting}[language=ms_sql,caption={Den
          logischen Namen einer Datendatei ändern},label=admin03_11]
ALTER DATABASE demo_grafisch
MODIFY FILE (
  NAME    = 'PRIMARY',
  NEWNAME = 'PRIMARY01')          
          \end{lstlisting}
        \subsubsection{Hinzufügen einer neuen Dateigruppe}
          Die \identifier{demo\_grafisch}-Datenbank benötigt eine neue
          Dateigruppe namens \identifier{hrm}.
          \begin{lstlisting}[language=ms_sql,caption={Eine
          neue Dateigruppe hinzufügen},label=admin03_12]
ALTER DATABASE demo_grafisch
ADD FILEGROUP HRM  
          \end{lstlisting}
        \subsubsection{Hinzufügen einer Datendatei}
          Die neu hinzugefügte Dateigruppe \identifier{HRM} benötigt eine
          Datendatei.
          \begin{lstlisting}[language=ms_sql,caption={Eine
          Datendatei einer Dateigruppe hinzufügen},label=admin03_13]
ALTER DATABASE demo_grafisch
ADD FILE ( 
  NAME       = 'HRM01', 
  FILENAME   = 'D:\data\demo_grafisch\hrm01.ndf', 
  SIZE       = 500 MB, 
  MAXSIZE    = 4 GB, 
  FILEGROWTH = 100 MB ) 
TO FILEGROUP HRM
          \end{lstlisting}
          \begin{literaturinternet}
            \item \cite{bb522469}
          \end{literaturinternet}
        \subsubsection{Ändern der Standarddateigruppe}
          Wenn in einer SQL Server Datenbank in neues Objekt (Tabelle, Index,
          Datendatei, \ldots) angelegt wird, muss es einer Dateigruppe
          zugeordnet werden. Dies kann auf zwei unterschiedliche Arten
          geschehen:
          \begin{itemize}
              \item Durch manuelle Angabe des Administrators, oder
              \item ohne Angabe des Administrators mit der Standarddateigruppe.
          \end{itemize}
          \begin{merke}
            Alle Objekte, f\"ur die keine Dateigruppenzuordnung gemacht wird,
            werden automatisch in der Standarddateigruppe angelegt.
          \end{merke}
          Standardmässig wird die Dateigruppe \identifier{primary} als
          Standarddateigruppe angegeben. Dies kann aber nach Belieben geändert
          werden.
\clearpage
          \begin{lstlisting}[language=ms_sql,caption={Ändern
          der Standarddateigruppe},label=admin03_14]
IF NOT EXISTS (SELECT name 
               FROM sys.filegroups 
               WHERE is_default=1 AND name = N'SALES')
  ALTER DATABASE demo_grafisch
  MODIFY FILEGROUP SALES DEFAULT
GO
          \end{lstlisting}
      \subsection{Datenbankoptionen verändern}
        Datenbankoptionen sind dazu da, um das Verhalten einer Datenbank zu
        steuern. Der Datenbankadministrator hat mit Hilfe dieser Optionen die
        Möglichkeit umfangreichen Einfluss auf das Verhalten der Datenbank zu
        nehmen. Im Folgenden werde einige Beispiele f\"ur verschiedene
        Datenbankoptionen gegeben. Ein vollständige Liste dieser Optionen kann
        auf der MSDN eingesehen werden.
        \begin{literaturinternet}
          \item \cite{bb522682}
        \end{literaturinternet}
        \subsubsection{Die Option ANSI\_WARNINGS}
          Mit der Datenbankoption \identifier{ansi\_warnings} wird entschieden,
          ob Fehlermeldungen ausgegeben werden, wenn z. B. eine Division durch 0
          geschieht oder wenn NULL-Werte in einer Aggregatfunktion verarbeitet
          werden soll.
          \begin{itemize}
              \item \textbf{ON}: Es werden ANSI-Standardkonforme Fehlermeldungen
              ausgegeben.
              \item \textbf{OFF}: Es werden keine Fehlermeldungen erzeugt und im
              Falle einer Division durch 0 wird der Wert NULL als Ergebnis der
              Berechnung angenommen.
          \end{itemize}
          \begin{lstlisting}[language=ms_sql,caption={ANSI-Standardkonforme
            Warnmeldungen ausgeben},label=admin03_15]
  ALTER DATABASE demo_grafisch
  SET ANSI_WARNINGS  ON;
            \end{lstlisting}
        \subsubsection{Die Option ARITHABORT}
          Mit dieser Option wird gesteuert was im Falle eines arithmetischen
          Überlaufs oder einer Division durch 0 geschieht.
\clearpage
          \begin{itemize}
              \item \textbf{ON}: Das SQL-Statement wird abgebrochen.
              \item \textbf{OFF}: Es wird lediglich eine Warnmeldung ausgegeben,
              wenn einer der genannten Fehler auftritt. Das Statement wird aber
              zuendegeführt, so als hätte es keinen Fehler gegeben.
          \end{itemize}
          \begin{lstlisting}[language=ms_sql,caption={Arithmetische
          Überläufe und Divisionen durch 0 kontrollieren},label=admin03_16]
ALTER DATABASE demo_grafisch 
SET ARITHABORT ON;
          \end{lstlisting}
        \subsubsection{Die Option CONCAT\_NULL\_YIELDS\_NULL}
          \begin{itemize}
              \item \textbf{ON}: Eine Verkettungsoperation gibt NULL zurück,
              wenn einer der beiden Operanden NULL war. Wird z. B. der Text
              \enquote{Hallo Welt} mit dem Wert NULL verkettet, ist das
              Ergebnis NULL.
              \item \textbf{OFF}: Bei Verkettungsoperationen werden NULL-Werte
              wie leere Zeichenfolgen behandelt. Wird z. B. der Text
              \enquote{Hallo Welt} mit dem Wert NULL verkettet, ist das Ergebnis
              \enquote{Hallo Welt}.
          \end{itemize}
          \begin{lstlisting}[language=ms_sql,caption={Behandlung
          von NULL-Werten bei Verkettungsoperationen steuern},label=admin03_17]
ALTER DATABASE demo_grafisch 
SET CONCAT_NULL_YIELDS_NULL ON;
          \end{lstlisting}          
        \subsubsection{Das Kompatibilitätslevel der Datenbank ändern}        
          Das Kompatibilitätslevel, gibt an welche Features in einer SQL
          Server-Datenbank aktiviert werden können bzw. wie sich die Datenbank
          in verschiedenen Situationen verhält. In einer SQL Server 2014-Instanz
          können Datenbanken mit dem Kompatibilitätsgrad 90 (SQL Server 2005),
          100 (SQL Server 2008), 110 (SQL Server 2012) oder 120 (SQL
          Server 2014) betrieben werden. Dadurch wird die Migration einer
          älteren Datenbank auf eine neue SQL Server-Version ermöglicht.
          
          Durch das ändern des Kompatibilitätslevels ist es auch möglich, dass
          sich eine in SQL Server 2014 erstellte Datenbank verhält, als währe
          sie mit dem SQL Server 2005 erstellt worden.
          \begin{lstlisting}[language=ms_sql,caption={Ändern
          des Kompatibilitätslevels einer Datenbank},label=admin03_18]
ALTER DATABASE demo_grafisch2005
SET COMPATIBILITY_LEVEL 120;
          \end{lstlisting}
          In \beispiel{admin03_18} wird die fiktive Datenbank
          \identifier{demo\_grafisch2005}, die noch mit dem SQL Server 2005
          erstellt wurde, auf eine SQL Server 2014-Instanz migriert und der
          Kompatibilitätsgrad wird auf 120 angehoben. Dadurch werden f\"ur diese
          Datenbank alle SQL Server 2014-Features aktiviert.          
      \subsection{Vergrößern und schrumpfen einer Datenbank}
        \subsubsection{Automatisches Wachstum und manuelles Vergrößern}
          Das automatische Wachstum f\"ur eine Datendatei wird, wie bereits
          demonstriert, über die beiden Dateieigenschaften
          \identifier{filegrowth} und \identifier{maxsize} kontrolliert. Um eine
          Datendatei manuell zu vergrößern muss deren Eigenschaft
          \identifier{size} angepasst werden.
          \begin{lstlisting}[language=ms_sql,caption={Eine
          Datendatei vergrößern},label=admin03_19]
ALTER DATABASE demo_grafisch
MODIFY FILE (
  NAME = 'SALES01'
  SIZE = 2 GB
)
          \end{lstlisting}
        \subsubsection{Schrumpfen der gesamten Datenbank}
          Das Schrumpfen einer ganzen Datenbank kann auf einfachem Wege über das
          SSMS erledigt werden. Über das Kontextmenü der Datenbank kann unter
          \identifier{tasks}, \identifier{verkleinern}, \identifier{datenbank}
          der Assistent zum Verkleinern der Datenbank erreicht werden.
          \bild{Den Assistenten zum Schrum\-pfen der
          Datenbank starten}{shrink_database_1}{0.33}
          Nach einem Klick auf \identifier{datenbank} öffnet sich der Assistent.
          \bild{Der Assistenten zum Schrum\-pfen der
          Datenbank}{shrink_database_2}{0.33}
          Der Assistent zeigt an, wie viel Speicherplatz die Datenbank aktuell
          belegt und wie viel er davon freigeben könnte. Zusätzlich zum reinen
          Verkleinern kann die Datenbank auch noch reorganisiert
          (defragmentiert) werden.
          \begin{merke}
            Das Schrumpfen einer ganzen Datenbank kann ein sehr zeitaufwendiger
            und resourcenintensiver Vorgang sein. Es wird daher dringend davon
            abgerate eine Datenbank zu Schrumpfen, es sei denn, dass diese
            Aktion unumgänglich ist.
          \end{merke}
        \subsubsection{Schrumpfen einzelner Datendateien}
          SQL Server bietet auch die Möglichkeit nur einzelne Daten- und
          Protokolldateien zu schrumpfen. Der hierf\"ur zuständige Assistent wird,
          genau wie beim Schrumpfen der Datenbank auch, über das Kontextmenü der Datenbank
          erreicht (\identifier{tasks}, \identifier{verkleinern},
          \identifier{datendatei}).
          \bild{Der Assistenten zum Schrum\-pfen einzelner
          Datendateien}{shrink_datafile}{0.33}
          f\"ur den Vorgang des Schrumpfens stehen drei verschiedene Optionen zur
          Verfügung:
          \begin{itemize}
              \item \textbf{Nicht verwendeten Speicherplatz freigeben}: Es wird
              nur der Speicherplatz am Ende der Datendatei freigegeben. Es
              werden keine Speicherseiten umsortiert.
              \item \textbf{Seiten vor dem Freigeben von nicht verwendetem
              Speicherplatz neu organisieren}: Vor dem Verkleinern werden zuerst
              alle Speicherseiten reorganisiert, so dass der Speicherplatz am
              Ende der Datendatei möglichst frei wird. Anschließend wird
              verkleinert. Wird zusätzlich die Option \identifier{Datei
              verkleiner auf} aktiviert versucht SQL Server die Datendatei auf
              den angegebenen Wert zu verkleiner, sofern dies möglich ist.
              \item \textbf{Datei durch Migrieren ihrer Daten zu anderen Dateien
              in der gleichen Dateigruppe leeren}: Es werden alle Speicherseiten
              in andere Datendateien der gleichen Dateigruppe verschoben, so
              dass die betroffene Datendatei am Ende leer ist.
          \end{itemize}
          \begin{merke}
            Da beim Verkleineren einer Datendatei die Speicherseiten am Ende der
            Datei nach vorne umsortiert werden, entsteht als Resultat eine hohe
            Indexfragmentierung. Dies kann die Performance von Abfragen massiv
            beeinflussen.
          \end{merke}
      \subsection{Datenbanken trennen und einhängen}
      \label{movedatabases}
        \subsubsection{Datenbanken trennen}
          Dem Administrator ist es möglich, eine Datenbank vollständig von einer
          SQL Server-Instanz zu trennen. Dies kann in verschiedenen Situationen
          nützlich sein, z. B. wenn eine Datenbank auf einen anderen Server
          umgezogen werden soll oder wenn eine Datenbank aus der Nutzung genommen
          werden muss, um sie außerhalb der Arbeitszeiten dann zu löschen.
          Außerdem ist es möglich, eine zugroß gewordene Log-Datei durch das
          Trennen und Anfügen der Datenbank auf ihre Mindestgröße zu schrumpfen.
          
          Beim Vorgang des trennens werden alle Spuren der Datenbank aus der
          Instanz gelöscht. Einzig ein Sicherungsverlauf in der
          \identifier{msdb}-Datenbank kann übrig bleiben. Das Trennen der
          Datenbank von ihrer Instanz geschieht mit der Stored Procedure
          \identifier{sp\_detach\_db}, die f\"ur Ihre Arbeit nur den Datenbanknamen
          als Parameter benötigt.
          
          Um eine Datenbank trennen zu können, müssen die folgenden
          Voraussetzungen gegeben sein:
          \begin{itemize}
            \item Die Datenbank darf aktuell nicht in Verwendung sein,
            \item sie darf keiner Replikation angehören,
            \item es darf keinen Database-Snapshot geben,
            \item sie darf nicht gespiegelt sein,
            \item es darf keine Systemdatenbank sein,
            \item die Datenbank darf nicht \enquote{fehlerverdächtig} sein.
          \end{itemize}
          \begin{lstlisting}[language=ms_sql,caption={Trennen der Datenbank
          demo\_grafisch},label=admin03_20]
USE master
GO

EXEC §sp_detach_db§ @dbname = 'demo_grafisch'
GO
          \end{lstlisting}
          \begin{literaturinternet}
            \item \cite{ms188031}
          \end{literaturinternet}
        \subsubsection{Datenbanken anfügen}
          Zum Anfügen einer Datenbank existieren in SQL Server 2014 zwei
          Möglichkeiten:
          \begin{itemize}
              \item Die Stored Procedure \identifier{sp\_attach\_db}. Sie
              gilt als veraltet und sollte daher nicht mehr benutzt werden.
              Außerdem hat sie ein Limit von max. 16 Datendateien, aus denen die
              anzuhängende Datenbank bestehen darf.
              \item Das SQL-Kommando \languagemssql{CREATE DATABASE ... FOR
              ATTACH} existiert seit SQL Server 2008 und stellt nun die von
              Microsoft empfohlene Methode zum Anfügen von Datenbanken dar. f\"ur
              diese Kommando existieren keinerlei Einschränkungen bezüglich der
              Datenbankgröße oder der Anzahl der Datendateien einer Datenbank.
          \end{itemize}
          In \beispiel{admin03_21} wird die Datenbank
          \identifier{demo\_grafisch} wieder an ihre SQL Server-Instanz
          angefügt. Obwohl sie aus vier Datendateien und einer Log-Datei
          besteht, genügt es, Pfad + Dateiname der primären Datendatei
          anzugeben, da dort die Metadaten der anderen Dateien gespeichert sind.
          \begin{merke}
            Sollte beim Trennen der Datenbank deren Log-Datei beschädigt oder
            nicht vorhanden gewesen sein, wird beim wiederanfügen der Datebank
            eine neue Log-Datei erstellt. Voraussetzung ist, dass die Datenbank
            nicht schreibgeschützt ist.
          \end{merke}
\clearpage
          \begin{lstlisting}[language=ms_sql,caption={Anfügen der Datenbank
          demo\_grafisch},label=admin03_21]
USE master 
GO

CREATE DATABASE demo_grafisch
ON 
(
 FILENAME = 'D:\data\demo_grafisch\primary01.mdf'
)
FOR ATTACH
GO
          \end{lstlisting}
          \begin{merke}
            Um eine Datenbank anfügen zu können, müssen alle \oscommand{.mdf}-
            und \oscommand{.ndf}-Dateien verfügbar sein!
          \end{merke}
          f\"ur schreibgeschützte Datenbanken gilt eine Besonderheit: Da die
          \oscommand{.mdf}-Datendatei nicht geändert werden kann, kann auch
          keine Log-Datei neu erstellt werden. In so einem Fall muss zwingend
          die bestehende Log-Datei mit angegeben werden.
          \begin{lstlisting}[language=ms_sql,caption={Anfügen der
          schreibgeschützten Datenbank demo\_grafisch},label=admin03_22]
USE master 
GO

CREATE DATABASE demo_grafisch
ON 
(
  FILENAME = 'D:\data\demo_grafisch\primary01.mdf'
),
(
  FILENAME = 'D:\data\demo_grafisch\demo_grafisch_log.ldf'
)
FOR ATTACH
GO
          \end{lstlisting}    
          \begin{literaturinternet}
            \item \cite{ms190794}
          \end{literaturinternet}
      \subsection{Systemdatenbanken verschieben}
        \subsubsection{Verschieben Systemdatenbanken}
          Da Systemdatenbanken nicht einfach ausgehängt werden können, muss ein
          anderes Verfahren angewandt werden, wenn sie verschoben werden
          sollen.
          \begin{merke}
            Die einzige Systemdatenbank die nicht verschoben werden kann ist die
            \identifier{resource}-Datenbank.
          \end{merke}
          Der folgende Ablauf zeigt das Verschieben einer Systemdatenbank am
          Beispiel der \identifier{msdb}-Datenbank.
          \begin{enumerate}
              \item Führen Sie ein \languagemssql{ALTER DATABASE ... MODIFY
              FILE}-Kommando f\"ur alle Daten- und Log-Dateien der
              \identifier{msdb}-Datenbank durch und geben Sie dabei den neuen
              Pfad der Dateien an.
              \item Stop Sie die SQL Server-Instanz
              \item Verschieben Sie die Dateien an ihren neuen Speicherort
              \item Starten Sie die SQL Server-Instanz erneut
              \item Benutzen Sie die Katalogsicht
              \identifier{sys.master\_files}, um das Ergebnis des
              Verschiebevorganges zu verifizieren.
                \begin{lstlisting}[language=ms_sql,caption={Abfragen
                der View \identifier{sys.master\_files}},label=admin03_23]
USE
master GO

SELECT name AS logical_name, physical_name AS new_location, 
       state_desc as state
FROM   sys.master_files
WHERE  database_id = DB_ID('msdb')
                \end{lstlisting}
          \end{enumerate}
        \subsubsection{Verschieben der Master-Datenbank}
          Die Technik zum Verschieben der \identifier{master}-Datenbank ähnelt
          dem Verschiebevorgang f\"ur alle anderen Systemdatenbanken sehr. Der
          wesentliche Unterschied besteht darin, dass nach dem Herunterfahren
          des SQL Server-Dienstes der Startparameter \oscommand{-d} geändert
          werden muss, so dass dieser den neuen Speicherort der Datenbank
          angibt.
          \begin{enumerate}
              \item Stoppen Sie die SQL Server-Instanz im Database Configuration
              Assistant.
              \item Verschieben Sie die Dateien an ihren neuen Speicherort
              \item Ändern Sie den Startparameter \oscommand{-d} so, dass er den
              neuen Speicherort der \identifier{master}-Datenbank angibt.
              \item Starten Sie die SQL Server-Instanz neu.
              \item Benutzen Sie Katalogsicht die
              \identifier{sys.master\_files}, um das Ergebnis des
              Verschiebevorganges zu verifizieren.
                \begin{lstlisting}[language=ms_sql,caption={Abfragen
                der View \identifier{sys.master\_files}},label=admin03_24]
USE
master GO

SELECT name AS logical_name, physical_name AS new_location, 
       state_desc as state
FROM   sys.master_files
WHERE  database_id = DB_ID('master')
                \end{lstlisting}
          \end{enumerate}
      \subsection{Datenbanken löschen}
        \subsubsection{Löschen von Datenbanken und Snapshots}
          SQL Server bietet das SQL-Kommando \languagemssql{DROP DATABASE} zum
          löschen von Datenbanken und Datenbank-Snapshots an. Es kann sowohl Datenbanken
          von einem Microsoft SQL Server, als auch von einem Windows Azure SQL-Datenbank-Server löschen.
          Die Syntax dieses Kommandos ist sehr einfach, wie \beispiel{admin03_23}
          zeigt.
          \begin{lstlisting}[language=ms_sql,caption={Löschen der
            Datenbank demo\_sql},label=admin03_25]
USE master 
GO

DROP DATABASE demo_sql
GO 
          \end{lstlisting}
          Lediglich der Datenbankname ist zum Löschen der Datenbank notwendig. Der
          Status einer Datenbank (offline, fehlerverdächtig, \ldots) ist f\"ur das
          Löschen nicht von Bedeutung. Soll eine einmal gelöschte Datenbank wieder
          an den Server angefügt werden, kann dies nur mittels eines Backups
          geschehen.
          \begin{merke}
            Das Löschen einer Datenbank beeinhaltet auch das löschen der
            Daten- und der Log-Dateien. Befindet sich eine Datenbank im Status
            \identifier{offline}, werden deren Datendateien nicht automatisch
            gelöscht.
          \end{merke}
          Als Alternative zum \languagemssql{DROP DATABASE}-Kommando kann auch
          das SSMS genutzt werden. Der Dialog zum Löschen der Datenbank ist über
          das Kontextmenü verfügbar.
          \bild{Löschen einer Datebank im SSMS}{drop_database_ssms}{0.3}          
          \subsubsection{Einschränkungen beim Löschen
          von Datenbanken und Snapshots}
          \begin{itemize}
              \item Es können keine Systemdatenbanken gelöscht werden.
              \item Datenbanken, zu denen Benutzerverbindungen bestehen, können
              nicht gelöscht werden.
              \item Vor dem Löschen einer Datenbank müssen zuerst alle
              Datenbank-Snapshots gelöscht werden.
              \item Das \languagemssql{DROP DATABASE}-Kommando muss im
              Autocommitt-Modus ausgeführt werden.
          \end{itemize}
    \section{Der DBCC - Database Console Commands}
      Hinter der Abkürzung DBCC - Database Console Commands - verbergen sich
      eine ganze Reihe von Befehlen zu unterschiedlichesten Zwecken.
      Ursprünglich waren diese Kommandos f\"ur die Softwaretesting Teams von
      Microsoft geschaffen worden. Mit ihrer Hilfe sollte es möglich
      werden, Aufgaben wie Informationsgewinnung, Debbuging und Testing auf
      einfache und schnelle Art und Weise zu erledigen. Nach und nach wuchs
      der Vorrat an DBCC-Kommandos und einige wurden sogar dokumentiert und
      f\"ur die Nutzung durch SQL Server-Administratoren freigegeben. Viele
      DBCC-Befehl sind aber auch heute noch undokumentiert und somit nur der
      Benutzung durch das Supportpersonal von Microsoft vorbehalten.
      \begin{merke}
        Die Benutzung eines undokumentierten DBCC-Kommandos durch einen SQL
        Server-Administrator kann, im Falle eines Fehlers, den Verlust der
        Supportberechtigung zur Folge haben. Deshalb sollten niemals
        undokumentierte DBCC-Kommandos auf einer operativen Datenbank
        eingesetzt werden.
      \end{merke}
      In SQL Server 2014 sind aktuell 30 verschiedene DBCC-Befehle
      dokumentiert. Diese können in vier verschiedene Kategorien unterteilt
      werden:
      \begin{itemize}
          \item Verwaltung
          \item Information
          \item Überprüfung
          \item Sonstiges
      \end{itemize}
      Alle DBCC-Kommandos nehmen Argumente entgegen, mit deren Hilfe das
      Verhalten des jeweiligen Kommandos gesteuert werden kann.
      \subsection{Überprüfungskommandos}
        DBCC kennt  insgesamt sieben dokumentierte Anweisungen zur Überprüfung
        der Datenbank bzw. von Teilen der Datenbank. Hier eine kleine Auswahl.
        \begin{itemize}
          \item \languagemssql{CHECKDB}: Dient zur Überprüfung der
          physischen und der logischen Integrität aller Objekte innerhalb
          einer Datenbank.
          \item \languagemssql{CHECKFILEGROUP}: Verhält sich genauso, wie
          \languagemssql{?DBCC? CHECKDB}, nur dass die Auswirkungen auf eine
          Dateigruppe begrenzt bleiben.
          \item \languagemssql{CHECKTABLE}: Wie \languagemssql{?DBCC?
          CHECKDB}, aber nur f\"ur die angegebene Tabelle.
          \item \languagemssql{CHECKCONSTRAINTS}: Überprüft die
          Funktionsfähigkeit eines bestimmten oder aller Constraints auf
          einer Tabelle.
        \end{itemize}
        \subsubsection{Konsistenzprüfung der Datenbank mit CHECKDB}
          Mit dem Kommando \languagemssql{CHECKDB} wird die Datenkonsistenz aller
          Objekte einer Datenbank geprüft. Um dieses Kommando ausführen zu können,
          muss der Benutzer entweder Mitglied in der festen Serverrolle
          \identifier{sysadmin} oder der festen Datenbankrolle
          \identifier{db\_owner} sein.
          
          Der Befehl \languagemssql{DBCHECK} ist ein \enquote{Sammelkommando}, was
          bedeutet, dass bei seiner Ausführung im Hintergrund 
          \begin{itemize}
            \item \languagemssql{?DBCC? CHECKCATALOG},
            \item \languagemssql{?DBCC? CHECKALLOC} und
            \item \languagemssql{?DBCC? CHECKTABLE}
          \end{itemize}
          ausgeführt werden. Zusätzlich zu diesen drei Anweisungen werden noch
          einige weitere Prüfungen durchgeführt, die hier unerwähnt bleiben.
          
          Die Syntax zur Ausführung von \languagemssql{?DBCC? CHECKDB} lautet
          (es handelt sich hier um ein auf das Wesentliche gekürztes
          Syntaxdiagram:
          \begin{lstlisting}[language=ebnf, caption={Die Syntax zu CHECKDB},
          label=admin03_28]
DBCC CHECKDB
  [
    [ |(| database_name / database_id / 0 
      [ , ( §REPAIR_ALLOW_DATA_LOSS§ / §REPAIR_FAST§ / §REPAIR_REBUILD§ ) ]
    |)| ]
    [ WITH
      {
        [ §ESTIMATEONLY§ ]
        [ , ( §PHYSICAL_ONLY§ / §DATA_PURITY§ ) ]
      }
    ]
  ]        
          \end{lstlisting}
          \begin{itemize}
            \item \textbf{database\_name / database\_id / 0}: Legt anhand des
            Namens oder einer ID fest, welche Datenbank geprüft werden soll.
            Wird dieses Argument nicht angegeben, oder wird eine 0 angegeben, so
            wird die aktuelle Datenbank überprüft.
            \item \textbf{ESTIMATEONLY}: Es wird berechnet, wie viel
            Speicherplatz in der \identifier{tempdb} benötigt wird, um das
            \languagemssql{CHECKDB}-Kommando auszuführen. Durch die Angabe
            dieses Parameters wird die Anweisung nicht ausgeführt.
            \item \textbf{PHYSICAL\_ONLY}: Es werden nur physikalische
            Überprüfungen durchgeführt. Dadurch wird die Ausführungszeit des
            DBCC-Kommandos deutlich reduziert.
            \item \textbf{DATA\_PURITY}: Kommando zur Durchführung einer
            logischen Konsistenzprüfung von Spaltenwerten, z. B. ob ein
            Datums-/Zeitwert innerhalb der gültigen Grenzen einer Spalte vom Typ
            \identifier{datetime} liegt. Diese Option wird nur benötigt, wenn
            eine Datenbank mit einer früheren Version von SQL Server erstellt
            wurde. Nach einmaliger und erfolgreicher Ausführung dieser Prüfung
            muss diese Option nie wieder angegeben werden.
          \end{itemize}
          \begin{lstlisting}[language=ms_sql,caption={Physikalische
          Konsistenzprüfung einer Datenbank mit DBCC},label=admin03_29]
  USE [master]
  GO
  
  ?DBCC? CHECKDB ('demo_grafisch') WITH PHYSICAL_ONLY
  
  Von CHECKDB wurden 0 Zuordnungsfehler und 0 Konsistenzfehler 
  in der demo_grafisch-Datenbank gefunden.
  Die DBCC-Ausf\"uhrung wurde abgeschlossen. Falls DBCC Fehlermeldungen
  ausgegeben hat, wenden Sie sich an den Systemadministrator.
          \end{lstlisting}         
          \begin{lstlisting}[language=ms_sql,caption={Konsistenzprüfung einer
          Datenbank mit DBCC},label=admin03_30]
  USE [master]
  GO
  
  ?DBCC? CHECKDB ('demo_grafisch')
  
  DBCC-Ergebnis fuer 'demo_grafisch'.
  Service Broker-Meldung 9675, Status 1: Analysierte Nachrichtentypen: 14.
  Service Broker-Meldung 9676, Status 1: Analysierte Dienstvertraege: 6.
  Service Broker-Meldung 9667, Status 1: Analysierte Dienste: 3.
  Service Broker-Meldung 9668, Status 1: Analysierte Dienstwarteschlangen: 3.
  Service Broker-Meldung 9669, Status 1: Analysierte Konversationsendpunkte: 0.
  Service Broker-Meldung 9674, Status 1: Analysierte Konversationsgruppen: 0.
  Service Broker-Meldung 9670, Status 1: Analysierte Remotedienstbindungen: 0.
  Service Broker-Meldung 9605, Status 1: Analysierte Konversationspriorit\"aten:0.
  DBCC-Ergebnis f\"ur 'sys.sysrscols'.
  Es sind 1086 Zeilen auf 14 Seiten f\"ur das sys.sysrscols-Objekt vorhanden.
  DBCC-Ergebnis f\"ur 'sys.sysrowsets'.
  Es sind 146 Zeilen auf 3 Seiten f\"ur das sys.sysrowsets-Objekt vorhanden.
  DBCC-Ergebnis f\"ur 'sys.sysclones'.
  Es sind 0 Zeilen auf 0 Seiten f\"ur das sys.sysclones-Objekt vorhanden.
  DBCC-Ergebnis f\"ur 'sys.sysallocunits'.
  Es sind 169 Zeilen auf 2 Seiten f\"ur das sys.sysallocunits-Objekt vorhanden.
  
  Ergebnis gek\"urzt!!!
  
  Von CHECKDB wurden 0 Zuordnungsfehler und 0 Konsistenzfehler 
  in der demo_grafisch-Datenbank gefunden.
  Die DBCC-Ausf\"uhrung wurde abgeschlossen. Falls DBCC Fehlermeldungen
  ausgegeben hat, wenden Sie sich an den Systemadministrator.
          \end{lstlisting}
          Zur Ausführung von \languagemssql{CHECKDB} wird ein interner Database
          Snapshot angelegt, eine Art \enquote{Abbildung der Datenbank},
          vergleichbar mit den Snapshots im Bereich von Virtualisierungssoftware.
          Dies ermöglicht es dem SQL Server die Konsistanzprüfung der Datenbank so
          durchzuführen, dass die arbeitenden Benutzer nicht gestört werden.
          
          Sollte \languagemssql{CHECKDB} Fehler in der Datenbank gefunden haben,
          wird eine Logdatei mit dem Namen \oscommand{SQLDUMPnnnn.txt} im
          \oscommand{LOG}-Verzeichnis angelegt.
        
          Die drei im Sytaxdiagram angegebenen Optionen
          \languagemssql{REPAIR_ALLOW_DATA_LOSS}, \languagemssql{REPAIR_FAST} und
          \languagemssql{REPAIR_REBUILD} bieten die Möglichkeit gefundene Fehler
          direkt zu reparieren. Es wird jedoch empfohlen, diese Optionen als das
          allerletzte Mittel zu betrachten, da durch ihre Anwendung weitere
          Dateninkonsistenzen entstehen könnten. Statt dessen sollten Backups
          genutzt werden, um fehlerhafte Datendateien wiederherzustellen.
        \subsubsection{Konsistenzprüfung einer einzelnen Dateigruppe}
          Die beiden Anweisungen \languagemssql{CHECKDB} und
          \languagemssql{CHECKFILEGROUP} sind sich in ihrer Anwendung sehr
          ähnlich. Der wesentliche Unterschied besteht darin, dass
          \languagemssql{CHECKDB} alle Dateigruppen, inklusive der Protokolldatei,
          einer Datenbank prüft, während mit \languagemssql{CHECKFILEGROUP}
          gezielt eine Dateigruppe geprüft werden kann. Die Syntax for
          \languagemssql{CHECKFILEGROUP} lautet wie folgt:
          \begin{lstlisting}[language=ebnf, caption={Die Syntax zu
          CHECKFILEGROUP}, label=admin03_31]
DBCC §CHECKFILEGROUP§
  [
    [ |(| filegroup_name / filegroup_id / 0 
    |)| ]
    [ §WITH§
      {
        [ §ESTIMATEONLY§ ]
        [ , §PHYSICAL_ONLY§ ]
      }
    ]
  ]        
        \end{lstlisting}
        \begin{lstlisting}[language=ms_sql,caption={Konsistenzprüfung
        einer Dateigruppe},label=admin03_32] 
USE [demo_grafisch]
GO

?DBCC? CHECKFILEGROUP ('CRM')

DBCC-Ergebnis f\"ur 'demo_grafisch'.
DBCC-Ergebnis f\"ur 'sys.sysrscols'.
Es sind 1086 Zeilen auf 14 Seiten f\"ur das sys.sysrscols-Objekt vorhanden.
DBCC-Ergebnis f\"ur 'sys.sysrowsets'.
Es sind 146 Zeilen auf 3 Seiten f\"ur das sys.sysrowsets-Objekt vorhanden.

Ergebnis gek\"urzt!!!

Von CHECKFILEGROUP wurden 0 Zuordnungsfehler und 0 Konsistenzfehler
in der demo_grafisch-Datenbank gefunden.
Die DBCC-Ausf\"uhrung wurde abgeschlossen. Falls DBCC Fehlermeldungen 
ausgegeben hat, wenden Sie sich an den Systemadministrator.
        \end{lstlisting}
        \subsubsection{Einzelne Tabellen überprüfen}
          Mit \languagemssql{?DBCC? CHECKTABLE} kann eine einzelne Tabelle auf
          ihre Konsistenz geprüft werden. Die Syntax dieser Anweisung gleicht
          der Syntax der \languagemssql{CHECKDB}-Abweisung.
\clearpage
          \begin{lstlisting}[language=ms_sql,caption={Konsistenzprüfung
          einer einzelnen Tabelle},label=admin03_33] 
USE [demo_grafisch]
GO

?DBCC? CHECKTABLE ('demo_table')

DBCC-Ergebnis f\"ur 'demo_grafisch.demo_table'.
Es sind 99 Zeilen auf 2 Seiten f\"ur das demo_table-Objekt vorhanden.
Die DBCC-Ausf\"uhrung wurde abgeschlossen. Falls DBCC Fehlermeldungen 
ausgegeben hat, wenden Sie sich an den Systemadministrator.
          \end{lstlisting}         
          \begin{literaturinternet}
            \item \cite{ms176061}
            \item \cite{ms187332}
            \item \cite{ms174338}
            \item \cite{ms189496}
          \end{literaturinternet}
      \subsection{Verwaltungsanweisungen}
        Mit den DBCC-Befehlen aus der Gruppe der Verwaltungsanweisungen lassen
        sich administrative Tätigkeiten an Datenbank, Datendatei, Tabellen und
        Indizes durchführen. Die zwei wichtigsten Vertreter dieser Gruppe
        sind:
        \begin{itemize}
          \item \languagemssql{SHRINKDATABASE}: Dient zur Reduzierung der
          Größe von Daten- und Protokolldateien einer Datenbank.
          \item \languagemssql{SHRINKFILE}: Reduziert die Größe einer Daten-
          oder Protokolldatei.
       \end{itemize}
        \subsubsection{Schrumpfen der Datenbank}
          Mit dem Kommando \languagemssql{SHRINKDATABASE} kann der Administrator
          sowohl die Daten- als auch die Protokolldateien einer Datenbank
          schrumpfen.
          \begin{lstlisting}[language=ebnf, caption={Die Syntax zu
          SHRINKDATABASE}, label=admin03_34]
DBCC §SHRINKDATABASE§
    |(| database_name / database_id / 0 
      [ , target_percent ]
      [ , ( §NOTRUNCATE§ / §TRUNCATEONLY§ ) ]
    |)| 
          \end{lstlisting}
          \begin{itemize}
            \item \textbf{database\_name / database\_id / 0}: Legt anhand des
            Namens oder einer ID fest, welche Datenbank geschrumpft werden soll.
            Wird dieses Argument nicht angegeben, oder wird eine 0 angegeben, so
            wird die aktuelle Datenbank bearbeitet.  
            \item \textbf{NOTRUNCATE}: \enquote{Defragmentiert} den
            Inhalt Datendatei, verkleinert die Datei aber nicht. Es wird
            kein Speicherplatz freigegeben, es werden lediglich Datenseite
            verschoben. Ein mit der Option \languagemssql{NOTRUNCATE}
            ausgeführtes \languagemssql{?DBCC? SHRINKDATABASE} wirkt sich nur
            auf die Datendateien, aber nicht auf die Protokolldatei aus.
            \item \textbf{TRUNCATEONLY}: Der gesamte am Ende einer
            Datei befindliche Speicherplatz wird freigegeben/abgeschnitten.
            Dateien werden somit verkleinert, es findet jedoch keine
            Defragmentierung innerhalb einer Datei statt. Durch die Angabe des
            Schlüsselwortes \languagemssql{TRUNCATEONLY} wirkt sich
            \languagemssql{?DBCC? SHRINKDATABASE} auch auf die Protokolldatei aus.
            \item \textbf{target\_percent}: Menge an freiem Speicher, der nach
            der Verkleinerung in der Datendatei enthalten sein soll.
          \end{itemize}
          \begin{merke}
            Die Größe einer Datenbank kann niemals unter ihre Mindestgröße
            geschrumpft werden. Wurde eine Datenbank z. B. mit einer Größe von 10
            MB angelegt, kann sie niemals kleiner als 10 MB werden.
          \end{merke}
          Das Schrumpfen einer Datenbank ist ein Online-Vorgang, d. h. während die
          Datenbank verkleinert wird können andere Nutzer ihre Arbeit fortsetzen.
          
          Weiterhin ist zu beachten, das ein Ausführen von
          \languagemssql{SHRINKDATABASE} ohne die Angabe von
          \languagemssql{TRUNCATEONLY} und \languagemssql{NOTRUNCATE} wie folgt
          abgearbeitet wird:
          \begin{enumerate}
            \item Ausführen von \languagemssql{SHRINKDATABASE} mit
            \languagemssql{NOTRUNCATE}
            \item Ausführen von \languagemssql{SHRINKDATABASE} mit
            \languagemssql{TRUNCATEONLY}
          \end{enumerate}
          \begin{merke}
            Werden die beiden Optionen \languagemssql{TRUNCATEONLY} und
            \languagemssql{NOTRUNCATE} weggelassen, wird das
            \languagemssql{SHRINKDATABASE}-Kommando zweimal ausgeführt. Zuerst mit
            \languagemssql{NOTRUNCATE} und anschließend mit
            \languagemssql{TRUNCATEONLY}
          \end{merke}        
          \begin{lstlisting}[language=ms_sql,caption={Verkleinern einer
          Datenbank},label=admin03_35]
USE [master] GO

?DBCC? SHRINKDATABASE(demo_grafisch, 25)
          \end{lstlisting}          
          Das \languagemssql{SHRINKDATABASE}-Kommando aus \beispiel{admin03_35}
          verkleinert die Datendateien der Datenbank \identifier{demo\_grafisch}
          um 25 \%. Wenn z. B. eine Datendatei eine Größe von 20 MB und
          eine Mindestgröße von 15 MB hat und mit 8 MB Daten gefüllt ist, wird SQL
          Server die Datei auf 16 MB verkleinern. Somit verbleiben 8 MB freier
          Speichern in der Datendatei. Soll diese Datei jedoch um 30 \%
          geschrumpft werden, kann SQL Server dies nicht, da sonst die
          Mindestgröße unterschritten werden würde.
        
          Bei Protokolldateien verhält sich \languagemssql{SHRINKDATABASE} etwas
          anders. Dort gibt \languagemssql{target\_percent} die Endgröße des
          Transaktionsprotokolls an und nicht die Menge an freiem Speicher nach
          dem Verkleinern. Besitzt die Datenbank mehr als nur eine Protokolldatei,
          wird f\"ur jede einzelne Protokolldatei deren Zielgröße berechnet, damit
          das Protokoll insgesamt auf die von \languagemssql{target\_percent}
          definierte Größe kommt.
          \begin{merke}
            Eine Protokolldatei kann immer nur bis zu den Grenzen eines Virtual
            Log Files verkleinert werden. Insgesamt sollte es vermieden werden
            eine Protokolldatei zu schrumpfen, da sich dadurch die Anzahl und die
            Größe der VLFs in ihr verändert.
          \end{merke}
          Am effektivsten ist ein Verkleinerungsvorgang, wenn erst kürzlich viel
          Speicherplatz in einer Datenbank (z. B. durch \languagemssql{DELETE}
          oder \languagemssql{TRUNCATE}) freigegeben wurde. Da das Schrumpfen
          einer Datenbank aber auch entscheidende negative Aspekte besitzt sollte
          es vermieden werden eine Datenbank wiederholt zu schrumpfen.
        \subsubsection{Schrumpfen einer Datendatei}
          Wenn gezielt einzelne Daten- oder Protokolldateien geschrumpft werden
          sollen kann dies mit der \languagemssql{SHRINKFILE}-Anweisung des DBCC
          geschehen. Sie unterscheidet sich in einigen Punkten von der
          \languagemssql{SHRINKDATABASE}-Anweisung:
          \begin{itemize}
              \item Die Zielgröße wird bei \languagemssql{SHRINKFILE} in MB und
              nicht in Prozent angegeben.
              \item Die Zielgröße bezeichnet bei \languagemssql{SHRINKFILE} die
              Tatsächliche Größe der Datei nach dem Verkleinerungsvorgang.
              \item Mit \languagemssql{SHRINKFILE} kann die Mindestgröße einer
              Datendatei verringert werden.
          \end{itemize}
          Die Syntax f\"ur \languagemssql{SHRINKFILE} lautet:
          \begin{lstlisting}[language=ebnf, caption={Die Syntax zu
          SHRINKFILE}, label=admin03_36]
DBCC §SHRINKFILE§
    |(| { file_name / file_id }
        { , [ §EMPTYFILE§ ] / 
            [ 
              [ , target_size ] [ , { §NOTRUNCATE§ / §TRUNCATEONLY§ } ]
            ]
         }
    |)|
          \end{lstlisting}
          \begin{itemize}
            \item \textbf{file\_name / file\_id}: Legt anhand des
            Namens oder einer ID fest, welche Datei geschrumpft werden soll.
            \item \textbf{EMPTYFILE}: Verlagert, wenn möglich, alle Daten aus
            einer Datendatei in die anderen, um eine Datendatei komplett
            freizugeben.
            \item \textbf{NOTRUNCATE}: \enquote{Defragmentiert} den
            Inhalt einer Datendatei, verkleinert die Datei aber nicht. Es wird
            kein Speicherplatz freigegeben, es werden lediglich Datenseite
            verschoben. Ein mit der Option \languagemssql{NOTRUNCATE}
            ausgeführtes \languagemssql{?DBCC? SHRINKDATABASE} wirkt sich nur
            auf die Datendateien, aber nicht auf die Protokolldatei aus.
            \item \textbf{TRUNCATEONLY}: Der gesamte, am Ende einer
            Datei befindliche Speicherplatz wird freigegeben/abgeschnitten.
            Dateien werden somit verkleinert, es findet jedoch keine
            Defragmentierung innerhalb einer Datei statt. Durch die Angabe
            des Schlüsselwortes \languagemssql{TRUNCATEONLY} wirkt sich
            \languagemssql{?DBCC? SHRINKDATABASE} auch auf die Protokolldatei aus.
            \item \textbf{target\_size}: Größe der Datei nach der
            Verkleinerung.
          \end{itemize}
          Das Schrumpfen einer Protokolldatei verläuft analog zum Kommando
          \languagemssql{SHRINKDATABASE}.
          \begin{literaturinternet}
            \item \cite{ms190488}
            \item \cite{ms189493}
          \end{literaturinternet}
      \subsection{Informationsanweisungen}
        In der Gruppe der Informationsanweisungen gibt es eine Reihe von
        DBCC-Kommandos die Informationen zu offenen Transaktionen, Auslastung
        des Transaktionsprotokolls, SQL Statistiken uvm. anzeigen.
        \subsubsection{Die Auslastung des Transaktionsprotokolls anzeigen}
          Mit dem Kommando \languagemssql{SQLPERF} kann zu jeder Datenbank einer
          SQL Server-Instanz die Auslastung ihrer Protokolldatei angezeigt
          werden.
          \begin{lstlisting}[language=ebnf, caption={Die Syntax zu
          SQLPERF}, label=admin03_37]
DBCC SQLPERF |(| §LOGSPACE§ |)|
          \end{lstlisting}
          \begin{lstlisting}[language=ms_sql,caption={Anzeigen
          der Protokolldateiauslastung},label=admin03_36]
USE [master]
GO

?DBCC? (LOGSPACE)
          \end{lstlisting}
\clearpage
          \begin{center}
            \begin{small}
              \changefont{pcr}{m}{n}
              \tablefirsthead {
                \multicolumn{1}{l}{\textbf{Database Name}} &
                \multicolumn{1}{l}{\textbf{Log Size (MB)}} & 
                \multicolumn{1}{l}{\textbf{Log Space Used (\%)}} & 
                \multicolumn{1}{l}{\textbf{Status}} \\
                \cmidrule(l){1-1}\cmidrule(l){2-2}\cmidrule(l){3-3}\cmidrule(l){4-4}
                }
              \tablehead{}
              \tabletail {
              }
              \tablelasttail {
              }
              \begin{mssql}
                \begin{supertabular}{llll}
                  master             & 2,242188  & 34,77787 & 0 \\
                  tempdb             & 0,7421875 & 48,09211 & 0 \\
                  model              & 0,9921875 & 34,49803 & 0 \\
                  msdb               & 19,61719  & 11,27788 & 0 \\
                  ReportServer       & 6,867188  & 22,7318  & 0 \\
                  ReportServerTempDB & 1,039063  & 60,05639 & 0 \\
                  demo\_grafisch     & 5,554688  & 48,31224 & 0 \\
                \end{supertabular}
              \end{mssql}
            \end{small}
          \end{center}
          \begin{literaturinternet}
            \item \cite{ms189768}
          \end{literaturinternet}
      \subsection{Sonstige Anweisungen}
        Mit dem Kommando \languagemssql{?DBCC? HELP} kann zu jedem anderen
        DBCC-Befehl eine Syntax-Hilfe abgerufen werden.
        \begin{lstlisting}[language=ms_sql, caption={DBCC Help - Syntaxhilfe
        aufrufen}, label=admin03_38] 
DBCC HELP |(| 'CHECKDB' |)|

dbcc CHECKDB 
(
    { 'database_name' | database_id | 0 }
    [ , NOINDEX
    | { REPAIR_ALLOW_DATA_LOSS
    | REPAIR_FAST
    | REPAIR_REBUILD
    } ]
)
    [ WITH
        {
            [ ALL_ERRORMSGS ]
            [ , [ NO_INFOMSGS ] ]
            [ , [ TABLOCK ] ]
            [ , [ ESTIMATEONLY ] ]
            [ , [ PHYSICAL_ONLY ] ]
            [ , [ DATA_PURITY ] ]
            [ , [ EXTENDED_LOGICAL_CHECKS  ] ]
        }
    ]

Die DBCC-Ausf\"uhrung wurde abgeschlossen. 
Falls DBCC Fehlermeldungen ausgegeben hat, wenden Sie sich 
an den Systemadministrator.
          \end{lstlisting}          
          Mit dem Befehl aus \beispiel{admin03_38} wird eine Syntax-Hilfe zum
          Kommando \languagemssql{?DBCC? CHECKDB} angezeigt.
          \begin{literaturinternet}
            \item \cite{ms176040}
          \end{literaturinternet}
