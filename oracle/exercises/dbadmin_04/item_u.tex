      \item Ändern Sie die Parameter der Archivierung wie folgt:
        \begin{center}
          \begin{small}
            \tablefirsthead{
              \multicolumn{1}{c}{\textbf{Optional}} &
              \multicolumn{1}{c}{\textbf{/u02}} &
              \multicolumn{1}{c}{\textbf{/u03}} &
              \multicolumn{1}{c}{\textbf{/u04}} &
              \multicolumn{1}{c}{\textbf{/u05}} \\
              \hline
            }
            \tablefirsthead{
              \multicolumn{1}{c}{\textbf{Optional}} &
              \multicolumn{1}{c}{\textbf{/u02}} &
              \multicolumn{1}{c}{\textbf{/u03}} &
              \multicolumn{1}{c}{\textbf{/u04}} &
              \multicolumn{1}{c}{\textbf{/u05}} \\
              \hline
            }
            \tabletail{
              \hline
            }
            \tablelasttail {
              \hline
            }
            \begin{supertabular}{| >{\centering\arraybackslash}m{1cm}| >{\centering\arraybackslash}m{2cm}| >{\centering\arraybackslash}m{2cm}| >{\centering\arraybackslash}m{2cm}| >{\centering\arraybackslash}m{2cm}|}
               Ja &  dest 1 &  - &  - & - \\
              \hline
               Nein &  - &  dest 2 &  - & - \\
              \hline
               Ja &  - &  - &  dest 3 & - \\
            \end{supertabular}
          \end{small}
        \end{center}
        \begin{itemize}
          \item Die Archivierung muss an mindestens zwei Speicherorten erfolgreich sein.
          \item Das Dateinamensformat der Archive Log Files muss wie folgt aufgebaut sein: \oscommand{archive\_\%d\_\%t\_\%s\_\%r.arc}
        \end{itemize}
