\section{Lösungen - Recovery mit dem RMAN}
  \begin{enumerate}
    \item Bereiten Sie Ihre SQL Server-Instanz auf die Benutzung von FileStream vor.
Der Bezeichner für die Filestream-Freigabe lautet: \identifier{mssqlserver}. Der
Bezeichner für das Verzeichnis der Datenbank \identifier{bank\_2014 }lautet:
\identifier{bank\_2014}. Aktivieren Sie den nichttransaktionalen Zugriff auf den
Filestream.
      \begin{lstlisting}[caption={Ermitteln des betroffenen Redo Log Members im Alert Log},language=terminal]
Errors in file /u01/app/oracle/diag/rdbms/orcl/orcl/trace/orcl_arc0_3513.trc
ORA-00313: &open& failed for members of log group 5 of thread 1
ORA-00312: online log 5 thread 1: '/u01/app/oracle/oradata/orcl/redo05a.log'
ORA-27037: unable to obtain file status
Linux-x86_64 Error: 2: No such file or directory
      \end{lstlisting}
      \begin{lstlisting}[caption={Recovern des Members},language=terminal]
[oracle@FEA11-119SRV ~]$ cd /u01/app/oracle/oradata/orcl
[oracle@FEA11-119SRV ~]$ cp /u02/oradata/orcl/redo05b.log redo05a.log
      \end{lstlisting}
        \item Registrieren Sie Ihre Datenbank in Ihrem neuen Recovery Katalog!

        \item Wechseln Sie den Undo-Tablespace zurück zum Original Undo-Tablespace und löschen Sie \identifier{undotbs02} und \identifier{undotbs03}.

        \item Erstellen Sie ein unkomprimiertes Backup Set des Tablespaces
    \identifier{bank}! Das Backup Set soll in der FRA (Fast Recovery Area)
    abgelegt werden!

      \begin{lstlisting}[caption={Ermitteln des betroffenen Tempfiles im Alert Log},language=terminal]
Errors in file /u01/app/oracle/diag/rdbms/orcl/orcl/trace/orcl_m000_7480.trc
ORA-01116: error in opening database file 201
ORA-01110: data file 201: '/u01/app/oracle/oradata/orcl/temp01.dbf'
ORA-27041: unable to &open& file
Linux-x86_64 Error: 2: No such file or directory
Additional information: 3
      \end{lstlisting}
      \begin{lstlisting}[caption={Neues Tempfile erstellen},language=oracle_sql]
SQL> ALTER TABLESPACE temp
  2  ADD TEMPFILE '/u02/oradata/ORCL/temp02.dbf'
  3  SIZE 20 M AUTOEXTEND ON MAXSIZE 500M;
      \end{lstlisting}
\clearpage
      \begin{lstlisting}[caption={Beschädigtes Tempfile löschen},language=oracle_sql]
SQL> ALTER TABLESPACE temp
  2  DROP TEMPFILE '/u02/oradata/ORCL/temp01.dbf';
      \end{lstlisting}
    \item Legen Sie die Tabelle \identifier{buchung} gemäß Ihrem ER-Modell als SOT
an und fügen Sie zwischen die beiden Spalten \identifier{buchungsdatum} und
\identifier{konto\_id} die Spalte \identifier{buchungstext} VARCHAR(200) ein!
Legen Sie auf die Spalte \identifier{buchungsdatum} einen Index. Legen Sie auch
auf die Spalte \identifier{buchungstext} einen Index.
      \begin{lstlisting}[caption={Ermitteln der betroffenen Kontrolldatei im Alert Log},language=terminal]
Errors in file /u01/app/oracle/diag/rdbms/orcl/orcl/trace/orcl_ora_2765.trc:
ORA-00210: cannot &open& the specified control file
ORA-00202: control file: '/u01/app/oracle/oradata/orcl/control01.ctl'
ORA-27041: unable to &open& file
Linux-x86_64 Error: 2: No such file or directory
      \end{lstlisting}
      \begin{lstlisting}[caption={Recovern der Kontrolldatei},language=terminal]
[oracle@FEA11-119SRV ~]$ cd /u01/app/oracle/oradata/orcl
[oracle@FEA11-119SRV ~]$ cp control02.ctl control01.ctl
      \end{lstlisting}
          \item Starten und öffnen Sie Ihre Datenbank.

      \begin{lstlisting}[caption={Recovern der Datendatei},language=rman]
RMAN> VALIDATE database;
Starting validate at 03-NOV-13
using target database control file instead of recovery catalog
allocated channel: ORA_DISK_1
channel ORA_DISK_1: SID=21 device type=&DISK&
allocated channel: ORA_DISK_2
channel ORA_DISK_2: SID=144 device type=&DISK&
RMAN-06169: could not read file header for datafile 6 error reason 5
RMAN-00571: ===========================================================
RMAN-00569: =============== ERROR MESSAGE STACK FOLLOWS ===============
RMAN-00571: ===========================================================
RMAN-03002: failure of validate command at 11/03/2013 09:24:26
RMAN-06056: could not access datafile 6
      \end{lstlisting}
\clearpage
      \begin{lstlisting}[language=rman,emph={[9]ALTER,DATABASE, DATAFILE,OFFLINE,ONLINE},emphstyle={[9]\color{magenta}\bfseries}]
RMAN> SQL 'ALTER DATABASE DATAFILE 6 OFFLINE';

RMAN> RESTORE datafile 6;

RMAN> RECOVER datafile 6;

RMAN> SQL 'ALTER DATABASE DATAFILE 6 ONLINE';
      \end{lstlisting}
        \item Konfigurieren Sie die Benennungsmethode Directory Naming für Ihre
    Datenbank. Der Directory Server ist ein OID und wird unter der IP-Adresse
    192.168.111.250 zufinden sein (Ports: 389 und 636). Der LDAP-Context
    lautet: \enquote{cn=OracleContext}. Nutzen Sie für diese Aufgabe den
    Oracle Net Configuration Assistant (\oscommand{netca} - Konfigurieren von
    Directoy-Verwendung).

      \begin{lstlisting}[caption={Recovern der Systemdatendatei},language=rman,alsolanguage=sqlplus,emph={[9]ALTER,DATABASE,OPEN},emphstyle={[9]\color{magenta}\bfseries}]
RMAN> VALIDATE database; 
Starting validate at 03-NOV-13
released channel: ORA_SBT_TAPE_1
using channel ORA_DISK_1
using channel ORA_DISK_2
channel ORA_DISK_1: starting validation of datafile
channel ORA_DISK_1: specifying datafile(s) for validation
RMAN-00571: ===========================================================
RMAN-00569: =============== ERROR MESSAGE STACK FOLLOWS ===============
RMAN-00571: ===========================================================
RMAN-03009: failure of validate command on ORA_DISK_1 channel at 11/03/2013
ORA-01122: database file 1 failed verification check
ORA-01110: data file 1: '/u01/app/oracle/oradata/orcl/system01.dbf'
ORA-01565: error in identifying file 
           '/u01/app/oracle/oradata/orcl/system01.dbf'
ORA-27037: unable to obtain file status
Linux-x86_64 Error: 2: No such file or directory
Additional information: 3

RMAN> shutdown abort

RMAN> startup mount

RMAN> RESTORE datafile 1;

RMAN> RECOVER datafile 1;

RMAN> ALTER DATABASE OPEN;
      \end{lstlisting}
\clearpage
        \item Welche Kosten hat das Statement verursacht und wie viel Zeit wurde für die Ausführung benötigt?

      \begin{lstlisting}[caption={Herunterfahren der Datenbank in SQL*Plus},language=oracle_sql,alsolanguage=sqlplus]
SQL> shutdown abort
      \end{lstlisting}
      \begin{lstlisting}[caption={Recovern der Kontrolldateien},language=rman,alsolanguage=sqlplus,emph={[9]ALTER,DATABASE, MOUNT,OPEN,RESETLOGS},emphstyle={[9]\color{magenta}\bfseries}]
RMAN> SET DBID 1351916467;
RMAN> startup nomount
RMAN> RESTORE controlfile
2>    FROM AUTOBACKUP
3>    RECOVERY AREA '/u05/fast_recovery_area'
4>    DB_NAME = 'orcl';

Starting restore at 03-NOV-13
allocated channel: ORA_DISK_1
channel ORA_DISK_1: SID=134 device type=&DISK&

recovery area destination: /u05/fast_recovery_area
database name (or database unique name) used for search: ORCL
channel ORA_DISK_1: &AUTOBACKUP& /u05/fast_recovery_area/ORCL/autobackup/
channel ORA_DISK_1: looking for &AUTOBACKUP& on day: 20131103
channel ORA_DISK_1: restoring control file from &AUTOBACKUP&
channel ORA_DISK_1: control file restore from &AUTOBACKUP& complete
output file name=/u01/app/oracle/oradata/orcl/control01.ctl
output file name=/u05/fast_recovery_area/orcl/control02.ctl
Finished restore at 03-NOV-13

RMAN> SQL 'ALTER DATABASE MOUNT';

RMAN> RECOVER database;

RMAN> SQL 'ALTER DATABASE OPEN RESETLOGS';
      \end{lstlisting}

        \item Verschieben Sie jetzt das Backup Set des \identifier{bank}-Tablespaces
    auf das SBT-Gerät!

        \item Löschen Sie das Backup Set des \identifier{bank}-Tablespaces von der Festplatte, so dass es nur noch auf SBT verfügbar ist!

      \begin{lstlisting}[caption={Validieren der Datenbank},language=rman]
RMAN> VALIDATE database;

      \end{lstlisting}
        \item Der Tablespace \identifier{bank} liegt derzeit in unverschlüsselter Form vor. Dies muss zukünftig anders sein. Bereiten Sie einen Tablespace \identifier{bank\_encrypted} vor, der mittels \identifier{AES256} verschlüsselung gesichert ist und der die gleichen Dimensionen hat, wie der Originaltablespace \identifier{bank}! Rufen Sie alle notwendigen Informationen über den Tablespace \identifier{bank} aus dem Data Dictionary ab! Welche Views helfen Ihnen dabei?

      \begin{lstlisting}[caption={Validieren der Datenbank},language=rman]
RMAN> RUN {
2>      SET MAXCORRUPT FOR datafile 7 TO 999;
3>      BACKUP AS BACKUPSET DATAFILE 7;
4>    }
      \end{lstlisting}
       \item Führen Sie mit Hilfe von RMAN ein Block-Media-Recovery durch, um die be\-schä\-dig\-ten Blöcke zu reparieren.

      \begin{lstlisting}[caption={Recovern der Datendatei},language=rman]
RMAN> BLOCKRECOVER CORRUPTION LIST;
      \end{lstlisting}
  \end{enumerate}