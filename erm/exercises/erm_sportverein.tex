\subsection{Übungsaufgabe Sportverein}
Entwerfen Sie, basierend auf der folgenden Lage, ein ER-Modell inklusive der Beziehungen zwischen
den Entit\"{a}ten.

Ein Sportverein will zur besseren Verwaltung seiner eigenen Sportabteilungen, Trainer und
Sportler eine Datenbank entwerfen. In der Datenbank soll ersichtlich werden, welcher Trainer
welche Sportart trainiert und welcher Sportler in der jeweiligen Abteilung aktiv ist.

Im Folgenden werden die angesprochenen Datenbankinhalte spezifiziert:
\begin{itemize}
    \item Zu jeder Sportabteilung muss eine eindeutige ID und deren vereinsinterne
          Bezeichnung gespeichert werden.
    \item Für jede Sportart ist eine ID und deren Bezeichnung wichtig. Eine Sportart wird in genau einer
          Abteilung durchgeführt, wobei in einer Abteilung mindestens eine Sportart durchgeführt wird.
    \item Für den jeweiligen Trainer ist der Vor- und Nachname, das Geburtsdatum
          und das Eintrittsdatum in den Verein zu speichern. Ein Trainer trainiert mindestens eine Sportart.
          Eine Sportart wird von höchstens einem Trainer trainiert, wobei es vorkommen kann, dass beim
          Ausscheiden eines Trainers aus dem Verein, eine Sportart kurzzeitig keinen Trainer hat.
    \item Jede Sportart wird von mindestens einem Sportler ausgeübt. Ein Sportler hingegen kann mehrere
          Sportarten ausüben, wobei es keine passiven Mitglieder/Sportler im Verein gibt. Bis auf die
          Trainernummer sind für den Sportler die selben Daten zu erheben, wie für die Trainer.
    \item Jede Sportabteilung hat mindestens einem Trainingsort (Adresse). Es kann sein, dass an
          einem Trainingsort mehrere Sportabteilungen trainieren. Bei der Adressspeicherung sind die
          Postleitzahl (PLZ), der Ort, die Straß e und die Hausnummer relevant.
    \item Jeder Trainer und jeder Sportler wohnen bei genau einer Adresse. Es ist auch möglich,
          dass mehrere Trainer oder Sportler an der selben Adresse wohnen.
\end{itemize}
\clearpage
