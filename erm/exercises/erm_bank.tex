\clearpage
      \subsection{Übungsaufgabe Bank}
        Der Manager einer neu gegründeten Bank beauftragt Sie mit der
        Erstellung eines Datenmodells für die Verwaltung der Bankgeschäfte.
        In der vor kurzem erfolgten Besprechung wurden folgende Eckpunkte
        festgelegt:

        Meine Bank hat mehrere Kunden aber mindestens Einen sonst würde meine
        Bank nicht existieren. Alle meine Kunden besitzen eine Kunden ID, einen
        Vorname, einen Nachnamen, ein Geburtsdatum und eine Rechnungsadresse.

        Kunden können eines oder mehrere Konten besitzen. Jedes Konto hat
        jedoch genau einen Besitzer. Ein Konto ist entweder ein Sparbuch, ein
        Depot oder ein Girokonto. Andere Arten von Konten werden bei uns nicht
        geführt und es gibt auch keine Mischformen dieser drei Kontoarten.
        Jedes Konto soll in unserer Datenbank mit einer IBAN (international bank
        account number) versehen werden. Für die Sparbücher und die
        Girokonten ist auch das aktuelle Guthaben zu speichern. Auf unsere
        Sparbücher gibt es einen Habenzinssatz, für die Girokonten wird ein
        Sollzinssatz geführt, der bei Überziehung des Kontos zum Tragen
        kommt. Wer ein Girokonto besitzt, für den wird jährlich eine
        Kontoführungsgebühr fällig. Bei einem Depot muss eine
        Eröffnungsgebühr gezahlt werden.

        Unsere Kunden können einer anderen Person, die ebenfalls bei uns Kunde
        sein muss, höchstens eine Vollmacht über ein Konto geben. Ein Kunde
        kann mehrere Vollmachten über verschiedene Konten besitzen.

        Ein wesentlicher Bestandteil der Datenbank ist unser Buchungssystem. Auf
        den Konten unserer Kunden erfolgen Buchungen (Einzahlungen,
        Auszahlungen, Überweisungen, usw. NICHT SPEZIALISIEREN!!!), die mit
        einem Betrag und einem Buchungsdatum gespeichert werden müssen. Jede
        Buchung ist immer genau einem Konto zuordenbar, während es für ein
        Konto mehrere Buchungen geben kann. Es muss auch der Fall
        berücksichtigt werden, dass z. B. auf einem neu eröffneten Konto
        noch keine Buchung vorgenommen wurde.

        Damit unsere Kunden auch Geschäfte mit Kunden anderer Banken tätigen
        können, müssen von diesen fremden Personen der Vorname, der Nachname
        und eine IBAN gespeichert werden. Auch benötigen wir den Namen und den
        BIC (bank identity code) der fremden Bank (Hinweis: Hier bietet es sich
        an, die Kunden in Eigenkunden unserer Bank und Fremdkunden zu
        unterteilen!). Es kommt auch vor, dass einer unserer Kunden Kunde bei
        einer anderen Bank ist und Geld von einem seiner Konten auf ein Anderes,
        bei einer anderen Bank, transferieren möchte.

        Die Eigenkunden können von mehreren Mitarbeitern unserer Bank betreut
        werden. Ein Mitarbeiter kann mehrere Kunden betreuen. Die Vorgesetzten
        der Filialleiter, die ebenfalls als Mitarbeiter geführt werden,  haben
        keinen Kundenkontakt. Jeder Mitarbeiter hat genau einen Vorgesetzten,
        auß{}er mir selbst. Ein vorgesetzter Mitarbeiter hat aber mehrere
        Mitarbeiter, die er führt. Meine Mitarbeiter sollen in der Datenbank
        mit Vorname, Nachname und einer Mitarbeiter ID gespeichert werden.

        Jeder Mitarbeiter der Bank arbeitet in einer Bankfiliale, auß{}er den
        Vorgesetzten der Filialleiter. In einer Bankfiliale arbeiten mindestens
        ein aber maximal zehn Mitarbeiter. Die Bankfiliale soll mit ihrer 
        Adresse in der Datenbank gespeichert werden.
\clearpage