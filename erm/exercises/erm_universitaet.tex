\subsection{Übungsaufgabe Universität}
Der Leiter der Abteilung für Verwaltungsangelegenheiten einer
Universität beauftragt Sie mit der Erstellung eines Datenmodells, um
die Verwaltungsstruktur effizienter zu gestalten. In einer Besprechung
wurden folgende Eckpunkte festgelegt:

Meine Universität gliedert sich in Fakultäten. Diese umfassen immer
mindestens ein Institut. Ein Institut kann niemals zu mehreren
Fakultäten gleichzeitig gehören. Für beide müssen deren
Bezeichungen gespeichert werden.

In der Datenbank sollen drei unterschiedliche Personentypen eingetragen
werden. Diese sind mit ihren Vornamen, Nachnamen, der Adresse und dem
Geburtsdatum zu hinterlegen. Der erste Personentyp sind Professoren,
welche genau ein Institut leiten. Umgekehrt kann ein Institut auch nur
von genau einem Professor geleitet werden.

Professoren halten Vorlesungen. Jeder Professor hält mindestens eine
Vorlesung. Der Fall, dass eine Vorlesung von mehreren Professoren
gehalten wird, tritt nicht auf. Vorlesungen werden immer in
Hörsälen gehalten. In einem Hörsaal können mehrere Vorlesungen
gehalten werden, jedoch wird eine Vorlesung immer nur in genau einem
Hörsaal abgehalten.

Eine weitere Aufgabe der Professoren ist es, Übungen bereitzustellen.
Dies geschieht jedoch auf freiwilliger Basis, so dass nicht jeder
Professor Übungen für seine Studenten zur Verfügung stellt. Eine
Übung wird immer von genau einem Professor erstellt.

In bestimmten Zeitabständen wird ein Professor zum \enquote{Dekan}
gewählt. Der Dekan ist der Leiter einer Fakultät. Ein Professor kann
nur höchstens eine Dekanstelle besetzen und eine Fakultät wird immer
von genau einem Dekan geleitet.

Eine zweite Personengruppe in der Datenbank sind die Wissenschaftlichen
Mitarbeiter (im Folgenden nur noch WiMa genannt). WiMas betreuen immer
wieder Übungen, haben aber auch andere Aufgaben. Eine Übung kann nur
von einem WiMa betreut werden. Zusätzlich zu den genannten Daten, die
allgemein für einen Personentyp gespeichert werden, wird für WiMas
und Professoren das jeweilige Gehalt in der Datenbank abgelegt.

Der dritte Personentyp sind Studenten. Diese können an Übungen und
Vorlesungen teilnehmen. Damit eine Übung oder Vorlesung stattfindet,
muss sich mindestens ein Student dafür eingeschrieben haben. Studenten
erhalten, neben den angesprochenen Personenparametern, noch zusätzlich
eine Matrikelnummer.

Die an der Universität gehaltenen Vorlesungen und Übungen sind mit
einem Thema versehen. Die Übungen unterteilen sich in Rechen- bzw.
Laborübungen und werden mit einer Aufgabennummer versehen.
Rechenübungen finden in Hörsälen und Laborübungen in Laboren
statt. Nicht jeder Hörsaal bzw. jedes Labor ist immer besetzt. Eine
Übung findet allerdings immer nur in einem Raum statt. Für jeden
Hörsaal und jedes Labor muss die Anzahl der Sitzplätze, die
Raumnummer sowie die Gebäudenummer gespeichert werden.
\clearpage